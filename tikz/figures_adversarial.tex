% TikZ figures (limited palette, ERCOT-style)
% Use with: % TikZ figures (limited palette, ERCOT-style)
% Use with: % TikZ figures (limited palette, ERCOT-style)
% Use with: % TikZ figures (limited palette, ERCOT-style)
% Use with: \input{tikz/figures_adversarial}

% ----------------------------------------------------------------------
% 1) Price-Blind Stress Packet Illustration
% ----------------------------------------------------------------------
\begin{figure}[!ht]
\centering
\begin{tikzpicture}[x=1cm,y=1cm, font=\small]
  \draw[very thick] (0,0) -- (14,0);
  \node[below] at (0,0) {Time};

  \fill[LinkBlue!10] (3,-0.55) rectangle (11,0.55);
  \draw[thick] (3,-0.55) rectangle (11,0.55);

  \draw[thick] (3,-0.85) -- (3,0.85);
  \draw[thick] (11,-0.85) -- (11,0.85);
  \node[above] at (3,0.85) {Stress onset $t_0$};
  \node[above] at (11,0.85) {Stress end $t_1$};

  \draw[decorate,decoration={brace,amplitude=5pt}] (3,-1.2) -- (11,-1.2);
  \node[below] at (7,-1.2) {Duration $t_1-t_0$};

  \node[align=left,anchor=west] (cov) at (11.3,0.4) {\textbf{Covariates (price-blind)}\\
  Weather bin (temp / wind chill)\\
  Load level / net-load proxy\\
  Sustained reserve stress\\
  Persistence / contiguity};

  \draw[-{Latex[length=2mm]}, thick] (10.5,0.3) -- (11.2,0.35);

  \node[align=left,anchor=west] at (0,-2.05) {\textbf{Definition:} Stress packet is defined without prices; prices are analyzed \emph{within} the packet.\\
  (Axes/terms consistent with Appendix~H definitions.)};
\end{tikzpicture}
\caption{Price-blind stress packet illustration: a contiguous stress interval defined by physical triggers (weather/load + reserve stress) with annotated onset, duration, and covariates.}
\label{fig:price_blind_stress_packet}
\end{figure}

% ----------------------------------------------------------------------
% 2) Deployment Re-Entry CDF (conceptual)
% ----------------------------------------------------------------------
\begin{figure}[!ht]
\centering
\begin{tikzpicture}
\begin{axis}[
  width=0.88\textwidth,
  height=0.42\textwidth,
  xmin=0, xmax=6,
  ymin=0, ymax=1,
  axis lines=left,
  xlabel={Re-entry latency $\rho$ (normalized units)},
  ylabel={Empirical probability $\Pr(\rho\le x)$},
  xtick={0,1,2,3,4,5,6},
  ytick={0,0.25,0.5,0.75,1},
  grid=both,
  grid style={draw=black!12},
  tick label style={font=\small},
  label style={font=\small},
  legend style={draw=none, fill=none, font=\small, at={(0.02,0.98)}, anchor=north west},
]

\addplot[very thick, LinkBlue] coordinates {
  (0,0) (0.5,0.30) (1.0,0.55) (1.5,0.75) (2.0,0.88) (3.0,0.97) (6.0,1.0)
};
\addlegendentry{Strategic withholding (fast re-entry)}

\addplot[very thick, red!70!black, dashed] coordinates {
  (0,0) (1.0,0.08) (2.0,0.22) (3.0,0.45) (4.0,0.70) (5.0,0.90) (6.0,1.0)
};
\addlegendentry{Physical exhaustion (slow re-entry)}

\node[align=left,anchor=south west] at (axis cs:3.2,0.12) {\footnotesize Diagnostic idea:\\
Left-shifted CDF $\Rightarrow$ rapid re-entry\\
Right-shifted CDF $\Rightarrow$ prolonged absence};

\end{axis}
\end{tikzpicture}
\caption{Conceptual CDF schematic for deployment re-entry latency $\rho$: strategic withholding implies shorter re-entry latencies than physical exhaustion.}
\label{fig:deployment_reentry_cdf}
\end{figure}

% ----------------------------------------------------------------------
% 3) Degradation Envelope (canonical ESR)
% ----------------------------------------------------------------------
\begin{figure}[!ht]
\centering
\begin{tikzpicture}[x=1cm,y=1cm, font=\small]
  \draw[->, thick] (0,0) -- (13.2,0) node[below] {Degradation-cost case $c_{\mathrm{deg}}$};
  \draw[->, thick] (0,0) -- (0,5.2) node[left, rotate=90] {Revenue proxy $\Pi$};

  \def\xA{2.5}
  \def\xB{6.5}
  \def\xC{10.5}

  \fill[LinkBlue!10] (\xA-1,1.2) rectangle (\xA+1,4.2);
  \draw[thick] (\xA-1,1.2) rectangle (\xA+1,4.2);
  \draw[thick] (\xA-1,2.0) -- (\xA+1,2.0);
  \node[below] at (\xA,0) {$0.5\,c_{\mathrm{deg}}^{\mathrm{base}}$};

  \fill[LinkBlue!14] (\xB-1,1.0) rectangle (\xB+1,3.6);
  \draw[thick] (\xB-1,1.0) rectangle (\xB+1,3.6);
  \draw[thick] (\xB-1,1.8) -- (\xB+1,1.8);
  \node[below] at (\xB,0) {$c_{\mathrm{deg}}^{\mathrm{base}}$};

  \fill[LinkBlue!18] (\xC-1,0.6) rectangle (\xC+1,3.0);
  \draw[thick] (\xC-1,0.6) rectangle (\xC+1,3.0);
  \draw[thick] (\xC-1,1.4) -- (\xC+1,1.4);
  \node[below] at (\xC,0) {$2.0\,c_{\mathrm{deg}}^{\mathrm{base}}$};

  \node[left] at (\xA-1,1.2) {$\Pi_{\min}$};
  \node[left] at (\xA-1,4.2) {$\Pi_{\max}$};

  \draw[black!70, dotted, thick] (0.2,1.2) -- (13.0,1.2) node[right] {P10 zone};
  \draw[black!70, dotted, thick] (0.2,2.4) -- (13.0,2.4) node[right] {P50 zone};
  \draw[black!70, dotted, thick] (0.2,3.8) -- (13.0,3.8) node[right] {P90 zone};

  \node[align=left,anchor=west] at (0.4,4.85) {\textbf{Investor risk zones:}\\
  Downside (P10) sensitive to feasibility truncation\\
  Median (P50) baseline underwriting\\
  Upside (P90) tail capture};

\end{tikzpicture}
\caption{Degradation envelope schematic: revenue proxy envelopes $[\Pi_{\min},\Pi_{\max}]$ under alternate degradation-cost assumptions $c_{\mathrm{deg}}$, with conceptual P10/P50/P90 risk zones.}
\label{fig:degradation_envelope_schematic}
\end{figure}

% ----------------------------------------------------------------------
% 4) Geometry FP/FN Test Matrix
% ----------------------------------------------------------------------
\begin{figure}[!ht]
\centering
\begin{tikzpicture}[font=\small]
  \draw[thick] (0,0) rectangle (10,6);
  \draw[thick] (5,0) -- (5,6);
  \draw[thick] (0,3) -- (10,3);

  \node[align=center] at (2.5,6.45) {Geometry \textbf{Low}};
  \node[align=center] at (7.5,6.45) {Geometry \textbf{High}};
  \node[align=center,rotate=90] at (-0.65,1.5) {Scarcity \textbf{Absent}};
  \node[align=center,rotate=90] at (-0.65,4.5) {Scarcity \textbf{Present}};

  \fill[black!6] (0,0) rectangle (5,3);
  \fill[orange!12] (5,0) rectangle (10,3);
  \fill[red!10] (0,3) rectangle (5,6);
  \fill[LinkBlue!12] (5,3) rectangle (10,6);

  \node[align=left] at (2.5,1.5) {\textbf{TN}\\Geometry low,\\Scarcity absent};
  \node[align=left, text=orange!70!black] at (7.5,1.5) {\textbf{FP}\\Geometry high,\\Scarcity absent\\(``dog didn't bark'')};
  \node[align=left, text=red!70!black] at (2.5,4.5) {\textbf{FN}\\Geometry low,\\Scarcity present\\(missed stress)};
  \node[align=left, text=LinkBlue!90!black] at (7.5,4.5) {\textbf{TP}\\Geometry high,\\Scarcity present\\(signal confirmed)};

  \node[align=left,anchor=west] at (0, -0.7) {\footnotesize Diagnostic objective: quantify FP/FN rates under Appendix~H definitions of geometry and scarcity tails.};
\end{tikzpicture}
\caption{Geometry false-positive/false-negative diagnostic matrix: geometry state vs scarcity outcome state.}
\label{fig:geometry_fp_fn_matrix}
\end{figure}

% ----------------------------------------------------------------------
% 5) Basis Premium Tail Map (stylized)
% ----------------------------------------------------------------------
\begin{figure}[!ht]
\centering
\begin{tikzpicture}[x=1cm,y=1cm, font=\small]
  \draw[->, thick] (0,0) -- (13.2,0) node[below] {Node / settlement point $n$ (stylized)};
  \draw[->, thick] (0,-3.0) -- (0,3.2) node[left, rotate=90] {Basis premium $\beta_{n,t}$};

  \draw[black!70, dashed, thick] (0.2,2.0) -- (13.0,2.0) node[right] {$+\tau_{\beta}$ (tail)};
  \draw[black!70, dashed, thick] (0.2,-2.0) -- (13.0,-2.0) node[right] {$-\tau_{\beta}$ (tail)};

  \foreach \i/\h in {1/0.6,2/1.2,3/2.4,4/0.9,5/-0.8,6/-2.3,7/-1.1,8/0.4,9/1.8,10/2.6,11/-0.5,12/-1.9} {
    \ifdim \h pt > 0pt
      \fill[LinkBlue!22] (\i,0) rectangle (\i+0.6,\h);
      \draw[black!60] (\i,0) rectangle (\i+0.6,\h);
    \else
      \fill[red!12] (\i,0) rectangle (\i+0.6,\h);
      \draw[black!60] (\i,0) rectangle (\i+0.6,\h);
    \fi
  }

  \node[align=left,anchor=west] at (0.4,2.55) {\textbf{Upper-tail nodes:}\\$\beta_{n,t} > +\tau_{\beta}$};
  \node[align=left,anchor=west] at (0.4,-2.75) {\textbf{Lower-tail nodes:}\\$\beta_{n,t} < -\tau_{\beta}$};

  \node[align=left,anchor=west] at (0.4,1.55) {\footnotesize Use with: basis tail frequency and directional skew diagnostics (Appendix~H).};
\end{tikzpicture}
\caption{Stylized basis premium tail map: node-level basis premiums $\beta_{n,t}$ with symmetric tail thresholds $\pm\tau_{\beta}$ for nodal basis-risk diagnostics.}
\label{fig:basis_premium_tail_map}
\end{figure}


% ----------------------------------------------------------------------
% 1) Price-Blind Stress Packet Illustration
% ----------------------------------------------------------------------
\begin{figure}[!ht]
\centering
\begin{tikzpicture}[x=1cm,y=1cm, font=\small]
  \draw[very thick] (0,0) -- (14,0);
  \node[below] at (0,0) {Time};

  \fill[LinkBlue!10] (3,-0.55) rectangle (11,0.55);
  \draw[thick] (3,-0.55) rectangle (11,0.55);

  \draw[thick] (3,-0.85) -- (3,0.85);
  \draw[thick] (11,-0.85) -- (11,0.85);
  \node[above] at (3,0.85) {Stress onset $t_0$};
  \node[above] at (11,0.85) {Stress end $t_1$};

  \draw[decorate,decoration={brace,amplitude=5pt}] (3,-1.2) -- (11,-1.2);
  \node[below] at (7,-1.2) {Duration $t_1-t_0$};

  \node[align=left,anchor=west] (cov) at (11.3,0.4) {\textbf{Covariates (price-blind)}\\
  Weather bin (temp / wind chill)\\
  Load level / net-load proxy\\
  Sustained reserve stress\\
  Persistence / contiguity};

  \draw[-{Latex[length=2mm]}, thick] (10.5,0.3) -- (11.2,0.35);

  \node[align=left,anchor=west] at (0,-2.05) {\textbf{Definition:} Stress packet is defined without prices; prices are analyzed \emph{within} the packet.\\
  (Axes/terms consistent with Appendix~H definitions.)};
\end{tikzpicture}
\caption{Price-blind stress packet illustration: a contiguous stress interval defined by physical triggers (weather/load + reserve stress) with annotated onset, duration, and covariates.}
\label{fig:price_blind_stress_packet}
\end{figure}

% ----------------------------------------------------------------------
% 2) Deployment Re-Entry CDF (conceptual)
% ----------------------------------------------------------------------
\begin{figure}[!ht]
\centering
\begin{tikzpicture}
\begin{axis}[
  width=0.88\textwidth,
  height=0.42\textwidth,
  xmin=0, xmax=6,
  ymin=0, ymax=1,
  axis lines=left,
  xlabel={Re-entry latency $\rho$ (normalized units)},
  ylabel={Empirical probability $\Pr(\rho\le x)$},
  xtick={0,1,2,3,4,5,6},
  ytick={0,0.25,0.5,0.75,1},
  grid=both,
  grid style={draw=black!12},
  tick label style={font=\small},
  label style={font=\small},
  legend style={draw=none, fill=none, font=\small, at={(0.02,0.98)}, anchor=north west},
]

\addplot[very thick, LinkBlue] coordinates {
  (0,0) (0.5,0.30) (1.0,0.55) (1.5,0.75) (2.0,0.88) (3.0,0.97) (6.0,1.0)
};
\addlegendentry{Strategic withholding (fast re-entry)}

\addplot[very thick, red!70!black, dashed] coordinates {
  (0,0) (1.0,0.08) (2.0,0.22) (3.0,0.45) (4.0,0.70) (5.0,0.90) (6.0,1.0)
};
\addlegendentry{Physical exhaustion (slow re-entry)}

\node[align=left,anchor=south west] at (axis cs:3.2,0.12) {\footnotesize Diagnostic idea:\\
Left-shifted CDF $\Rightarrow$ rapid re-entry\\
Right-shifted CDF $\Rightarrow$ prolonged absence};

\end{axis}
\end{tikzpicture}
\caption{Conceptual CDF schematic for deployment re-entry latency $\rho$: strategic withholding implies shorter re-entry latencies than physical exhaustion.}
\label{fig:deployment_reentry_cdf}
\end{figure}

% ----------------------------------------------------------------------
% 3) Degradation Envelope (canonical ESR)
% ----------------------------------------------------------------------
\begin{figure}[!ht]
\centering
\begin{tikzpicture}[x=1cm,y=1cm, font=\small]
  \draw[->, thick] (0,0) -- (13.2,0) node[below] {Degradation-cost case $c_{\mathrm{deg}}$};
  \draw[->, thick] (0,0) -- (0,5.2) node[left, rotate=90] {Revenue proxy $\Pi$};

  \def\xA{2.5}
  \def\xB{6.5}
  \def\xC{10.5}

  \fill[LinkBlue!10] (\xA-1,1.2) rectangle (\xA+1,4.2);
  \draw[thick] (\xA-1,1.2) rectangle (\xA+1,4.2);
  \draw[thick] (\xA-1,2.0) -- (\xA+1,2.0);
  \node[below] at (\xA,0) {$0.5\,c_{\mathrm{deg}}^{\mathrm{base}}$};

  \fill[LinkBlue!14] (\xB-1,1.0) rectangle (\xB+1,3.6);
  \draw[thick] (\xB-1,1.0) rectangle (\xB+1,3.6);
  \draw[thick] (\xB-1,1.8) -- (\xB+1,1.8);
  \node[below] at (\xB,0) {$c_{\mathrm{deg}}^{\mathrm{base}}$};

  \fill[LinkBlue!18] (\xC-1,0.6) rectangle (\xC+1,3.0);
  \draw[thick] (\xC-1,0.6) rectangle (\xC+1,3.0);
  \draw[thick] (\xC-1,1.4) -- (\xC+1,1.4);
  \node[below] at (\xC,0) {$2.0\,c_{\mathrm{deg}}^{\mathrm{base}}$};

  \node[left] at (\xA-1,1.2) {$\Pi_{\min}$};
  \node[left] at (\xA-1,4.2) {$\Pi_{\max}$};

  \draw[black!70, dotted, thick] (0.2,1.2) -- (13.0,1.2) node[right] {P10 zone};
  \draw[black!70, dotted, thick] (0.2,2.4) -- (13.0,2.4) node[right] {P50 zone};
  \draw[black!70, dotted, thick] (0.2,3.8) -- (13.0,3.8) node[right] {P90 zone};

  \node[align=left,anchor=west] at (0.4,4.85) {\textbf{Investor risk zones:}\\
  Downside (P10) sensitive to feasibility truncation\\
  Median (P50) baseline underwriting\\
  Upside (P90) tail capture};

\end{tikzpicture}
\caption{Degradation envelope schematic: revenue proxy envelopes $[\Pi_{\min},\Pi_{\max}]$ under alternate degradation-cost assumptions $c_{\mathrm{deg}}$, with conceptual P10/P50/P90 risk zones.}
\label{fig:degradation_envelope_schematic}
\end{figure}

% ----------------------------------------------------------------------
% 4) Geometry FP/FN Test Matrix
% ----------------------------------------------------------------------
\begin{figure}[!ht]
\centering
\begin{tikzpicture}[font=\small]
  \draw[thick] (0,0) rectangle (10,6);
  \draw[thick] (5,0) -- (5,6);
  \draw[thick] (0,3) -- (10,3);

  \node[align=center] at (2.5,6.45) {Geometry \textbf{Low}};
  \node[align=center] at (7.5,6.45) {Geometry \textbf{High}};
  \node[align=center,rotate=90] at (-0.65,1.5) {Scarcity \textbf{Absent}};
  \node[align=center,rotate=90] at (-0.65,4.5) {Scarcity \textbf{Present}};

  \fill[black!6] (0,0) rectangle (5,3);
  \fill[orange!12] (5,0) rectangle (10,3);
  \fill[red!10] (0,3) rectangle (5,6);
  \fill[LinkBlue!12] (5,3) rectangle (10,6);

  \node[align=left] at (2.5,1.5) {\textbf{TN}\\Geometry low,\\Scarcity absent};
  \node[align=left, text=orange!70!black] at (7.5,1.5) {\textbf{FP}\\Geometry high,\\Scarcity absent\\(``dog didn't bark'')};
  \node[align=left, text=red!70!black] at (2.5,4.5) {\textbf{FN}\\Geometry low,\\Scarcity present\\(missed stress)};
  \node[align=left, text=LinkBlue!90!black] at (7.5,4.5) {\textbf{TP}\\Geometry high,\\Scarcity present\\(signal confirmed)};

  \node[align=left,anchor=west] at (0, -0.7) {\footnotesize Diagnostic objective: quantify FP/FN rates under Appendix~H definitions of geometry and scarcity tails.};
\end{tikzpicture}
\caption{Geometry false-positive/false-negative diagnostic matrix: geometry state vs scarcity outcome state.}
\label{fig:geometry_fp_fn_matrix}
\end{figure}

% ----------------------------------------------------------------------
% 5) Basis Premium Tail Map (stylized)
% ----------------------------------------------------------------------
\begin{figure}[!ht]
\centering
\begin{tikzpicture}[x=1cm,y=1cm, font=\small]
  \draw[->, thick] (0,0) -- (13.2,0) node[below] {Node / settlement point $n$ (stylized)};
  \draw[->, thick] (0,-3.0) -- (0,3.2) node[left, rotate=90] {Basis premium $\beta_{n,t}$};

  \draw[black!70, dashed, thick] (0.2,2.0) -- (13.0,2.0) node[right] {$+\tau_{\beta}$ (tail)};
  \draw[black!70, dashed, thick] (0.2,-2.0) -- (13.0,-2.0) node[right] {$-\tau_{\beta}$ (tail)};

  \foreach \i/\h in {1/0.6,2/1.2,3/2.4,4/0.9,5/-0.8,6/-2.3,7/-1.1,8/0.4,9/1.8,10/2.6,11/-0.5,12/-1.9} {
    \ifdim \h pt > 0pt
      \fill[LinkBlue!22] (\i,0) rectangle (\i+0.6,\h);
      \draw[black!60] (\i,0) rectangle (\i+0.6,\h);
    \else
      \fill[red!12] (\i,0) rectangle (\i+0.6,\h);
      \draw[black!60] (\i,0) rectangle (\i+0.6,\h);
    \fi
  }

  \node[align=left,anchor=west] at (0.4,2.55) {\textbf{Upper-tail nodes:}\\$\beta_{n,t} > +\tau_{\beta}$};
  \node[align=left,anchor=west] at (0.4,-2.75) {\textbf{Lower-tail nodes:}\\$\beta_{n,t} < -\tau_{\beta}$};

  \node[align=left,anchor=west] at (0.4,1.55) {\footnotesize Use with: basis tail frequency and directional skew diagnostics (Appendix~H).};
\end{tikzpicture}
\caption{Stylized basis premium tail map: node-level basis premiums $\beta_{n,t}$ with symmetric tail thresholds $\pm\tau_{\beta}$ for nodal basis-risk diagnostics.}
\label{fig:basis_premium_tail_map}
\end{figure}


% ----------------------------------------------------------------------
% 1) Price-Blind Stress Packet Illustration
% ----------------------------------------------------------------------
\begin{figure}[!ht]
\centering
\begin{tikzpicture}[x=1cm,y=1cm, font=\small]
  \draw[very thick] (0,0) -- (14,0);
  \node[below] at (0,0) {Time};

  \fill[LinkBlue!10] (3,-0.55) rectangle (11,0.55);
  \draw[thick] (3,-0.55) rectangle (11,0.55);

  \draw[thick] (3,-0.85) -- (3,0.85);
  \draw[thick] (11,-0.85) -- (11,0.85);
  \node[above] at (3,0.85) {Stress onset $t_0$};
  \node[above] at (11,0.85) {Stress end $t_1$};

  \draw[decorate,decoration={brace,amplitude=5pt}] (3,-1.2) -- (11,-1.2);
  \node[below] at (7,-1.2) {Duration $t_1-t_0$};

  \node[align=left,anchor=west] (cov) at (11.3,0.4) {\textbf{Covariates (price-blind)}\\
  Weather bin (temp / wind chill)\\
  Load level / net-load proxy\\
  Sustained reserve stress\\
  Persistence / contiguity};

  \draw[-{Latex[length=2mm]}, thick] (10.5,0.3) -- (11.2,0.35);

  \node[align=left,anchor=west] at (0,-2.05) {\textbf{Definition:} Stress packet is defined without prices; prices are analyzed \emph{within} the packet.\\
  (Axes/terms consistent with Appendix~H definitions.)};
\end{tikzpicture}
\caption{Price-blind stress packet illustration: a contiguous stress interval defined by physical triggers (weather/load + reserve stress) with annotated onset, duration, and covariates.}
\label{fig:price_blind_stress_packet}
\end{figure}

% ----------------------------------------------------------------------
% 2) Deployment Re-Entry CDF (conceptual)
% ----------------------------------------------------------------------
\begin{figure}[!ht]
\centering
\begin{tikzpicture}
\begin{axis}[
  width=0.88\textwidth,
  height=0.42\textwidth,
  xmin=0, xmax=6,
  ymin=0, ymax=1,
  axis lines=left,
  xlabel={Re-entry latency $\rho$ (normalized units)},
  ylabel={Empirical probability $\Pr(\rho\le x)$},
  xtick={0,1,2,3,4,5,6},
  ytick={0,0.25,0.5,0.75,1},
  grid=both,
  grid style={draw=black!12},
  tick label style={font=\small},
  label style={font=\small},
  legend style={draw=none, fill=none, font=\small, at={(0.02,0.98)}, anchor=north west},
]

\addplot[very thick, LinkBlue] coordinates {
  (0,0) (0.5,0.30) (1.0,0.55) (1.5,0.75) (2.0,0.88) (3.0,0.97) (6.0,1.0)
};
\addlegendentry{Strategic withholding (fast re-entry)}

\addplot[very thick, red!70!black, dashed] coordinates {
  (0,0) (1.0,0.08) (2.0,0.22) (3.0,0.45) (4.0,0.70) (5.0,0.90) (6.0,1.0)
};
\addlegendentry{Physical exhaustion (slow re-entry)}

\node[align=left,anchor=south west] at (axis cs:3.2,0.12) {\footnotesize Diagnostic idea:\\
Left-shifted CDF $\Rightarrow$ rapid re-entry\\
Right-shifted CDF $\Rightarrow$ prolonged absence};

\end{axis}
\end{tikzpicture}
\caption{Conceptual CDF schematic for deployment re-entry latency $\rho$: strategic withholding implies shorter re-entry latencies than physical exhaustion.}
\label{fig:deployment_reentry_cdf}
\end{figure}

% ----------------------------------------------------------------------
% 3) Degradation Envelope (canonical ESR)
% ----------------------------------------------------------------------
\begin{figure}[!ht]
\centering
\begin{tikzpicture}[x=1cm,y=1cm, font=\small]
  \draw[->, thick] (0,0) -- (13.2,0) node[below] {Degradation-cost case $c_{\mathrm{deg}}$};
  \draw[->, thick] (0,0) -- (0,5.2) node[left, rotate=90] {Revenue proxy $\Pi$};

  \def\xA{2.5}
  \def\xB{6.5}
  \def\xC{10.5}

  \fill[LinkBlue!10] (\xA-1,1.2) rectangle (\xA+1,4.2);
  \draw[thick] (\xA-1,1.2) rectangle (\xA+1,4.2);
  \draw[thick] (\xA-1,2.0) -- (\xA+1,2.0);
  \node[below] at (\xA,0) {$0.5\,c_{\mathrm{deg}}^{\mathrm{base}}$};

  \fill[LinkBlue!14] (\xB-1,1.0) rectangle (\xB+1,3.6);
  \draw[thick] (\xB-1,1.0) rectangle (\xB+1,3.6);
  \draw[thick] (\xB-1,1.8) -- (\xB+1,1.8);
  \node[below] at (\xB,0) {$c_{\mathrm{deg}}^{\mathrm{base}}$};

  \fill[LinkBlue!18] (\xC-1,0.6) rectangle (\xC+1,3.0);
  \draw[thick] (\xC-1,0.6) rectangle (\xC+1,3.0);
  \draw[thick] (\xC-1,1.4) -- (\xC+1,1.4);
  \node[below] at (\xC,0) {$2.0\,c_{\mathrm{deg}}^{\mathrm{base}}$};

  \node[left] at (\xA-1,1.2) {$\Pi_{\min}$};
  \node[left] at (\xA-1,4.2) {$\Pi_{\max}$};

  \draw[black!70, dotted, thick] (0.2,1.2) -- (13.0,1.2) node[right] {P10 zone};
  \draw[black!70, dotted, thick] (0.2,2.4) -- (13.0,2.4) node[right] {P50 zone};
  \draw[black!70, dotted, thick] (0.2,3.8) -- (13.0,3.8) node[right] {P90 zone};

  \node[align=left,anchor=west] at (0.4,4.85) {\textbf{Investor risk zones:}\\
  Downside (P10) sensitive to feasibility truncation\\
  Median (P50) baseline underwriting\\
  Upside (P90) tail capture};

\end{tikzpicture}
\caption{Degradation envelope schematic: revenue proxy envelopes $[\Pi_{\min},\Pi_{\max}]$ under alternate degradation-cost assumptions $c_{\mathrm{deg}}$, with conceptual P10/P50/P90 risk zones.}
\label{fig:degradation_envelope_schematic}
\end{figure}

% ----------------------------------------------------------------------
% 4) Geometry FP/FN Test Matrix
% ----------------------------------------------------------------------
\begin{figure}[!ht]
\centering
\begin{tikzpicture}[font=\small]
  \draw[thick] (0,0) rectangle (10,6);
  \draw[thick] (5,0) -- (5,6);
  \draw[thick] (0,3) -- (10,3);

  \node[align=center] at (2.5,6.45) {Geometry \textbf{Low}};
  \node[align=center] at (7.5,6.45) {Geometry \textbf{High}};
  \node[align=center,rotate=90] at (-0.65,1.5) {Scarcity \textbf{Absent}};
  \node[align=center,rotate=90] at (-0.65,4.5) {Scarcity \textbf{Present}};

  \fill[black!6] (0,0) rectangle (5,3);
  \fill[orange!12] (5,0) rectangle (10,3);
  \fill[red!10] (0,3) rectangle (5,6);
  \fill[LinkBlue!12] (5,3) rectangle (10,6);

  \node[align=left] at (2.5,1.5) {\textbf{TN}\\Geometry low,\\Scarcity absent};
  \node[align=left, text=orange!70!black] at (7.5,1.5) {\textbf{FP}\\Geometry high,\\Scarcity absent\\(``dog didn't bark'')};
  \node[align=left, text=red!70!black] at (2.5,4.5) {\textbf{FN}\\Geometry low,\\Scarcity present\\(missed stress)};
  \node[align=left, text=LinkBlue!90!black] at (7.5,4.5) {\textbf{TP}\\Geometry high,\\Scarcity present\\(signal confirmed)};

  \node[align=left,anchor=west] at (0, -0.7) {\footnotesize Diagnostic objective: quantify FP/FN rates under Appendix~H definitions of geometry and scarcity tails.};
\end{tikzpicture}
\caption{Geometry false-positive/false-negative diagnostic matrix: geometry state vs scarcity outcome state.}
\label{fig:geometry_fp_fn_matrix}
\end{figure}

% ----------------------------------------------------------------------
% 5) Basis Premium Tail Map (stylized)
% ----------------------------------------------------------------------
\begin{figure}[!ht]
\centering
\begin{tikzpicture}[x=1cm,y=1cm, font=\small]
  \draw[->, thick] (0,0) -- (13.2,0) node[below] {Node / settlement point $n$ (stylized)};
  \draw[->, thick] (0,-3.0) -- (0,3.2) node[left, rotate=90] {Basis premium $\beta_{n,t}$};

  \draw[black!70, dashed, thick] (0.2,2.0) -- (13.0,2.0) node[right] {$+\tau_{\beta}$ (tail)};
  \draw[black!70, dashed, thick] (0.2,-2.0) -- (13.0,-2.0) node[right] {$-\tau_{\beta}$ (tail)};

  \foreach \i/\h in {1/0.6,2/1.2,3/2.4,4/0.9,5/-0.8,6/-2.3,7/-1.1,8/0.4,9/1.8,10/2.6,11/-0.5,12/-1.9} {
    \ifdim \h pt > 0pt
      \fill[LinkBlue!22] (\i,0) rectangle (\i+0.6,\h);
      \draw[black!60] (\i,0) rectangle (\i+0.6,\h);
    \else
      \fill[red!12] (\i,0) rectangle (\i+0.6,\h);
      \draw[black!60] (\i,0) rectangle (\i+0.6,\h);
    \fi
  }

  \node[align=left,anchor=west] at (0.4,2.55) {\textbf{Upper-tail nodes:}\\$\beta_{n,t} > +\tau_{\beta}$};
  \node[align=left,anchor=west] at (0.4,-2.75) {\textbf{Lower-tail nodes:}\\$\beta_{n,t} < -\tau_{\beta}$};

  \node[align=left,anchor=west] at (0.4,1.55) {\footnotesize Use with: basis tail frequency and directional skew diagnostics (Appendix~H).};
\end{tikzpicture}
\caption{Stylized basis premium tail map: node-level basis premiums $\beta_{n,t}$ with symmetric tail thresholds $\pm\tau_{\beta}$ for nodal basis-risk diagnostics.}
\label{fig:basis_premium_tail_map}
\end{figure}


% ----------------------------------------------------------------------
% 1) Price-Blind Stress Packet Illustration
% ----------------------------------------------------------------------
\begin{figure}[!ht]
\centering
\begin{tikzpicture}[x=1cm,y=1cm, font=\small]
  \draw[very thick] (0,0) -- (14,0);
  \node[below] at (0,0) {Time};

  \fill[LinkBlue!10] (3,-0.55) rectangle (11,0.55);
  \draw[thick] (3,-0.55) rectangle (11,0.55);

  \draw[thick] (3,-0.85) -- (3,0.85);
  \draw[thick] (11,-0.85) -- (11,0.85);
  \node[above] at (3,0.85) {Stress onset $t_0$};
  \node[above] at (11,0.85) {Stress end $t_1$};

  \draw[decorate,decoration={brace,amplitude=5pt}] (3,-1.2) -- (11,-1.2);
  \node[below] at (7,-1.2) {Duration $t_1-t_0$};

  \node[align=left,anchor=west] (cov) at (11.3,0.4) {\textbf{Covariates (price-blind)}\\
  Weather bin (temp / wind chill)\\
  Load level / net-load proxy\\
  Sustained reserve stress\\
  Persistence / contiguity};

  \draw[-{Latex[length=2mm]}, thick] (10.5,0.3) -- (11.2,0.35);

  \node[align=left,anchor=west] at (0,-2.05) {\textbf{Definition:} Stress packet is defined without prices; prices are analyzed \emph{within} the packet.\\
  (Axes/terms consistent with Appendix~H definitions.)};
\end{tikzpicture}
\caption{Price-blind stress packet illustration: a contiguous stress interval defined by physical triggers (weather/load + reserve stress) with annotated onset, duration, and covariates.}
\label{fig:price_blind_stress_packet}
\end{figure}

% ----------------------------------------------------------------------
% 2) Deployment Re-Entry CDF (conceptual)
% ----------------------------------------------------------------------
\begin{figure}[!ht]
\centering
\begin{tikzpicture}
\begin{axis}[
  width=0.88\textwidth,
  height=0.42\textwidth,
  xmin=0, xmax=6,
  ymin=0, ymax=1,
  axis lines=left,
  xlabel={Re-entry latency $\rho$ (normalized units)},
  ylabel={Empirical probability $\Pr(\rho\le x)$},
  xtick={0,1,2,3,4,5,6},
  ytick={0,0.25,0.5,0.75,1},
  grid=both,
  grid style={draw=black!12},
  tick label style={font=\small},
  label style={font=\small},
  legend style={draw=none, fill=none, font=\small, at={(0.02,0.98)}, anchor=north west},
]

\addplot[very thick, LinkBlue] coordinates {
  (0,0) (0.5,0.30) (1.0,0.55) (1.5,0.75) (2.0,0.88) (3.0,0.97) (6.0,1.0)
};
\addlegendentry{Strategic withholding (fast re-entry)}

\addplot[very thick, red!70!black, dashed] coordinates {
  (0,0) (1.0,0.08) (2.0,0.22) (3.0,0.45) (4.0,0.70) (5.0,0.90) (6.0,1.0)
};
\addlegendentry{Physical exhaustion (slow re-entry)}

\node[align=left,anchor=south west] at (axis cs:3.2,0.12) {\footnotesize Diagnostic idea:\\
Left-shifted CDF $\Rightarrow$ rapid re-entry\\
Right-shifted CDF $\Rightarrow$ prolonged absence};

\end{axis}
\end{tikzpicture}
\caption{Conceptual CDF schematic for deployment re-entry latency $\rho$: strategic withholding implies shorter re-entry latencies than physical exhaustion.}
\label{fig:deployment_reentry_cdf}
\end{figure}

% ----------------------------------------------------------------------
% 3) Degradation Envelope (canonical ESR)
% ----------------------------------------------------------------------
\begin{figure}[!ht]
\centering
\begin{tikzpicture}[x=1cm,y=1cm, font=\small]
  \draw[->, thick] (0,0) -- (13.2,0) node[below] {Degradation-cost case $c_{\mathrm{deg}}$};
  \draw[->, thick] (0,0) -- (0,5.2) node[left, rotate=90] {Revenue proxy $\Pi$};

  \def\xA{2.5}
  \def\xB{6.5}
  \def\xC{10.5}

  \fill[LinkBlue!10] (\xA-1,1.2) rectangle (\xA+1,4.2);
  \draw[thick] (\xA-1,1.2) rectangle (\xA+1,4.2);
  \draw[thick] (\xA-1,2.0) -- (\xA+1,2.0);
  \node[below] at (\xA,0) {$0.5\,c_{\mathrm{deg}}^{\mathrm{base}}$};

  \fill[LinkBlue!14] (\xB-1,1.0) rectangle (\xB+1,3.6);
  \draw[thick] (\xB-1,1.0) rectangle (\xB+1,3.6);
  \draw[thick] (\xB-1,1.8) -- (\xB+1,1.8);
  \node[below] at (\xB,0) {$c_{\mathrm{deg}}^{\mathrm{base}}$};

  \fill[LinkBlue!18] (\xC-1,0.6) rectangle (\xC+1,3.0);
  \draw[thick] (\xC-1,0.6) rectangle (\xC+1,3.0);
  \draw[thick] (\xC-1,1.4) -- (\xC+1,1.4);
  \node[below] at (\xC,0) {$2.0\,c_{\mathrm{deg}}^{\mathrm{base}}$};

  \node[left] at (\xA-1,1.2) {$\Pi_{\min}$};
  \node[left] at (\xA-1,4.2) {$\Pi_{\max}$};

  \draw[black!70, dotted, thick] (0.2,1.2) -- (13.0,1.2) node[right] {P10 zone};
  \draw[black!70, dotted, thick] (0.2,2.4) -- (13.0,2.4) node[right] {P50 zone};
  \draw[black!70, dotted, thick] (0.2,3.8) -- (13.0,3.8) node[right] {P90 zone};

  \node[align=left,anchor=west] at (0.4,4.85) {\textbf{Investor risk zones:}\\
  Downside (P10) sensitive to feasibility truncation\\
  Median (P50) baseline underwriting\\
  Upside (P90) tail capture};

\end{tikzpicture}
\caption{Degradation envelope schematic: revenue proxy envelopes $[\Pi_{\min},\Pi_{\max}]$ under alternate degradation-cost assumptions $c_{\mathrm{deg}}$, with conceptual P10/P50/P90 risk zones.}
\label{fig:degradation_envelope_schematic}
\end{figure}

% ----------------------------------------------------------------------
% 4) Geometry FP/FN Test Matrix
% ----------------------------------------------------------------------
\begin{figure}[!ht]
\centering
\begin{tikzpicture}[font=\small]
  \draw[thick] (0,0) rectangle (10,6);
  \draw[thick] (5,0) -- (5,6);
  \draw[thick] (0,3) -- (10,3);

  \node[align=center] at (2.5,6.45) {Geometry \textbf{Low}};
  \node[align=center] at (7.5,6.45) {Geometry \textbf{High}};
  \node[align=center,rotate=90] at (-0.65,1.5) {Scarcity \textbf{Absent}};
  \node[align=center,rotate=90] at (-0.65,4.5) {Scarcity \textbf{Present}};

  \fill[black!6] (0,0) rectangle (5,3);
  \fill[orange!12] (5,0) rectangle (10,3);
  \fill[red!10] (0,3) rectangle (5,6);
  \fill[LinkBlue!12] (5,3) rectangle (10,6);

  \node[align=left] at (2.5,1.5) {\textbf{TN}\\Geometry low,\\Scarcity absent};
  \node[align=left, text=orange!70!black] at (7.5,1.5) {\textbf{FP}\\Geometry high,\\Scarcity absent\\(``dog didn't bark'')};
  \node[align=left, text=red!70!black] at (2.5,4.5) {\textbf{FN}\\Geometry low,\\Scarcity present\\(missed stress)};
  \node[align=left, text=LinkBlue!90!black] at (7.5,4.5) {\textbf{TP}\\Geometry high,\\Scarcity present\\(signal confirmed)};

  \node[align=left,anchor=west] at (0, -0.7) {\footnotesize Diagnostic objective: quantify FP/FN rates under Appendix~H definitions of geometry and scarcity tails.};
\end{tikzpicture}
\caption{Geometry false-positive/false-negative diagnostic matrix: geometry state vs scarcity outcome state.}
\label{fig:geometry_fp_fn_matrix}
\end{figure}

% ----------------------------------------------------------------------
% 5) Basis Premium Tail Map (stylized)
% ----------------------------------------------------------------------
\begin{figure}[!ht]
\centering
\begin{tikzpicture}[x=1cm,y=1cm, font=\small]
  \draw[->, thick] (0,0) -- (13.2,0) node[below] {Node / settlement point $n$ (stylized)};
  \draw[->, thick] (0,-3.0) -- (0,3.2) node[left, rotate=90] {Basis premium $\beta_{n,t}$};

  \draw[black!70, dashed, thick] (0.2,2.0) -- (13.0,2.0) node[right] {$+\tau_{\beta}$ (tail)};
  \draw[black!70, dashed, thick] (0.2,-2.0) -- (13.0,-2.0) node[right] {$-\tau_{\beta}$ (tail)};

  \foreach \i/\h in {1/0.6,2/1.2,3/2.4,4/0.9,5/-0.8,6/-2.3,7/-1.1,8/0.4,9/1.8,10/2.6,11/-0.5,12/-1.9} {
    \ifdim \h pt > 0pt
      \fill[LinkBlue!22] (\i,0) rectangle (\i+0.6,\h);
      \draw[black!60] (\i,0) rectangle (\i+0.6,\h);
    \else
      \fill[red!12] (\i,0) rectangle (\i+0.6,\h);
      \draw[black!60] (\i,0) rectangle (\i+0.6,\h);
    \fi
  }

  \node[align=left,anchor=west] at (0.4,2.55) {\textbf{Upper-tail nodes:}\\$\beta_{n,t} > +\tau_{\beta}$};
  \node[align=left,anchor=west] at (0.4,-2.75) {\textbf{Lower-tail nodes:}\\$\beta_{n,t} < -\tau_{\beta}$};

  \node[align=left,anchor=west] at (0.4,1.55) {\footnotesize Use with: basis tail frequency and directional skew diagnostics (Appendix~H).};
\end{tikzpicture}
\caption{Stylized basis premium tail map: node-level basis premiums $\beta_{n,t}$ with symmetric tail thresholds $\pm\tau_{\beta}$ for nodal basis-risk diagnostics.}
\label{fig:basis_premium_tail_map}
\end{figure}
