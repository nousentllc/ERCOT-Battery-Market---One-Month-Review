% Stack Geometry Visual (side-by-side supply stacks)

\begin{figure}[!ht]
\centering
\begin{tikzpicture}
\begin{groupplot}[
  group style={group size=2 by 1, horizontal sep=1.2cm},
  width=0.46\textwidth,
  height=0.42\textwidth,
  xmin=0, xmax=100,
  ymin=0, ymax=100,
  axis lines=left,
  grid=both,
  grid style={draw=black!12},
  tick label style={font=\small},
  label style={font=\small},
  xlabel={Awarded / cleared quantity (MW, stylized)},
  ylabel={Marginal award price (\$/MW-hr, stylized)},
]

% -----------------------------
% Panel A: Healthy / broad-based
% -----------------------------
\nextgroupplot[
  title={Panel A: Broad-based (``healthy'') stack},
]

% A smooth, gradually steepening curve
\addplot[very thick, LinkBlue] coordinates {
  (0,5) (10,8) (20,12) (30,18) (40,26) (50,36) (60,48) (70,62) (80,78) (90,92) (100,98)
};

\node[anchor=north west, align=left, font=\footnotesize, text=LinkBlue!90!black] at (axis cs:2,98)
{Many blocks\\Low $H_k$\\Small shocks $\Rightarrow$\\small price moves};

% A small demand shock arrow
\draw[-{Latex[length=2mm]}, thick, LinkBlue!80!black] (axis cs:65,48) -- (axis cs:72,55);
\node[font=\footnotesize, anchor=west, text=LinkBlue!80!black] at (axis cs:73,55) {$\Delta q$ small};

% -----------------------------
% Panel B: Fragile / hockey stick
% -----------------------------
\nextgroupplot[
  title={Panel B: Concentrated (``hockey stick'') stack},
  ylabel={},
]

% Flat then near-vertical wall
\addplot[very thick, red!70!black] coordinates {
  (0,6) (20,6) (40,7) (60,8) (75,10) (85,18) (90,35) (92,55) (94,75) (96,90) (98,98) (100,100)
};

% Feasibility wall marker
\draw[orange!70!black, dashed, thick] (axis cs:92,0) -- (axis cs:92,100);
\node[font=\footnotesize, anchor=south west, align=left, text=orange!70!black] at (axis cs:92.5,12)
{``Feasibility wall''\\(vertical jump)};

\node[anchor=north west, align=left, font=\footnotesize, text=red!70!black] at (axis cs:2,98)
{Few blocks\\High $H_k$\\Small shocks $\Rightarrow$\\large price jump};

% Same small demand shock arrow (causes big price move)
\draw[-{Latex[length=2mm]}, thick, red!70!black] (axis cs:89,35) -- (axis cs:93,75);
\node[font=\footnotesize, anchor=west, text=red!70!black] at (axis cs:93.2,75) {$\Delta q$ small};

\end{groupplot}
\end{tikzpicture}
\caption{Figure 9.1: The evolution of stack geometry. During Fern, the day-ahead ancillary award stack can transition from a robust, gradual slope (left) to a brittle, high-concentration ``hockey stick'' (right), increasing the sensitivity of prices to moderate quantity shocks. This figure illustrates the convexity/concentration logic behind the Herfindahl index $H_k$ used in the stack-shape diagnostics.}
\label{fig:stack_geometry_evolution}
\end{figure}% Stack Geometry Visual (side-by-side supply stacks)

\begin{figure}[!ht]
\centering
\begin{tikzpicture}
\begin{groupplot}[
  group style={group size=2 by 1, horizontal sep=1.2cm},
  width=0.46\textwidth,
  height=0.42\textwidth,
  xmin=0, xmax=100,
  ymin=0, ymax=100,
  axis lines=left,
  grid=both,
  grid style={draw=black!12},
  tick label style={font=\small},
  label style={font=\small},
  xlabel={Awarded / cleared quantity (MW, stylized)},
  ylabel={Marginal award price (\$/MW-hr, stylized)},
]

% -----------------------------
% Panel A: Healthy / broad-based
% -----------------------------
\nextgroupplot[
  title={Panel A: Broad-based (``healthy'') stack},
]

% A smooth, gradually steepening curve
\addplot[very thick, black] coordinates {
  (0,5) (10,8) (20,12) (30,18) (40,26) (50,36) (60,48) (70,62) (80,78) (90,92) (100,98)
};

\node[anchor=north west, align=left, font=\footnotesize] at (axis cs:2,98)
{Many blocks\\Low $H_k$\\Small shocks $\Rightarrow$\\small price moves};

% A small demand shock arrow
\draw[-{Latex[length=2mm]}, thick] (axis cs:65,48) -- (axis cs:72,55);
\node[font=\footnotesize, anchor=west] at (axis cs:73,55) {$\Delta q$ small};

% -----------------------------
% Panel B: Fragile / hockey stick
% -----------------------------
\nextgroupplot[
  title={Panel B: Concentrated (``hockey stick'') stack},
  ylabel={},
]

% Flat then near-vertical wall
\addplot[very thick, black] coordinates {
  (0,6) (20,6) (40,7) (60,8) (75,10) (85,18) (90,35) (92,55) (94,75) (96,90) (98,98) (100,100)
};

% Feasibility wall marker
\draw[black!70, dashed, thick] (axis cs:92,0) -- (axis cs:92,100);
\node[font=\footnotesize, anchor=south west, align=left] at (axis cs:92.5,12)
{``Feasibility wall''\\(vertical jump)};

\node[anchor=north west, align=left, font=\footnotesize] at (axis cs:2,98)
{Few blocks\\High $H_k$\\Small shocks $\Rightarrow$\\large price jump};

% Same small demand shock arrow (causes big price move)
\draw[-{Latex[length=2mm]}, thick] (axis cs:89,35) -- (axis cs:93,75);
\node[font=\footnotesize, anchor=west] at (axis cs:93.2,75) {$\Delta q$ small};

\end{groupplot}
\end{tikzpicture}
\caption{Figure 9.1: The evolution of stack geometry. During Fern, the day-ahead ancillary award stack can transition from a robust, gradual slope (left) to a brittle, high-concentration ``hockey stick'' (right), increasing the sensitivity of prices to moderate quantity shocks. This figure illustrates the convexity/concentration logic behind the Herfindahl index $H_k$ used in the stack-shape diagnostics.}
\label{fig:stack_geometry_evolution}
\end{figure}
