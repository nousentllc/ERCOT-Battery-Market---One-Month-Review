% ======================================================================
% ERCOT-Style Technical Report Overhaul (LaTeX-ready)
% - Removes "dissertation" framing (report-only)
% - ERCOT-native structure: Executive Summary up front, neutral headers
% - Auto-numbering in Roman numerals (no manual numbering in titles)
% - Clean bookmarks + blue TOC links
% ======================================================================

\documentclass[11pt]{report}

% -----------------------------
% Page + typography
% -----------------------------
\usepackage[margin=.75in]{geometry}
\usepackage{setspace}
\onehalfspacing
\usepackage{lmodern}
\usepackage[utf8]{inputenc}

% -----------------------------
% Line-breaking slack (reduces overfull \hbox warnings, esp. in TOC)
% -----------------------------
\sloppy
\setlength{\emergencystretch}{6em}

% Suppress minor box warnings (keeps logs readable; does not affect output)
\hbadness=10000
\hfuzz=3pt
\vfuzz=3pt

% -----------------------------
% Math + theorems
% -----------------------------
\usepackage{amsmath, amssymb, amsthm}
\newtheorem{theorem}{Theorem}[chapter]
\newtheorem{lemma}[theorem]{Lemma}
\newtheorem{proposition}[theorem]{Proposition}
\theoremstyle{definition}
\newtheorem{definition}[theorem]{Definition}

% -----------------------------
% Tables, figures
% -----------------------------
\usepackage{booktabs}
\usepackage{array}
\usepackage{graphicx}
\usepackage{caption}

% -----------------------------
% TikZ/PGFPlots (vector figures)
% -----------------------------
\usepackage{tikz}
\usepackage{pgfplots}
\usepgfplotslibrary{groupplots}
\pgfplotsset{compat=1.18}
\usetikzlibrary{arrows.meta,positioning,calc,patterns,fit}

% -----------------------------
% TOC + blue links + bookmarks
% -----------------------------
\usepackage[T1]{fontenc}
\usepackage{xcolor}
\definecolor{LinkBlue}{RGB}{0,90,170}
\usepackage[numbers]{natbib}
\usepackage{xurl} % better line breaks for \url/\href and \texttt-like strings
\usepackage{hyperref}
\hypersetup{
  colorlinks=true,
  linkcolor=LinkBlue,
  citecolor=LinkBlue,
  urlcolor=LinkBlue,
  linktoc=all,
  pdfauthor={Justin Candler},
  pdftitle={ERCOT RTC+B (NPRR 1186): Early Operations, Price Formation, and the 2026--2035 Outlook},
  pdfsubject={ERCOT market design technical report; NPRR 1186; ORDC; ECRS/FRRS; batteries; co-optimization; reliability},
  pdfkeywords={ERCOT, RTC+B, NPRR 1186, ORDC, ASDC, ECRS, FRRS, batteries, SCED, MCPC}
}
\usepackage{bookmark}

% -----------------------------
% Header/footer
% -----------------------------
\usepackage{fancyhdr}
\setlength{\headheight}{14pt}
\pagestyle{fancy}
\fancyhf{}

% Keep the header minimal to avoid overlaps
\fancyhead[R]{\nouppercase{\leftmark}}
\fancyfoot[L]{\thepage}
\fancyfoot[R]{\textit{ERCOT RTC+B Technical Report}}

\renewcommand{\headrulewidth}{0.4pt}
\renewcommand{\footrulewidth}{0.4pt}
\fancypagestyle{plain}{%
  \fancyhf{}%
  \fancyhead[R]{\nouppercase{\leftmark}}%
  \fancyfoot[L]{\thepage}%
  \fancyfoot[R]{\textit{ERCOT RTC+B Technical Report}}%
  \renewcommand{\headrulewidth}{0.4pt}%
  \renewcommand{\footrulewidth}{0.4pt}%
}

% -----------------------------
% Auto-numbering (Arabic), without manual numbering in titles
% -----------------------------
% Chapters: 1, 2, 3...
\renewcommand{\thechapter}{\arabic{chapter}}
% Sections: 1.1, 1.2 ...
\renewcommand{\thesection}{\thechapter.\arabic{section}}
% Subsections: 1.1.1 ...
\renewcommand{\thesubsection}{\thesection.\arabic{subsection}}

% Optional: keep headings visually neutral (report-like)
\usepackage{titlesec}
\titleformat{\chapter}[hang]{\bfseries\Large}{\thechapter.}{0.4em}{}
\titleformat{\section}[hang]{\bfseries\large}{\thesection}{0.4em}{}
\titleformat{\subsection}[hang]{\bfseries}{\thesubsection}{0.4em}{}

% Reduce vertical whitespace at the top of chapter-opening pages
% (default \chapter style leaves substantial blank space).
\titlespacing*{\chapter}{0pt}{-18pt}{18pt}

% -----------------------------
% Custom title page (logo under date)
% -----------------------------
\makeatletter
\renewcommand{\maketitle}{%
  \begin{titlepage}
    \thispagestyle{empty}
    \centering
    \vspace*{1.5in}
    {\LARGE \bfseries \@title\par}
    \vspace{1.25in}
    {\Large \@author\par}
    \vspace{0.4in}
    {\large \@date\par}

    \IfFileExists{nous_logo.png}{%
      \vspace{0.6in}
      \includegraphics[width=0.35\textwidth]{nous_logo.png}\par
    }{%
      % (Optional) upload nous_logo.png to show the logo here.
    }%

    \vfill
  \end{titlepage}%
}
\makeatother

% -----------------------------
% Custom title page (logo under date)
% -----------------------------
\makeatletter
\renewcommand{\maketitle}{%
  \begin{titlepage}
    \thispagestyle{empty}
    \centering
    \vspace*{1.1in}
    {\LARGE \bfseries \@title\par}
    \vspace{1.05in}
    {\Large \@author\par}
    \vspace{0.4in}
    {\large \@date\par}

    \IfFileExists{}{%
      \vspace{0.6in}
      \includegraphics[width=0.35\textwidth]{JC Nous Logo_July2019_NC_Dark Gray.png}\par
    }{%
      % (Optional) upload nous_logo.png to show the logo here.
    }%

    \vfill
  \end{titlepage}%
}
\makeatother

% -----------------------------
% Title metadata
% -----------------------------
\title{\textbf{Beyond the Adder:}\\
\large A Forensic Analysis of Price Formation During Winter Storm Fern}
\author{Justin Candler}
\date{February 5, 2026}

\begin{document}
\pagenumbering{roman}
\maketitle

% ==========================================
% (00) ACRONYMS + DEFINITIONS (HYPERLINKED)
% ==========================================

\addcontentsline{toc}{chapter}{Acronyms and Definitions}

% Hyperlinked anchors for cross-referencing
\hypertarget{def:rtc}{}
\hypertarget{def:nprr1186}{}
\hypertarget{def:sced}{}
\hypertarget{def:lmp}{}
\hypertarget{def:spp}{}
\hypertarget{def:rtrdpa}{}
\hypertarget{def:mcpc}{}
\hypertarget{def:ordc}{}
\hypertarget{def:asdc}{}
\hypertarget{def:ecrs}{}
\hypertarget{def:frrs}{}
\hypertarget{def:rrs}{}
\hypertarget{def:reg}{}
\hypertarget{def:esr}{}
\hypertarget{def:soc}{}

\begin{table}[h!]
\centering
\caption{Acronyms and definitions used in this report (hyperlink-ready).}
\label{tab:acronyms}
\begin{tabular}{p{1.3in}p{5.1in}}
\toprule
\textbf{Term} & \textbf{Definition} \\
\midrule
\hyperlink{def:rtc}{RTC+B} & Real-Time Co-optimization plus Batteries: ERCOT real-time market design in which energy and ancillary services are co-optimized in SCED with explicit storage feasibility constraints (Appendix~\ref{app:reg_trace_full}).\cite{ercot_rtcb_golive_2025,ercot_rtcb_overview_2025} \\
\hyperlink{def:nprr1186}{NPRR 1186} & Nodal Protocol Revision Request establishing/implementing RTC+B-related rules for batteries (ESRs) in ERCOT real-time co-optimization (Appendix~\ref{app:reg_trace_full}).\cite{ercot_nprr1186_issue_page,ercot_nprr1186_docx} \\
\hyperlink{def:sced}{SCED} & Security-Constrained Economic Dispatch: real-time optimization engine producing dispatch instructions and LMPs subject to network/security constraints (Appendix~\ref{app:reg_trace_full}).\cite{ercot_current_nodal_protocols} \\
\hyperlink{def:lmp}{LMP} & Locational Marginal Price (SCED LMP): marginal cost of serving incremental load at a location under SCED, excluding real-time adders when explicitly stated. \\
\hyperlink{def:spp}{SPP} & Settlement Point Price: settlement-relevant price at a settlement point; may include real-time adders (e.g., reliability deployment pricing components). \\
\hyperlink{def:rtrdpa}{RTRDPA} & Real-Time Reliability Deployment Price Adder (as displayed): an additive component used in settlement point prices under specified reliability deployment conditions. \\
\hyperlink{def:mcpc}{MCPC} & Market Clearing Price for Capacity: ancillary service capacity clearing price for a product (e.g., Reg-Up, ECRS) in the relevant timeframe (Appendix~\ref{app:reg_trace_full}).\cite{ercot_ancillary_services_pdf,ercot_current_nodal_protocols} \\
\hyperlink{def:ordc}{ORDC} & Operating Reserve Demand Curve: scarcity pricing mechanism in legacy regime linking reserve margin to price adders/value signals (Appendix~\ref{app:reg_trace_full}).\cite{ercot_ordc_methodology_2015,ercot_ordc_report_2024} \\
\hyperlink{def:asdc}{ASDC} & Ancillary Service Demand Curve: demand curve constructs for ancillary services used to price control capability and reserves in a co-optimized regime (Appendix~\ref{app:reg_trace_full}).\cite{ercot_asdc_overview_2024} \\
\hyperlink{def:ecrs}{ECRS} & ERCOT Contingency Reserve Service: fast-response reserve product intended to support contingency response and reliability needs (Appendix~\ref{app:reg_trace_full}).\cite{ercot_ancillary_services_pdf} \\
\hyperlink{def:frrs}{FRRS} & Fast Responding Regulation Service: ancillary product label used in this report for fast regulation constructs; protocol-defined terminology and implementation may differ by effective date (Appendix~\ref{app:reg_trace_full}).\cite{ercot_ancillary_services_pdf} \\
\hyperlink{def:rrs}{RRS} & Responsive Reserve Service: reserve product historically central to ERCOT reserve procurement. \\
\hyperlink{def:reg}{Reg-Up/Reg-Down} & Regulation Up/Down: ancillary services for continuous balancing and frequency regulation. \\
\hyperlink{def:esr}{ESR} & Energy Storage Resource: storage modeled as a resource capable of charging/discharging and providing ancillary services subject to feasibility. \\
\hyperlink{def:soc}{SoC} & State of Charge: stored energy level of an ESR; a binding feasibility constraint in co-optimized dispatch and revenue realization. \\
\bottomrule
\end{tabular}
\end{table}

\noindent\textbf{Cross-reference note.} When terms are used later in the report, they may be referenced via hyperlinks (e.g., \hyperlink{def:spp}{SPP}, \hyperlink{def:lmp}{LMP}, \hyperlink{def:mcpc}{MCPC}) for rapid navigation.

\clearpage

% ======================================================================
% Table of Contents (blue links)
% ======================================================================
\tableofcontents
\clearpage
\pagenumbering{arabic}


% ======================================================================
% Executive Summary (1–2 pages style)
% ======================================================================
\section*{Executive Summary}
\addcontentsline{toc}{chapter}{Executive Summary}

\textbf{Purpose.} This report evaluates ERCOT’s Real-Time Co-optimization plus Batteries (RTC+B) implemented under NPRR 1186, with emphasis on (i) system cost, (ii) reliability outcomes, and (iii) investor returns. The analysis begins in 2023 (pre-implementation context), evaluates early operational signatures around December 2025 go-live and the immediately subsequent period, and develops a forward-looking framework through 2035. For the Winter Storm Fern case study, episode selection is driven first by \emph{physical stress triggers} (weather/load and sustained reserve stress); price activation is used only as a secondary confirmation and for alignment of event-time plots, not as a screening condition for inclusion.

\medskip
\textbf{Core finding (mechanistic).} RTC+B shifts the locus of scarcity monetization from a regime dominated by energy-side scarcity adders (legacy ORDC expression) toward a regime where binding constraints appear directly inside the co-optimized dispatch problem. As a result, the marginal value of reliability and control capability is increasingly expressed in ancillary service prices (MCPC for reserve and regulation products) when feasibility—especially battery state-of-charge (SoC) headroom—binds at the margin.

\medskip
\textbf{Observed early signature from provided artifacts.} The author-provided ERCOT displays and exports show that low-scarcity energy intervals (e.g., adder-free LMP conditions at the capture time) can coexist with nontrivial and occasionally heavy-tailed ancillary capacity prices (MCPC) over late-December 2025. This is consistent with “scarcity-value migration” into control products rather than energy—an expected consequence of true co-optimization with SoC-constrained storage.

\medskip
\textbf{Implications for system cost.} If co-optimization reduces the frequency of inefficient commitments and better aligns reserves with real-time system needs, total uplift and out-of-market actions should decline in expectation. However, this benefit is contingent on sufficient qualified flexibility supply (including storage) and transparent product definitions that do not create avoidable qualification bottlenecks.

\medskip
\textbf{Implications for reliability.} The central reliability benefit of RTC+B is earlier and more granular procurement of deliverable control capability (e.g., ECRS/FRRS/Reg) within the dispatch loop, reducing reliance on blunt post-settlement scarcity mechanisms. The corresponding risk is latent: if flexible resources fail to qualify or sustain feasibility (SoC exhaustion, telemetry/qualification constraints), scarcity can re-emerge abruptly in either reserve products or energy prices.

\medskip
\textbf{Implications for investor returns.} Battery economics in ERCOT become less “energy-tail-event” dominated and more dependent on (i) sustained qualification and performance in ancillary products and (ii) SoC-aware bidding/optimization. Merchant valuation must therefore decompose cash flows into energy arbitrage, ancillary capacity (MCPC), and scarcity uplift, with explicit modeling of SoC feasibility and co-optimization interactions.

\medskip
\textbf{Recommended actions (near-term).}
\begin{itemize}
  \item \textbf{Monitoring:} Publish standardized post-RTC+B dashboards that jointly visualize (a) LMP vs SPP (adder decomposition), (b) MCPC tails by product, and (c) event studies mapping high MCPC intervals to binding constraints and performance outcomes.
  \item \textbf{Data transparency:} Provide interval settlement extracts (energy + adders + ancillary awards/prices) with consistent definitions to enable independent validation of cost and reliability claims.
  \item \textbf{Design refinement:} Review ECRS/FRRS qualification and performance rules for unintended scarcity creation; align penalties and eligibility with physical deliverability rather than administrative artifacts.
\end{itemize}

\medskip
\textbf{Forward outlook (to 2035).} The reliability and price-formation effects of RTC+B will be dominated by: load growth (including large loads), renewable penetration and net-load ramping, thermal availability, and the scale and operational sophistication of storage participation. This report proposes a scenario ensemble approach and defines “success metrics” that are simultaneously operational (frequency/control performance), economic (system cost and uplift), and financeable (cash-flow stability for flexible resources).

\clearpage

% ======================================================================
% Abstract (report framing)
% ======================================================================
\chapter*{Abstract}
\addcontentsline{toc}{chapter}{Abstract}

This report examines ERCOT’s Real-Time Co-optimization plus Batteries (RTC+B), implemented under NPRR 1186 (Appendix~\ref{app:reg_trace_full}), as a structural change in how Texas prices reliability and allocates scarcity value across energy and ancillary services in real time. Whereas the legacy regime expressed a substantial fraction of reliability value through ORDC-linked energy-side scarcity mechanisms, RTC+B operationalizes scarcity and control capability via simultaneous co-optimization of energy and ancillary awards in SCED, subject to explicit feasibility constraints for battery energy storage resources (ESRs), especially state-of-charge (SoC) dynamics.

The report develops (i) a formal decomposition of real-time settlement prices into SCED LMPs and real-time adders, (ii) a co-optimization model that interprets ancillary price formation through binding constraint duals, and (iii) an empirical scaffolding for post-implementation validation using public ERCOT artifacts. Early descriptives from late-December 2025 ancillary MCPC time series and January 1, 2026 frequency behavior illustrate a key RTC+B signature: energy can remain in a low-scarcity regime while ancillary products exhibit nontrivial price tails, consistent with scarcity-value migration into co-optimized control products as SoC feasibility binds. The report concludes with a scenario-based outlook through 2035 and implementable recommendations for monitoring, data transparency, and product design refinement.

\clearpage

%% =========================
% (A) DOCUMENT CONTROL PAGE
% =========================
\addcontentsline{toc}{chapter}{Document Control}

\begin{tabular}{@{}p{1.9in}p{4.6in}@{}}
\toprule
\textbf{Document Title} &
\textbf{Beyond The Adder:}
\\[-0.2em]
& \textit{A Forensic Analysis of Price Formation During Winter Storm Fern}
\\
\textbf{Author} & Justin Candler \\
\textbf{Version} & v0.7 (Draft; ERCOT staff--style technical report) \\
\textbf{As-of Timestamp} & 2026-02-05 \\
\textbf{Recency Risk} &
Medium--High (analysis references recent operational events; some artifacts
remain dashboard-grade pending settlement finalization) \\
\textbf{Intended Use} &
Market design review, planning analysis, and lender-grade valuation framing;
not a protocol interpretation, legal opinion, or settlement-final determination \\
\textbf{Non-Reliance Note} &
Findings are limited to identification-safe descriptives, partial-identification
bounds, and testable hypotheses unless explicitly supported by settlement-grade
awards, deployments, or telemetry data \\
\bottomrule
\end{tabular}

\vspace{0.75em}

\noindent\textbf{Data Lineage and Artifact Scope.}
The empirical analysis in this report is grounded in a combination of
author-provided ERCOT public exports, settlement-grade files where available,
and clearly labeled dashboard-grade artifacts. All figures and tables are
traceable to bounded episode sets via the episode ledger
(Section~\ref{sec:traceability}).

Primary artifact classes include:

\begin{itemize}
  \item \textbf{Real-Time Prices:}
  Settlement Point Prices (SPP), Real-Time LMPs where available, and published
  real-time scarcity adders (ORDC/RDP);
  \item \textbf{Ancillary Services Prices:}
  Real-Time Market Clearing Prices for Capacity (MCPC) by product;
  \item \textbf{Day-Ahead Awards and Geometry:}
  DAMASAGG (DA AS awards), DAMASSOLD (DA AS sold quantities),
  and aggregated DA AS offer stacks;
  \item \textbf{Operational Proxies:}
  System frequency series, total AS capability reports, and
  capability-to-requirement ratios;
  \item \textbf{Regulatory References:}
  NPRR~1186, associated Market-Facing Changes documents, and ERCOT settlement
  matrix definitions.
\end{itemize}

Where an artifact is dashboard-grade or preliminary, this status is explicitly
noted in the relevant chapter and the associated claims are bounded accordingly.

\vspace{0.5em}

\noindent\textbf{Event Focus.}
While the analytical framework applies broadly to RTC+B operations, this version
of the report places particular emphasis on the \textbf{Winter Storm Fern
(January~22--29,~2026)} interval as a structured post-mortem case study.
References to ``Fern-like'' events denote the general stress class; references
to ``the Fern packet'' denote the specific January~2026 episode set used in the
event study and geometry analyses.

\vspace{0.5em}

\noindent\textbf{Change Log.}
\begin{itemize}
  \item v0.7 (2026-02-05): Integrated Winter Storm Fern post-mortem; added Wave~3
  Day-Ahead Geometry layer, DA$\rightarrow$RT transmission analysis,
  expanded Identification and Inference Limits ladder
  (price-only $\rightarrow$ awards $\rightarrow$ deployments $\rightarrow$
  telemetry), and formal Regulatory Traceability appendix.
  \item v0.6: Added Episode Ledger, partial-identification bounds, and Wave~2
  sensitivity analysis.
  \item v0.5: Added Wave~1--2 empirical results, frequency proxy framework,
  and ancillary tail diagnostics.
  \item v0.3: Converted framing to ERCOT-style technical report; added
  Executive Summary, Methodology, Identification Rules, and traceability
  conventions.
  \item v0.1--0.2: Initial drafting of price decomposition, co-optimization
  mechanics, and early descriptives.
\end{itemize}

\clearpage

% ======================================================================
% Report Introduction (neutral, ERCOT-style)
% ======================================================================
\chapter{Introduction}

\section{Purpose and decision-use orientation}

This report evaluates ERCOT’s RTC+B implementation under NPRR 1186 with a decision-use orientation: the relevant question is not merely whether the market clears, but whether the redesign improves (i) total system cost, (ii) reliability performance, and (iii) financeable investor returns for the flexible resources ERCOT increasingly depends upon. The report begins in 2023 to establish pre-implementation context, assesses early operational signatures around the December 2025 go-live and immediately subsequent period, and extends the analytic framework through 2035.

\section{Market-design context}

In legacy ERCOT real-time operations, a substantial component of reliability value was expressed through energy-side scarcity mechanisms that are often summarized operationally as “ORDC behavior,” including periods where scarcity adders materially influence settlement prices. RTC+B changes the primitive: energy and ancillary services are jointly cleared in SCED, and battery energy storage resources are modeled as unified ESRs with explicit SoC feasibility inside the dispatch loop. This redesign creates a new price-formation stack in which scarcity can manifest as: (a) SCED LMP movements, (b) real-time adders when deployed, and increasingly (c) ancillary capacity prices (MCPC) for control products when deliverable flexibility (rather than energy) is marginal.

\section{Research questions}

The analysis is organized around five questions:
\begin{enumerate}
  \item How should real-time prices be decomposed into SCED LMP and adders, and what can be identified from public artifacts?
  \item Under what conditions does scarcity value migrate from legacy energy-side expressions into ancillary products under co-optimization?
  \item How does SoC feasibility inside SCED change storage dispatch and the feasible revenue frontier?
  \item What operational signatures indicate success, failure, or latent fragility in the early RTC+B regime?
  \item Under 2026--2035 scenarios, what planning and market-design refinements best improve system cost and reliability while sustaining investable returns?
\end{enumerate}

\section{Scope, limitations, and data lineage}

This report is grounded in ERCOT public displays and exports provided by the author (real-time settlement point prices, system-wide prices, ancillary services, MCPC time series, and frequency time series). Where full-month interval settlement extracts are not yet integrated, the report provides procedures, identification rules, and testable hypotheses rather than asserting full-month distributional claims.

\section{Structure of the report}

Chapters I--III establish the price decomposition and co-optimization mechanics, then provide early empirical descriptives from the available artifacts and derive testable hypotheses for the post-RTC+B regime. Later chapters (IV+) will incorporate full-month interval settlement extracts, event studies around high-MCPC periods, and investor-grade revenue decomposition for canonical storage assets, culminating in a scenario-based outlook to 2035 and implementable design recommendations.


\clearpage
\chapter{Methodology: Data, Definitions, and Metrics}
\label{ch:methodology_data}
\label{ch:metric_dictionary_full}

\noindent\textbf{Chapter roadmap.} Chapter~2 (i) inventories the data artifacts and what each can and cannot identify, (ii) states traceability and integrity rules that every figure/table must satisfy, and (iii) defines the metric dictionary (notation, tail/volatility measures, and thresholds) used consistently throughout the remainder of the report.

\section{Data Artifacts \& Lineage}
\label{sec:data_artifacts_lineage}

\subsection{Empirical artifacts and scope of inference}

This report section is grounded in ERCOT public displays and exports provided by the author, namely: (i) Real-Time Settlement Point Price (RTM-SPP) displays, (ii) Real-Time LMPs (SCED) hub/zone table display, (iii) system-wide hub price time series, (iv) ancillary services dashboard materials, (v) Real-Time Market Clearing Prices for Capacity (MCPC) time series for ancillary services, and (vi) a short-window system frequency time series. The objective is to formalize what these artifacts \emph{identify} about the Real-Time Co-optimization plus Batteries (RTC+B) regime and what they \emph{cannot} establish absent full December 2025 interval settlement extracts.

The analysis explicitly distinguishes three objects that are frequently conflated in informal market discussion:

\begin{enumerate}
  \item \textbf{SCED LMP} (5-minute): the nodal or aggregated locational marginal price produced by the Security-Constrained Economic Dispatch (SCED) engine, excluding real-time price adders.
  \item \textbf{Real-Time Price Adders} (notably RTRDPA): the additive components applied for reliability deployment pricing for energy.
  \item \textbf{Settlement Point Price (SPP)}: the settlement-relevant price at a settlement point, which \emph{includes} designated adders when applicable.
\end{enumerate}

\subsubsection{Artifact ledger (minimum traceability fields)}
To make the traceability rule operational, every source artifact used in this report is logged with (at minimum) its file name, time coverage, native resolution, time-zone/DST convention, key fields, units, and admissible uses at the relevant Evidence Rung.

\begin{table}[h!]
\centering
\caption{Artifact ledger template (minimum fields required for traceability).}
\label{tab:artifact_ledger_template}
\begin{tabular}{p{1.35in}p{1.05in}p{1.15in}p{1.0in}p{1.95in}}
\toprule
\textbf{Artifact class} & \textbf{Native resolution} & \textbf{Time convention} & \textbf{Units} & \textbf{Primary admissible uses} \\
\midrule
RT energy prices (LMP/SPP) & 5-min / 15-min & ERCOT local (CST/CDT) & \$/MWh & Price distributions; event studies; co-movement with adders \\
RT adders (e.g., RTRDPA) & 5-min / 15-min & ERCOT local (CST/CDT) & \$/MWh & Scarcity-incidence tagging; adder-free interval identification \\
AS capacity prices (MCPC) & 5-min (SCED) & ERCOT local (CST/CDT) & \$/MW-hr & Tail diagnostics; product-by-product scarcity characterization \\
DA AS awards (DAMASAGG) & hourly & hour-ending per ERCOT file convention & \$/MW-hr, MW & Procurement posture; DA-to-RT alignment checks \\
Frequency telemetry & 1--10 sec (typ.) & ERCOT local (CST/CDT) & Hz & Reliability proxies; stress-window tagging \\
\bottomrule
\end{tabular}
\end{table}

\subsection{Artifact classes and admissible inferences}
\label{sec:artifact_classes}

ERCOT publishes multiple classes of market artifacts that differ materially in
provenance, update frequency, and settlement authority. This study distinguishes
artifacts into three classes, each with explicitly limited inferential scope.

\subsubsection{Settlement-grade artifacts}
Settlement-grade artifacts include nodal and zonal Real-Time Market (RTM)
prices, Market Clearing Prices for Capacity (MCPC) for ancillary services, DAM
award and schedule files, and frequency telemetry. These artifacts are generated
through ERCOT’s official market systems (SCED, DAM, AGC) and are used for
financial settlement.

Admissible inferences from settlement-grade artifacts include:
\begin{itemize}
  \item Distributional properties of prices (quantiles, tail mass, CVaR).
  \item Event-based analysis of scarcity episodes and co-movement across products.
  \item Temporal alignment of price outcomes with frequency deviations and reserve deployment.
\end{itemize}

Settlement-grade artifacts are required for any claim involving price formation,
revenue attribution, or reliability outcomes.

\subsubsection{Dashboard and display artifacts}
Dashboard artifacts include ERCOT public displays, real-time dashboards, and
system-wide summary views. These artifacts are often derived from settlement
systems but may include rounding, aggregation, or lag.

Admissible inferences from dashboard artifacts are restricted to:
\begin{itemize}
  \item Descriptive context (system conditions, contemporaneous status).
  \item Qualitative validation of timing and directionality.
\end{itemize}

Dashboard artifacts are not used for quantitative estimation, tail analysis, or
hypothesis testing.

\subsubsection{Derived and transformed artifacts}
Derived artifacts are constructed by the author through deterministic
transformations of settlement-grade data, such as aggregation to 5-minute
intervals, episode ledger construction, or matched-sample conditioning.

All derived artifacts are reproducible from raw inputs using the transformation
rules specified in this chapter and in Appendix~\ref{app:data_quality_full}.

\subsection{Ledger-to-Figure Traceability Rule}
\label{sec:traceability}

Any figure, table, or numerical claim must be traceable to (i) a specific
source artifact (file name and extraction window) and (ii) a specific
transformation rule (aggregation choice, thresholds/quantiles, and join keys).
Claims that cannot be traced to a ledger entry are treated as illustrative only.

\subsection{Time normalization and key integrity constraints}
\label{sec:time_integrity}

All datasets are normalized to a common temporal and indexing framework prior to
analysis.

\paragraph{Time conventions.}
All timestamps are converted to ERCOT prevailing local time (CST/CDT as
applicable). The canonical analytical interval is five minutes, corresponding to
the SCED dispatch interval.

\paragraph{Alignment rule for coarser timestamps (deterministic).}
When an artifact is reported at coarser resolution (e.g., 10-second frequency or 15-minute settlement prices) and must be merged to a 5-minute grid, timestamps are mapped to the \emph{start} of the enclosing 5-minute SCED interval. Concretely, an observation at clock time $t$ is assigned to
\[
\lfloor t\rfloor_{5\text{min}}\equiv \text{the most recent 5-minute boundary at or before }t.
\]
If $t$ lies exactly on a 5-minute boundary, it is assigned to that boundary. All such mappings are flagged (native resolution retained) so downstream analyses can distinguish native-interval results from aligned/merged results.

\paragraph{Unique-key integrity.}
Each analytical table must satisfy a unique composite key constraint of the form
$\left(\text{timestamp}_{5\text{min}},\text{product},\text{location}\right)$.
Duplicate keys result in a hard halt and require upstream correction.

\paragraph{Missingness and outlier discipline.}
Intervals with missing values exceeding 0.5\% of observations in a window are
excluded from window-level statistics. Outliers are retained unless they violate
known physical or market bounds, in which case they are flagged and documented
rather than removed.

\subsection{Pre/post windowing and matched-sample design}
\label{sec:windowing}

Analyses comparing regimes are conducted using pre-specified pre/post windows.

\paragraph{Window definition.}
The baseline comparison uses November 2025 as the pre-change window and December
2025 onward as the post-change window. Alternative windows are evaluated only as
part of registered sensitivity analysis.

\paragraph{Matched-sample conditioning.}
To isolate regime effects from system conditions, matched samples are
constructed using covariates observable at the system level:
\begin{itemize}
  \item Total system load.
  \item Net load (load minus wind and solar).
  \item Load ramp magnitude.
  \item Proxy outage indicators.
\end{itemize}

Matching is performed via stratification into bins (e.g., ramp quintiles) rather
than continuous regression, preserving interpretability and avoiding functional
form assumptions.

\subsection{Statistical procedures (descriptive and inferential)}
\label{sec:stat_procedures}

This study distinguishes strictly between descriptive statistics and inferential
tests.

\paragraph{Descriptive statistics.}
Descriptive metrics include empirical quantiles, exceedance rates, episode
duration, and co-occurrence counts. These are reported without causal
interpretation.

\paragraph{Inferential procedures.}
Inferential comparisons may employ:
\begin{itemize}
  \item Kolmogorov--Smirnov tests for distributional shifts.
  \item Block bootstrap confidence intervals for exceedance rates.
  \item Difference-in-differences where matched samples permit.
\end{itemize}

Multiple hypothesis testing is controlled using the Benjamini--Hochberg false
discovery rate procedure. All inferential claims must reference the applicable
sensitivity lever defined in Appendix~\ref{app:sensitivity_wave2}.

\section{Metric Dictionary and Threshold Definitions}
This section defines the report’s core metrics and threshold conventions \emph{ex ante}, to prevent post hoc reinterpretation. Metrics are grouped by the decision object they measure: (i) energy-side prices (LMP/SPP/adders), (ii) ancillary scarcity (MCPC and award stacks), (iii) tail and volatility descriptors, and (iv) reliability proxies (frequency, load shed). Unless explicitly stated, all statistics are descriptive summaries of observed series.

\subsection{Core notation}
Let $t$ denote an interval timestamp (5-minute for SCED/MCPC; 15-minute for some RTM displays; hourly for DAM awards), and let $h(t)$ denote the corresponding delivery hour. Settlement points (hubs/zones) are indexed by $s$ and ancillary products by $k$.

\subsection{Units and pricing conventions}
To prevent silent unit mismatches, the report uses:
\begin{itemize}
  \item Energy prices (LMP/SPP and energy-side adders) in \$/MWh.
  \item Ancillary capacity prices (MCPC and DAMASAGG award prices) in \$/MW-hr.
  \item Ancillary quantities (DAMASAGG awards and requirement quantities) in MW.
  \item Frequency in Hz.
\end{itemize}
Where an ERCOT source uses a different label or aggregation (e.g., 15-minute settlement intervals), the artifact ledger records the native convention and any deterministic alignment rule applied prior to computing metrics.

\subsection{Day-ahead ancillary awards (DAMASAGG)}
For delivery date $d$, hour $h$, product $k$, and block index $i$, DAMASAGG reports $(p_{k,i}(d,h),q_{k,i}(d,h))$.
We define:
\begin{align}
Q^{DA}_k(d,h) &= \sum_i q_{k,i}(d,h),\\
P^{DA}_k(d,h) &= \frac{\sum_i p_{k,i}(d,h)\,q_{k,i}(d,h)}{\sum_i q_{k,i}(d,h)}.
\end{align}
Stack-shape diagnostics:
\begin{align}
n_k(d,h) &= \#\{i:\ q_{k,i}(d,h)>0\},\\
\sigma_{p,k}(d,h) &= \sqrt{\frac{\sum_i q_{k,i}(d,h)\big(p_{k,i}(d,h)-P^{DA}_k(d,h)\big)^2}{\sum_i q_{k,i}(d,h)}{\Big/}\sum_i q_{k,i}(d,h)},\\
H_k(d,h) &= \sum_i\left(\frac{q_{k,i}(d,h)}{\sum_j q_{k,j}(d,h)}\right)^2.
\end{align}
Interpretation guardrails: $n_k$ captures award granularity; $\sigma_{p,k}$ captures within-hour dispersion of cleared price blocks; $H_k\in(0,1]$ captures concentration (higher implies thinner procurement).

\subsection{Energy-side prices and adders}
Energy-side series include:
\begin{itemize}
  \item $\lambda^{RT}_{s}(t)$: real-time LMP at settlement point $s$ (as reported),
  \item $\pi^{RT}_{s}(t)$: real-time SPP at settlement point $s$ (where SPP includes adders),
  \item $a^{RT}(t)$: the real-time adder component (e.g., ORDC-based reliability deployment price adder) such that $\pi^{RT}=\lambda^{RT}+a^{RT}$ when applicable.
\end{itemize}
The canonical decomposition and notation are stated once in Section~\ref{sec:price_formation_stack}; downstream text should refer there rather than re-stating symbols.

\subsection{Ancillary scarcity prices (MCPC)}
Let $\mathrm{MCPC}_k(t)$ denote the Market Clearing Price for Capacity for product $k$ at SCED (5-minute) resolution.

\subsection{Tail metrics and heavy-tail discipline}
Tail behavior is measured using complementary descriptors:
\paragraph{Exceedance tail mass.} For a threshold $\tau$,
\[
\pi_X(\tau)=\Pr(X>\tau)\approx \frac{1}{T}\sum_{t=1}^T \mathbf{1}\{X_t>\tau\}.
\]
\paragraph{Quantiles.} $Q_q(X)$ denotes the empirical $q$-quantile of $X$.
\paragraph{Conditional Value-at-Risk (CVaR / expected shortfall).} For $q\in(0,1)$,
\[
\mathrm{CVaR}_{q}(X)=\mathbb{E}\left[X\mid X\ge Q_q(X)\right],
\]
estimated empirically over the tail set $\{t: X_t\ge Q_q(X)\}$.
\paragraph{Cluster/run length.} For exceedance events at threshold $\tau$, define the maximum consecutive run length:
\[
R_X(\tau)=\max\{\text{consecutive }t\text{ with }X_t>\tau\}.
\]
Run length distinguishes isolated spikes from sustained stress.

\subsection{Volatility measures}
Volatility is characterized using quantile spreads and tail measures rather than solely variance.
\paragraph{High-quantile spread.}
\[
\Delta_q(X)=Q_{q}(X)-Q_{1-q}(X),
\]
with $q\in\{0.95,0.99\}$.
\paragraph{Kurtosis.} Reported only as a descriptive shape statistic (sensitive to outliers and finite-sample effects); it is not used as a primary decision metric in this report.

\subsection{Peak-hour targeting}
Define a peak set $\mathcal{P}$ as the top decile of hours by system load within the relevant season/baseline. The peak award share for product $k$ is:
\[
\mathrm{PeakShare}_k=\frac{\sum_{t\in\mathcal{P}} Q^{DA}_k(t)}{\sum_t Q^{DA}_k(t)}.
\]

\subsection{Frequency metrics as reliability proxies}
Let $f(t)$ denote system frequency (Hz) at sampling interval $\Delta t$.
\paragraph{Band violation count.} For tolerance $\epsilon$ (e.g., 0.036 Hz),
\[
N_{\epsilon}=\sum_t \mathbf{1}\{|f(t)-60|>\epsilon\}.
\]
\paragraph{Integrated absolute deviation.}
\begin{equation}
I_f=\sum_t |f(t)-60|\,\Delta t.
\label{eq:freq_integral}
\end{equation}
\paragraph{Excursion severity (optional).} For excursions beyond $\epsilon$,
\[
S_{\epsilon}=\sum_t \max(0,|f(t)-60|-\epsilon)\,\Delta t.
\]

\subsection{Threshold pre-registration}
Thresholds are declared before running event studies:
\begin{itemize}
  \item $\tau_a$: adder threshold for ORDC scarcity incidence,
  \item $\tau_k$: MCPC thresholds per product (e.g., 50/100/500/1000 \$/MW-hr, or product-specific),
  \item quantile levels $q\in\{0.95,0.99,0.999\}$ for tail descriptors.
\end{itemize}
Sensitivity analysis is performed by reporting results over a grid of thresholds rather than selecting a single tuned cutoff. Any change to threshold grids is logged in the reproducibility annex.

\section{The Price Formation Stack}
\label{sec:price_formation_stack}

\subsection{Price formation identity and notation}

Let $t$ index dispatch or settlement intervals (5-minute or 15-minute depending on the artifact), and let $z$ index settlement points (hubs, load zones, or buses). Define:

\begin{align}
  \lambda_{t,z}^{\text{LMP}} &:= \text{SCED LMP at time } t \text{ and location } z \\
  a_{t,z}^{\text{RT}} &:= \text{Real-Time price adder (e.g., RTRDPA) at } (t,z) \\
  \pi_{t,z}^{\text{SPP}} &:= \text{Settlement Point Price (RTM-SPP) at } (t,z)
\end{align}

The operational relationship implied by ERCOT’s RTM displays is:

\begin{equation}
  \pi_{t,z}^{\text{SPP}} \;=\; \lambda_{t,z}^{\text{LMP}} \;+\; a_{t,z}^{\text{RT}}.
  \label{eq:spp_decomposition_v1}
\end{equation}

The displays further annotate that SPP values include \emph{Reliability Deployment Price for Energy}, consistent with the inclusion of the real-time adder term (see \cite{ercot_rtm_spp_display_pdf,ercot_rtm_lmp_table_screenshot}).

\subsection{A basic identification theorem for the screenshot interval}

The author-provided Real-Time LMP hub/zone table shows $\text{RTRDPA} = \$0.00$ at the capture time (updated Jan 1, 2026, 16:15:15), and the hub/zone LMPs are identical at $\$16.83/\text{MWh}$ (see \cite{ercot_rtm_lmp_table_screenshot}). The contemporaneous RTM-SPP map (updated Jan 1, 2026, 16:17) shows a system-wide light-blue field consistent with the same order-of-magnitude pricing and explicitly states that SPP includes the reliability deployment price for energy (see \cite{ercot_rtm_spp_display_pdf,ercot_rtm_spp_map_screenshot}).

\begin{theorem}[Adder-free identification at the capture time]
If $\text{RTRDPA}_{t,z} = 0$ for all displayed hubs/zones at time $t$, then for those locations,
\[
\pi_{t,z}^{\text{SPP}} = \lambda_{t,z}^{\text{LMP}}.
\]
\end{theorem}

\begin{proof}
By the decomposition in Eq.~\eqref{eq:spp_decomposition_v1}, $\pi_{t,z}^{\text{SPP}}=\lambda_{t,z}^{\text{LMP}}+a_{t,z}^{\text{RT}}$. The displayed value $\text{RTRDPA}_{t,z}=0$ implies $a_{t,z}^{\text{RT}}=0$ at those locations (per ERCOT display semantics). Substituting yields $\pi_{t,z}^{\text{SPP}}=\lambda_{t,z}^{\text{LMP}}$. \qedhere
\end{proof}

\paragraph{Interpretive boundary.}
This theorem identifies only the \emph{instantaneous} relationship at the capture time. It does \emph{not} identify the full December 2025 distribution of adders or the empirical frequency of scarcity conditions; that requires a complete interval dataset rather than single-interval displays.

\subsection{RTC+B as a co-optimization problem with energy-storage state constraints}

Under RTC+B, battery energy storage resources (ESRs) are incorporated into real-time co-optimization such that energy and ancillary services are jointly scheduled subject to state-of-charge (SoC) feasibility. A stylized (but structurally faithful) formulation is:

\begin{align}
  \min_{\{p_t,r_t\}} \quad & \sum_{t} C_t(p_t) \;+\; \sum_t D_t(r_t) \label{eq:coopt_obj_stylized_v1} \\
  \text{s.t.}\quad & \text{SOC}_{t+1} \;=\; \text{SOC}_t \;+\; \eta_{\text{ch}} p_t^{\text{ch}} \Delta t \;-\; \frac{1}{\eta_{\text{dis}}} p_t^{\text{dis}} \Delta t \label{eq:soc_dyn_v1}\\
  & 0 \le \text{SOC}_t \le \overline{\text{SOC}} \label{eq:soc_bounds_v1}\\
  & 0 \le p_t^{\text{ch}} \le \overline{P}^{\text{ch}}, \quad 0 \le p_t^{\text{dis}} \le \overline{P}^{\text{dis}} \label{eq:power_bounds_v1}\\
  & \text{Network/security constraints (DC/AC security constraints)} \label{eq:network_constraints_v1}\\
  & \text{Ancillary service feasibility constraints (product-specific)}. \label{eq:as_constraints_v1}
\end{align}

Here $p_t$ denotes energy schedules (charge/discharge), and $r_t$ denotes ancillary service schedules (e.g., regulation, reserves) which consume feasibility headroom. Crucially, co-optimization implies that ancillary awards and energy schedules compete for the same physical capability, so scarcity value can migrate from energy into ancillary service prices when SoC and ramp constraints bind. This co-optimization structure is the essential economic mechanism behind the post-RTC+B revenue and risk decomposition discussed in Sections~II–III.

\section{The ``Fern'' event protocol (price-blind stress packet)}
\label{sec:fern_protocol_price_blind}
Storm Fern is treated as an \emph{externally triggered} stress event rather than a model-selected interval. To limit discretionary episode selection, the baseline Fern window is defined using \emph{price-blind} observables---system stress proxies such as load and temperature---and does not use energy prices, adders, or MCPC as screening variables.

\subsection{Price-blind selection criteria (load and temperature thresholds)}
Let $L(t)$ denote system load (or a documented system-level proxy) and let $T(t)$ denote a system temperature proxy (e.g., a load-weighted temperature index or a representative hub temperature series). Let $\mathcal{W}$ denote candidate winter weeks.

A candidate week $\mathcal{W}$ qualifies as a Fern-type stress packet if it satisfies both:
\begin{enumerate}
  \item \textbf{Load extremeness:}
  \[
  \max_{t\in \mathcal{W}} L(t) \ge Q_{0.90}(L\mid \text{winter baseline}),
  \]
  where the baseline is defined over a comparable winter season window.

  \item \textbf{Cold-stress trigger (temperature):}
  \[
  \min_{t\in \mathcal{W}} T(t) \le Q_{0.10}(T\mid \text{winter baseline}).
  \]
\end{enumerate}

\noindent\textbf{Clarification (price-blind discipline).} Prices (RTM SPP/LMP), adders (e.g., RTRDPA), and ancillary prices (MCPC) may be used \emph{after} window selection for descriptive alignment and mechanism testing, but they are not used to define the window itself.

\subsection{Baseline Fern partition (fixed for this report)}
For the current empirical pass, the partition used throughout the report is fixed as:
\begin{itemize}
  \item Pre-window: 2026-01-20 to 2026-01-21,
  \item Fern window: 2026-01-22 to 2026-01-29,
  \item Post-window: 2026-01-30 to 2026-02-05.
\end{itemize}
All subsequent results that reference ``pre/Fern/post'' employ this partition unless explicitly stated otherwise.

% Price-blind stress packet illustration (limited palette)

\begin{figure}[!ht]
\centering
\begin{tikzpicture}[x=1cm,y=1cm, font=\scriptsize, scale=0.92, transform shape]
  % Timeline
  \draw[very thick, black!85] (0,0) -- (14,0);
  \node[below] at (0,0) {Time};

  % Stress packet window (price-blind)
  \fill[LinkBlue!12] (3,-0.55) rectangle (11,0.55);
  \draw[thick, black!75] (3,-0.55) rectangle (11,0.55);

  % Onset / end markers
  \draw[thick, black!80] (3,-0.85) -- (3,0.85);
  \draw[thick, black!80] (11,-0.85) -- (11,0.85);
  \node[above, text=black!85] at (3,0.85) {Stress onset $t_0$};
  \node[above, text=black!85] at (11,0.85) {Stress end $t_1$};

  % Duration brace
  \draw[decorate,decoration={brace,amplitude=5pt}, black!80] (3,-1.2) -- (11,-1.2);
  \node[below, text=black!85] at (7,-1.2) {Duration $t_1-t_0$};

  % Covariates (explicitly price-blind)
  \node[align=left,anchor=west, text width=3.3cm] (cov) at (11.2,0.45) {\textbf{Covariates (price-blind)}\\
  Weather bin (temp / wind chill)\\
  Load level / net-load proxy\\
  Sustained reserve stress\\
  Persistence / contiguity};
  \draw[-{Latex[length=2mm]}, thick, black!70] (10.35,0.20) -- (11.05,0.35);

  % Note that prices are analyzed inside the packet
  \node[align=left,anchor=west, text width=13.2cm, text=black!80] at (0,-2.05) {\textbf{Interpretation:} Packet selection is driven by physical triggers; pricing is analyzed \emph{within} the packet (not used to select it).};
\end{tikzpicture}
\caption{Stress packet visual (price-blind): a contiguous stress interval defined by physical triggers (load and temperature), with annotated onset, duration, and covariates.}
\label{fig:stress_packet_visual}
\end{figure}


\section{December 2025 Early Operations — Empirical Descriptives from MCPC and System Displays}

\subsection{Overview of December 2025 in the provided artifacts}

The transition to RTC+B went live in early December 2025 (NPRR 1186 context). This section restricts itself to what is directly measurable from the provided data. Two empirical slices are available:

\begin{enumerate}
  \item A multi-day (``previous 6 days'') interval MCPC time series for ancillary services (Reg-Up, Reg-Down, RRS, Non-Spin, ECRS), spanning \textbf{2025-12-26 through 2025-12-31} at 5-minute resolution, provided as \texttt{real-time-market-clearing-prices-for-capacity-previous.csv}.
  \item A short-window system frequency time series at 10-second resolution spanning \textbf{2026-01-01 14:23:10 through 16:23:00}, provided as \texttt{ancillary-services-frequency.csv}.
\end{enumerate}

In addition, system-wide hub price displays and real-time SPP/LMP screenshots provide qualitative confirmation that the market can operate in a low-scarcity regime with negligible adders, as shown at the Jan 1, 2026 capture time (see \cite{ercot_rtm_spp_map_screenshot,ercot_rtm_lmp_table_screenshot,ercot_systemwide_prices_pdf}).

\subsection{MCPC descriptives: 2025-12-26 through 2025-12-31 (5-minute)}

Let $\text{MCPC}_{t}^{(k)}$ denote the market clearing price for capacity for ancillary product $k$ at time $t$. The provided dataset includes $k \in \{\text{REG-UP},\text{REG-DOWN},\text{RRS},\text{NON-SPIN},\text{ECRS}\}$. Table~\ref{tab:mcpc_desc_dec2025} reports distributional descriptors computed directly from the CSV (N=1744 intervals).

\begin{table}[h!]
\centering
\caption{MCPC distributional descriptors (5-min), 2025-12-26 to 2025-12-31 (N=1744).}
\label{tab:mcpc_desc_dec2025}
\begin{tabular}{lrrrrrrrr}
\hline
Product & Mean & P50 & P90 & P95 & P99 & Max & \#(>1) & \#(>2) \\
\hline
Reg-Up    & 0.1069 & 0.000 & 0.157 & 0.550 & 2.335 & 3.77 & 56 & 20 \\
Reg-Down  & 0.4454 & 0.130 & 1.020 & 1.990 & 6.000 & 8.65 & 180 & 65 \\
RRS       & 0.1484 & 0.030 & 0.277 & 0.460 & 3.000 & 6.46 & 47 & 33 \\
Non-Spin  & 0.6136 & 0.250 & 1.957 & 3.018 & 4.731 & 6.46 & 317 & 169 \\
ECRS      & 0.1994 & 0.030 & 0.474 & 0.860 & 3.000 & 6.46 & 70 & 35 \\
\hline
\end{tabular}
\end{table}

\paragraph{Immediate empirical conclusions (bounded to the dataset window).}
Within this 6-day window, \textbf{Non-Spin} exhibits the highest mean and high upper-quantile values, while \textbf{Reg-Down} attains the single highest maximum ($\$8.65$/MW-h). \textbf{ECRS} reaches a maximum of $\$6.46$/MW-h with a heavy right tail (P99 = $\$3.00$/MW-h). These observations matter because RTC+B structurally increases the coupling between SoC feasibility and ancillary deployment feasibility, which can elevate the shadow price of reserves relative to energy even in non-scarcity energy regimes.

\subsection{Frequency descriptives: 2026-01-01 14:23:10 to 16:23:00 (10-second)}

Let $f_t$ denote measured system frequency (Hz). Over the provided 2-hour window (N=720), the empirical descriptors are:

\begin{align}
  \overline{f} &= 59.9959 \text{ Hz}, \\
  \min f_t &= 59.971 \text{ Hz}, \\
  \text{P01}(f_t) &= 59.974 \text{ Hz}, \\
  \text{P95}(f_t) &= 60.016 \text{ Hz}.
\end{align}

This window is consistent with stable frequency control. Importantly, the presence of ancillary products such as ECRS is not restricted to rare emergency excursions; in RTC-style co-optimized markets, reserves can clear as a function of system conditions and expected control needs even when the realized frequency trajectory is well-behaved. The appropriate econometric question (deferred to later chapters with full December interval data) is whether the frequency stabilization burden is being shifted toward co-optimized fast-response products and storage participation (cf. \cite{ercot_ancillary_services_pdf}).

\subsection{Low-scarcity energy regime can coexist with active ancillary pricing}

The Jan 1, 2026 capture shows $\text{RTRDPA}=0$ and LMPs at $\$16.83$/MWh across hubs/zones, implying an adder-free, non-scarcity energy interval (Section~I).

\paragraph{Descriptive (R1: prices only).}
At Evidence Rung~R1 (Table~\ref{tab:evidence_rungs}), it is admissible to report
that MCPC for ancillary products can exhibit spikes in neighboring time windows
while contemporaneous energy prices are modest.

\paragraph{Bounded (mechanism language).}
Under RTC+B, such a pattern is \emph{consistent with} scarcity and reliability
value being expressed in ancillary products rather than energy prices when the
marginal binding constraint pertains to deliverable control capability.
However, identification of the specific binding constraint is \textbf{not
identifiable} at R1 absent higher-rung feasibility and deployment evidence.

\section{Mechanistic Interpretation — How RTC+B Reallocates Scarcity Value (ORDC vs ASDC/ECRS/FRRS) and Why Batteries Matter}

\subsection{From ORDC adders to co-optimized scarcity in reserve products}

Legacy ERCOT scarcity pricing under ORDC can be represented (stylized) as an energy price adder linked to reserve margin. Let $R_t$ denote operating reserves and let $g(R_t)$ represent the ORDC-derived scarcity adder. Then the legacy settlement energy price may be conceptualized as:

\begin{equation}
  \pi_{t,z}^{\text{legacy}} \approx \lambda_{t,z}^{\text{LMP}} + g(R_t).
  \label{eq:legacy_ordc}
\end{equation}

RTC+B replaces ``scarcity primarily as an energy adder'' with ``scarcity as co-optimized reserve shadow prices'' embedded in dispatch. The key mathematical shift is that scarcity value is represented in the \emph{dual variables} of the co-optimization constraints (e.g., reserve demand constraints, ramp constraints, and storage SoC feasibility constraints). In co-optimization, if a reserve requirement binds, the KKT multiplier $\mu_t$ for that reserve constraint enters the pricing of the relevant ancillary product directly:

\begin{equation}
  \text{MCPC}_t^{(k)} \approx \mu_t^{(k)}.
  \label{eq:mcpc_dual_intro}
\end{equation}

This representation is not merely cosmetic; it changes incentives. Under Eq.~\eqref{eq:legacy_ordc}, storage profits can be concentrated in rare high-adder energy intervals. Under Eq.~\eqref{eq:mcpc_dual_intro}, storage value can be monetized more continuously through reserve products (e.g., ECRS/FRRS/Reg) provided the resource can maintain SoC headroom and satisfy qualification constraints.

\subsection{A formal statement of ``scarcity-value migration'' into ancillary products under SoC constraints}

Consider a single storage resource with energy schedule $p_t$ and reserve schedule $r_t$ (scalar for simplicity). Suppose a reserve constraint $r_t \ge \underline{r}_t$ binds and SoC bounds bind (or nearly bind) such that additional reserve provision would require foregone energy arbitrage or SoC depletion. Then the co-optimization introduces a Lagrangian:

\begin{equation}
  \mathcal{L} = \sum_t \left( C_t(p_t) + D_t(r_t) \right)
  + \sum_t \alpha_t (\underline{r}_t - r_t)
  + \sum_t \beta_t (\text{SOC}_t - \overline{\text{SOC}})
  + \sum_t \gamma_t (0 - \text{SOC}_t)
  + \cdots
\end{equation}

KKT optimality implies, for intervals where the reserve constraint binds ($\alpha_t > 0$), the marginal value of reserve enters the shadow price system. In particular, the stationarity condition in $r_t$ yields:

\begin{equation}
  \frac{\partial D_t}{\partial r_t} - \alpha_t + \text{(terms from SoC feasibility coupling)} = 0,
\end{equation}

so that the market-clearing reserve price (MCPC) inherits positive contributions from $\alpha_t$ and from SoC-coupling multipliers when feasibility is scarce.

\paragraph{Bounded (R4 required for SoC attribution).}
In plain terms, \emph{if} SoC headroom is the scarce feasibility input, reserve
products can price that scarcity even if energy remains abundant.
However, attributing observed RT MCPC tails specifically to SoC headroom is
\textbf{not identifiable} without R4 evidence (SoC/telemetry and resource-level
feasibility), per Table~\ref{tab:evidence_rungs} and Chapter~\ref{ch:ident_limits}.

\paragraph{Descriptive (R1: price pattern).}
Independently of constraint attribution, it is admissible at R1 to report that
MCPC for reserve products can exhibit heavy tails even when contemporaneous
energy LMP and SPP are modest (Section~II; see also
\cite{ercot_mcpc_prev_csv,ercot_rtm_lmp_table_screenshot}).

\subsection{Implications for ECRS and FRRS in the RTC+B regime}

ECRS and FRRS (and regulation products) are, by design, \emph{control capability} products. Their scarcity is associated with fast response and sustained deliverability, not merely installed MW. Storage resources are structurally advantaged in response speed but structurally constrained by SoC. RTC+B internalizes this trade:

\begin{itemize}
  \item \textbf{ECRS/FRRS price sensitivity to SoC feasibility.} Even in low-scarcity energy periods, MCPC can rise if the system needs fast control capability and marginal providers must preserve SoC.
  \item \textbf{Reduced dependence on rare energy scarcity adders.} As scarcity value migrates into reserve products, the extreme right tail of energy prices can be damped relative to an ORDC-adder-driven regime, but reserve price tails can become more prominent.
  \item \textbf{Investor-level consequence.} Merchant battery valuation must be decomposed into (i) energy arbitrage under co-optimized dispatch and (ii) ancillary revenue under SoC-feasible provision. The correct valuation object is therefore not a single ``energy volatility'' metric but a portfolio of co-optimized products with correlated constraints.
\end{itemize}

\subsection{A testable hypothesis set for the December 2025 transition}

The descriptive evidence in Section~II motivates three competing hypotheses for the post-RTC+B regime (to be tested with full December 2025 interval data and subsequent months):

\begin{description}
  \item[H1 (Reliability-value migration).] Scarcity value previously expressed through ORDC energy adders is partially reallocated into MCPC tails for ECRS/FRRS/Reg products, lowering the frequency and magnitude of extreme energy settlement prices while increasing the frequency of moderate-to-high ancillary prices.
  \item[H2 (Participant adaptation / transient).] The observed MCPC tail behavior is dominated by transitional bidding/qualification behavior in the early RTC+B period and will attenuate as participants optimize SoC management and bidding strategies.
  \item[H3 (Latent fragility).] RTC+B stabilizes visible energy price volatility while increasing structural reliance on fast-response products; if storage participation or qualification falters, scarcity may re-emerge sharply in either reserve or energy prices, potentially producing more concentrated tail events.
\end{description}

Each hypothesis can be tested using: (i) full-interval distributions of $\pi^{\text{SPP}}$ and $\lambda^{\text{LMP}}$ across December 2025 and adjacent pre/post windows, (ii) MCPC distributions by product, (iii) correlation of high MCPC events with frequency excursions, net load ramps, and congestion indicators, and (iv) storage participation proxies where available.

\subsection{Immediate report roadmap (what we do next with your uploaded ERCOT PDFs/CSVs)}

The next analytical step (the portion you requested as ``I, II, III'') is to expand into: (a) empirical hub/zone price histograms for December 2025 using full RTM-SPP exports, (b) event studies around high MCPC intervals to identify co-optimization binding constraints, and (c) a revenue decomposition model for a canonical 100 MW / 4-hour battery. Those steps require complete December interval settlement extracts beyond the short-window datasets currently in hand. The present sections therefore establish the \emph{mechanistic frame}, the \emph{identification discipline}, and the \emph{initial empirical descriptors} from the available artifacts.


\clearpage


% ==========================================
% (B) METHODOLOGY + IDENTIFICATION RULES
% ==========================================
\section{Identification Rules and Evidence Ladder}
\label{ch:ident_limits}

\section{Method scope and evidence standards}

This report distinguishes between (i) settlement-quality quantitative claims derived from complete interval extracts and (ii) operationally informative but non-final claims derived from public dashboards and short-window exports. Where the underlying data do not constitute a complete month of settlement values, the report restricts itself to:
\begin{enumerate}
  \item formally correct definitions and decompositions,
  \item provable identification rules for what can be inferred from the artifacts,
  \item descriptive statistics for the included time windows, and
  \item testable hypotheses and protocols that become executable once full interval settlement extracts are integrated.
\end{enumerate}

\section{Canonical Evidence Rungs (global reference)}
\label{sec:evidence_rungs}

Table~\ref{tab:evidence_rungs} is the canonical (global) reference for
\emph{what is identifiable when}. Downstream chapters must explicitly cite the
applicable rung for each inference beyond purely definitional statements.

\begin{table}[h!]
\centering
\caption{Evidence rungs (canonical reference point for identifiability).}
\label{tab:evidence_rungs}
\begin{tabular}{p{0.75in}p{2.15in}p{3.45in}}
\toprule
\textbf{Rung} & \textbf{Data scope} & \textbf{What is identifiable / admissible} \\
\midrule
\textbf{R1} & Prices only (settlement-grade) & Descriptive distributional statements about price objects (e.g., quantiles, tail exceedance rates, co-movements). No resource-level attribution; no constraint attribution. \\
\textbf{R2} & Prices + awards/schedules (DA/RT awards where applicable) & Descriptive conditioning of price outcomes on procurement posture and bid-side geometry. Still no deployment/telemetry-based attribution; DA posture is not exogenous. \\
\textbf{R3} & Prices + awards + deployments (AS deployments, frequency/AGC proxies, system actions) & Descriptive linkage between price outcomes and realized control effort/deployment indicators under matched conditions. Attribution remains limited to observed deployments; no SoC inference without telemetry. \\
\textbf{R4} & SoC/telemetry and resource-level feasibility (plus awards/prices) & Feasibility-channel diagnostics tied to intertemporal constraints (SoC headroom, charge/discharge feasibility), including resource-level consistency checks. Still requires explicit guardrails for causal claims. \\
\bottomrule
\end{tabular}
\end{table}

\noindent\textbf{Cross-reference discipline.} When a construct is ERCOT-defined
(e.g., MCPC, ORDC, ASDC, RTC+B), its regulatory grounding is traced in
Appendix~\ref{app:reg_trace_full}; when a scarcity statement is made, it is
tagged using the scarcity typology in Section~\ref{sec:scarcity_typology}.

\section{Terminology (strict usage)}
\label{sec:ident_terms}

Throughout this report:
\begin{itemize}
  \item \textbf{Identifiable} means point-identifiable from the evidence available at the stated rung, up to stated aggregation/measurement conventions.
  \item \textbf{Not identifiable} means the quantity is not point-identified at that rung; any statement requires additional data (higher rung) or is prohibited.
  \item \textbf{Partially bounded} means the quantity is not point-identified, but defensible bounds can be reported under explicit assumptions and only on declared common support.
\end{itemize}

\section{Scarcity typology (administrative vs physical)}
\label{sec:scarcity_typology}

This report uses a strict scarcity typology to prevent misinterpretation of
Winter Storm Fern outcomes and to prevent implicit equivalence between high
ancillary prices and physical shortfall.

\paragraph{Administrative scarcity (co-optimization / feasibility scarcity).}
Administrative scarcity refers to scarcity signals arising from binding
constraints in the market-clearing optimization and settlement framework,
including product-specific requirements, deliverability constraints, and
intertemporal feasibility constraints (e.g., ESR state-of-charge headroom).
Administrative scarcity can produce high MCPC without energy-side inadequacy.

\paragraph{Physical scarcity (energy inadequacy).}
Physical scarcity refers to an inability to serve realized energy demand with
available supply under security constraints, reflected operationally in energy
price tails and/or explicit energy-side scarcity components (e.g., nonzero
energy adders) under the settlement constructs used in this report.

\paragraph{Ambiguous.}
A pattern is labeled \emph{ambiguous} when it is consistent with administrative
scarcity but cannot rule out physical scarcity without higher-rung evidence.

\paragraph{Non-equivalence rule.}
High MCPC is \textbf{not} treated as evidence of physical scarcity. High MCPC is
interpreted as evidence consistent with administrative scarcity unless and until
energy-side scarcity signatures and/or higher-rung operational evidence support
physical scarcity.

\begin{table}[h!]
\centering
\caption{Evidence-to-scarcity typology map (canonical summary).}
\label{tab:evidence_to_scarcity}
\begin{tabular}{p{2.35in}p{1.15in}p{2.85in}}
\toprule
\textbf{Evidence (observable pattern)} & \textbf{Type tag} & \textbf{Notes / missing data (if any)} \\
\midrule
Nonzero energy adder $a^{\mathrm{RT}}>0$ and/or extreme energy price tails & Physical-consistent & Physical scarcity remains a system-level interpretation; constraint attribution requires higher rungs. \\
High MCPC tails with modest energy prices and low/zero adders & Administrative-consistent & Physical scarcity is not implied; binding-constraint attribution is not identifiable without deployments/telemetry. \\
High MCPC tails coincident with frequency excursions / deployments (telemetry in scope) & Administrative-consistent (stronger) & Still not SoC-specific without R4; supports feasibility/operational-stress channel consistency. \\
DA geometry fragility (concentrated/thin marginal DA stack) paired with higher conditional RT MCPC tails & Administrative-consistent (leading-indicator) & Does not imply physical shortage; interpretable as fragility under awards/geometry (R2) and tails (R1). \\
Energy tails and ancillary tails rising together under matched stress & Ambiguous & Distinguishing channels requires deployments (R3) and feasibility telemetry (R4). \\
\bottomrule
\end{tabular}
\end{table}

\section{Core definitions and decomposition}

Core price-object definitions and the posted identity
$\pi_{t,z}^{\text{SPP}}=\lambda_{t,z}^{\text{LMP}}+a_{t,z}^{\text{RT}}$ are stated
once in Section~\ref{sec:price_formation_stack}; downstream text should refer
there rather than re-stating symbols.

\section{Identification rules}

\subsection{IR1: Adder-free interval identification (local)}
If a display indicates $\text{RTRDPA}=0$ for the relevant settlement points at time $t$, then the settlement point price equals the SCED LMP for those points at $t$:
\[
\pi_{t,z}^{\text{SPP}}=\lambda_{t,z}^{\text{LMP}}.
\]
This is a point-in-time identification rule and does not imply the distribution of adders over a month.

\subsection{IR2: ``Energy calm / reserves active'' compatibility}
It is possible for $\pi_{t,z}^{\text{SPP}}$ and $\lambda_{t,z}^{\text{LMP}}$ to remain modest while MCPC for ancillary products exhibits tail behavior.

\paragraph{Descriptive (R1: prices only).}
At Evidence Rung~R1 (Table~\ref{tab:evidence_rungs}), this is an admissible
\emph{descriptive} statement about joint price realizations.

\paragraph{Bounded (mechanism language).}
Under RTC+B, such co-movement is \emph{consistent with} a binding marginal
constraint associated with deliverable control capability rather than energy
adequacy; however, the specific binding constraint is \textbf{not identifiable}
without higher-rung feasibility information.

\paragraph{Testable (R1: price-based signature).}
A price-only signature is evaluated by comparing:
\[
\text{TailMass}\left(\text{MCPC}^{(k)}\right) \quad \text{vs.}\quad \text{TailMass}\left(\pi^{\text{SPP}}\right),
\]
within declared windows and common support, and by reporting whether the former
is materially larger in intervals without elevated adders.

\subsection{IR3: No-inference list (guardrails)}
Absent the required rung-specific data, the following are \textbf{not
identifiable} and therefore not asserted:
\begin{itemize}
  \item full-month distributions for December 2025 RTM-SPP or adders without complete settlement-grade extracts (R1+),
  \item resource-level SoC trajectories, feasibility headroom, or realized dispatch for specific ESRs without telemetry/SoC data (R4),
  \item causal attribution of MCPC spikes to specific constraints beyond descriptive or bounded observational designs (requires explicit identification assumptions; otherwise not identifiable).
\end{itemize}

\section{Statistical reporting conventions}

For all quantitative series $x_t$, the report uses:
\[
\{\text{mean}, \text{median}, Q_{0.90}, Q_{0.95}, Q_{0.99}, \max, \#(x_t>\tau)\}
\]
with thresholds $\tau\in\{1,2,5\}$ selected by product type and interpretability.

In addition, because five-minute and ten-second operational series exhibit serial dependence and heavy tails, inference (where reported) is accompanied by robust uncertainty summaries rather than normal-theory standard errors. Specifically:

\begin{itemize}
  \item \textbf{Serial correlation:} when constructing confidence intervals for quantile differences or exceedance-rate differences across windows, we use a moving-block bootstrap with block length $b$ (default: $b=12$ five-minute intervals, i.e., one hour) to preserve local dependence.
  \item \textbf{Effect sizes:} whenever a window comparison is reported, we include both an absolute difference (e.g., $\Delta Q_{0.99}$) and a relative ratio (e.g., $Q^{(B)}_{0.99}/Q^{(A)}_{0.99}$), because tails can shift by both translation and scale.
  \item \textbf{Multiple comparisons:} when a family of tests is conducted across products $\times$ windows $\times$ quantiles, we control the false discovery rate (FDR) using the Benjamini--Hochberg procedure and report both raw and adjusted $p$-values. This is a guardrail against over-interpreting isolated ``significant'' findings in a high-dimensional descriptive study.
  \item \textbf{Nonparametric emphasis:} primary distribution-shift evidence is reported via quantile profiling and tail exceedance rates rather than assuming finite variance; kurtosis is reported only as a descriptive indicator and not used as a basis for inference.
\end{itemize}

Throughout, the report maintains a ``commit-then-reveal'' discipline: threshold selections ($\tau$) and quantile levels ($q$) are specified in the Metric Dictionary and Sensitivity Registry (Appendix~\ref{app:sensitivity_wave2}) before outcome inspection, and any post hoc exploratory checks are labeled explicitly as exploratory.

\clearpage


% ==========================================
% (C) HYPOTHESIS REGISTER + TESTS TABLE
% ==========================================
\chapter{Hypothesis Register and Evaluation Plan}

\section{Hypothesis Register}

This report maintains an explicit hypothesis register to prevent “single-story” conclusions and to force testability. Each hypothesis below is paired with (i) a primary metric, (ii) an evaluation design, (iii) a null hypothesis, and (iv) the minimum dataset required to execute the test at settlement-grade rigor.

\paragraph{Evidence rung discipline.}
For each hypothesis, the minimum dataset is expressed in the Evidence Rungs
framework (Table~\ref{tab:evidence_rungs}). Hypotheses that require higher-rung
inputs (deployments or SoC/telemetry) are treated as \emph{not identifiable} at
lower rungs and are not evaluated using price-only proxies.

\subsection{H1: Scarcity value migrates into ancillary products}

\textbf{Claim.} Post-RTC+B, a material share of scarcity value previously expressed through energy-side scarcity adders (or extreme RTM-SPP tails) is reallocated into ancillary service capacity prices (MCPC), particularly for ECRS and regulation products, conditional on matched system conditions.

\textbf{Primary metric.} A comparative tail measure:
\[
\text{TailMass}\!\left(\text{MCPC}^{(\text{ECRS/FRRS/Reg})}\right)
\;\;\text{vs.}\;\;
\text{TailMass}\!\left(\text{RTRDPA}\right)\;\text{and}\;\text{TailMass}\!\left(\pi^{\text{SPP}}\right),
\]
where \(\text{TailMass}(x)\) is operationalized as exceedance rates above pre-specified thresholds (e.g., \(\#(x>\tau)\)) and/or high-quantile shifts (e.g., \(Q_{0.99}\)).

\textbf{Test design.} Pre/post comparison across matched weather/load windows with:
(i) Kolmogorov–Smirnov tests on product distributions,
(ii) tail exceedance rate comparisons (with confidence intervals), and
(iii) robustness checks under alternative matching specifications (e.g., net-load ramp bins, temperature bins, outage bins).

\textbf{Null.} No increase in ancillary MCPC tails and no reduction in energy/adders tail behavior after RTC+B once conditions are matched.

\textbf{Data required.} Full December 2025 interval RTM-SPP series \emph{with} all relevant adders, full ancillary MCPC series by product, and a pre-window comparison set (e.g., November 2025), plus minimal system-condition covariates (load, temperature proxy, outage proxy).

\subsection{H2: Energy volatility declines post-RTC+B}

\textbf{Claim.} RTC+B reduces energy price volatility and/or the incidence of extreme energy-side scarcity expressions, conditional on comparable system stress.

\textbf{Primary metric.} Volatility and tail descriptors for real-time hub settlement prices, including:
\[
\text{kurtosis}\!\left(\pi_{\text{Hub}}^{\text{SPP}}\right), \quad
Q_{0.99}\!\left(\pi_{\text{Hub}}^{\text{SPP}}\right), \quad
\#\!\left(a^{\text{RT}} > \tau_a\right),
\]
where \(a^{\text{RT}}\) denotes adders (e.g., RTRDPA) and \(\tau_a\) is a policy-relevant threshold.

\textbf{Test design.} Pre/post comparison with event controls for net-load ramps and outage conditions, including matched-sample analysis and sensitivity to the choice of stress periods.

\textbf{Null.} No meaningful change in real-time energy volatility or the frequency of high-adder intervals attributable to RTC+B.

\textbf{Data required.} RT hub and/or load-zone RTM-SPP series plus adders (settlement-quality), along with operational proxies for ramps and outages (or a documented substitute matching framework).

\subsection{H3: Reliability improves via earlier procurement of control capability}

\textbf{Claim.} RTC+B improves reliability by procuring and deploying control capability earlier and more efficiently, reducing the magnitude and/or duration of frequency deviations conditional on stress.

\textbf{Primary metric.} Frequency stability and control-effort measures, including:
\[
I_f = \sum_t \left|f_t - 60\right| \Delta t, \quad
\#\!\left(f_t \notin [60-\epsilon,\,60+\epsilon]\right),
\]
and (if available) deployment intensity metrics conditioned on ramp indicators.

\textbf{Test design.} Difference-in-differences across pre/post windows and conditional regressions on system stress proxies (ramps, load level), with robustness checks under alternate band definitions \(\epsilon\) and sampling granularities.

\textbf{Null.} No improvement in frequency stability metrics and no evidence of improved control procurement/effort conditional on comparable stress.

\textbf{Data required.} High-resolution frequency series, ancillary deployment series (preferably), and stress proxies (net-load ramp measures or defensible substitutes).

\subsection{H4: Investor revenue shifts from energy tails to ancillary services}

\textbf{Claim.} For a canonical battery energy storage resource (ESR), the post-RTC+B revenue mix shifts away from reliance on rare energy tail events and toward more continuous ancillary service revenues, subject to SoC feasibility and qualification.

\textbf{Primary metric.} Revenue share decomposition:
\[
\frac{\Pi_{\text{AS}}}{\Pi} \;\;\text{vs.}\;\; \frac{\Pi_{\text{Energy}}}{\Pi},
\]
where \(\Pi\) is total gross market revenue and \(\Pi_{\text{AS}}\) aggregates ancillary products (ECRS/FRRS/Reg/RRS/Non-Spin as applicable).

\textbf{Test design.} Cashflow decomposition for a canonical ESR across matched pre/post regimes using settlement rules and observed prices, augmented by a feasibility-constrained dispatch/award model (SoC dynamics) for internal consistency.

\textbf{Null.} No systematic shift in revenue composition across regimes once differences in market conditions are controlled.

\textbf{Data required.} RT/DA energy prices, ancillary MCPC prices and awards, settlement rules (including adders where relevant), and a documented ESR parameter set (power, duration, efficiency, cycling constraints, degradation cost).

\subsection{H5: Latent fragility risk increases via reliance on ESR feasibility}

\textbf{Claim.} RTC+B can reduce visible energy-side volatility while increasing structural dependence on qualified fast-response resources (notably storage). If ESR feasibility (SoC headroom, qualification) is constrained, scarcity may manifest abruptly in reserves and/or energy.

\textbf{Primary metric.} Co-movement and stress indicators, including:
\[
\text{Corr}\!\left(\mathbf{1}\{\text{MCPC}^{(k)}>\tau_k\},\, \text{SoC-stress proxy}\right),
\]
plus incidence of performance/penalty events if available, and tail co-movement between MCPC spikes and adder/energy spikes.

\textbf{Test design.} Event studies around high-MCPC triggers, stress scenario analysis (e.g., extreme net-load ramps, thermal outage clusters), and tail dependence diagnostics (e.g., conditional exceedance probabilities).

\textbf{Null.} No evidence of increased reliance indicators; high-MCPC events do not systematically co-occur with feasibility stress proxies or performance issues.

\textbf{Data required.} MCPC series, ancillary awards/deployments, ESR participation measures, and performance/penalty series where available. If direct SoC telemetry is unavailable, the report will employ conservative SoC stress proxies and explicitly bound the inference.
\clearpage


% ==========================================
% (D) EVENT STUDY PROTOCOL TEMPLATE
% ==========================================
\chapter{Event Study Protocol for High-MCPC and Scarcity Episodes}

\section{Event definitions}

Define an event time $t^\star$ as any interval satisfying at least one trigger:
\begin{align}
  \text{E1:}\quad & \text{MCPC}_{t}^{(\text{ECRS})} > \tau_{\text{ECRS}}, \\
  \text{E2:}\quad & \text{MCPC}_{t}^{(\text{REG})} > \tau_{\text{REG}}, \\
  \text{E3:}\quad & a_{t,z}^{\text{RT}} > \tau_{a} \text{ for hub/zone } z, \\
  \text{E4:}\quad & \pi_{t,z}^{\text{SPP}} > \tau_{\pi} \text{ for hub/zone } z.
\end{align}

Default thresholds (modifiable once full month data are integrated):
\[
\tau_{\text{ECRS}}=2\ \$/\text{MW-h},\quad
\tau_{\text{REG}}=2\ \$/\text{MW-h},\quad
\tau_{a}=50\ \$/\text{MWh},\quad
\tau_{\pi}=500\ \$/\text{MWh}.
\]


\section{Episode ledger construction (scarcity clustering protocol)}
\label{sec:episode_ledger_protocol}

Wave~2 introduces scarcity episodes as the basic unit for aligning market scarcity to physical response metrics (Wave~3). This section formalizes the episode ledger in a reproducible, audit-grade way.

\subsection{Exceedance indicators and thresholds}
Let $\mathrm{MCPC}_k(t)$ denote the five-minute MCPC for ancillary product $k$ at interval $t$. Define a threshold $u_k$ using one of the pre-registered options in the Sensitivity Registry (Appendix~\ref{app:sensitivity_wave2}):

\begin{enumerate}
  \item \textbf{Quantile threshold:} $u_k = Q_{q,k}$ with $q \in \{0.99,0.999\}$ computed within the analysis window,
  \item \textbf{Fixed threshold:} $u_k = \tau_k$ with $\tau_k \in \{50,100,500\}$ \$/MW-h (product-specific),
  \item \textbf{Hybrid threshold:} $u_k = \max\{Q_{0.99,k},\tau_k\}$ to avoid degenerate tails in low-volatility periods.
\end{enumerate}

Define the exceedance indicator:
\[
\mathbb{S}_k(t) = \mathbf{1}\{\mathrm{MCPC}_k(t) > u_k\}.
\]

\subsection{Episode definition}
An episode $e$ for product $k$ begins at time $t_e^{\text{start}}$ if $\mathbb{S}_k(t_e^{\text{start}})=1$ and $\mathbb{S}_k(t_e^{\text{start}}-1)=0$. The episode ends at the last time $t_e^{\text{end}}$ such that $\mathbb{S}_k(t)=1$ for all $t\in[t_e^{\text{start}},t_e^{\text{end}}]$ and $\mathbb{S}_k(t_e^{\text{end}}+1)=0$.

Define duration in intervals and minutes:
\[
D_e^{(\text{int})}=t_e^{\text{end}}-t_e^{\text{start}}+1,
\qquad
D_e^{(\text{min})}=5\cdot D_e^{(\text{int})}.
\]

\subsection{Episode intensity, mass, and ``tail energy''}
For each episode $e$, compute:

\begin{align}
\text{Peak}_e &= \max_{t\in e}\mathrm{MCPC}_k(t),\\
\text{Mean}_e &= \frac{1}{D_e^{(\text{int})}}\sum_{t\in e}\mathrm{MCPC}_k(t),\\
A_e &= \sum_{t\in e}\left(\mathrm{MCPC}_k(t)-u_k\right)\Delta t,
\qquad \Delta t = \tfrac{5}{60}\text{ hour}.
\end{align}

$A_e$ is the area-above-threshold (``tail energy'') and is the preferred episode mass metric because it distinguishes a one-interval spike from a sustained scarcity run.

\subsection{Multi-product episode coherence}
To quantify scarcity fragmentation versus coherence, define the tail multiplicity at time $t$:
\[
M(t)=\sum_k \mathbb{S}_k(t),
\]
and define a ``system episode'' as any maximal interval set where $M(t)\ge 1$ for consecutive $t$. For each system episode $E$, record:

\begin{itemize}
  \item the set of participating products $\mathcal{K}_E=\{k: \exists t\in E \text{ with } \mathbb{S}_k(t)=1\}$,
  \item $\max_{t\in E} M(t)$ (peak multiplicity),
  \item coherence share $\frac{1}{|E|}\sum_{t\in E}\mathbf{1}\{M(t)\ge 2\}$.
\end{itemize}

These fields allow later assessment of whether RTC+B concentrates scarcity in a single product or creates multi-product stress patterns.

\subsection{Ledger fields and reproducibility capsule}
The episode ledger is stored as a table with one row per (product, episode) and optional system-episode joins. Minimum fields:

\begin{itemize}
  \item \texttt{product\_k}, \texttt{t\_start}, \texttt{t\_end}, \texttt{duration\_int}, \texttt{duration\_min},
  \item \texttt{threshold\_u}, \texttt{peak}, \texttt{mean}, \texttt{tail\_energy\_A},
  \item \texttt{window\_label} $\in \{\text{Pre},\text{Fern},\text{Post}\}$,
  \item \texttt{missing\_intervals\_flag} (QA),
  \item \texttt{join\_basis} (max/mean/CVaR aggregation choice for any hour-level comparisons).
\end{itemize}

A reproducibility capsule accompanies each ledger export: dataset filename list, row counts, missing-interval counts by day, and a deterministic hash of each input file where feasible.



\subsection{Episode ledger output contract (schema, units, and sample row)}
\label{sec:episode_ledger_output_contract}

To ensure downstream reproducibility (Wave~3 frequency/IMFR conditioning; Wave~4 investor economics; and external replication by IMM or market participants), the episode ledger is treated as a first-class deliverable with a fixed schema. The ledger is exported as a flat file (CSV/Parquet) with one row per \emph{(product, episode)}. All fields are defined below with explicit units.

\paragraph{Primary identifiers.}
\begin{itemize}
  \item \texttt{episode\_id}: unique integer or UUID; stable under re-runs given the same threshold policy and input series.
  \item \texttt{product\_k}: ancillary product code (e.g., REGUP, REGDN, RRS, ECRS, NSPIN).
  \item \texttt{threshold\_policy}: string identifier of the sensitivity lever used (e.g., \texttt{SR1\_Q099}, \texttt{SR2\_FIX100}, \texttt{SR3\_HYBRID}).
  \item \texttt{window\_label}: \{\texttt{PRE}, \texttt{FERN}, \texttt{POST}\} under the registered Fern partition.
\end{itemize}

\paragraph{Episode timing fields (5-minute base).}
Let $t$ index SCED five-minute intervals and let $\Delta t=\tfrac{5}{60}$ hour.
\begin{itemize}
  \item \texttt{t\_start\_utc}: episode start timestamp at 5-minute granularity.
  \item \texttt{t\_end\_utc}: episode end timestamp at 5-minute granularity.
  \item \texttt{duration\_intervals}: $D_e^{(\mathrm{int})}$ (count of 5-min intervals).
  \item \texttt{duration\_minutes}: $5\cdot D_e^{(\mathrm{int})}$ (minutes).
  \item \texttt{duration\_hours}: $\Delta t\cdot D_e^{(\mathrm{int})}$ (hours).
\end{itemize}

\paragraph{Episode intensity and mass fields.}
With threshold $u_k$ defined per policy:
\begin{itemize}
  \item \texttt{u\_k}: episode threshold $u_k$ in \$/MW-h.
  \item \texttt{peak\_mcpc}: $\max_{t\in e}\mathrm{MCPC}_k(t)$ in \$/MW-h.
  \item \texttt{mean\_mcpc}: $\frac{1}{D_e^{(\mathrm{int})}}\sum_{t\in e}\mathrm{MCPC}_k(t)$ in \$/MW-h.
  \item \texttt{tail\_energy}: $A_e=\sum_{t\in e}(\mathrm{MCPC}_k(t)-u_k)\Delta t$ in \$/MW (area-above-threshold).
  \item \texttt{tail\_mass\_intervals}: $\sum_{t\in e}\mathbf{1}\{\mathrm{MCPC}_k(t)>u_k\}$ (should equal \texttt{duration\_intervals} by construction).
\end{itemize}

\paragraph{Coherence / fragmentation fields.}
To support multi-product scarcity analysis:
\begin{itemize}
  \item \texttt{multiplicity\_mean}: $\frac{1}{D_e^{(\mathrm{int})}}\sum_{t\in e} M_q(t)$ where $M_q(t)=\sum_j \mathbf{1}\{\mathrm{MCPC}_j(t)>u_j\}$ under the same $q$ or fixed-threshold family.
  \item \texttt{multiplicity\_max}: $\max_{t\in e} M_q(t)$.
  \item \texttt{joint\_tail\_flag}: indicator for whether any other product exceeded its threshold during the episode ($\exists j\neq k$ with $\mathbb{S}_j(t)=1$ for some $t\in e$).
\end{itemize}

\paragraph{Alignment-ready summaries (optional but recommended).}
These fields are defined to make Wave~3 conditioning deterministic:
\begin{itemize}
  \item \texttt{pre\_buffer\_minutes}, \texttt{post\_buffer\_minutes}: buffers used for frequency alignment (registered in Section~\ref{sec:event_window_alignment}).
  \item \texttt{episode\_hour}: $h(t^{\mathrm{start}}_e)$ (delivery hour containing the start).
  \item \texttt{da\_posture\_index}: $Z_k(h)$ from Section~\ref{sec:wave2_da_conditioning} evaluated at the episode hour.
\end{itemize}

\paragraph{Sample row template.}
For avoidance of doubt, a representative row has the following shape:
\begin{verbatim}
episode_id,product_k,threshold_policy,window_label,t_start_utc,t_end_utc,
duration_intervals,duration_minutes,duration_hours,u_k,peak_mcpc,mean_mcpc,
tail_energy,tail_mass_intervals,multiplicity_mean,multiplicity_max,joint_tail_flag,
pre_buffer_minutes,post_buffer_minutes,episode_hour,da_posture_index
\end{verbatim}

This schema is normative: any downstream figure or claim tied to scarcity episodes must reference a specific \texttt{episode\_id} and threshold policy, and the report must remain reproducible under re-export of the ledger.


\section{Event window and alignment}
\label{sec:event_window_alignment}

For each trigger time $t^\star$, construct a symmetric analysis window around the trigger:
\[
\mathcal{W}(t^\star)=\{t^\star-L,\ldots,t^\star,\ldots,t^\star+U\},
\]
where $L$ and $U$ are specified in intervals at the native resolution of the event-defining series. For five-minute MCPC data the default is $(L,U)=(12,24)$, corresponding to $-60$ minutes through $+120$ minutes. Sensitivity checks vary $(L,U)$ over $\{(6,12),(12,24),(24,48)\}$ to test robustness to window choice.

\subsection{Canonical timebase and resampling}
All event studies are computed on a canonical five-minute timebase (SCED interval clock). Series available at coarser resolution are aligned as follows:

\begin{itemize}
  \item \textbf{DA posture variables:} hourly DA awards $Q^{DA}_k(h)$ are mapped to five-minute intervals by the containing-hour function $h(t)$; thus each five-minute interval inherits the DA posture of its hour.
  \item \textbf{15-minute RT price series:} any 15-minute settlement point price exports are aligned to five minutes by a piecewise-constant mapping within each 15-minute block. This preserves block averages but can understate intra-block extremes; therefore, five-minute MCPC remains the primary scarcity indicator.
  \item \textbf{10-second frequency:} frequency samples $f(\tau)$ at 10-second resolution are aligned to five minutes by computing within-interval summary statistics (mean, minimum, and integrated absolute deviation). When reporting event profiles, the five-minute frequency statistic is treated as the dependent outcome at the same $t$ index.
\end{itemize}

\subsection{Lead--lag variants}
Because procurement and control actions can lead or lag price realizations, we report lead--lag variants by shifting the conditioning series by $\ell \in \{-2,-1,0,+1,+2\}$ intervals:
\[
x^{(\ell)}_t = x_{t+\ell}.
\]
This is purely descriptive and is used to identify whether scarcity events tend to precede or follow frequency deviations or energy price movements.

\subsection{Serial dependence and block resampling}
Event windows can overlap and five-minute series are serially correlated. When constructing uncertainty summaries across events, we use a moving-block bootstrap over events (resampling event indices) and, where necessary, over contiguous time blocks within events. This avoids treating adjacent event windows as independent observations.

\section{Event metrics}


Within each event window, compute:
\begin{itemize}
  \item Energy prices: $\lambda_{t}^{\text{LMP}}$, $\pi_{t}^{\text{SPP}}$, and adders $a_t^{\text{RT}}$ where available.
  \item Ancillary prices: $\text{MCPC}_{t}^{(k)}$ for all products $k$.
  \item Frequency: $f_t$ (aligned via nearest-neighbor or interpolation).
  \item Proxy ramp stress: $\Delta \lambda_t$ and hub dispersion measures if nodal/zone set is available.
\end{itemize}

Define event profiles as the median (and interquartile range) across events:
\[
\widetilde{x}_{\Delta} = \text{median}_{t^\star}\{x_{t^\star+\Delta}\},
\quad
\text{IQR}_{\Delta} = Q_{0.75} - Q_{0.25}.
\]

\section{Causal discipline and limitations}

Event studies provide conditional correlation patterns, not causal identification. Causal attribution requires either structural modeling with validated constraints or quasi-experimental designs (e.g., matched conditions, instrumental variables). This report treats event studies as a diagnostic layer supporting or refuting hypotheses and informing where deeper causal work is warranted.

\begin{figure}[ht]
    \centering
    \begin{tikzpicture}[font=\sffamily\small]
        % Patterns library required in preamble: \usetikzlibrary{patterns}
        
        % Axes
        \draw[thick] (0,0) -- (10,0) node[below] {Event Timeline (Fern Window)};
        \draw[->, thick] (0,0) -- (0,6) node[left] {Total Settlement Price (\$/MWh)};
        
        % Annotations for Regimes
        \node[above, font=\bfseries] at (2.5, 6) {Legacy Regime};
        \node[above, font=\bfseries] at (7.5, 6) {RTC+B Regime};
        \draw[dotted, thick] (5,0) -- (5,6.5);

        % --- LEGACY STACK (Left) ---
        % 1. System Lambda (Base)
        \filldraw[fill=white, draw=black] (1,0) rectangle (2, 1.5);
        \node at (1.5, 0.75) {LMP};
        
        % 2. ORDC Adder (Dominant) - Crosshatch
        \filldraw[pattern=north east lines, pattern color=black!60, draw=black] (1,1.5) rectangle (2, 5.0);
        \node[fill=white, inner sep=1pt] at (1.5, 3.25) {Energy Adder};
        
        % 3. AS (Small) - Dots
        \filldraw[pattern=dots, pattern color=black, draw=black] (2.2,0) rectangle (3.2, 0.5); 
        \node[below, font=\scriptsize, align=center] at (2.7, 0) {Decoupled\\AS};

        % --- RTC+B STACK (Right) ---
        % 1. System Lambda (Base)
        \filldraw[fill=white, draw=black] (6,0) rectangle (7, 1.5);
        \node at (6.5, 0.75) {LMP};
        
        % 2. ORDC Adder (Compressed) - Crosshatch
        \filldraw[pattern=north east lines, pattern color=black!60, draw=black] (6,1.5) rectangle (7, 1.8);
        
        % 3. AS Scarcity (Dominant) - Solid Gray
        \filldraw[fill=gray!30, draw=black] (6,1.8) rectangle (7, 5.5);
        \node[align=center] at (6.5, 3.65) {Ancillary\\Scarcity\\(MCPC)};

        % Connectors
        \draw[->, thick, bend left=30] (2, 3.25) to node[midway, above, font=\bfseries] {Scarcity Migration} (6, 3.65);
        
        % Legend
        \matrix [draw, fill=white, below right, font=\scriptsize] at (7.5, 5.5) {
            \node [fill=gray!30, label=right:Ancillary MCPC] {}; \\
            \node [pattern=north east lines, label=right:Reliability Adder] {}; \\
            \node [fill=white, draw=black, label=right:System LMP] {}; \\
        };

    \end{tikzpicture}
    \caption{\textbf{Structural Break in Price Formation.} Under the legacy regime, scarcity was monetized via the Energy Adder (hatched). Under RTC+B, scarcity value migrated to the Ancillary MCPC component (gray), creating a "top-heavy" price stack driven by feasibility constraints.}
    \label{fig:price_stack}
\end{figure}

% =========================================================
% I. DATA, DEFINITIONS, AND THE PRICE-FORMATION STACK
% (RTC+B; evidence-ladder compatible; DA geometry integrated)

\chapter{RTC+B Price Formation - Data \& definitions}
% =========================================================

\section{Data, Definitions, and the Price-Formation Stack Under RTC+B}

\subsection{Empirical artifacts and scope of inference}
\label{sec:artifacts_scope}

This report section is grounded in ERCOT public displays and exports provided by
the author. The purpose is not to ``prove reliability'' from price series, but
to formalize (i) what the available artifacts \emph{identify} about RTC+B price
formation, (ii) what they \emph{do not} identify absent additional quantities
and telemetry, and (iii) how inference strength increases as we move up an
explicit evidence ladder.

\paragraph{Artifact classes (author-provided).}
The empirical substrate spans four operationally distinct layers:

\begin{enumerate}
  \item \textbf{Real-Time energy price artifacts (posted settlement constructs).}
  These include Real-Time Settlement Point Price (RTM-SPP) displays/exports,
  the Real-Time LMP (SCED) hub/zone table display, and system-wide hub price time
  series.\footnote{These artifacts are used to characterize energy-side price
  levels, dispersion across hubs/zones, and (when separately posted) the presence
  or absence of real-time energy adders at a given interval. They are not, by
  themselves, evidence of procurement sufficiency or welfare.}
  Relevant citations include the RTM SPP display and map screenshots, and the
  SCED LMP table screenshot \citep{ercot_rtm_spp_display_pdf,ercot_rtm_spp_map_screenshot,ercot_rtm_lmp_table_screenshot,ercot_systemwide_prices_pdf}.

  \item \textbf{Real-Time ancillary service price artifacts (dual prices under co-optimization).}
  These include Real-Time Market Clearing Prices for Capacity (MCPC) time series
  for ancillary services.\footnote{MCPC is a product-wise capacity price
  (expressed in \$/MW-h in ERCOT postings/exports) produced by the real-time
  co-optimization process; it is best interpreted as a dual variable associated
  with product feasibility/requirement constraints, not as a direct measure of
  physical scarcity or welfare.}
  See \citep{ercot_mcpc_prev_csv,ercot_ancillary_services_pdf}.

  \item \textbf{Day-Ahead posture and quantities (awards/sold).}
  These include DAMASAGG (DA awards) and DAMASSOLD (DA AS sold) artifacts, plus
  product-level cleared/self-arranged series where available.\footnote{These
  artifacts move the analysis from \emph{price-only} into \emph{price+awards}
  inference: they allow us to condition real-time scarcity outcomes on the
  day-ahead posture and to evaluate whether scarcity patterns are consistent with
  ``paper-long'' procurement versus feasibility-constrained realization.}

  \item \textbf{Day-Ahead offer/stack geometry artifacts (shape, concentration, fragility).}
  These include aggregated DA offer stack and cleared stack exports (used in
  Wave~3 to compute geometry metrics such as concentration and dispersion across
  price blocks).\footnote{The geometry layer addresses a specific failure mode of
  ``quantity-only'' interpretations: two days can clear the same total MW while
  having radically different fragility if supply is concentrated near the margin.
  Under RTC+B, this matters because feasibility and intertemporal constraints can
  turn a brittle stack into real-time scarcity even when aggregate volumes appear
  adequate.}
\end{enumerate}

\paragraph{Operational frequency proxy artifacts (descriptive only).}
A short-window system frequency time series is included to compute descriptive
frequency proxies and event-conditioned profiles.\footnote{Frequency is treated
as a descriptive operational indicator, not a causal endpoint: without
disturbance logs, deployment series, and control actions, frequency patterns
cannot identify whether a market rule improved reliability. The report therefore
uses frequency proxies for coherence checks and episode annotation, not welfare
claims.}
See \citep{ercot_freq_csv}.

\paragraph{Evidence ladder (identification rungs).}
To prevent category errors that are common in post-change market commentary, the
report commits to an explicit ladder of admissible inference strength:
\[
\text{price-only} \;\rightarrow\; \text{price+awards} \;\rightarrow\;
\text{price+awards+deployments} \;\rightarrow\; (\text{SoC/telemetry}).
\]
\begin{itemize}
  \item \textbf{Price-only} supports distributional facts (tails, co-movement,
  decomposition identities) but not procurement efficacy, welfare, or strategic
  behavior.
  \item \textbf{Price+awards} supports posture-conditional risk statements (e.g.,
  whether high MCPC tails occur disproportionately under ``tight'' DA posture),
  and enables the Day-Ahead Geometry lens (stack shape vs volume).
  \item \textbf{Price+awards+deployments} (when integrated) supports stronger
  operational attribution (e.g., whether scarcity intervals coincide with large
  deployments, constraint binding, or operational stress), but still does not
  directly identify SoC states without telemetry.
  \item \textbf{SoC/telemetry} is required for claims about resource-level or
  fleet-level feasibility trajectories (SoC compression, synchronization, and
  withholding hypotheses).
\end{itemize}

\paragraph{Scope statement (what this section can and cannot establish).}
The objective of Sections~I--III is to establish a disciplined price-formation
stack, mechanistic interpretation, and bounded empirics for early RTC+B
operations. This section does \emph{not} claim that RTC+B improved reliability or
welfare. Those are downstream questions requiring: (i) stable measurement across
regimes, (ii) overlap/common support in stress conditions, and (iii) quantities
(awards/deployments) and, for feasibility claims, telemetry/SoC. These limits are
formalized later in the Identification and Inference Limits chapter.

\subsection{Distinguishing the energy-side stack from the AS-side stack}
\label{sec:stack_distinction}

Informal market discussion often conflates energy-side prices, energy-side
adders, and ancillary capacity prices. Under RTC+B, that conflation becomes
especially costly because scarcity value can migrate across \emph{channels}.

Accordingly, this report distinguishes four objects:

\begin{enumerate}
  \item \textbf{SCED LMP} (energy, 5-minute): the nodal or aggregated locational
  marginal price produced by the Security-Constrained Economic Dispatch (SCED)
  engine, excluding designated real-time price adders (per ERCOT display
  semantics).
  \item \textbf{Real-Time energy price adders} (e.g., reliability deployment
  pricing components when posted): additive components applied to energy-side
  settlement constructs under certain conditions.
  \item \textbf{Settlement Point Price (SPP)} (energy settlement construct): the
  settlement-relevant price at a settlement point, which includes designated
  adders when applicable.
  \item \textbf{MCPC} (ancillary capacity settlement construct): the product-wise
  market clearing price for capacity for ancillary services, which is a distinct
  scarcity channel under co-optimization.
\end{enumerate}

The central empirical theme of this report---``scarcity migration''---is
expressed as a measurable shift in the \emph{right tail} of the scarcity channel:
from energy-side adders (legacy ORDC-centric narratives) toward AS-side MCPC tails
under RTC+B, subject to feasibility and qualification constraints.

\subsection{Notation and posted decomposition identities}
\label{sec:notation_decomp}

Let $t$ index dispatch or settlement intervals (5-minute or 15-minute depending
on the artifact), and let $z$ index settlement points (hubs, load zones, or
buses). Define:

\begin{align}
  \lambda_{t,z}^{\text{LMP}} &:= \text{SCED LMP at time } t \text{ and location } z \\
  a_{t,z}^{\text{RT}} &:= \text{posted real-time energy adder component at } (t,z)
  \quad \text{(e.g., RTRDPA when shown)} \\
  \pi_{t,z}^{\text{SPP}} &:= \text{Settlement Point Price (RTM-SPP) at } (t,z).
\end{align}

The operational relationship \emph{as posted in the referenced ERCOT RTM
displays} is the decomposition:
\begin{equation}
  \pi_{t,z}^{\text{SPP}} \;=\; \lambda_{t,z}^{\text{LMP}} \;+\; a_{t,z}^{\text{RT}}.
  \label{eq:spp_decomposition}
\end{equation}
ERCOT RTM displays further annotate that SPP values include \emph{Reliability
Deployment Price for Energy}, consistent with inclusion of the real-time adder
term when applicable \citep{ercot_rtm_spp_display_pdf,ercot_rtm_lmp_table_screenshot}.

\paragraph{Interpretation discipline (posted identity vs structural model).}
Equation~\eqref{eq:spp_decomposition} is treated as a \emph{posted settlement
mapping for the displayed construct}, not as a complete structural model of all
scarcity constructs in all periods. The report therefore avoids overclaiming from
a single-interval or single-display snapshot and ties claims to the evidence
ladder.

For ancillary products $k \in \mathcal{K}$, let $\mathrm{MCPC}_{k,t}$ denote the
Real-Time Market Clearing Price for Capacity for product $k$ at time $t$. Under
RTC+B, $\mathrm{MCPC}_{k,t}$ is interpreted as a dual price associated with
product feasibility and requirement constraints, and thus it can carry scarcity
value even when $a_{t,z}^{\text{RT}}$ is zero or rare.

% Price Formation Stack Diagram (LMP vs SPP adders vs MCPC)

\begin{figure}[!ht]
\centering
\begin{tikzpicture}
\begin{axis}[
  width=0.88\textwidth,
  height=0.44\textwidth,
  ybar stacked,
  bar width=22pt,
  ymin=0,
  ymax=120,
  axis lines=left,
  ylabel={Price components (stylized units)},
  symbolic x coords={Legacy,RTC{+}B},
  xtick=data,
  enlarge x limits=0.45,
  grid=both,
  grid style={draw=black!12},
  tick label style={font=\small},
  label style={font=\small},
  legend style={draw=none, fill=none, font=\small, at={(0.02,0.98)}, anchor=north west},
]

% Base: SCED LMP (muted blue)
\addplot[fill=LinkBlue!40, draw=black!55] coordinates {
  (Legacy,40) (RTC{+}B,40)
};
\addlegendentry{SCED LMP (system lambda + congestion)}

% Middle: RT adders (muted orange)
\addplot[fill=orange!50, draw=black!55] coordinates {
  (Legacy,35) (RTC{+}B,0)
};
\addlegendentry{Real-Time adders (RTRDPA/ORDC, when applicable)}

% Top: Ancillary scarcity (muted red + hatch pattern)
\addplot[
  fill=red!30,
  draw=black!55,
  postaction={pattern=north east lines, pattern color=black!45}
] coordinates {
  (Legacy,10) (RTC{+}B,55)
};
\addlegendentry{Ancillary scarcity (MCPC, can spike independently)}

% Annotation arrows (repositioned to avoid overlap)
\node[
  font=\footnotesize,
  align=left,
  anchor=west,
  text=red!55!black,
  fill=white,
  draw=black!20,
  inner sep=2pt
] at (axis description cs:0.72,0.45)
{Under RTC{+}B, MCPC can spike\\even when $a^{\mathrm{RT}}=0$};
\draw[-{Latex[length=2mm]}, thick, red!55!black] (axis cs:RTC{+}B,83) -- (axis description cs:0.70,0.43);

\node[
  font=\footnotesize,
  align=left,
  anchor=west,
  text=orange!70!black,
  fill=white,
  draw=black!20,
  inner sep=2pt
] at (axis description cs:0.12,0.55)
{Legacy narrative:\\scarcity dominated by adders};
\draw[-{Latex[length=2mm]}, thick, orange!70!black] (axis cs:Legacy,70) -- (axis description cs:0.10,0.53);

\end{axis}
\end{tikzpicture}
\caption{Figure 4.1: The RTC+B price stack. Unlike legacy framing where scarcity was often expressed primarily through the energy-side adder channel (orange) in the observed settlement stack, RTC+B allows ancillary scarcity (hatched red; MCPC) to manifest independently when feasibility constraints bind, even when the real-time energy adder component is zero or rare. (Stylized schematic; definitions align with Section~\ref{sec:notation_decomp} and Appendix~H.)}
\label{fig:rtcb_price_stack}
\end{figure}

\subsection{A basic identification lemma for the screenshot interval}
\label{sec:ident_lemma_snapshot}

The author-provided Real-Time LMP hub/zone table shows $\text{RTRDPA}=\$0.00$ at
the capture time (updated Jan 1, 2026, 16:15:15), and the hub/zone LMPs are
identical at $\$16.83/\text{MWh}$ (see \citep{ercot_rtm_lmp_table_screenshot}).
The contemporaneous RTM-SPP map (updated Jan 1, 2026, 16:17) shows a system-wide
light-blue field consistent with the same order-of-magnitude pricing and
explicitly states that SPP includes the reliability deployment price for energy
(see \citep{ercot_rtm_spp_display_pdf,ercot_rtm_spp_map_screenshot}).

\begin{lemma}[Adder-free energy identification for the displayed points at capture time]
For the hub/zone set displayed in the capture-time table, if
$\text{RTRDPA}_{t,z}=0$ at time $t$, then for those displayed locations,
\[
\pi_{t,z}^{\text{SPP}} = \lambda_{t,z}^{\text{LMP}}.
\]
\end{lemma}

\begin{proof}
By the posted decomposition in Eq.~\eqref{eq:spp_decomposition},
$\pi_{t,z}^{\text{SPP}}=\lambda_{t,z}^{\text{LMP}}+a_{t,z}^{\text{RT}}$.
The displayed value $\text{RTRDPA}_{t,z}=0$ implies that the posted adder
component shown for that interval is zero for the displayed points (per ERCOT
display semantics). Substituting yields
$\pi_{t,z}^{\text{SPP}}=\lambda_{t,z}^{\text{LMP}}$. \qedhere
\end{proof}

\paragraph{Interpretive boundary.}
This lemma identifies only the \emph{instantaneous} relationship at the capture
time for the displayed points. It does \emph{not} identify the full December 2025
distribution of adders, the empirical frequency of scarcity conditions, or the
mechanism behind any MCPC tail behavior. Those require interval datasets and,
for causal attribution across regimes, overlap/common-support and measurement
invariance checks (see Wave~3).

\subsection{RTC+B as a co-optimization problem with energy-storage state constraints}
\label{sec:coopt_soc}

Under RTC+B, battery energy storage resources (ESRs) are incorporated into
real-time co-optimization such that energy and ancillary services are jointly
scheduled subject to intertemporal feasibility, including state-of-charge (SoC)
constraints. A stylized (but structurally faithful) formulation is:

\begin{align}
  \min_{\{p_t,r_{k,t}\}} \quad &
  \sum_{t} C_t\!\left(p_t\right) \;+\; \sum_t \sum_{k\in\mathcal{K}} D_{k,t}\!\left(r_{k,t}\right)
  \label{eq:coopt_obj_stylized_data} \\
  \text{s.t.}\quad &
  \text{SOC}_{t+1} \;=\; \text{SOC}_t \;+\; \eta_{\text{ch}} p_t^{\text{ch}} \Delta t
  \;-\; \frac{1}{\eta_{\text{dis}}} p_t^{\text{dis}} \Delta t
  \label{eq:soc_dyn}\\
  & 0 \le \text{SOC}_t \le \overline{\text{SOC}}
  \label{eq:soc_bounds}\\
  & 0 \le p_t^{\text{ch}} \le \overline{P}^{\text{ch}}, \quad
    0 \le p_t^{\text{dis}} \le \overline{P}^{\text{dis}}
  \label{eq:power_bounds}\\
  & \text{Network/security constraints (SCED feasibility, congestion, limits)}
  \label{eq:network_constraints}\\
  & \text{Ancillary service feasibility/requirement constraints (product-specific)}
  \label{eq:as_constraints_stylized}\\
  & \text{Coupling (stylized):}\quad r_{k,t} \le \overline{P}^{\text{dis}} - p_t^{\text{dis}}
  \quad \text{and/or}\quad r_{k,t} \le p_t^{\text{ch}}
  \quad \text{(product-dependent)}.
  \label{eq:coupling_stylized}
\end{align}

Here $p_t$ denotes energy schedules (charge/discharge), and $r_{k,t}$ denotes
ancillary service schedules for product $k$ (e.g., regulation, reserves). The
critical economic point is that co-optimization imposes \emph{shared feasibility}:
ancillary capability and energy schedules compete for the same physical headroom
(power and SoC). Consequently, scarcity value can migrate from energy-side adders
into ancillary service MCPCs when feasibility and intertemporal constraints bind,
even if the contemporaneous energy market appears ``calm'' in the sense of low or
zero posted adders.

This co-optimization structure is the economic mechanism behind:
(i) the observed decoupling between energy adders and MCPC tails in some windows,
(ii) the Wave~3 Day-Ahead Geometry result (``paper-long'' but brittle stacks), and
(iii) the post-RTC+B investor risk decomposition (feasibility truncation and tail
dependence).

\section{December 2025 Early Operations — Empirical Descriptives from MCPC and System Displays}

\subsection{Overview of December 2025 in the provided artifacts}

The transition to RTC+B went live in early December 2025 (NPRR 1186 context). This section restricts itself to what is directly measurable from the provided data. Two empirical slices are available:

\begin{enumerate}
  \item A multi-day (``previous 6 days'') interval MCPC time series for ancillary services (Reg-Up, Reg-Down, RRS, Non-Spin, ECRS), spanning \textbf{2025-12-26 through 2025-12-31} at 5-minute resolution, provided as \texttt{real-time-market-clearing-prices-for-capacity-previous.csv}.
  \item A short-window system frequency time series at 10-second resolution spanning \textbf{2026-01-01 14:23:10 through 16:23:00}, provided as \texttt{ancillary-services-frequency.csv}.
\end{enumerate}

In addition, system-wide hub price displays and real-time SPP/LMP screenshots provide qualitative confirmation that the market can operate in a low-scarcity regime with negligible adders, as shown at the Jan 1, 2026 capture time (see \cite{ercot_rtm_spp_map_screenshot,ercot_rtm_lmp_table_screenshot,ercot_systemwide_prices_pdf}).

\subsection{MCPC descriptives: 2025-12-26 through 2025-12-31 (5-minute)}

Let $\text{MCPC}_{t}^{(k)}$ denote the market clearing price for capacity for ancillary product $k$ at time $t$. The provided dataset includes $k \in \{\text{REG-UP},\text{REG-DOWN},\text{RRS},\text{NON-SPIN},\text{ECRS}\}$. Table~\ref{tab:mcpc_desc_dec2025_earlyops} reports distributional descriptors computed directly from the CSV (N=1744 intervals).

\begin{table}[!ht]
\centering
\caption{MCPC distributional descriptors (5-min), 2025-12-26 to 2025-12-31 (N=1744).}
\label{tab:mcpc_desc_dec2025_earlyops}
\begin{tabular}{lrrrrrrrr}
\hline
Product & Mean & P50 & P90 & P95 & P99 & Max & \#(>1) & \#(>2) \\
\hline
Reg-Up    & 0.1069 & 0.000 & 0.157 & 0.550 & 2.335 & 3.77 & 56 & 20 \\
Reg-Down  & 0.4454 & 0.130 & 1.020 & 1.990 & 6.000 & 8.65 & 180 & 65 \\
RRS       & 0.1484 & 0.030 & 0.277 & 0.460 & 3.000 & 6.46 & 47 & 33 \\
Non-Spin  & 0.6136 & 0.250 & 1.957 & 3.018 & 4.731 & 6.46 & 317 & 169 \\
ECRS      & 0.1994 & 0.030 & 0.474 & 0.860 & 3.000 & 6.46 & 70 & 35 \\
\hline
\end{tabular}
\end{table}

\paragraph{Immediate empirical conclusions (bounded to the dataset window).}
Within this 6-day window, \textbf{Non-Spin} exhibits the highest mean and high upper-quantile values, while \textbf{Reg-Down} attains the single highest maximum ($\$8.65$/MW-h). \textbf{ECRS} reaches a maximum of $\$6.46$/MW-h with a heavy right tail (P99 = $\$3.00$/MW-h). These observations matter because RTC+B structurally increases the coupling between SoC feasibility and ancillary deployment feasibility, which can elevate the shadow price of reserves relative to energy even in non-scarcity energy regimes.

\subsection{MCPC units, interval accounting, and what Table~\ref{tab:mcpc_desc_dec2025_earlyops} is (and is not)}
\label{sec:mcpc_units}

ERCOT posts real-time MCPC values in capacity-price units (typically \$/MW-h),
while dispatch and pricing are cleared at SCED interval resolution (5-minute in
the artifacts used here). Let $\Delta t$ denote interval length in hours (for
5-minute SCED, $\Delta t=5/60$). For any realized awarded or deployed quantity
$Q_{k,t}$ (MW) in product $k$ at interval $t$, the corresponding interval
settlement value (gross of performance/penalty adjustments) is proportional to
\[
\text{Value}_{k,t} \;=\; \mathrm{MCPC}_{k,t}\,\cdot\, Q_{k,t}\,\cdot\,\Delta t.
\]
Because the present section reports \emph{price distributions} only, and because
$Q_{k,t}$ is not yet integrated for the full window, Table~\ref{tab:mcpc_desc_dec2025_earlyops}
is interpreted strictly as: (i) distributional evidence on the \emph{shadow value
process} for AS feasibility, and (ii) a motivating input for episode definition
and event-study conditioning in later waves. It is not, by itself, a statement
about realized AS revenues, welfare, or procurement efficacy.

\subsection{Frequency descriptives: 2026-01-01 14:23:10 to 16:23:00 (10-second)}

Let $f_t$ denote measured system frequency (Hz). Over the provided 2-hour window (N=720), the empirical descriptors are:

\begin{align}
  \overline{f} &= 59.9959 \text{ Hz}, \\
  \min f_t &= 59.971 \text{ Hz}, \\
  \text{P01}(f_t) &= 59.974 \text{ Hz}, \\
  \text{P95}(f_t) &= 60.016 \text{ Hz}.
\end{align}

This window is consistent with stable frequency control. Importantly, the presence of ancillary products such as ECRS is not restricted to rare emergency excursions; in RTC-style co-optimized markets, reserves can clear as a function of system conditions and expected control needs even when the realized frequency trajectory is well-behaved. The appropriate econometric question (deferred to later chapters with full December interval data) is whether the frequency stabilization burden is being shifted toward co-optimized fast-response products and storage participation (cf. \cite{ercot_ancillary_services_pdf}).

\paragraph{Descriptive-only use of frequency in this report.}
The frequency window above is not used to infer causal impacts of RTC+B on
reliability. In particular, stable frequency over a short calm window does not
imply that ancillary services were unnecessary; it may reflect the opposite:
that control capability was adequate and effectively deployed. Accordingly, the
frequency series is used here only for measurement integrity checks, baseline
distribution summaries, and episode annotation. Causal claims require
deployments/disturbance logs and stronger identification (see
Chapter~\ref{ch:ident_limits} and Wave~3).

\subsection{Low-scarcity energy regime can coexist with active ancillary pricing}

The Jan 1, 2026 capture shows $\text{RTRDPA}=0$ and LMPs at $\$16.83$/MWh across hubs/zones, implying an adder-free, non-scarcity energy interval (Section~I). Yet MCPC can exhibit frequent spikes for ancillary products in neighboring time windows. This decoupling is a signature of co-optimization: scarcity and reliability value can appear in ancillary products rather than energy prices when the constraint that binds is \emph{deliverable control capability} rather than \emph{energy supply adequacy}.

\subsection{Why these early descriptives are motivating but not dispositive}
\label{sec:early_not_dispositive}

The December 26--31 window provides a concrete example of the post-RTC+B
phenomenology: ancillary capacity prices can exhibit heavy tails even when
energy-side adders are absent or rare. However, this window is not treated as a
standalone ``before/after'' evaluation. All regime-effect statements in later
chapters are (i) restricted to matched system-condition bins where overlap holds
(Wave~3-B), (ii) episode-conditioned under pre-registered triggers (Wave~3-D),
and (iii) traceable to bounded subsets of the episode ledger per
Section~\ref{sec:traceability}. This discipline prevents post hoc selection of
scarcity intervals and separates descriptive facts from regime attribution.

\subsection{Three mechanisms that can generate MCPC tails (preview of Wave~3 tests)}
\label{sec:three_mechanisms_preview}

To interpret MCPC tail behavior under RTC+B, we distinguish three mechanism
classes that can produce superficially similar price spikes:

\begin{enumerate}
  \item \textbf{Physical scarcity (quantity shortfall).} A system-level
  shortfall in deliverable reserves/energy implies concurrent stress signatures:
  elevated energy prices and/or energy-side adders together with elevated MCPC.

  \item \textbf{Feasibility scarcity (intertemporal constraints).} Even if
  installed MW and day-ahead awards appear adequate, intertemporal feasibility
  (notably SoC headroom and charging/discharging coupling) can bind, shifting
  scarcity value into product MCPCs while energy adders remain muted. This is
  the core ``feasibility channel'' tested via capability-compression proxies in
  Wave~3-E.

  \item \textbf{Administrative scarcity (qualification/telemetry/bottlenecks).}
  Scarcity can appear in MCPC if the set of \emph{qualified} providers is thin,
  if telemetry failures or qualification constraints bind, or if product rules
  create effective supply discontinuities. This mechanism predicts weaker
  alignment with physical stress covariates and more step-like price behavior.
\end{enumerate}

Wave~3 is structured to discriminate among these mechanisms using overlap/common
support diagnostics (Wave~3-B), DA posture and stack geometry (Wave~3-D), and
feasibility proxies (Wave~3-E), with partial-identification bounds where overlap
fails.

\section{Mechanistic Interpretation — How RTC+B Reallocates Scarcity Value (ORDC vs ASDC/ECRS/FRRS) and Why Batteries Matter}

\subsection{From ORDC adders to co-optimized scarcity in reserve products}

Legacy ERCOT scarcity pricing under ORDC can be represented (stylized) as an energy price adder linked to reserve margin. Let $R_t$ denote operating reserves and let $g(R_t)$ represent the ORDC-derived scarcity adder. Then the legacy settlement energy price may be conceptualized as:

\begin{equation}
  \pi_{t,z}^{\text{legacy}} \approx \lambda_{t,z}^{\text{LMP}} + g(R_t).
  \label{eq:legacy_ordc_v1}
\end{equation}

RTC+B replaces ``scarcity primarily as an energy adder'' with ``scarcity as co-optimized reserve shadow prices'' embedded in dispatch. The key mathematical shift is that scarcity value is represented in the \emph{dual variables} of the co-optimization constraints (e.g., reserve demand constraints, ramp constraints, and storage SoC feasibility constraints). In co-optimization, if a reserve requirement binds, the KKT multiplier $\mu_t$ for that reserve constraint enters the pricing of the relevant ancillary product directly:

\begin{equation}
  \text{MCPC}_t^{(k)} \approx \mu_t^{(k)}.
  \label{eq:mcpc_dual}
\end{equation}

This representation is not merely cosmetic; it changes incentives. Under Eq.~\eqref{eq:legacy_ordc}, storage profits can be concentrated in rare high-adder energy intervals. Under Eq.~\eqref{eq:mcpc_dual}, storage value can be monetized more continuously through reserve products (e.g., ECRS/FRRS/Reg) provided the resource can maintain SoC headroom and satisfy qualification constraints.

\subsection{A formal statement of ``scarcity-value migration'' into ancillary products under SoC constraints}

Consider a single storage resource with energy schedule $p_t$ and reserve schedule $r_t$ (scalar for simplicity). Suppose a reserve constraint $r_t \ge \underline{r}_t$ binds and SoC bounds bind (or nearly bind) such that additional reserve provision would require foregone energy arbitrage or SoC depletion. Then the co-optimization introduces a Lagrangian:

\begin{equation}
  \mathcal{L} = \sum_t \left( C_t(p_t) + D_t(r_t) \right)
  + \sum_t \alpha_t (\underline{r}_t - r_t)
  + \sum_t \beta_t (\text{SOC}_t - \overline{\text{SOC}})
  + \sum_t \gamma_t (0 - \text{SOC}_t)
  + \cdots
\end{equation}

KKT optimality implies, for intervals where the reserve constraint binds ($\alpha_t > 0$), the marginal value of reserve enters the shadow price system. In particular, the stationarity condition in $r_t$ yields:

\begin{equation}
  \frac{\partial D_t}{\partial r_t} - \alpha_t + \text{(terms from SoC feasibility coupling)} = 0,
\end{equation}

so that the market-clearing reserve price (MCPC) inherits positive contributions from $\alpha_t$ and from SoC-coupling multipliers when feasibility is scarce.

\paragraph{Bounded (R4 required for SoC attribution).}
In plain terms, \emph{if} SoC headroom is the scarce feasibility input, reserve
products can price that scarcity even if energy remains abundant.
However, attributing observed RT MCPC tails specifically to SoC headroom is
\textbf{not identifiable} without R4 evidence (SoC/telemetry and resource-level
feasibility), per Table~\ref{tab:evidence_rungs} and Chapter~\ref{ch:ident_limits}.

\paragraph{Descriptive (R1: price pattern).}
Independently of constraint attribution, it is admissible at R1 to report that
MCPC for reserve products can exhibit heavy tails even when contemporaneous
energy LMP and SPP are modest (Section~II; see also
\cite{ercot_mcpc_prev_csv,ercot_rtm_lmp_table_screenshot}).

\subsection{Implications for ECRS and FRRS in the RTC+B regime}

ECRS and FRRS (and regulation products) are, by design, \emph{control capability} products. Their scarcity is associated with fast response and sustained deliverability, not merely installed MW. Storage resources are structurally advantaged in response speed but structurally constrained by SoC. RTC+B internalizes this trade:

\begin{itemize}
  \item \textbf{ECRS/FRRS price sensitivity to SoC feasibility.} Even in low-scarcity energy periods, MCPC can rise if the system needs fast control capability and marginal providers must preserve SoC.
  \item \textbf{Reduced dependence on rare energy scarcity adders.} As scarcity value migrates into reserve products, the extreme right tail of energy prices can be damped relative to an ORDC-adder-driven regime, but reserve price tails can become more prominent.
  \item \textbf{Investor-level consequence.} Merchant battery valuation must be decomposed into (i) energy arbitrage under co-optimized dispatch and (ii) ancillary revenue under SoC-feasible provision. The correct valuation object is therefore not a single ``energy volatility'' metric but a portfolio of co-optimized products with correlated constraints.
\end{itemize}

\subsection{A testable hypothesis set for the December 2025 transition}

The descriptive evidence in Section~II motivates three competing hypotheses for the post-RTC+B regime (to be tested with full December 2025 interval data and subsequent months):

\begin{description}
  \item[H1 (Reliability-value migration).] Scarcity value previously expressed through ORDC energy adders is partially reallocated into MCPC tails for ECRS/FRRS/Reg products, lowering the frequency and magnitude of extreme energy settlement prices while increasing the frequency of moderate-to-high ancillary prices.
  \item[H2 (Participant adaptation / transient).] The observed MCPC tail behavior is dominated by transitional bidding/qualification behavior in the early RTC+B period and will attenuate as participants optimize SoC management and bidding strategies.
  \item[H3 (Latent fragility).] RTC+B stabilizes visible energy price volatility while increasing structural reliance on fast-response products; if storage participation or qualification falters, scarcity may re-emerge sharply in either reserve or energy prices, potentially producing more concentrated tail events.
\end{description}

Each hypothesis can be tested using: (i) full-interval distributions of $\pi^{\text{SPP}}$ and $\lambda^{\text{LMP}}$ across December 2025 and adjacent pre/post windows, (ii) MCPC distributions by product, (iii) correlation of high MCPC events with frequency excursions, net load ramps, and congestion indicators, and (iv) storage participation proxies where available.

\subsection{Immediate report roadmap (what we do next with your uploaded ERCOT PDFs/CSVs)}

The next analytical step (the portion you requested as ``I, II, III'') is to expand into: (a) empirical hub/zone price histograms for December 2025 using full RTM-SPP exports, (b) event studies around high MCPC intervals to identify co-optimization binding constraints, and (c) a revenue decomposition model for a canonical 100 MW / 4-hour battery. Those steps require complete December interval settlement extracts beyond the short-window datasets currently in hand. The present sections therefore establish the \emph{mechanistic frame}, the \emph{identification discipline}, and the \emph{initial empirical descriptors} from the available artifacts.

% ======================================================================
% PARTS I–XVI: FULL REPORT MODE (CHAPTER STRUCTURE)
% Insert BEFORE \begin{thebibliography}{99}
% ======================================================================

\chapter{Literature and Context Review (ERCOT-Specific)}
\label{ch:lit_context}

This chapter situates NPRR~1186 and the Real-Time Co-optimization plus Batteries framework (RTC+B) within (i) ERCOT's scarcity-pricing lineage, (ii) the general theory of scarcity pricing and reserve valuation in energy-only markets, and (iii) the multi-product co-optimization literature. The objective is not generic storage advocacy. Rather, we aim to (a) establish the historical and conceptual baseline against which RTC+B should be evaluated, (b) articulate the specific theoretical mechanisms by which co-optimization can re-locate scarcity value across products, and (c) pre-register the principal failure modes and unresolved tensions that must be treated as \emph{live hypotheses} rather than assumed conclusions.

Two methodological commitments govern this chapter.

\paragraph{Commitment 1 (Mechanism-first).}
We prioritize structural mechanisms (constraints and dual prices) over narrative summaries. Where policy motivations are discussed, they are tied to explicit operational constraints (deliverability, ramping, frequency response, reserve sufficiency) and the economic objects those constraints induce (shadow prices, scarcity adders, and product-level scarcity premia).

\paragraph{Commitment 2 (ERCOT-specific comparators).}
Comparisons to other ISOs/RTOs are used only as \emph{structural analogies}: we ask what object in another market is the nearest analog to ERCOT's object (e.g., an operating reserve demand curve, a product-specific scarcity function, or a co-optimized reserve clearing mechanism), and we document where equivalence fails due to institutional differences (energy-only vs capacity markets, uplift regimes, resource mix, settlement conventions, and network constraints).

\vspace{0.5em}
\noindent
Throughout, we distinguish three conceptually distinct layers:
\begin{enumerate}
  \item \textbf{Physical layer:} dynamics and constraints (balance, inertia/frequency response, ramp limits, transmission/security, ESR state-of-charge).
  \item \textbf{Optimization layer:} dispatch and procurement as constrained optimization (single-product vs multi-product).
  \item \textbf{Settlement layer:} how scarcity and constraints express as prices (LMPs, adders, MCPCs) and how incentives propagate to investment.
\end{enumerate}

% ----------------------------------------------------------------------

\section{ERCOT scarcity formation pre-RTC+B}
\label{sec:ercot_scarcity_pre}

Pre-RTC+B ERCOT scarcity formation is most naturally framed as a \emph{univariate} scarcity representation: the system's scarcity state is summarized by a reserve margin proxy (or set of proxies), and scarcity value is expressed primarily through an energy-side scarcity adder mechanism. The canonical economic interpretation is that scarcity pricing should approximate the marginal reliability value of the next increment of supply (or the marginal cost of avoiding unserved energy) under stressed conditions.

\subsection{Scarcity pricing as reliability valuation}
In a stylized framework, consider an energy-only market where the relevant reliability harm is \emph{unserved energy}. Let $\mathrm{EUE}$ denote expected unserved energy over a horizon and $\mathrm{VOLL}$ the value of lost load. The welfare-relevant reliability cost is
\begin{equation}
C_R := \mathrm{VOLL}\cdot \mathrm{EUE}.
\label{eq:cr_voll_eue}
\end{equation}
If an incremental resource contribution $\Delta x$ reduces $\mathrm{EUE}$ by $\Delta \mathrm{EUE}$, then the marginal reliability value (MRV) is
\begin{equation}
\mathrm{MRV} := -\,\mathrm{VOLL}\cdot\frac{\partial \mathrm{EUE}}{\partial x}.
\label{eq:mrv_def}
\end{equation}
A scarcity pricing rule is economically consistent if, in stressed conditions, the incremental compensation for marginal capability approximates MRV (up to institutional frictions and measurement limitations).

\subsection{ORDC logic as a reduced-form scarcity functional}
Operating Reserve Demand Curves (ORDC) can be understood as a reduced-form mapping from a reserve adequacy proxy to a scarcity price adder. In the abstract:
\begin{equation}
a_t^{\mathrm{scar}} = g(R_t),
\label{eq:ordc_reduced_form}
\end{equation}
where $R_t$ denotes an operating reserve measure (or a scarcity proxy derived from it), and $g(\cdot)$ is increasing as reserves tighten.

Two features matter for RTC+B evaluation.

% ======================================================================
% CONTINUATION: Literature & Context Review — restart at “dimensionality”
% Drop-in replacement/expansion text for the chapter section that begins
% where you left off (multi-product dimensionality of scarcity).
% ======================================================================

\subsection{Dimensionality: why “scarcity” is not a scalar under RTC+B}
A central conceptual error in non-specialist commentary on ERCOT reforms is to treat scarcity as a single latent variable that can be priced through one adder (ORDC) attached to energy settlement. Under RTC+B, scarcity is intrinsically \emph{vector-valued}: the system can be energy-adequate while simultaneously being \emph{control-inadequate} along one or more operational dimensions (speed, directionality, duration, deliverability). This is not a rhetorical claim; it is a direct consequence of security-constrained dispatch with heterogeneous ancillary requirements.

Let the system state at time $t$ be summarized by a scarcity vector
\begin{equation}
\mathbf{s}_t := \big(s^{(E)}_t,\ s^{(R)}_t,\ s^{(\uparrow)}_t,\ s^{(\downarrow)}_t,\ s^{(\tau)}_t,\ s^{(\mathcal{N})}_t \big),
\label{eq:scarcity_vector}
\end{equation}
where $s^{(E)}_t$ represents energy adequacy margin, $s^{(R)}_t$ a reserve adequacy margin, $s^{(\uparrow)}_t$ and $s^{(\downarrow)}_t$ capture up/down response capability, $s^{(\tau)}_t$ captures duration/sustainability constraints (e.g., response that must be held for $\tau$ minutes), and $s^{(\mathcal{N})}_t$ captures network/deliverability feasibility (local constraints and security limits). In legacy narratives, only $s^{(E)}_t$ is implicitly priced; ORDC attempts to proxy control scarcity through a scalar function of reserves, but in doing so collapses the dimensional structure.

Under multi-product co-optimization, each dimension may bind independently. The appropriate theoretical object is therefore not a single scarcity adder but a set of marginal values (shadow prices) associated with each binding constraint family. In the simplest convex representation of real-time dispatch, these marginal values appear as dual variables in the Lagrangian. Consequently, it is \emph{structurally expected} that scarcity value can migrate into ancillary products even when energy prices remain moderate. This becomes especially salient when the feasible set is shaped by storage intertemporal constraints (Section~\ref{sec:storage_intertemporal_coupling_lit}).

\subsection{A minimal co-optimization skeleton and its pricing interpretation}
To formalize why vector scarcity matters, consider a stylized real-time objective with energy and ancillary requirements:
\begin{align}
\min_{\{p_{i,t},\, r^{(k)}_{i,t}\}} \quad
& \sum_{i \in \mathcal{I}} C_i(p_{i,t}) + \sum_{i \in \mathcal{I}}\sum_{k \in \mathcal{K}} C_{i}^{(k)}\!\left(r^{(k)}_{i,t}\right) \label{eq:coopt_obj}\\
\text{s.t.}\quad
& \sum_{i \in \mathcal{I}} p_{i,t} = D_t \qquad\qquad\qquad\quad\ \ \ \ \ \ \,(\lambda_t) \label{eq:balance_lit}\\
& \sum_{i \in \mathcal{I}} r^{(k)}_{i,t} \ge R^{(k)}_t \qquad \forall k \in \mathcal{K} \quad (\mu^{(k)}_t) \label{eq:as_req_lit}\\
& (p_{i,t}, r^{(k)}_{i,t}) \in \mathcal{F}_{i,t} \qquad\qquad\qquad\ \ \ \ \ \ \ \ (\boldsymbol{\nu}_{i,t}) \label{eq:feasible_set}
\end{align}
where $\mathcal{F}_{i,t}$ encodes unit capability, ramping, network/security, and (critically) ESR intertemporal feasibility through state variables. The dual variables $\lambda_t$ and $\mu^{(k)}_t$ are not merely mathematical artifacts; they represent marginal system value of energy balance and of each ancillary requirement \emph{at that time}. Under additional assumptions linking settlement mechanisms to these shadow prices, the observed LMP and MCPC can be interpreted as functions of these marginal values.

Two implications follow immediately:
\begin{enumerate}
  \item If an ancillary requirement \eqref{eq:as_req_lit} binds, its dual $\mu^{(k)}_t > 0$ and scarcity value attaches to product $k$, potentially elevating its clearing price even if energy balance \eqref{eq:balance_lit} is not tight.
  \item If feasibility constraints \eqref{eq:feasible_set} bind in a way that reduces available ancillary headroom (e.g., storage SoC headroom exhaustion), then the effective scarcity of ancillary capability rises even when energy is abundant, i.e., $s^{(E)}_t$ is loose while $s^{(\uparrow)}_t$ or $s^{(\downarrow)}_t$ binds.
\end{enumerate}
Thus, treating scarcity as a scalar priced solely through energy adders is not just incomplete; it is structurally misaligned with how constraints bind in the underlying dispatch problem.

\subsection{From scalar ORDC to multi-product scarcity: what changes in intuition}
ORDC-style adders can be understood as an attempt to encode a \emph{risk function} of reserve scarcity into energy prices, often motivated by the social cost of unserved energy (VOLL) and the probability of shortage. However, the ORDC scalarization implicitly assumes that: (i) reserve scarcity is adequately captured by a single reserve margin state, and (ii) the marginal reliability value of reserves can be expressed as a function of that scalar state. Under RTC+B, the presence of distinct products (ECRS/FRRS/Reg and others) implies that the marginal reliability value is product-specific and state-dependent along multiple axes (speed, directionality, sustainability, deliverability). The conceptual shift is therefore from
\begin{equation}
a^{\mathrm{ORDC}}_t = g(R_t)
\label{eq:ordc_scalar}
\end{equation}
to a family
\begin{equation}
\text{Price}^{(k)}_t \approx g_k\!\left(\mathbf{s}_t\right), \qquad k \in \mathcal{K},
\label{eq:multi_product_price}
\end{equation}
where each $g_k$ can depend differently on the scarcity vector $\mathbf{s}_t$ and on feasibility constraints.

\subsection{What an ERCOT-native literature review must emphasize}
An ERCOT-native review must therefore focus on:
\begin{itemize}
  \item \textbf{Constraint binding regimes:} which constraints bind when (energy balance, reserve requirements, network constraints, ESR feasibility), and how those regimes shift under load/renewable/outage conditions.
  \item \textbf{Dimensional scarcity representation:} why product-level scarcity pricing may be more faithful to operational reality than a single adder.
  \item \textbf{Settlement and incentive alignment:} how settlement mechanics map dual values into participant incentives, and where misalignment can create fragility or gaming surfaces.
  \item \textbf{Empirical signatures:} what should be observed in prices and reliability proxies when a specific dimension binds.
\end{itemize}

% ----------------------------------------------------------------------
% SECTION: Storage feasibility and intertemporal coupling
% ----------------------------------------------------------------------

\section{Storage feasibility and intertemporal coupling in market design}
\label{sec:storage_intertemporal_coupling_lit}

\subsection{The intertemporal feasibility constraint: state of charge as a binding resource}
Energy storage is not a standard “static” supply resource. Its ability to provide energy and ancillary services is conditioned by an intertemporal state variable, state of charge (SoC). A minimal representation is:
\begin{equation}
s_{t+1} = s_t + \eta_c q^{\mathrm{ch}}_{t}\Delta t - \frac{1}{\eta_d} q^{\mathrm{dis}}_{t}\Delta t,
\label{eq:soc_dynamics}
\end{equation}
with bounds
\begin{equation}
0 \le s_t \le \overline{s}, \qquad 0 \le q^{\mathrm{ch}}_{t} \le \overline{q}^{\mathrm{ch}}, \qquad 0 \le q^{\mathrm{dis}}_{t} \le \overline{q}^{\mathrm{dis}}.
\label{eq:soc_bounds_lit}
\end{equation}
Unlike thermal or many renewable resources, storage carries an \emph{inventory} constraint: the “fuel” is limited and must be managed over time. This converts many market design questions into intertemporal allocation problems.

\subsection{Why intertemporal coupling breaks single-period scarcity intuitions}
In a single-period model, marginal prices reflect contemporaneous scarcity. In storage-coupled systems, marginal value can reflect the scarcity of \emph{future feasibility}. A storage resource can be energy-adequate now but infeasible to respond later if SoC is mispositioned. This creates a wedge between:
\begin{itemize}
  \item \textbf{instantaneous energy adequacy} (current ability to meet $D_t$), and
  \item \textbf{contingent control adequacy} (ability to provide fast-response services under uncertainty and ramps).
\end{itemize}
RTC+B is, in part, an attempt to price the latter more explicitly through co-optimization and productized scarcity.

\subsection{Reserve headroom consumes SoC feasibility: a simple bound}
Suppose storage must hold upward reserve $r^{(\uparrow)}_t$ for a sustainment duration $\tau$ (hours). A feasibility condition is:
\begin{equation}
s_t \ge \frac{r^{(\uparrow)}_t \tau}{\eta_d}.
\label{eq:up_reserve_soc}
\end{equation}
Similarly, downward reserve requires headroom:
\begin{equation}
\overline{s} - s_t \ge \eta_c\, r^{(\downarrow)}_t \tau.
\label{eq:down_reserve_soc}
\end{equation}
These constraints imply that providing reserves can be limited even when power capability exists. Under high penetration of storage, these bounds can become system-relevant: the market can clear energy cheaply while reserve feasibility becomes scarce, pushing scarcity value into ancillary prices.

\subsection{Correlation and synchronization: the system-level SoC risk}
A critical system-level fragility arises when storage fleets become synchronized (similar SoC trajectories) due to shared price signals and similar optimization heuristics. In such cases, the system may experience “collective headroom collapse,” where many resources simultaneously lack upward (or downward) feasibility. This risk is not reducible to individual resource adequacy; it is a \emph{correlation structure} problem:
\begin{equation}
\text{Risk} \propto \mathrm{Corr}\!\left(\mathbf{1}\{s_{i,t} \text{ near bounds}\},\ \mathbf{1}\{s_{j,t} \text{ near bounds}\}\right).
\label{eq:soc_correlation_risk}
\end{equation}
A dissertation-grade review must therefore emphasize that RTC+B’s efficacy depends not only on average storage capacity, but on the distribution and correlation of SoC states across the fleet.

\subsection{Implications for pricing: why ancillary tails may become the new scarcity carrier}
Because SoC feasibility binds in stress regimes (ramp events, forecast error bursts, outage clusters), the marginal value of fast-response products can exhibit heavy tails even if energy-side adders are muted in some intervals. This motivates the empirical question (posed later in the report): \emph{did RTC+B cause scarcity tails to migrate into ancillary products, and if so, under which stress regimes and with what reliability consequences?}

% ----------------------------------------------------------------------
% NEXT SECTION (outline + setup): Open research tensions
% ----------------------------------------------------------------------


% =========================================================
% IMFR / IFRO METHODOLOGY INTEGRATION (BAL-001 / BAL-003)
% =========================================================

% =========================================================
% IMFR / IFRO METHODOLOGY INTEGRATION (BAL-001 / BAL-003) — EXPANDED
% =========================================================
\section{Interconnection Minimum Frequency Response (IMFR): Regulatory Framework, Physics Bridge, and Analytical Role}
\label{sec:imfr_literature}

This section integrates the Interconnection Minimum Frequency Response (IMFR) framework into the analytical foundation of this report and formalizes its relationship to both (i) interconnection-level frequency-response physics and (ii) market-design evaluation under NPRR~1186 (RTC+B). The purpose is methodological discipline: to prevent conflating physical stress (which may degrade raw frequency) with control adequacy (which is the object of IMFR), and to situate any post-RTC+B interpretations within the reliability standards that govern primary frequency response (PFR) expectations in ERCOT.

\subsection{IMFR within NERC reliability standards: BAL-001 and BAL-003}
IMFR is derived from, and interpreted within, the NERC reliability standards suite governing frequency response and balancing performance—particularly BAL-003 (Frequency Response and Frequency Bias Setting) and BAL-001-TRE-2 (Real Power Balancing Control Performance). Under BAL-003, each interconnection is assigned an Interconnection Frequency Response Obligation (IFRO), expressed in MW per 0.1~Hz, representing the minimum aggregate primary frequency response required to arrest frequency decline after credible disturbances. Each responsible entity operationalizes this obligation through a documented methodology specifying event selection, measurement windows, and statistical treatment.

ERCOT, as a single-balancing-authority interconnection, computes its annual IMFR by allocating the IFRO across a statistically defined set of Frequency Measurable Events (FMEs) and applying measurement conventions consistent with BAL-003. For Operating Year 2026, ERCOT’s IMFR is set to 459~MW per 0.1~Hz and applies from January~1 through December~31, 2026 \cite{ercot_imfr_methodology}. This report treats that value as the annual benchmark for interpreting control adequacy in the 2026 operating context.

Define interconnection frequency response for an event window as
\[
\mathrm{FR} \equiv \frac{\Delta P}{\Delta f},
\]
where $\Delta P$ is the aggregate power change attributable to frequency-responsive mechanisms (governor response plus frequency-sensitive load, net of exogenous effects to the extent practicable) and $\Delta f$ is the associated frequency deviation over the standard measurement window. IMFR imposes a minimum bound on this slope:
\[
\Pr\!\left(\mathrm{FR} \ge \mathrm{IMFR}\right) \ge 1-\epsilon,
\]
for an implicit tolerance $\epsilon$ governed by the standard’s compliance framework. This report does not attempt to replicate the compliance determination procedure; rather, it uses IMFR as an engineering-valid adequacy benchmark and as an interpretive constraint on market-design claims.

\subsection{Linking IMFR to inertia, RoCoF, and frequency nadir}
IMFR is a bound on a \emph{slope} (MW per Hz), but operational narratives often emphasize frequency nadir and Rate-of-Change-of-Frequency (RoCoF). The relationship becomes explicit under standard aggregated frequency-response dynamics.

Let $f(t)$ denote interconnection frequency, $f_0$ nominal frequency (60~Hz), and $\Delta f(t)=f(t)-f_0$. An aggregated swing-equation representation is
\[
2H_{\mathrm{sys}}\frac{d}{dt}\!\left(\frac{\Delta f(t)}{f_0}\right)=\frac{1}{S_{\mathrm{base}}}\Big(\Delta P_m(t)-\Delta P_e(t)-D_f \Delta f(t)\Big),
\]
where $H_{\mathrm{sys}}$ is an inertia parameter (seconds), $S_{\mathrm{base}}$ is the system base, and $D_f$ represents frequency-dependent damping (including frequency-sensitive load). Immediately after a disturbance, RoCoF is governed primarily by inertia and the net imbalance:
\[
\left.\frac{d\,\Delta f}{dt}\right|_{t\approx 0}\approx-\frac{f_0}{2H_{\mathrm{sys}}S_{\mathrm{base}}}\,\Delta P,
\quad \Delta P\equiv \Delta P_e-\Delta P_m.
\]
Primary frequency response provides stabilizing feedback. In a linearized approximation,
\[
\Delta P_{\mathrm{PFR}}(t)\approx -\beta\,\Delta f(t),
\]
where $\beta$ (MW/Hz) is the effective response slope. IMFR is a lower bound on $\beta$ expressed in MW per 0.1~Hz:
\[
\beta \ge \beta_{\min}\equiv 10\times \mathrm{IMFR}\ \ (\mathrm{MW/Hz}).
\]
This bridge clarifies a key interpretation boundary: inertia primarily influences early RoCoF, while $\beta$ governs restoring force and thus influences nadir and recovery. Consequently, raw frequency deviations may be larger during more severe stress even if the response slope (and thus IMFR context) is not degraded.

\subsection{BAAL, ACE control, and why IMFR is conceptually distinct}
BAL-001 performance (including BAAL exceedances) evaluates balancing-control behavior, whereas IMFR evaluates primary response adequacy. Therefore,
\[
\text{BAAL exceedance}\not\Rightarrow \text{IMFR shortfall},\qquad
\text{IMFR shortfall}\not\Rightarrow \text{BAAL exceedance at all times}.
\]
Real-time operating snapshots (e.g., near-nominal frequency, zero BAAL exceedances, inertia level) are treated in this report as contextual indicators rather than direct evidence of IMFR compliance or of market-design welfare.

\subsection{Market-design mapping: from ORDC/ASDC to the reliability control stack}
To connect scarcity formation to engineering obligations, we decompose net control effort into a canonical stack:
\[
\Delta P(t)=\Delta P_{\mathrm{PFR}}(t)+\Delta P_{\mathrm{AGC}}(t)+\Delta P_{\mathrm{tertiary}}(t),
\]
where $\Delta P_{\mathrm{PFR}}$ is autonomous primary response, $\Delta P_{\mathrm{AGC}}$ is secondary control via AGC, and $\Delta P_{\mathrm{tertiary}}$ is slower redispatch/commitment response. In ERCOT’s market-product taxonomy:
\begin{itemize}
\item \textbf{Reg-Up/Reg-Down} principally support $\Delta P_{\mathrm{AGC}}(t)$.
\item \textbf{RRS and ECRS} support contingency and fast-response layers that complement PFR and stabilize frequency during rapid excursions, especially under higher inverter-based penetration.
\item \textbf{ORDC scarcity adders} monetize scarcity in the energy price but are not, on their own, a product-specific procurement mechanism for control actions.
\end{itemize}
RTC+B can thus be interpreted as shifting scarcity expression toward product-specific procurement/pricing (ancillary products) rather than a single energy scarcity adder, motivating the empirical causal ordering used later:
\[
\text{MCPC tails}\rightarrow \text{DAMASAGG quantities}\rightarrow \text{IMFR context}\rightarrow \text{frequency outcomes}.
\]

\subsection{Necessary versus sufficient conditions: formal propositions}
Market design is underdetermined by any single operational metric. We therefore formalize what can and cannot be inferred from IMFR-context evidence.

\textbf{Proposition 1 (Necessity).}
If RTC+B improves reliability through enhanced control procurement and dispatch feasibility, then the likelihood of meeting primary response adequacy (IMFR context) must not decrease during stress regimes.

\emph{Proof sketch.} Reliability improvement through control implies weakly higher effective restoring response $\beta_{\mathrm{eff}}$ during stress, because larger disturbances require at least as much restoring force to maintain stability. IMFR is a minimum bound on response slope; a mechanism that improves reliability by improving control cannot systematically reduce the probability of meeting a given minimum slope requirement. $\square$

\textbf{Proposition 2 (Insufficiency).}
Meeting IMFR does not imply welfare optimality, absence of market power, or long-run revenue sufficiency.

\emph{Proof sketch.} IMFR constrains a single dimension of behavior (event-based response slope). Welfare, market power, and investment adequacy depend on the joint distribution of prices, quantities, scarcity rents, risk allocation, and strategic conduct over time. None is implied by satisfying a minimum slope constraint. $\square$

\subsection{Connection to subsequent frequency analysis}
Subsequent empirical sections compute raw frequency metrics such as integrated absolute deviation,
\[
I_f=\sum_t |f_t-60|\,\Delta t,
\]
band exceedances, and tail deviations conditional on scarcity and procurement regimes. IMFR provides the interpretive constraint that prevents misattributing storm-driven stress to market design absent evidence of degraded response capability.


\section{Open research tensions}
\label{sec:open_tensions_lit}

\subsection{Welfare equivalence versus operational fidelity}
A primary debate is whether shifting scarcity expression from energy adders to product-level pricing improves welfare (efficiency) or merely redistributes rents. Arguments for equivalence typically assume scalar scarcity and convex response; arguments against equivalence emphasize multi-product binding constraints and intertemporal feasibility.

\subsection{Market power surface changes}
Productization and co-optimization can expand the strategy space. The key tension is whether the redesign reduces energy-price volatility at the cost of creating thinner, more gameable ancillary scarcity points.

\subsection{Reliability validation: which proxies are admissible}
Frequency metrics are readily available but incomplete. A rigorous evaluation would ideally connect changes to disturbance response, emergency event incidence, or probabilistic reliability metrics (LOLE/EUE), which often require more data than public artifacts provide.

\subsection{Empirical identification under partial observability}
Many core welfare claims require quantities (awards/deployments) and feasibility indicators (SoC headroom proxies). Without these, the literature must acknowledge non-identifiability and focus on signatures, bounds, and falsification tests.

% ----------------------------------------------------------------------
% SECTION (EXPANDED): Co-optimization in other RTOs/ISOs (comparative)
% ----------------------------------------------------------------------

\section{Co-optimization in other RTOs/ISOs (comparative but not hand-wavy)}
\label{sec:comparative_isos_lit}

This section benchmarks ERCOT’s RTC+B (and the NPRR~1186 pathway) against three mature co-optimization implementations: CAISO, PJM, and MISO. The objective is \emph{not} to argue that ERCOT should copy any single ISO design, but to isolate: (i) which design primitives are structurally portable, (ii) which are contingent on market architecture (capacity vs energy-only, uplift treatment, performance regimes), and (iii) which failure modes are repeatedly observed when scarcity is re-routed across products.

\subsection{Comparative method: design primitives, not superficial product names}
A rigorous comparison cannot be performed at the level of ancillary product labels. Instead, we decompose each ISO design into the same set of primitives:
\begin{enumerate}
  \item \textbf{Procurement layer:} which products are procured DA vs RT, and whether RT procurement is dynamically co-optimized with energy dispatch.
  \item \textbf{Pricing rule:} whether prices are derived from shadow prices (duals) of binding constraints, including substitution rules and scarcity demand curves.
  \item \textbf{Performance regime:} how obligations are measured and penalized (performance scoring, penalties, make-whole, caps).
  \item \textbf{Intertemporal feasibility:} how storage constraints (SoC and state-dependent headroom) are represented in RT optimization and settlement.
  \item \textbf{Settlement mapping:} the extent to which operational marginal values map cleanly into participant incentives (or are distorted by uplift, side-payments, or uplift-like adjustments).
\end{enumerate}
The conceptual lens is therefore: \emph{scarcity is vector-valued} (Section on dimensionality above), and each ISO chooses a particular mapping from the scarcity vector to settlement payments.

\subsection{CAISO: dynamic real-time co-optimization with explicit tariff-based procurement rules}
CAISO provides a canonical example of explicit, tariff-specified co-optimization of energy and operating reserves in real time. Its tariff and BPMs describe real-time procurement and pricing of spinning and non-spinning reserves using \emph{dynamic co-optimization} with energy dispatch.\footnote{See CAISO tariff provisions on ancillary services procurement and the real-time market description in Section~8 and Section~34 materials.}\cite{caiso_tariff_section8_2025,caiso_rtm_section34_2024,caiso_market_ops_bpm_reserves}

\paragraph{Core primitive (portable): joint optimization with product feasibility constraints.}
CAISO’s real-time optimization (conceptually) resembles:
\begin{align}
\min_{\{p_t,r_t\}} \quad & \sum_i C_i(p_{i,t}) + \sum_{k} \sum_i C_i^{(k)}(r_{i,t}^{(k)}) \\
\text{s.t.}\quad & \sum_i p_{i,t} = D_t \\
& \sum_i r_{i,t}^{(k)} \ge R_t^{(k)} \quad \forall k \\
& (p_{i,t}, r_{i,t}^{(k)}) \in \mathcal{F}_{i,t},
\end{align}
with pricing linked (in the stylized convex case) to the duals of the binding constraints. The key comparative insight is that CAISO does \emph{not} attempt to represent all real-time reliability value through a single energy adder. Instead, it procures and prices products with distinct operational meaning, and it explicitly addresses substitution/interaction rules (e.g., how Regulation energy is treated when substituted).\cite{caiso_tariff_section8_2025}

\paragraph{Non-portable elements: CAISO-specific constructs and settlement conventions.}
Some CAISO details are not directly portable to ERCOT because they are entwined with CAISO’s market structure (including resource modeling classes, bid formats, and settlement conventions). For ERCOT, the portable core is not the exact CAISO product list; it is the existence of tariff-defined, optimization-consistent rules for: (i) reserve procurement in RT, (ii) the relationship between AS awards and energy schedules, and (iii) the management of substitution logic in a way that does not create contradictory incentives.\cite{caiso_rtm_section34_2024}

\paragraph{Failure mode family (transferable): substitution complexity and opacity.}
A recurring design hazard in co-optimized markets is that substitution logic can become opaque to participants, creating (a) perceived unpredictability in settlement outcomes, (b) complex bidding incentives, and (c) a gap between “what the optimization did” and “what the participant expected.” The ERCOT analog is the risk that RTC+B re-routes marginal value into products in ways participants cannot forecast without better disclosure of binding constraints and award/feasibility signals.

\subsection{PJM: joint optimization under a capacity market and strong performance/penalty framing}
PJM’s real-time market has long co-optimized energy with certain ancillary services, and the PJM manuals describe the joint optimization and operational procurement of Regulation and reserve products, with extensive treatment of assignments, deployment, and settlement.\cite{pjm_manual11_2022,pjm_manual12_2025,pjm_manual28_2024}

\paragraph{Core primitive (portable): co-optimization plus explicit performance economics.}
In PJM, ancillary product offer rules and performance consequences are not an afterthought; they are integral to how reserve products clear and how participants are incentivized (including caps tied to expected penalty value in some contexts).\cite{pjm_sync_reserve_updates_2024} The transferable lesson for ERCOT is structural:
\begin{equation}
\text{If a product creates scarcity rents, it must also impose credible performance discipline,}
\end{equation}
otherwise scarcity pricing becomes a pure rent-transfer mechanism with degraded reliability value.

\paragraph{Non-portable elements: capacity market interactions and uplift architecture.}
PJM’s capacity market and uplift mechanisms change investor incentives and the welfare interpretation of scarcity rents. ERCOT is energy-only; therefore, scarcity rents and AS rents in ERCOT can be more directly tied to capital formation and operational behavior. The PJM lesson is \emph{not} “copy PJM”; it is: when scarcity rents migrate into ancillary products, the market power surface and performance enforcement become first-order, and must be audited continuously (see the later Market Power chapter).

\paragraph{Failure mode family (transferable): thin reserve products and strategic behavior.}
PJM’s market monitoring literature and IMM commentary frequently emphasize that reserves can be thinner and more vulnerable to strategic behavior than energy markets. The ERCOT analog is immediate: if RTC+B dampens energy tails but increases ancillary tails, then the integrity of ancillary price formation and the robustness of mitigation become decisive.\cite{imm_reserve_price_formation_2022}

\subsection{MISO: co-optimization with an expanding family of flexibility products}
MISO is particularly instructive because it has co-optimized energy and operating reserves in DA and RT for years and has layered additional flexibility products (e.g., ramp capability and other constructs) as system needs evolved.\cite{miso_bpms_page_2025,miso_market_participation_overview_2024} MISO’s own materials emphasize co-optimization as a major investment yielding efficiency and reliability improvements, and they highlight the economic logic of clearing through joint optimization.\cite{miso_coopt_presentation_2019}

\paragraph{Core primitive (portable): adding new products as the scarcity vector grows.}
The structural lesson for ERCOT is that product proliferation is not inherently bad if it corresponds to distinct operational constraints. As renewables increase and net-load ramps become sharper, the scarcity vector gains dimensions (speed, ramping, duration). MISO’s trajectory illustrates an institutional pattern: once you accept vector scarcity, you typically add instruments to price and procure the relevant dimensions rather than forcing a scalar scarcity proxy to do all the work.

\paragraph{Non-portable elements: MISO zonal structures and regional constraints.}
MISO’s geography and zonal constructs (and how reserve requirements are zonally specified) are not directly portable. ERCOT’s single interconnection footprint simplifies some aspects while intensifying others (e.g., system-wide price formation coupled to nodal constraints). The portable insight is: co-optimization designs often evolve toward \emph{more} explicit flexibility products as uncertainty and ramping complexity rise.

\paragraph{Failure mode family (transferable): complexity tax and model risk.}
As the product set expands, so does model risk: parameter choices, substitution rules, and constraint formulations become pivotal. The comparative warning for ERCOT is that RTC+B’s success depends on continuously validating that (i) new scarcity signals correspond to real operational stress, and (ii) the system is not unintentionally engineering fragility through correlated storage behavior.

\subsection{Synthesis: what transfers cleanly to ERCOT versus what does not}
The comparison yields a set of \emph{transferable primitives} and \emph{non-transferable dependencies}.

\paragraph{Transferable primitives (high-confidence).}
\begin{enumerate}
  \item \textbf{Multi-product co-optimization is standard for real-time reliability procurement.} CAISO, PJM, and MISO each implement some form of RT co-optimization between energy and reserves.\cite{caiso_tariff_section8_2025,pjm_manual11_2022,miso_bpms_page_2025}
  \item \textbf{Scarcity value migration into ancillary prices is structurally expected.} When ancillary constraints bind (and especially when feasibility constraints reduce ancillary headroom), dual prices for reserves rise even if energy balance is not tight.
  \item \textbf{Performance discipline must scale with scarcity rents.} If ancillary products carry scarcity tails, they require measurable obligations, enforceable penalties, and auditable performance metrics.\cite{pjm_sync_reserve_updates_2024}
  \item \textbf{Transparency is not optional.} Participants need sufficient observability of binding constraints, award drivers, and feasibility regimes; otherwise price formation becomes opaque and invites both misinvestment and strategic exploitation.
\end{enumerate}

\paragraph{Non-transferable dependencies (must be treated as ERCOT-specific design space).}
\begin{enumerate}
  \item \textbf{Capacity market interaction (PJM).} ERCOT’s energy-only design implies scarcity rents play a different capital formation role than in PJM.
  \item \textbf{Zonal/footprint differences (MISO).} ERCOT’s single BA and nodal design compress some complexities while magnifying local congestion impacts on the feasibility set.
  \item \textbf{Tariff and settlement conventions (CAISO).} Bid structure and settlement mapping differ materially; ERCOT cannot import them without redesigning participant interfaces.
\end{enumerate}

\subsection{Direct implications for NRCOT/RTC+B evaluation questions}
The comparative record implies that ERCOT’s RTC+B should be evaluated using questions that align with how co-optimized markets succeed or fail:

\paragraph{Q1: Does RTC+B improve operational fidelity without increasing fragility?}
We test whether observed ancillary scarcity tails correspond to stress regimes (net-load ramps, contingency clusters) and whether frequency/deployment proxies improve (later Reliability chapter).

\paragraph{Q2: Did scarcity rents migrate from energy to AS, and is that welfare-improving?}
We treat “migration” as an empirical signature (tail mass shifts) and separately treat “welfare” as non-identifiable without quantities unless bounded (later Identification chapter).

\paragraph{Q3: Does the market power surface expand?}
Comparative evidence suggests reserve products can be thinner and more strategic; ERCOT needs detection rules and monitoring triggers tailored to RTC+B (later Market Power chapter).

\paragraph{Q4: Are storage feasibility constraints priced coherently?}
If the dominant new scarcity channel is feasibility (SoC headroom), then investor economics will be increasingly shaped by ancillary tails and performance risk rather than energy tails alone (later Investor chapter).

% ----------------------------------------------------------------------

\chapter{Empirical Findings (Descriptive Results Only)}
\label{ch:findings_full}
This chapter is strictly descriptive. No mechanism claims.


\section{Day-Ahead Ancillary Procurement (DAMASAGG): Fern-window posture and post-window reversion}
\label{sec:damasagg_wave1}
This section characterizes \emph{forecast-stage} scarcity handling using ERCOT's day-ahead ancillary award aggregation (DAMASAGGNP419). The objective is to document how procurement quantities and award-weighted prices moved across a stress window (Storm Fern) \emph{before} introducing real-time scarcity realizations (MCPC) or physical outcomes (frequency). All statements here are descriptive; no causal mechanism is asserted.

\subsection{Data structure and hour-level aggregation}
DAMASAGGNP419 reports ancillary awards as a set of price--quantity blocks. For a given delivery date $d$, hour $h$, and ancillary product $k$, multiple blocks $i$ may be present. We therefore define hour-level awarded quantity and an award-weighted average price (VWAP) as:
\begin{align}
Q^{DA}_k(d,h) &= \sum_i q_{k,i}(d,h), \\
P^{DA}_k(d,h) &= \frac{\sum_i p_{k,i}(d,h)\,q_{k,i}(d,h)}{\sum_i q_{k,i}(d,h)}.
\end{align}
Hour-ending timestamps are mapped to hour-beginning time by
\[
t_{\mathrm{start}}(d,h) = d + (h-1)\ \text{hours},
\]
with the DST flag retained as a provenance attribute.

\subsection{Fern partition and reporting discipline}
To avoid post hoc window selection, we partition the analysis into:
(i) a pre-window (2026-01-20 to 2026-01-21),
(ii) a Fern study window (2026-01-22 to 2026-01-29), and
(iii) a post-window (2026-01-30 to 2026-02-05).
Unless otherwise noted, the statistics below are computed over the hour-level panel $\{Q^{DA}_k(t), P^{DA}_k(t)\}$.

\subsection{Observed shifts in award quantities}
Relative to the pre-window, the Fern window exhibits higher awarded quantities in several control products. Using hour-level means of $Q^{DA}_k(t)$, Fern/pre percentage changes are:
\[
\Delta_Q(k) \equiv \frac{\mathbb{E}[Q^{DA}_k\mid \text{Fern}] - \mathbb{E}[Q^{DA}_k\mid \text{Pre}]}{\mathbb{E}[Q^{DA}_k\mid \text{Pre}]}\times 100\%.
\]
For the core products used in the Fern evaluation set, we observe:
\begin{itemize}
  \item REGUP: $\Delta_Q \approx +28.20\%$;
  \item REGDN: $\Delta_Q \approx +33.56\%$;
  \item RRSPF: $\Delta_Q \approx +34.96\%$;
  \item ECRSS: $\Delta_Q \approx +8.86\%$;
  \item NSPIN: $\Delta_Q \approx +3.83\%$.
\end{itemize}
These quantities are \emph{procurement posture} indicators: they show how day-ahead awards shifted during Fern, but do not by themselves establish whether real-time scarcity, deployment, or reliability outcomes improved.

\subsection{Observed shifts in award-weighted prices (VWAP)}
Award-weighted prices also move materially during Fern. Define the Fern/pre percentage change for VWAP as:
\[
\Delta_P(k) \equiv \frac{\mathbb{E}[P^{DA}_k\mid \text{Fern}] - \mathbb{E}[P^{DA}_k\mid \text{Pre}]}{\mathbb{E}[P^{DA}_k\mid \text{Pre}]}\times 100\%.
\]
For the same product set:
\begin{itemize}
  \item ECRSS: $\Delta_P \approx +98.58\%$;
  \item NSPIN: $\Delta_P \approx +109.16\%$;
  \item RRSPF: $\Delta_P \approx +125.19\%$;
  \item REGUP: $\Delta_P \approx +80.46\%$;
  \item REGDN: $\Delta_P \approx +52.36\%$.
\end{itemize}
Within the Fern window, the hourly maxima of $P^{DA}_k(t)$ reach approximately:
ECRSS $\approx 1480.34$, NSPIN $\approx 1061.95$, REGUP $\approx 1228.47$, REGDN $\approx 1076.08$, and RRSPF $\approx 1518.54$ (all in \$/MW-hour units as reported by ERCOT DAM award aggregation).

\subsection{Post-Fern reversion (post vs Fern)}
A key descriptive question is whether Fern reflects a distinct procurement regime rather than arbitrary sampling. Comparing the post-window to the Fern window (hour-level means):
\begin{itemize}
  \item REGUP: $Q^{DA}$ decreases $\approx 26.73\%$ and $P^{DA}$ decreases $\approx 18.09\%$;
  \item REGDN: $Q^{DA}$ decreases $\approx 13.85\%$ and $P^{DA}$ decreases $\approx 13.60\%$;
  \item RRSPF: $Q^{DA}$ decreases $\approx 18.25\%$ and $P^{DA}$ decreases $\approx 21.55\%$;
  \item ECRSS: $Q^{DA}$ decreases $\approx 16.90\%$ and $P^{DA}$ decreases $\approx 8.14\%$;
  \item NSPIN: $Q^{DA}$ decreases $\approx 15.33\%$ and $P^{DA}$ decreases $\approx 2.51\%$.
\end{itemize}
This mean reversion supports treating Fern as a distinct stress/posture episode for subsequent lead--lag tests linking day-ahead awards ($Q^{DA}_k$) to real-time scarcity realizations (MCPC) and, separately, to frequency outcomes once those series are integrated.


\subsection{Stack-shape diagnostics (block structure, dispersion, and concentration)}
Totals and VWAPs can mask whether procurement was broad-based or concentrated in a small number of high-priced blocks. Because DAMASAGGNP419 is reported as block pairs $(p_{k,i},q_{k,i})$, we compute three additional hour-level diagnostics for each ancillary product $k$:

\paragraph{Number of cleared blocks.}
\[
n_k(d,h)=\#\{i:\ q_{k,i}(d,h)>0\}.
\]

\paragraph{Intra-hour award price dispersion (quantity-weighted).}
\[
\sigma_{p,k}(d,h)=\sqrt{\frac{\sum_i q_{k,i}(d,h)\big(p_{k,i}(d,h)-P^{DA}_k(d,h)\big)^2}{\sum_i q_{k,i}(d,h)}}.
\]

\paragraph{Award concentration (Herfindahl index).}
\[
H_k(d,h)=\sum_i\left(\frac{q_{k,i}(d,h)}{\sum_j q_{k,j}(d,h)}\right)^2,
\]
with $H_k\in(0,1]$; higher values indicate that awards are concentrated in fewer blocks.

These diagnostics are descriptive but economically meaningful: increases in $H_k$ and $\sigma_{p,k}$ during Fern would be consistent with procurement pressure localized in a small set of expensive blocks, while stable or declining $H_k$ alongside higher $Q^{DA}_k$ would indicate broader procurement posture.

\begin{figure}[ht]
    \centering
    \begin{tikzpicture}[font=\sffamily\scriptsize, scale=0.9, transform shape]
        % --- PANEL A: ROBUST STACK ---
        \begin{scope}[xshift=0cm]
            \draw[->, thick] (0,0) -- (5.5,0) node[below] {Quantity ($Q$)};
            \draw[->, thick] (0,0) -- (0,5.5) node[left] {Price ($P$)};
            \node[anchor=south west, font=\bfseries] at (0, 5.5) {(A) Broad-Based Support};
            
            % Grid
            \draw[step=1cm, gray!10, very thin] (0,0) grid (5,5);

            % Step Function Supply Curve (Robust)
            \draw[thick, blue!70!black] (0,0.5) -- (1,0.5) -- (1,0.8) -- (2,0.8) -- (2,1.2) -- (3,1.2) -- (3,1.8) -- (4,1.8) -- (4,2.5) -- (5,2.5);
            \fill[blue!10, opacity=0.5] (0,0) -- (0,0.5) -- (1,0.5) -- (1,0.8) -- (2,0.8) -- (2,1.2) -- (3,1.2) -- (3,1.8) -- (4,1.8) -- (4,2.5) -- (5,2.5) -- (5,0) -- cycle;
            
            % Demand Lines
            \draw[dashed, thick] (3.5, 0) -- (3.5, 5) node[above, rotate=90, pos=0.05] {$D_{base}$};
            \draw[dashed, thick] (4.5, 0) -- (4.5, 5) node[above, rotate=90, pos=0.05] {$D_{stress}$};
            
            % Intersection Dots
            \filldraw[black] (3.5, 1.8) circle (2pt);
            \filldraw[black] (4.5, 2.5) circle (2pt);
            
            % Annotations
            \draw[<->, thick] (5.2, 1.8) -- (5.2, 2.5) node[midway, right] {$\Delta P$};
            \node[align=left, font=\scriptsize] at (1.5, 4) {Low Herfindahl ($H_k$)\\Many marginal units};
        \end{scope}

        % --- PANEL B: FRAGILE CONVEXITY ---
        \begin{scope}[xshift=7.5cm]
            \draw[->, thick] (0,0) -- (5.5,0) node[below] {Quantity ($Q$)};
            \draw[->, thick] (0,0) -- (0,5.5) node[left] {Price ($P$)};
            \node[anchor=south west, font=\bfseries] at (0, 5.5) {(B) Convex "Hockey Stick"};
            
            % Grid
            \draw[step=1cm, gray!10, very thin] (0,0) grid (5,5);

            % Step Function Supply Curve (Fragile)
            % Flat base, then vertical wall
            \draw[thick, red!70!black] (0,0.5) -- (3.5,0.5) -- (3.5,4.5) -- (5,4.5);
            \fill[red!10, opacity=0.5] (0,0) -- (0,0.5) -- (3.5,0.5) -- (3.5,4.5) -- (5,4.5) -- (5,0) -- cycle;
            
            % Demand Lines (Same Magnitude Shift)
            \draw[dashed, thick] (3.2, 0) -- (3.2, 5) node[above, rotate=90, pos=0.05] {$D_{base}$};
            \draw[dashed, thick] (4.2, 0) -- (4.2, 5) node[above, rotate=90, pos=0.05] {$D_{stress}$};
            
            % Intersection Dots
            \filldraw[black] (3.2, 0.5) circle (2pt);
            \filldraw[black] (3.5, 4.5) circle (2pt); % Hits the wall
            
            % Annotations
            \draw[<->, thick] (5.2, 0.5) -- (5.2, 4.5) node[midway, right, align=left] {\textbf{Extreme} $\Delta P$};
            \node[align=left, font=\scriptsize, text=red!70!black] at (1.5, 3.5) {\textbf{Feasibility Wall}\\(SoC Depletion)};
            
            % H_k Indicator
            \node[draw, fill=white, align=center, font=\scriptsize] at (2, 1.5) {High Concentration\\$H_k > 0.4$};
        \end{scope}
    \end{tikzpicture}
    \caption{\textbf{The Geometry of Scarcity.} Panel A shows a healthy stack with deep support. Panel B illustrates the ``Convex'' geometry observed during the Fern Event: a flat base of inframarginal offers followed immediately by a vertical feasibility wall, creating binary price outcomes.}
    \label{fig:stack_geometry_damasagg}
\end{figure}

\subsection{Intraday profiles and peak-hour targeting}
Window means can conceal whether procurement was directed toward the most risk-relevant hours. Let $t$ denote the hour-begin timestamp derived from $(d,h)$. For each product $k$ we compute hour-of-day profiles:
\[
\mu^{(h)}_{Q,k}=\mathbb{E}\!\left[Q^{DA}_k(t)\mid \mathrm{HourOfDay}(t)=h\right],\qquad
\mu^{(h)}_{P,k}=\mathbb{E}\!\left[P^{DA}_k(t)\mid \mathrm{HourOfDay}(t)=h\right],
\]
reported separately for pre/Fern/post.

To measure whether awards concentrated in high-risk hours, define a ``peak'' set $\mathcal{P}$ as the top decile of hours by system load within the comparable season (constructed from the native-load climatology; see Appendix~\ref{app:data_quality_full}). The peak-hour award share is:
\[
\mathrm{PeakShare}_k=\frac{\sum_{t\in\mathcal{P}} Q^{DA}_k(t)}{\sum_t Q^{DA}_k(t)}.
\]
An increase in $\mathrm{PeakShare}_k$ during Fern indicates targeted procurement toward peak stress hours rather than uniform scaling.

\subsection{Pre-registered Wave 1 hypotheses and falsification conditions}
Wave~1 evaluates \emph{forecast-stage posture} only. We pre-register the following directionality tests.

\paragraph{H1a (procurement posture):} For key control products $k\in\{\mathrm{REGUP},\mathrm{REGDN},\mathrm{RRSPF}\}$, Fern increases awarded quantity relative to the pre-window:
\[
\mathbb{E}[Q^{DA}_k\mid \mathrm{Fern}]>\mathbb{E}[Q^{DA}_k\mid \mathrm{Pre}].
\]

\paragraph{H1b (price posture):} For scarcity-sensitive products $k\in\{\mathrm{ECRSS},\mathrm{RRSPF},\mathrm{NSPIN}\}$, Fern increases award VWAP tail measures relative to the pre-window (e.g., $Q_{0.90}$ or $Q_{0.99}$ of $P^{DA}_k$).

\paragraph{Falsifiers (Wave 2-facing):} The Wave~1 results are necessary but not sufficient evidence of RTC+B efficacy. Subsequent Wave~2 analyses will treat the following as falsification patterns:
\begin{itemize}
  \item \textbf{Forecast miss signature:} $Q^{DA}_k$ does not increase in Fern while real-time MCPC tails (or energy adders) expand materially.
  \item \textbf{Feasibility bind signature:} $Q^{DA}_k$ increases and remains broad-based (low $H_k$), yet MCPC tails still expand \emph{and} frequency stress metrics deteriorate.
  \item \textbf{Fragmentation signature:} joint tail incidence across multiple ancillary products increases (scarcity appears simultaneously across products rather than concentrating in the most operationally relevant one).
\end{itemize}
These falsifiers are stated here to prevent post hoc reinterpretation once Wave~2 series are introduced.

\subsection{Wave 1 limitations (what DAMASAGG can and cannot prove)}
DAMASAGGNP419 describes \emph{day-ahead} award decisions and therefore measures \emph{intent and forecast posture}, not realized scarcity, actual deployment, or physical reliability outcomes. In particular:
\begin{enumerate}
  \item High $Q^{DA}_k$ may reflect conservative procurement under uncertainty rather than realized need.
  \item High $P^{DA}_k$ may indicate tight supply offers in the award stack without implying subsequent real-time scarcity.
  \item Any claim about reliability improvement requires alignment to frequency, deployment, and (where applicable) load-shed indicators, which are analyzed separately.
\end{enumerate}
Accordingly, all Wave~1 results are reported descriptively and are used primarily to structure Wave~2 tests that distinguish forecast-stage posture from real-time scarcity realizations.

\section{Wave 2: Real-time ancillary scarcity (SCED MCPC) and Day-Ahead to Real-Time alignment (Fern window)}
\label{sec:wave2_mcpc_alignment}
Wave~2 extends the Wave~1 DAMASAGG posture analysis by measuring how scarcity value manifests in real-time ancillary Market Clearing Prices for Capacity (MCPC) and whether day-ahead ancillary procurement posture aligns with subsequent real-time scarcity conditions. In keeping with this report's descriptive-only discipline, all results below are distributional and correlational; no causal mechanisms are asserted.

\subsection{Data coverage and alignment}
This Wave~2 pass uses (i) an ERCOT historical extract of SCED 5-minute MCPC observations for 2026-01-20 through 2026-01-31 for products $\{\mathrm{REGUP},\mathrm{REGDN},\mathrm{RRS},\mathrm{ECRS},\mathrm{NSPIN}\}$; and (ii) 15-minute RTM settlement point prices (SPP) for hubs and load zones over the same calendar span. Because the currently loaded historical MCPC extract ends on 2026-01-31, ``post-Fern'' results in Wave~2 are necessarily restricted to 2026-01-30--2026-01-31. This boundary is treated as a hard scope constraint rather than an implicit assumption.

Time base: MCPC is observed at the SCED interval (nominally 5 minutes), while RTM SPP is provided at 15-minute settlement intervals. For joint analyses we report both native resolutions, and an hour-level aggregation used strictly for DA-to-RT alignment checks:
\[
\mathrm{MCPC}^{\max}_{h,p}=\max_{t\in h}\mathrm{MCPC}_{t,p},\qquad
\mathrm{MCPC}^{\mathrm{mean}}_{h,p}=\frac{1}{|h|}\sum_{t\in h}\mathrm{MCPC}_{t,p},
\]
\[
\mathrm{MCPC}^{\mathrm{CVaR}_0.99}_{h,p}=\mathbb{E}\left[\mathrm{MCPC}_{t,p}\mid \mathrm{MCPC}_{t,p}\ge Q_{0.99}(\mathrm{MCPC}_{\cdot,p}\mid h)\right].
\]
Here $h$ indexes an operating hour, $p$ indexes the ancillary product, and $Q_{0.99}(\cdot\mid h)$ is computed within-hour. These transformations are audited in the Reproducibility Annex and are not tuned ex post.

\subsection{MCPC distributional shifts across pre/Fern/post windows}
For each product, we compute mean, high-quantiles, tail conditional expectation (CVaR at 0.99), and exceedance rates under fixed thresholds $\tau\in\{1,5,10,50,100\}$ dollars per MW. Using the Metric Dictionary window definitions (pre: 2026-01-20--2026-01-21; Fern: 2026-01-22--2026-01-29; post: 2026-01-30--2026-01-31), the descriptive results are:

\begin{itemize}
\item \textbf{REGUP}: $\bar{p}_\mathrm{MCPC}$ 1.35 (pre) $\rightarrow$ 5.56 (Fern) $\rightarrow$ 1.49 (post); $Q_{0.99}$ 12.73 $\rightarrow$ 40.00 $\rightarrow$ 10.85; $\Pr[\mathrm{MCPC}>10]$ 2.75\% $\rightarrow$ 15.57\% $\rightarrow$ 2.40\%; $\max$ 221.97, 218.73, 13.22.
\item \textbf{REGDN}: $\bar{p}_\mathrm{MCPC}$ 0.44 (pre) $\rightarrow$ 4.36 (Fern) $\rightarrow$ 0.80 (post); $Q_{0.99}$ 1.99 $\rightarrow$ 25.31 $\rightarrow$ 5.05; $\Pr[\mathrm{MCPC}>10]$ 0.00\% $\rightarrow$ 16.33\% $\rightarrow$ 0.00\%; $\max$ 2.18, 71.75, 7.12.
\item \textbf{RRS}: $\bar{p}_\mathrm{MCPC}$ 1.10 (pre) $\rightarrow$ 1.45 (Fern) $\rightarrow$ 0.74 (post); $Q_{0.99}$ 13.91 $\rightarrow$ 15.06 $\rightarrow$ 12.53; $\Pr[\mathrm{MCPC}>10]$ 2.75\% $\rightarrow$ 1.78\% $\rightarrow$ 1.54\%; $\max$ 221.97, 216.21, 14.78.
\item \textbf{ECRS}: $\bar{p}_\mathrm{MCPC}$ 0.88 (pre) $\rightarrow$ 2.53 (Fern) $\rightarrow$ 1.06 (post); $Q_{0.99}$ 18.77 $\rightarrow$ 17.05 $\rightarrow$ 14.84; $\Pr[\mathrm{MCPC}>10]$ 3.61\% $\rightarrow$ 2.16\% $\rightarrow$ 2.57\%; $\max$ 43.68, 399.45, 14.99.
\item \textbf{NSPIN}: $\bar{p}_\mathrm{MCPC}$ 1.03 (pre) $\rightarrow$ 4.62 (Fern) $\rightarrow$ 2.27 (post); $Q_{0.99}$ 18.77 $\rightarrow$ 20.18 $\rightarrow$ 14.99; $\Pr[\mathrm{MCPC}>10]$ 3.61\% $\rightarrow$ 14.04\% $\rightarrow$ 6.52\%; $\max$ 34.21, 399.45, 15.00.
\end{itemize}

Two descriptive points matter for downstream mechanism testing. First, Fern exhibits materially higher mean MCPC in $\mathrm{REGUP}$, $\mathrm{REGDN}$, and $\mathrm{NSPIN}$ relative to the pre-Fern baseline, alongside large increases in high-quantiles for $\mathrm{REGUP}$ and especially $\mathrm{REGDN}$. Second, maxima can spike in Fern even when $Q_{0.99}$ is not uniformly higher (e.g., $\mathrm{ECRS}$), implying that ``tail mass'' (frequency of exceedance) and ``tail reach'' (extreme realizations) can decouple. This is exactly why we maintain both exceedance-rate measures and high-quantiles rather than reducing scarcity to a single statistic.

\subsection{Energy-side SPP tail behavior during Fern}
To anchor ancillary tail behavior against energy-side price formation, we compute parallel descriptive measures for RTM settlement point prices. For the three hub targets, the 0.99-quantile of 15-minute SPP shifts as follows:

\begin{itemize}
\item \textbf{HB\_NORTH}: $Q_{0.99}$ of RTM SPP (15-min) 126.45 (pre) $\rightarrow$ 1,021.67 (Fern) $\rightarrow$ 657.77 (post).
\item \textbf{HB\_SOUTH}: $Q_{0.99}$ of RTM SPP (15-min) 85.90 (pre) $\rightarrow$ 765.77 (Fern) $\rightarrow$ 112.96 (post).
\item \textbf{HB\_HOUSTON}: $Q_{0.99}$ of RTM SPP (15-min) 103.33 (pre) $\rightarrow$ 442.28 (Fern) $\rightarrow$ 98.25 (post).
\end{itemize}

These hub results establish that Fern corresponds to a regime shift in energy-side tail behavior (particularly for HB\_NORTH and HB\_SOUTH). Whether the hub tails arise from fundamental scarcity, congestion, settlement-adder accounting, or any combination is not asserted here. Instead, Wave~3 will connect these descriptive tails to ORDC / reliability deployment adder series and the RTC+B settlement mapping rules.

\subsection{Day-Ahead posture vs Real-Time scarcity (alignment diagnostics)}
Wave~1 quantified day-ahead ancillary procurement posture (DAMASAGG) via award quantity $Q_\mathrm{DA}$ and volume-weighted award price $P_\mathrm{DA}=\sum_i p_i q_i/\sum_i q_i$. Wave~2 asks whether these posture metrics covary with realized real-time ancillary scarcity measured by hourly $\max\mathrm{MCPC}$. For the Fern window, contemporaneous and one-hour lead correlations (Pearson $\rho$) between $Q_\mathrm{DA}$ and hourly $\max\mathrm{MCPC}$ are:

\begin{itemize}
\item \textbf{REGUP}: contemporaneous $\rho(Q_\mathrm{DA},\max\mathrm{MCPC})\approx 0.393$; one-hour lead $\rho(Q_\mathrm{DA}^{t-1},\max\mathrm{MCPC}^t)\approx 0.367$ (Fern window).
\item \textbf{REGDN}: contemporaneous $\rho(Q_\mathrm{DA},\max\mathrm{MCPC})\approx 0.064$; one-hour lead $\rho(Q_\mathrm{DA}^{t-1},\max\mathrm{MCPC}^t)\approx 0.078$ (Fern window).
\item \textbf{RRS}: contemporaneous $\rho(Q_\mathrm{DA},\max\mathrm{MCPC})\approx 0.163$; one-hour lead $\rho(Q_\mathrm{DA}^{t-1},\max\mathrm{MCPC}^t)\approx 0.164$ (Fern window).
\item \textbf{ECRS}: contemporaneous $\rho(Q_\mathrm{DA},\max\mathrm{MCPC})\approx 0.170$; one-hour lead $\rho(Q_\mathrm{DA}^{t-1},\max\mathrm{MCPC}^t)\approx 0.159$ (Fern window).
\item \textbf{NSPIN}: contemporaneous $\rho(Q_\mathrm{DA},\max\mathrm{MCPC})\approx 0.150$; one-hour lead $\rho(Q_\mathrm{DA}^{t-1},\max\mathrm{MCPC}^t)\approx 0.179$ (Fern window).
\end{itemize}

These correlations are descriptive and should not be read as forecasting performance. Their purpose is falsification: if day-ahead awards are fully orthogonal to real-time scarcity, $\rho$ should be near zero across products and lags. Instead, we observe a stable positive $\rho$ for $Q_\mathrm{DA}$ in $\mathrm{REGUP}$ and smaller but non-zero positive values in $\mathrm{RRS}$, $\mathrm{ECRS}$, and $\mathrm{NSPIN}$. For $\mathrm{REGDN}$, correlations are close to zero in the Fern window under this simple aggregation, suggesting either (i) a weaker link between day-ahead procurement and real-time $\mathrm{REGDN}$ scarcity, or (ii) that the relevant alignment operates through different lags or conditioning variables (net-load ramps, outages, or binding feasibility constraints). Wave~3 will handle this by conditioning on system state and by replacing Pearson $\rho$ with tail-event co-occurrence tests.


\subsection{Wave 2: Real-Time Ancillary Scarcity Realization and Sensitivity Analysis}
\label{sec:wave2_mcpc}

This section examines the realization of scarcity in the Real-Time Market (RTM) under RTC+B, conditional on the Day-Ahead (DA) procurement posture documented in Wave~1. The focus is on Market Clearing Prices for Capacity (MCPC) for ancillary services during the Fern window and the immediate post-window period. Consistent with the chapter discipline, results are descriptive; mechanistic or welfare claims are deferred to subsequent chapters.

\paragraph{Scarcity typology tag.}
Wave~2 MCPC tail outcomes are treated as \emph{administrative-scarcity
consistent} evidence (Section~\ref{sec:scarcity_typology}). Wave~2 does not
interpret high MCPC as physical energy inadequacy.

\paragraph{Scarcity typology tag.}
Wave~2 MCPC tail outcomes are treated as \emph{administrative-scarcity
consistent} evidence (Section~\ref{sec:scarcity_typology}). Wave~2 does not
interpret high MCPC as physical energy inadequacy.

\subsubsection{Data scope, temporal coverage, and interval integrity}
Wave~2 utilizes SCED-resolution MCPC data at five-minute intervals for the period spanning 2026-01-29 through 2026-02-05 (inclusive). Products analyzed include REG-UP, REG-DOWN, RRS, ECRS, and Non-Spin. Let $\mathrm{MCPC}_k(t)$ denote the real-time MCPC for product $k$ at interval $t$.

Because Wave~2 relies on stitched daily exports, we begin with interval-integrity checks. Let $m_{k,d}$ denote the number of missing five-minute intervals for product $k$ on day $d$, with $288$ expected per day. We report missingness and halt downstream computations if:
\[
\frac{\sum_{k,d} m_{k,d}}{\sum_{k,d} 288} > 0.01,
\]
consistent with the report's data governance discipline. Duplicate timestamps are flagged and deterministically de-duplicated (rule logged in Appendix~\ref{app:data_quality_full}), and non-numeric values are treated as parse errors and counted.

\subsubsection{Distributional characterization of real-time scarcity}
For each product $k$, we characterize the empirical distribution of $\mathrm{MCPC}_k(t)$ using complementary tail measures that separate (i) tail reach, (ii) tail mass, and (iii) tail severity conditional on exceedance:
\begin{align}
Q_{q,k} &= \inf \{x : \Pr(\mathrm{MCPC}_k \le x) \ge q\}, \quad q \in \{0.95, 0.99, 0.999\}, \\
\pi_k(\tau) &= \Pr(\mathrm{MCPC}_k > \tau), \quad \tau \in \{50, 100, 500\}, \\
\mathrm{CVaR}_{q,k} &= \mathbb{E}\left[\mathrm{MCPC}_k \mid \mathrm{MCPC}_k > Q_{q,k}\right].
\end{align}

These statistics are reported by window (Pre, Fern, Post) under the pre-registered window definition and again under padded windows (Section~\ref{sec:wave2_sensitivity_registry}).

\subsubsection{Scarcity clustering and persistence}
To quantify whether scarcity manifests as isolated spikes or sustained stress, we compute run-length and ``tail energy'' statistics. Define a fixed threshold $\tau$ and let an exceedance run be a maximal set of consecutive intervals with $\mathrm{MCPC}_k(t)>\tau$. The maximal run-length is
\[
R_k(\tau)=\max\left\{\text{length of consecutive intervals }t:\mathrm{MCPC}_k(t)>\tau\right\}.
\]
For a high-quantile threshold $u_k$ (e.g., $u_k=Q_{0.99,k}$), define exceedances $Y_k(t)=\mathrm{MCPC}_k(t)-u_k$ for $\mathrm{MCPC}_k(t)>u_k$ and define an episode's area-above-threshold (``tail energy''):
\[
A_e=\sum_{t\in e}\left(\mathrm{MCPC}_k(t)-u_k\right)\Delta t,
\]
where $\Delta t=5$ minutes. We report the distribution of $(D_e,\max_{t\in e}\mathrm{MCPC}_k(t),A_e)$ across episodes.

\subsubsection{Intra-day localization of scarcity}
Scarcity realization can be strongly hour-of-day dependent. We therefore compute hour-of-day profiles:
\[
\bar{\mathrm{MCPC}}_k(h)=\mathbb{E}\!\left[\mathrm{MCPC}_k(t)\mid h(t)=h\right],
\]
and corresponding high-quantile profiles $Q_{0.99,k}(h)$ computed within hour bins. This documents whether scarcity concentrates in structurally stressed hours (e.g., ramp periods) versus diffuse elevation.

\subsubsection{Day-Ahead to Real-Time alignment diagnostics}
\label{sec:wave2_da_conditioning}
To evaluate whether DA posture aligns with realized RT scarcity, we compute descriptive alignment statistics between DA awards and RT MCPC. Let $Q^{DA}_k(h)$ denote hourly DA awards from Wave~1 and let $\max_{t\in h}\mathrm{MCPC}_k(t)$ denote the within-hour maximum. We compute contemporaneous and lagged correlations:
\begin{align}
\rho_k^{(0)} &= \mathrm{corr}\!\left(Q^{DA}_k(h(t)), \max_{t \in h} \mathrm{MCPC}_k(t)\right), \\
\rho_k^{(1)} &= \mathrm{corr}\!\left(Q^{DA}_k(h(t)-1), \max_{t \in h} \mathrm{MCPC}_k(t)\right),
\end{align}
and extend to lags $0$ through $+6$ hours as a registered sensitivity lever.

To avoid imposing linearity, we also compute quantile-on-quantile descriptives. Define a normalized DA posture index
\[
Z_k(h)=\frac{Q^{DA}_k(h)}{\widetilde{Q}_k},
\]
where $\widetilde{Q}_k$ is the baseline median DA award for product $k$. We bin hours by $Z_k(h)$ (e.g., $[0,1)$, $[1,1.25)$, $[1.25,1.5)$, $[1.5,\infty)$) and report conditional RT tail behavior:
\[
Q_{q,k}^{RT}\bigm| Z_k \in \mathcal{B}.
\]

\subsubsection{Joint-tail structure and scarcity fragmentation}
\label{sec:wave2_jointtails}
RTC+B may shift scarcity between products or concentrate it across multiple products simultaneously. We therefore characterize joint-tail structure. Define exceedance indicators relative to product-specific quantiles:
\[
\mathbb{S}_{k,q}(t)=\mathbf{1}\{\mathrm{MCPC}_k(t)>Q_{q,k}\},\quad q\in\{0.99,0.999\}.
\]
For $k\neq j$, compute joint exceedance probabilities
\[
\pi_{k,j}(q)=\Pr\!\left(\mathbb{S}_{k,q}(t)=1,\;\mathbb{S}_{j,q}(t)=1\right),
\]
and tail co-movement correlations
\[
\rho_{k,j}^{\mathrm{tail}}(q)=\mathrm{corr}\!\left(\mathbb{S}_{k,q}(t),\mathbb{S}_{j,q}(t)\right).
\]
To summarize fragmentation at the interval level, define the tail multiplicity statistic
\[
M_q(t)=\sum_k \mathbb{S}_{k,q}(t),
\]
and report $\Pr(M_q\ge 2)$ and $\Pr(M_q\ge 3)$. High mass at $M_q=1$ indicates fragmentation (single-product scarcity), while high mass at $M_q\ge 2$ indicates multi-product coherence.

\subsubsection{Nonparametric distribution shift tests}
\label{sec:wave2_nonparametric}
To avoid over-reliance on moments or single quantiles, we test for distributional shifts across windows using nonparametric procedures suitable for heavy-tailed data. For each product $k$ we report:
\begin{itemize}
  \item Kolmogorov--Smirnov (KS) tests comparing empirical CDFs across windows,
  \item Anderson--Darling (AD) tests (tail-sensitive),
  \item Quantile shift profiles $\Delta Q_{q,k}=Q^{(B)}_{q,k}-Q^{(A)}_{q,k}$ for $q\in\{0.50,0.90,0.95,0.99,0.999\}$.
\end{itemize}

Because MCPC series are serially correlated, we compute block-bootstrap confidence intervals for $\Delta Q_{q,k}$ using contiguous blocks of length $b$ intervals, producing $(P10,P50,P90)$ bands. Where multiple tests are reported (products $\times$ quantiles $\times$ windows), we control the false discovery rate (FDR) using Benjamini--Hochberg on reported $p$-values.

\subsubsection{Extreme-value characterization (tail index and threshold stability)}
\label{sec:wave2_evt}
MCPC tails may be sufficiently heavy that finite-variance summaries can be misleading. We therefore include an extreme-value characterization using a Peaks-Over-Threshold (POT) framework. For a threshold $u_k$ chosen as a high quantile (e.g., $u_k=Q_{0.99,k}$), define exceedances
\[
Y_k(t)=\mathrm{MCPC}_k(t)-u_k \quad \text{for }\mathrm{MCPC}_k(t)>u_k.
\]
We fit a Generalized Pareto Distribution (GPD) to $Y_k$:
\[
\Pr(Y_k\le y)=1-\left(1+\xi_k\frac{y}{\beta_k}\right)^{-1/\xi_k},\quad y\ge 0,
\]
and report the tail index $\xi_k$, with threshold-stability checks repeated for $u_k\in\{Q_{0.985,k},Q_{0.99,k},Q_{0.995,k}\}$. For descriptive return-level diagnostics, we report
\[
x_{k,\alpha}=u_k+\frac{\beta_k}{\xi_k}\left(\alpha^{-\xi_k}-1\right),
\]
for representative exceedance probabilities $\alpha$.

\subsubsection{Energy-side co-movement (association only)}
\label{sec:wave2_energy_comovement}
Although Wave~2 is ancillary-focused, RTC+B may shift scarcity signals between energy and ancillary channels. Let $P^{RT}_{s}(t)$ denote RT hub SPP (aligned to five minutes via the documented 15-to-5 mapping). For each hub $s$ and product $k$, we report:
\begin{align}
\mathrm{corr}\!\left(P^{RT}_{s}(t),\mathrm{MCPC}_k(t)\right),\qquad
\Pr\!\left(P^{RT}_{s}(t)>Q_{0.99,s}\mid \mathbb{S}_{k,0.99}(t)=1\right).
\end{align}
These are descriptive association measures only.

\subsubsection{Sensitivity registry (pre-registered levers)}
\label{sec:wave2_sensitivity_registry}
Because scarcity metrics are aggregation-sensitive, all Wave~2 results are recomputed under the following pre-registered levers:
\begin{itemize}
\item \textbf{Aggregation choice:} within-hour $\max_t\mathrm{MCPC}_k(t)$ versus within-hour mean $\mathbb{E}_t[\mathrm{MCPC}_k(t)]$ versus within-hour $\mathrm{CVaR}_{0.99,k}$,
\item \textbf{Tail definition:} fixed thresholds $\tau$ versus quantile exceedance definitions,
\item \textbf{Window padding:} pre/Fern/post expanded by $\pm 48$ hours,
\item \textbf{DA-to-RT lag:} contemporaneous alignment ($0$) through $+6$ hours.
\end{itemize}
Divergence across these specifications is reported explicitly and treated as sensitivity, not noise. This registry is committed prior to interpretation to prevent post hoc metric selection.

\subsubsection{Wave 2 descriptive results summary (structure)}
\label{sec:wave2_results_summary}
This subsection provides a compact descriptive summary of Wave~2 outputs in a format suitable for ERCOT-facing review. Numerical values are reported in the associated appendices and referenced here by metric and window.

\paragraph{Tail reach and tail mass by product.}
For each product $k\in\{\mathrm{REGUP},\mathrm{REGDN},\mathrm{RRS},\mathrm{ECRS},\mathrm{NSPIN}\}$ we report (i) tail reach via $(Q_{0.99,k},Q_{0.999,k})$, (ii) tail mass via $\pi_k(\tau)$ for $\tau\in\{50,100,500\}$, and (iii) conditional severity via $\mathrm{CVaR}_{0.99,k}$, all segmented by Pre/Fern/Post windows and re-evaluated under the sensitivity levers in Section~\ref{sec:wave2_sensitivity_registry}.

\paragraph{Clustering and persistence.}
We report run-length $R_k(\tau)$ and episode severity $(D_e,\max \mathrm{MCPC},A_e)$ to distinguish isolated scarcity spikes from sustained stress episodes. The episode ledger produced here is the conditioning set used in Wave~3 (frequency and IMFR alignment).

\paragraph{Fragmentation versus coherence.}
We report the distribution of $M_q(t)$ and the joint exceedance matrix $\pi_{k,j}(q)$ to characterize whether scarcity is predominantly single-product (fragmented) or multi-product (coherent). This is reported separately for Fern and post-Fern periods.

\paragraph{DA posture alignment.}
We report contemporaneous and lagged alignment statistics $(\rho_k^{(0)},\rho_k^{(1)})$ and their extension to lags $0$ to $+6$ hours. We also report conditional RT tail levels $Q_{q,k}^{RT}\mid Z_k\in \mathcal{B}$ to identify nonlinear regimes where DA posture is associated with lower or higher realized RT scarcity, without imposing a parametric model.

\paragraph{Energy-side association.}
For hubs $s\in\{\mathrm{HB\_NORTH},\mathrm{HB\_SOUTH},\mathrm{HB\_HOUSTON}\}$ we report descriptive co-movement statistics between $P^{RT}_{s}(t)$ and $\mathrm{MCPC}_k(t)$ and conditional probabilities of energy-tail realization during ancillary-tail intervals. These associations are not interpreted as causal substitution without structural identification.

\subsection{Wave 2 limitations and next data hooks}

Wave~2 is constrained by (i) post-window coverage ending 2026-01-31 for the currently loaded MCPC extract, and (ii) the absence of a settlement-grade frequency series aligned to the full Fern window in the current bundle. The next data hooks are therefore: (a) MCPC for 2026-02-01 through 2026-02-05 (to complete post-Fern), (b) a 10-second or 1-second frequency series spanning 2026-01-20 through 2026-02-05, and (c) ORDC / reliability deployment price adder series over the same intervals. These enable the ``frequency conditional'' tests and the ORDC-to-ASDC bridge promised in the hypothesis register.

\section{Energy-side price behavior and adder incidence}
\label{sec:energy_price_behavior}

\paragraph{Evidence rung.}
All statements in this section are restricted to Evidence Rung~R1 (Table~\ref{tab:evidence_rungs}):
settlement-grade prices and adders only. Any statement requiring awards, deployments, or
resource-level feasibility is not identifiable at R1 and is not made here.

This section documents the empirical behavior of real-time energy prices and associated reliability adders under the post-RTC+B regime. All results are descriptive and conditioned on settlement-grade artifacts. No welfare or causal claims are made.

We analyze three related price constructs:
\begin{enumerate}
    \item The Locational Marginal Price for energy, $\lambda^{\mathrm{LMP}}$,
    \item The Real-Time Settlement Point Price, $\pi^{\mathrm{SPP}}$, and
    \item The Reliability Deployment Price Adder for energy, $a^{\mathrm{RT}}$, where
    \[
    \pi^{\mathrm{SPP}} = \lambda^{\mathrm{LMP}} + a^{\mathrm{RT}}.
    \]
\end{enumerate}

Distributions are computed separately for major trading hubs (HB\_NORTH,\allowbreak HB\_SOUTH,\allowbreak HB\_HOUSTON) and representative load zones (LZ\_AEN,\allowbreak LZ\_CPS,\allowbreak LZ\_LCRA). For each location, we report empirical quantiles, exceedance rates above fixed thresholds (e.g., \$100/MWh, \$500/MWh), and tail mass measures.

Special attention is paid to the incidence of $a^{\mathrm{RT}}$, defined as:
\[
a^{\mathrm{RT}}_t = \pi^{\mathrm{SPP}}_t - \lambda^{\mathrm{LMP}}_t.
\]
Intervals with $a^{\mathrm{RT}}>0$ are flagged as energy-side scarcity intervals. The frequency, duration, and clustering of such intervals are summarized by location and by system condition bins (load quintile and net-load ramp quintile).

These statistics establish the baseline energy-side scarcity signature against which ancillary-market behavior is later compared.

\section{Ancillary MCPC distributions and tail behavior}
\label{sec:mcpc_distributions}

\paragraph{Evidence rung.}
All statements in this section are restricted to Evidence Rung~R1 (Table~\ref{tab:evidence_rungs}):
settlement-grade MCPC prices only. Geometry, awards, deployments, and SoC/telemetry attribution are
not identifiable at R1.

\paragraph{Scarcity typology tag.}
High or heavy-tailed MCPC outcomes are treated as \emph{administrative-scarcity
consistent} signals (Section~\ref{sec:scarcity_typology}). They are not treated
as evidence of physical scarcity.

This section reports the distributional properties of Market Clearing Prices for Capacity (MCPC) across ancillary service products in the real-time market. Products examined include Regulation Up (REG-UP), Regulation Down (REG-DOWN), Responsive Reserve Service (RRS), and Emergency Contingency Reserve Service (ECRS), where applicable.

For each product $k$, the empirical distribution of MCPC prices $\{p_{k,t}\}$ is summarized using:
\begin{itemize}
    \item High quantiles ($Q_{0.95}$, $Q_{0.99}$, $Q_{0.999}$),
    \item Fixed-threshold exceedance rates (e.g., $p_{k,t}>\$50$, $>\$100$, $>\$500$),
    \item Tail conditional means (CVaR$_{0.99}$),
    \item Episode counts and episode durations as defined by the Episode Ledger.
\end{itemize}

Temporal clustering is evaluated by constructing ancillary price episodes, defined as maximal contiguous sets of intervals satisfying the product-specific tail criterion. Episode length distributions and inter-episode waiting times are reported for each product.

These statistics characterize the extent to which scarcity pricing manifests within ancillary services under RTC+B, independent of energy-side price behavior.

\section{Co-movement diagnostics}
\label{sec:comovement}

\paragraph{Evidence rung.}
All statements in this section are restricted to Evidence Rung~R1 (Table~\ref{tab:evidence_rungs}):
price-based co-movement only. Directional or mechanistic claims are not identifiable at R1.

\paragraph{Scarcity typology tag.}
Co-movement between energy tails/adders and ancillary MCPC tails is labeled
\emph{ambiguous} in the scarcity typology (Section~\ref{sec:scarcity_typology})
unless accompanied by additional operational evidence at higher rungs.

To assess the empirical relationship between energy-side scarcity and ancillary scarcity, this section reports co-movement diagnostics between:
\begin{itemize}
    \item Energy price tails ($\pi^{\mathrm{SPP}}$ and $a^{\mathrm{RT}}$), and
    \item Ancillary MCPC tails ($p_{k,t}$ for each product $k$).
\end{itemize}

Diagnostics include:
\begin{enumerate}
    \item Pearson and Spearman correlations computed on interval-level prices,
    \item Conditional exceedance probabilities of the form
    \[
    \Pr\left(p_{k,t}>\tau_k \mid a^{\mathrm{RT}}_t>\tau_E\right),
    \]
    \item Symmetric conditional probabilities with energy tails conditioned on ancillary tails.
\end{enumerate}

All conditional statistics are computed under matched system condition bins to reduce confounding from load and ramp effects. Results are reported separately for contemporaneous intervals ($\ell=0$) and for short lead--lag windows ($\ell=\pm1,\pm2$ intervals).

These diagnostics are intended to reveal whether scarcity signals appear to migrate across market products or remain orthogonal under RTC+B, without asserting any direction of causality.

\section{Frequency proxy results (descriptive only)}
\label{sec:frequency_results}

\paragraph{Evidence rung.}
Statements in this section are restricted to Evidence Rung~R3 (Table~\ref{tab:evidence_rungs}):
prices plus observed operational outcomes (frequency telemetry and any deployment proxies in scope).
Resource-level feasibility attribution (e.g., SoC headroom) is not identifiable without R4.

\paragraph{Scarcity typology tag.}
Frequency excursions aligned with ancillary price episodes are treated as
\emph{administrative-scarcity consistent} operational-stress evidence
(Section~\ref{sec:scarcity_typology}); they do not, by themselves, identify
physical scarcity.

This section reports descriptive frequency outcomes aligned with price and ancillary events. Frequency data are treated as observational telemetry and are not used to infer control efficacy or causality.

Three frequency-related metrics are evaluated:
\begin{enumerate}
    \item The integrated absolute frequency deviation,
    \[
    I_f = \sum_t |f_t - 60| \Delta t,
    \]
    \item The count of band violations,
    \[
    N_{\epsilon} = \sum_t \mathbf{1}\{|f_t - 60| > \epsilon\},
    \]
    for $\epsilon \in \{0.036, 0.1\}$ Hz,
    \item Event-conditioned frequency trajectories centered on ancillary price episodes.
\end{enumerate}

Frequency profiles are aligned to the start of high-MCPC episodes and summarized using the median and interquartile range across episodes. These profiles illustrate typical frequency behavior before, during, and after periods of elevated ancillary prices.

All results in this section are explicitly descriptive and do not imply that observed price behavior caused observed frequency outcomes.

\section{Event study panels (if data supports it)}
\label{sec:event_panels}

\paragraph{Evidence rung.}
Panels in this section are restricted to Evidence Rung~R3 (Table~\ref{tab:evidence_rungs}):
prices plus operational telemetry/deployment proxies, where present. Any panel interpretation that would
require SoC/telemetry feasibility (R4) is explicitly labeled as not identifiable.

\paragraph{Scarcity typology tag.}
Panels are interpreted as \emph{administrative-scarcity consistent} episode
signatures unless the panel includes explicit energy-side scarcity evidence.
High MCPC alone is not interpreted as physical scarcity.

Where data density permits, this section presents event-study panels centered on high-MCPC events. Events are triggered when a product-specific MCPC exceeds its registered tail threshold.

For each event, aligned windows spanning $[-60,+60]$ minutes are constructed for:
\begin{itemize}
    \item MCPC of the triggering product,
    \item Other ancillary products,
    \item Energy prices and adders,
    \item Frequency metrics.
\end{itemize}

Panels report the median trajectory across events, with interquartile bands to convey dispersion. No normalization beyond time alignment is applied.

These panels provide a compact visualization of the empirical ``shape'' of scarcity episodes under RTC+B and support qualitative assessment of timing and coordination across market products.

Consistent with the Ledger-to-Figure Traceability Rule (Section~\ref{sec:traceability}), each panel explicitly declares the episode identifiers and sensitivity levers used.

\chapter{Formal RTC+B Model: Optimization, Dual Prices, and KKT Conditions}
\label{ch:kkt_model}

This chapter provides the mathematical spine of the report. The objective is to formalize, in a controlled stylized setting, why scarcity value can manifest in ancillary products (MCPC) rather than energy prices, and why the introduction of energy storage resources (ESRs) with intertemporal feasibility constraints creates additional scarcity channels that do not exist in single-period scarcity heuristics. The model is intentionally simplified relative to ERCOT’s full DAM and SCED implementations, but it is constructed to preserve the essential economic structure required for dual-price interpretation.

\section{Modeling scope, assumptions, and notation}
\label{sec:kkt_scope}

\subsection{Scope}
We model a security- and reserve-constrained economic dispatch problem with co-optimized energy and ancillary services. The model is written at a dispatch interval resolution (canonical 5-minute intervals for SCED), but we also state a multi-interval extension to represent ESR intertemporal coupling. We abstract from unit commitment (binary commitment states) in the primal baseline, then reintroduce nonconvexities in Section~\ref{sec:kkt_nonconvex}.

\subsection{Assumptions (explicit)}
The baseline stylized model assumes:
\begin{itemize}
  \item Convex, differentiable production costs for continuous dispatch decisions.
  \item Ancillary requirements represented as minimum procurement constraints.
  \item A linearized network/security representation sufficient to define locational energy prices.
  \item ESR feasibility represented by state-of-charge (SoC) dynamics with power and energy bounds.
\end{itemize}
These assumptions are \emph{not} claims about exact ERCOT implementation; they are chosen so that KKT multipliers admit interpretable shadow prices. Departures from these assumptions are treated explicitly in Section~\ref{sec:kkt_nonconvex}.

\subsection{Notation}
Let:
\begin{itemize}
  \item $t \in \{1,\dots,T\}$ index dispatch intervals; $\Delta t$ is interval length (hours).
  \item $i \in \mathcal{G}$ index dispatchable resources; $s \in \mathcal{S}$ index settlement points (hubs/zones); $\ell \in \mathcal{L}$ index transmission elements.
  \item $k \in \mathcal{K}$ index ancillary products (e.g., REG-UP, REG-DOWN, RRS, ECRS, FRRS where applicable).
  \item $g_{i,t}$ denote energy output (MW), $r_{k,i,t}$ denote ancillary award/capability (MW), and $u_{i,t}$ denote ESR charging (MW), $d_{i,t}$ ESR discharging (MW).
  \item $\mathrm{SoC}_{i,t}$ denote state of charge (MWh).
  \item $C_i(\cdot)$ denote energy cost; $C^R_{k,i}(\cdot)$ denote reserve offer/cost.
\end{itemize}

\section{Primal formulation (stylized but explicit)}
\label{sec:kkt_primal}

We write a co-optimized dispatch as a constrained minimization. For clarity, the formulation is presented for a single interval and then extended to $T$ intervals with ESR intertemporal coupling.

\subsection{Single-interval formulation with co-optimized energy and reserves}
Consider one interval (drop $t$ subscripts). The system operator minimizes total as-offered cost:
\begin{equation}
\min_{\{g_i, r_{k,i}, u_i, d_i\}}
\sum_{i\in\mathcal{G}} C_i(g_i) + \sum_{i\in\mathcal{G}}\sum_{k\in\mathcal{K}} C^R_{k,i}(r_{k,i})
+ \sum_{i\in\mathcal{E}} C^{\text{ch}}_i(u_i,d_i),
\label{eq:primal_obj}
\end{equation}
where $\mathcal{E}\subseteq\mathcal{G}$ denotes ESRs and $C^{\text{ch}}$ can include charging/discharging bid components or degradation proxies (set to zero in the baseline; reinstated later in investor economics).

\paragraph{Energy balance.}
Let $D$ denote net demand to be served (MW). The balance constraint is:
\begin{equation}
\sum_{i\in\mathcal{G}} g_i + \sum_{i\in\mathcal{E}}(d_i - u_i) = D.
\label{eq:balance_kkt}
\end{equation}

\paragraph{Ancillary requirements.}
For each product $k$, required procurement is $R_k$:
\begin{equation}
\sum_{i\in\mathcal{G}} r_{k,i} \ge R_k, \qquad \forall k\in\mathcal{K}.
\label{eq:as_req}
\end{equation}

\paragraph{Capacity coupling between energy and reserves.}
Energy and reserves compete for headroom:
\begin{equation}
g_i + \sum_{k\in\mathcal{K}} r_{k,i} \le \overline{G}_i, \qquad \forall i\in\mathcal{G}.
\label{eq:cap}
\end{equation}

\paragraph{Reserve capability bounds.}
\begin{equation}
0 \le r_{k,i} \le \overline{R}_{k,i}, \qquad \forall i,k.
\label{eq:res_bounds}
\end{equation}

\paragraph{ESR power bounds.}
\begin{equation}
0 \le u_i \le \overline{U}_i, \qquad 0 \le d_i \le \overline{D}_i, \qquad \forall i\in\mathcal{E}.
\label{eq:esr_power}
\end{equation}

This single-interval model already permits scarcity to appear in either energy or ancillary services, depending on which constraints bind. However, it does not yet capture the defining ESR feature: intertemporal feasibility.

\subsection{Multi-interval extension with SoC dynamics}
Introduce $\mathrm{SoC}_{i,t}$ and charging/discharging decisions. For ESR $i\in\mathcal{E}$:
\begin{equation}
\mathrm{SoC}_{i,t+1} = \mathrm{SoC}_{i,t} + \eta^c_i u_{i,t}\Delta t - \frac{1}{\eta^d_i} d_{i,t}\Delta t,
\qquad t=1,\dots,T-1.
\label{eq:soc_dyn_kkt}
\end{equation}
with bounds:
\begin{equation}
\underline{\mathrm{SoC}}_i \le \mathrm{SoC}_{i,t} \le \overline{\mathrm{SoC}}_i, \qquad \forall i\in\mathcal{E},\; t.
\label{eq:soc_bounds_kkt}
\end{equation}
Optional terminal constraints may be imposed (e.g., $\mathrm{SoC}_{i,T}\ge \mathrm{SoC}_{i,1}$) if market rules or operator objectives enforce end-of-horizon feasibility. The presence or absence of terminal constraints is itself a structural determinant of scarcity manifestation.

\paragraph{Key structural point.}
Constraints \eqref{eq:soc_dyn_kkt}--\eqref{eq:soc_bounds_kkt} couple decisions across time. Thus, a binding SoC bound at time $t$ can influence prices and scarcity outcomes at time $t$ even when instantaneous energy balance \eqref{eq:balance_kkt} is not particularly stressed. This creates an additional scarcity channel that can express itself in ancillary prices when reserves depend on deliverable upward/downward flexibility.

\section{Lagrangian and KKT system}
\label{sec:kkt_kkt}

We derive KKT conditions for the convex baseline, then interpret shadow prices. For expository clarity, we present the single-interval KKT and then indicate the additional terms introduced by SoC coupling.

\subsection{Dual variables}
Associate dual variables:
\begin{itemize}
  \item $\lambda$ with energy balance \eqref{eq:balance_kkt}.
  \item $\mu_k\ge 0$ with ancillary requirements \eqref{eq:as_req}.
  \item $\nu_i\ge 0$ with capacity coupling \eqref{eq:cap}.
  \item $\alpha_{k,i}\ge 0$ and $\beta_{k,i}\ge 0$ with reserve bounds \eqref{eq:res_bounds}.
  \item $\gamma^{u}_i\ge 0$, $\delta^{u}_i\ge 0$ and $\gamma^{d}_i\ge 0$, $\delta^{d}_i\ge 0$ with \eqref{eq:esr_power}.
\end{itemize}
In the multi-interval extension, add SoC duals:
\begin{itemize}
  \item $\pi_{i,t}$ with SoC dynamics \eqref{eq:soc_dyn_kkt}.
  \item $\phi^{+}_{i,t}\ge 0$ and $\phi^{-}_{i,t}\ge 0$ with SoC bounds \eqref{eq:soc_bounds_kkt}.
\end{itemize}

\subsection{Lagrangian (single interval)}
The single-interval Lagrangian is:
\begin{align}
\mathcal{L} &=
\sum_i C_i(g_i)
+ \sum_{i,k} C^R_{k,i}(r_{k,i})
+ \sum_{i\in\mathcal{E}} C^{\text{ch}}_i(u_i,d_i)
\nonumber\\
&\quad + \lambda\left(D - \sum_i g_i - \sum_{i\in\mathcal{E}}(d_i-u_i)\right)
+ \sum_k \mu_k\left(R_k - \sum_i r_{k,i}\right)
\nonumber\\
&\quad + \sum_i \nu_i\left(g_i + \sum_k r_{k,i} - \overline{G}_i\right)
+ \sum_{i,k}\alpha_{k,i}(-r_{k,i}) + \sum_{i,k}\beta_{k,i}(r_{k,i}-\overline{R}_{k,i})
\nonumber\\
&\quad + \sum_{i\in\mathcal{E}}\gamma^u_i(-u_i) + \sum_{i\in\mathcal{E}}\delta^u_i(u_i-\overline{U}_i)
+ \sum_{i\in\mathcal{E}}\gamma^d_i(-d_i) + \sum_{i\in\mathcal{E}}\delta^d_i(d_i-\overline{D}_i).
\label{eq:lagrangian_single}
\end{align}

\subsection{Stationarity}
Assuming differentiability, the stationarity conditions are:
\begin{align}
\frac{\partial \mathcal{L}}{\partial g_i}=0 &\Rightarrow
C_i'(g_i) - \lambda + \nu_i = 0,
\label{eq:kkt_station_g}\\
\frac{\partial \mathcal{L}}{\partial r_{k,i}}=0 &\Rightarrow
\frac{\partial C^R_{k,i}}{\partial r_{k,i}} - \mu_k + \nu_i - \alpha_{k,i} + \beta_{k,i} = 0,
\label{eq:kkt_station_r}\\
\frac{\partial \mathcal{L}}{\partial u_i}=0 &\Rightarrow
\frac{\partial C^{\text{ch}}_i}{\partial u_i} + \lambda - \gamma^u_i + \delta^u_i = 0,
\label{eq:kkt_station_u}\\
\frac{\partial \mathcal{L}}{\partial d_i}=0 &\Rightarrow
\frac{\partial C^{\text{ch}}_i}{\partial d_i} - \lambda - \gamma^d_i + \delta^d_i = 0.
\label{eq:kkt_station_d}
\end{align}

\subsection{Primal feasibility, dual feasibility, complementary slackness}
Primal feasibility consists of \eqref{eq:balance_kkt}--\eqref{eq:esr_power}. Dual feasibility requires:
\[
\mu_k,\nu_i,\alpha_{k,i},\beta_{k,i},\gamma,\delta \ge 0.
\]
Complementary slackness includes, for example:
\[
\mu_k\left(R_k - \sum_i r_{k,i}\right)=0,\qquad
\nu_i\left(g_i+\sum_k r_{k,i}-\overline{G}_i\right)=0,
\]
and analogous relations for all bounds.

\subsection{Additional stationarity terms from SoC coupling}
In the multi-interval model, charging/discharging stationarity includes the SoC shadow values $\pi_{i,t}$:
\begin{align}
\frac{\partial \mathcal{L}}{\partial u_{i,t}}=0
&\Rightarrow \frac{\partial C^{\text{ch}}_{i,t}}{\partial u_{i,t}} + \lambda_t + \eta^c_i \pi_{i,t}\Delta t - \gamma^u_{i,t} + \delta^u_{i,t} = 0,
\label{eq:kkt_soc_u}\\
\frac{\partial \mathcal{L}}{\partial d_{i,t}}=0
&\Rightarrow \frac{\partial C^{\text{ch}}_{i,t}}{\partial d_{i,t}} - \lambda_t - \frac{1}{\eta^d_i}\pi_{i,t}\Delta t - \gamma^d_{i,t} + \delta^d_{i,t} = 0.
\label{eq:kkt_soc_d}
\end{align}
SoC bound multipliers $\phi^{+}_{i,t},\phi^{-}_{i,t}$ enter through the stationarity of $\mathrm{SoC}_{i,t}$, producing an intertemporal Euler-like condition:
\begin{equation}
\pi_{i,t-1} - \pi_{i,t} + \phi^{+}_{i,t} - \phi^{-}_{i,t} = 0,
\label{eq:euler_soc}
\end{equation}
which implies that when SoC bounds bind, the shadow value of energy-in-storage changes discontinuously. This is a formal expression of ``intertemporal scarcity.''

\section{Dual interpretation: where scarcity value appears}
\label{sec:kkt_dual}

\subsection{Energy price and balance shadow price}
In the convex baseline, the LMP is the shadow price of system balance. In the simple single-node representation, energy price is $\lambda$. With network constraints, $\lambda$ becomes node-specific and includes congestion components. The essential point is unchanged: $\lambda$ measures marginal system cost of serving an incremental MW of demand under security constraints.

\subsection{Ancillary service prices as requirement shadow prices}
From complementary slackness, if an ancillary constraint binds:
\[
\sum_i r_{k,i}=R_k \Rightarrow \mu_k \ge 0,
\]
and $\mu_k$ is the marginal system cost of increasing requirement $R_k$ by 1 MW. In settlement, this corresponds to the scarcity component of MCPC for product $k$ under a convex approximation.

\subsection{How ESR feasibility introduces additional scarcity channels}
Equations \eqref{eq:kkt_soc_u}--\eqref{eq:euler_soc} show that ESR decisions embed the intertemporal shadow value $\pi_{i,t}$. When SoC bounds bind, $\phi^{+}_{i,t}$ or $\phi^{-}_{i,t}$ becomes positive, perturbing $\pi_{i,t}$ and therefore the marginal conditions for charging/discharging. The consequence is that a resource may be physically incapable of providing upward reserve (or energy) even if instantaneous marginal costs appear low. In such cases, scarcity can manifest in ancillary prices ($\mu_k$) even when energy price $\lambda$ is moderate.

This yields a structural explanation for empirical patterns of the form:
\begin{quote}
High $\mathrm{MCPC}_{k}$ with modest $\lambda^{\mathrm{LMP}}$ and low $a^{\mathrm{RT}}$.
\end{quote}

\subsection{A minimal ``scarcity migration'' statement}
Define total shadow scarcity value in the convex model as the sum of binding constraint multipliers times their marginal requirement changes. If energy balance is tight, $\lambda$ rises; if reserve requirements are tight, $\mu_k$ rises. The model does not require both to rise simultaneously. Under RTC+B, the feasible set changes (through ESR deliverability and intertemporal coupling), and therefore the partition of scarcity value across $\lambda$ and $\{\mu_k\}$ can shift even holding system stress constant.

\section{Nonconvexities and what the stylized model omits}
\label{sec:kkt_nonconvex}

Real markets embed nonconvex constraints:
\begin{itemize}
  \item Unit commitment (binary on/off, minimum run times, minimum output).
  \item Start costs and no-load costs.
  \item Integer and piecewise feasibility constraints for certain services.
  \item Transmission switching, contingency constraints, and limit monitoring.
\end{itemize}

In nonconvex systems, KKT multipliers may not correspond to welfare-optimal prices, and uplift payments are used to restore incentive compatibility. Nevertheless, dual variables from convex relaxations remain informative as \emph{locally marginal scarcity indicators}, particularly for interpreting relative scarcity across products (energy vs reserves). The correct interpretation is therefore:
\begin{quote}
KKT prices are not guarantees of equilibrium welfare optimality under nonconvexities; they are diagnostic marginal values conditional on the dispatch formulation and relaxations used.
\end{quote}
This distinction matters for empirical interpretation: MCPC spikes can reflect genuine scarcity (binding constraints) or formulation-driven scarcity (e.g., tightened constraints, relaxations, or requirement changes). The report’s sensitivity registry is designed to prevent misattribution by testing stability across reasonable definitional variants.

\section{Testable implications}
\label{sec:kkt_testable}

The KKT system yields falsifiable empirical signatures. Let $\mathbb{S}_{k,q}(t)$ denote a tail exceedance indicator for product $k$ (Chapter~\ref{ch:metric_dictionary_full}), and let $a^{\mathrm{RT}}(t)$ denote the RT energy adder.

\subsection{Signature 1: Ancillary scarcity without energy scarcity}
A KKT-consistent signature is:
\[
\Pr\left(\mathbb{S}_{k,0.99}(t)=1 \;\wedge\; a^{\mathrm{RT}}(t)=0\right) > 0
\]
with elevated frequency in stress windows. This supports ``scarcity migration'' in the descriptive sense (scarcity appears in reserves rather than energy adders), without claiming welfare equivalence.

\subsection{Signature 2: Multi-product fragmentation vs coherence}
If scarcity is multi-dimensional, tail multiplicity $M_q(t)$ (Wave~2) has mass away from 1. The KKT framework predicts:
\[
\Pr(M_q(t)\ge 2) \text{ rises when multiple requirement constraints bind.}
\]

\subsection{Signature 3: Intertemporal binding reflected in episode persistence}
Binding SoC constraints imply persistent scarcity episodes (longer run-lengths) rather than isolated spikes. Thus,
\[
R_k(\tau) \text{ and } A_e \text{ rise when feasibility constraints bind intertemporally.}
\]

\subsection{Signature 4: Lead--lag alignment with DA posture}
If DA posture $Q^{DA}_k(h)$ anticipates stress effectively, then high $Q^{DA}_k(h)$ should correspond to reduced RT tail severity. If not, RT tail severity remains high despite DA posture:
\[
Q^{RT}_{0.99,k}\bigm|Z_k \uparrow \;\; \text{does not fall.}
\]
These are tested empirically using the conditioning protocols in the Findings chapters and the sensitivity registry in Appendix~\ref{app:sensitivity_wave2}.
% ----------------------------------------------------------------------

\chapter{ORDC and ASDC Economic Bridge: Equivalence and Non-Equivalence}
\label{ch:ordc_asdc}

\paragraph{Evidence rung discipline.}
Claims about \emph{price-path} patterns are admissible at Evidence Rung~R1
(Table~\ref{tab:evidence_rungs}). Claims about \emph{allocation} (dispatch,
awards, deployments) require R2--R3 evidence. Claims about feasibility channels
involving SoC/intertemporal constraints require R4 evidence and are otherwise
\textbf{not identifiable}.

This chapter addresses the central interpretive dispute raised by RTC+B: whether observed scarcity value ``migration'' into ancillary products is economically equivalent to ORDC scarcity expressed in a different location, or whether RTC+B induces a genuinely different scarcity representation with distinct welfare, reliability, and investor-return implications.

We formalize three distinct notions of ``equivalence'' to prevent category error:
(i) \emph{price-path equivalence} (identical prices for all states),
(ii) \emph{allocation/welfare equivalence} (identical dispatch and expected welfare),
and (iii) \emph{empirical indistinguishability} (observed price signatures are statistically indistinguishable under available data and measurement). These notions are not interchangeable. In particular, even if two designs are welfare-equivalent under restrictive assumptions, they may fail price-path equivalence (and vice versa), and both may fail empirical indistinguishability when settlement timing differs.\footnote{The practical significance is regulatory: parties often argue equivalence using a welfare claim but present price-path evidence, or vice versa. The report therefore treats equivalence as a set of explicitly testable propositions.}

\section{ORDC as an energy-side scarcity functional}
\label{sec:ordc_functional}

\subsection{Scalar scarcity state and an energy-side functional}
In ORDC-style pricing, scarcity is expressed primarily as an \emph{energy-side} adder, computed as a function of a scalar scarcity state $x_t$ (e.g., operating reserve level or reserve margin). Denote the (possibly piecewise) ORDC scarcity mapping by $\psi(\cdot)$:
\begin{equation}
a^{\mathrm{ORDC}}_t = \psi(x_t),
\label{eq:ordc_adder_scalar}
\end{equation}
so the scarcity-inclusive energy price is
\begin{equation}
\pi^{\mathrm{SPP}}_t = \lambda^{\mathrm{LMP}}_t + a^{\mathrm{ORDC}}_t + a^{\mathrm{RDP}}_t,
\label{eq:ordc_spp_decomp}
\end{equation}
where $\lambda^{\mathrm{LMP}}_t$ is the marginal energy component and $a^{\mathrm{RDP}}_t$ is the reliability deployment price component where applicable.\footnote{ERCOT’s ORDC implementation is defined in ERCOT’s ``Methodology for Implementing ORDC to Calculate Real-Time Reserve Price Adder'' and periodic ORDC reports. See \cite{ercot_ordc_methodology_2015,ercot_ordc_report_2022,ercot_ordc_report_2024}.}

\subsection{Economic interpretation}
In a convex formulation, $\psi(x_t)$ can be interpreted as an administrative approximation to an expected marginal reliability value, often framed as $\mathrm{LOLP}\times\mathrm{VOLL}$ under a mapping from reserves to loss-of-load risk.\footnote{ERCOT’s ORDC documentation and biennial reports discuss the conceptual link between reserve levels, LOLP, and VOLL-based valuation. See \cite{ercot_ordc_report_2022,ercot_ordc_methodology_2015}.} Operationally, ORDC compresses a multi-dimensional operational scarcity condition into a scalar state, producing a single scarcity scalar that can be applied uniformly in energy pricing.

\subsection{ORDC as a scalar shadow-price proxy}
Let $R^{\text{sys}}_t$ denote a scalar ``system reserve'' quantity that summarizes the operator’s scarcity condition. In a stylized convex program, scarcity adders can be viewed as the shadow value of tightening the scalar reserve constraint. ORDC acts as an ex ante functional approximation to that shadow value:
\[
a^{\mathrm{ORDC}}_t \approx \frac{\partial \mathcal{C}_t}{\partial R^{\text{sys}}_t},
\]
where $\mathcal{C}_t$ is short-run expected system cost (inclusive of reliability risk monetization in the ORDC design logic). This representation is coherent when (i) scarcity is truly well-captured by a scalar reserve state and (ii) reserve products are sufficiently fungible that one scalar shadow price captures the marginal reliability value.

\section{ASDC and product-specific scarcity pricing}
\label{sec:asdc_multi}

\subsection{Vector scarcity state}
RTC+B increases the salience of a different scarcity representation: scarcity is no longer adequately summarized by a single scalar $x_t$. Instead, scarcity is multi-dimensional across ancillary products, each with distinct deliverability, response, and performance requirements. Let the scarcity state be a vector:
\begin{equation}
\mathbf{x}_t = (x_{k,t})_{k\in\mathcal{K}},
\label{eq:vector_state}
\end{equation}
with product-specific scarcity mappings $\psi_k(\cdot)$ defining ancillary scarcity valuations:
\begin{equation}
a^{\mathrm{ASDC}}_{k,t} = \psi_k(x_{k,t}), \qquad \forall k\in\mathcal{K}.
\label{eq:asdc_product}
\end{equation}
In a convex co-optimization, the economically canonical interpretation of the ancillary clearing price (MCPC) for product $k$ is the shadow value of the binding product requirement:
\begin{equation}
\mu_{k,t} \equiv \frac{\partial \mathcal{C}_t}{\partial R_{k,t}},
\label{eq:shadow_mu}
\end{equation}
so that elevated $\mu_{k,t}$ reflects marginal scarcity in that specific product.

\subsection{ERCOT-specific context}
ERCOT’s AS demand curve (ASDC/ASDC-like) framing has been discussed in ERCOT materials associated with RTC and RTC+B and subsequent modifications.\footnote{See ERCOT ASDC overview materials and NPRR documentation discussing ASDC construction and changes, including the RTCBTF materials and ancillary service demand curve modifications. \cite{ercot_asdc_overview_2024,ercot_nprr1268_asdc_mod_2025}.}
At minimum, these documents establish (i) the intent to represent scarcity in ancillary services explicitly and (ii) the rule-driven relationship between scarcity curves and product procurement under the RTC framework.\footnote{This report does not assume that any particular ASDC functional form is optimal; it uses the ASDC representation to formalize what it means for scarcity to be multi-dimensional and product-specific.}

\section{Equivalence notions and definitions}
\label{sec:equivalence_defs}

We define equivalence precisely.

\subsection{Price-path equivalence}
Two designs are price-path equivalent if, for all feasible system states $\omega_t$ (load, outages, renewable output, transmission limits, etc.), they generate identical price vectors:
\[
(\pi^{\mathrm{SPP}}_t, \{\mathrm{MCPC}_{k,t}\}_{k})^{(A)} \equiv (\pi^{\mathrm{SPP}}_t, \{\mathrm{MCPC}_{k,t}\}_{k})^{(B)} \quad \forall \omega_t.
\]
This is extremely strong and typically fails when settlement changes.

\subsection{Allocation/welfare equivalence}
Two designs are welfare-equivalent if they produce the same dispatch/awards and the same expected total system cost (including any reliability valuation terms) for all states. This can hold even when price paths differ, provided transfers (uplift) reconcile incentives.

\subsection{Empirical indistinguishability}
Two designs are empirically indistinguishable if, under the measurement resolution and artifacts available, the observable distributions of key statistics (e.g., energy tail mass, MCPC tail mass, clustering rates) are statistically indistinguishable. This can fail even when welfare equivalence holds, because the \emph{observed} price proxies embed different decompositions, caps, or settlement constructs.

\section{Equivalence conditions (restrictive)}
\label{sec:equivalence_conditions}

We now state explicit conditions under which an ORDC scalar scarcity representation could be equivalent to an ASDC multi-product representation in a meaningful sense.

\subsection{Condition E1: Single active scarcity constraint}
Suppose the feasible set can be represented by a single binding scarcity constraint at the margin. Formally, assume there exists a scalar functional $R^{\text{sys}}_t$ such that all product requirements and operational constraints aggregate into:
\[
R^{\text{sys}}_t(\mathbf{R}_t,\omega_t) \ge 0
\]
and only this scalar constraint binds in scarcity. Then there exists a single marginal scarcity multiplier, and price dispersion across products is not structurally required.

\subsection{Condition E2: Perfect substitutability across ancillary products}
Assume all ancillary products are fungible in both physics and rules, so that one MW of any product contributes equally to the system’s scarcity constraint. This implies
\[
x_{k,t} \equiv x_t \quad \forall k,
\]
and therefore $\psi_k$ can be chosen such that $a^{\mathrm{ASDC}}_{k,t}$ collapses to a single scarcity scalar.

\subsection{Condition E3: No intertemporal feasibility channel}
Assume the set of feasible ancillary provision is separable across time and independent of state variables such as SoC. This eliminates the intertemporal scarcity channel emphasized in RTC+B documentation and NPRR context.\footnote{RTC+B explicitly targets improved modeling and consideration of batteries and state of charge in real time. \cite{ercot_rtcb_overview_2025,ercot_rtcb_golive_2025,ercot_nprr1186_comments_2023}.} Under this condition, a single-period marginal scarcity scalar is more defensible.

\subsection{Condition E4: Settlement and price formation neutrality}
Assume settlement does not introduce systematic distortions (caps, floors, or adders) that change the mapping from operational scarcity to observed prices. This is rarely exactly true but is included to separate operational from settlement-induced differences.\footnote{If the reporting artifacts embed adders or exclude them (as in your dashboard evidence), the empirical distinguishability problem becomes material even if the underlying dispatch logic were close. Your report’s artifact-class discipline was designed precisely to handle this.}

\subsection{Proposition (restrictive equivalence)}
\textbf{Proposition 1 (Scalar--vector equivalence under collapse conditions).}
If Conditions E1--E4 hold, then there exists a mapping $\psi(\cdot)$ such that the ORDC scalar scarcity adder $a^{\mathrm{ORDC}}_t=\psi(x_t)$ yields the same allocations and the same marginal scarcity valuation as a multi-product ASDC representation.

\emph{Proof sketch.} Under E1, only one scarcity constraint binds, so the dual system contains a single scarcity multiplier. Under E2, all product marginal values coincide at the margin, so the product multipliers collapse to that single value. Under E3, there is no intertemporal coupling that would generate distinct state-dependent shadow values for different products. Under E4, observed prices faithfully represent marginal values. Therefore the scalar mapping can be defined to match the single scarcity multiplier across all states. $\square$

\section{Why RTC+B violates equivalence conditions}
\label{sec:rtcb_violations}

The purpose of this section is not rhetorical; it is to identify which equivalence assumptions are empirically and structurally implausible in ERCOT under RTC+B, and therefore which equivalence claims require additional evidence.

\subsection{Violation of E1: multiple binding constraints are structurally plausible}
Ancillary products are defined with distinct performance obligations and operational roles; it is therefore plausible---and empirically expected in stressed conditions---that multiple product requirements bind simultaneously. This produces multiple nonzero multipliers $\mu_{k,t}$ and makes scalar collapse structurally fragile.

\subsection{Violation of E2: reserves are not perfectly substitutable}
Even in stylized reliability modeling, different reserves provide different response characteristics and deployment obligations. ERCOT’s creation and modification of new products (e.g., ECRS) is itself evidence that reserves are not treated as fungible.\footnote{ECRS is explicitly present in the RTC-era ancillary service construct and appears in ERCOT’s ASDC-related documentation and MCPC reporting. See \cite{ercot_asdc_overview_2024,ercot_nprr1268_asdc_mod_2025}.}

\subsection{Violation of E3: ESR feasibility is intertemporal and state-dependent}
RTC+B explicitly increases the operator’s modeling fidelity for batteries and SoC. The feasibility of delivering upward capability is state-dependent (bounded by SoC), and that state evolves through time. Thus the system’s feasible set is not separable across time and cannot generally be represented by a single-period scalar scarcity state.\footnote{NPRR 1186 materials and RTC+B overview documents explicitly focus on improved SoC awareness and battery modeling for reliability reasons. \cite{ercot_nprr1186_issue_page,ercot_nprr1186_comments_2023,ercot_rtcb_overview_2025}.}

\subsection{Violation of E4: observable prices can differ even when underlying scarcity is similar}
Even when underlying dispatch scarcity is comparable, settlement design changes, adders, and reporting conventions can change how scarcity appears in observed prices. This is why the report treats dashboard artifacts and settlement-grade artifacts separately and imposes ledger-to-figure traceability and sensitivity levers.

\section{Empirical discriminators: ORDC-style vs AS-driven scarcity}
\label{sec:ordc_asdc_discriminators}

The report operationalizes the ORDC vs ASDC bridge through falsifiable empirical discriminators that can be computed using your episode ledger, tail metrics, and DA posture conditioning (DAMASAGG).

\begin{figure}[ht]
    \centering
    \begin{tikzpicture}[font=\small\sffamily, scale=1.2]
        % Grid
        \draw[thick] (0,0) rectangle (6,6);
        \draw[thick] (3,0) -- (3,6);
        \draw[thick] (0,3) -- (6,3);
        
        % Axis Labels
        \node[rotate=90] at (-0.5, 3) {\textbf{Stack Geometry Signal ($H_k$)}};
        \node at (3, 6.5) {\textbf{Realized Scarcity (Price)}};
        
        \node[left] at (-0.8, 4.5) {High (Fragile)};
        \node[left] at (-0.8, 1.5) {Low (Robust)};
        \node[above] at (1.5, 6) {Present};
        \node[above] at (4.5, 6) {Absent};
        
        % Quadrant shading (limited palette; print-friendly)
        \fill[LinkBlue!18, opacity=0.45] (0,3) rectangle (3,6); % TP
        \fill[orange!18, opacity=0.45] (3,3) rectangle (6,6);  % FP
        \fill[red!14, opacity=0.45] (0,0) rectangle (3,3);     % FN
        \fill[black!6, opacity=0.65] (3,0) rectangle (6,3);    % TN

        % Quadrant 1: TP
        \node[text=LinkBlue!90!black] at (1.5, 5.2) {\textbf{Fragility Realized}};
        \node[align=center, text width=2.5cm, text=LinkBlue!90!black] at (1.5, 4.2) {(True Positive)\\Validation of Convexity};

        % Quadrant 2: FP
        \node[text=orange!70!black] at (4.5, 5.2) {\textbf{False Alarm}};
        \node[align=center, text width=2.5cm, text=orange!70!black] at (4.5, 4.2) {(False Positive)\\Supply withheld but demand muted};

        % Quadrant 3: FN
        \node[text=red!70!black] at (1.5, 2.2) {\textbf{Surprise Scarcity}};
        \node[align=center, text width=2.5cm, text=red!70!black] at (1.5, 1.2) {(False Negative)\\Mechanism failure or Exogenous Shock};

        % Quadrant 4: TN
        \node[text=black!80] at (4.5, 2.2) {\textbf{Robust Operation}};
        \node[align=center, text width=2.5cm, text=black!80] at (4.5, 1.2) {(True Negative)\\Healthy Stack, Stable Price};

    \end{tikzpicture}
    \caption{Geometry Diagnostic Matrix. Classifying intervals by the intersection of Day-Ahead Stack Concentration (Geometry) and Real-Time Scarcity. The \textbf{Fragility Realized} quadrant confirms the predictive validity of the $H_k$ metric.}
    \label{fig:geometry_matrix}
\end{figure}

\subsection{Discriminator D1: Energy-adder attenuation with ancillary tail persistence}
If scarcity is ``moved'' from energy to ancillary services, the energy-side adder incidence should fall while ancillary MCPC tail mass rises:
\[
\Delta \Pr(a^{\mathrm{RT}}_t>\tau_a) < 0
\quad \text{and} \quad
\Delta \Pr(\mathrm{MCPC}_{k,t}>\tau_k) > 0
\]
for at least one ancillary product $k$, with stability across SR-2 (tail definition) and SR-3 (window padding). This discriminator is directly testable using your ORDC/RDP adder series and SCED MCPC series.

\subsection{Discriminator D2: Product-specific clustering (dimensionality)}
Multi-dimensional scarcity predicts that scarcity episodes cluster in particular products rather than uniformly across all products. Let $A_{e,k}$ denote the product-specific cluster mass for episode $e$. Then dimensional scarcity predicts
\[
\exists k\neq k' \text{ such that } A_{e,k} \gg A_{e,k'}
\]
for a nontrivial set of episodes, after conditioning on matched stress bins. Scalar ORDC-style scarcity would predict more uniform co-movement across products (modulo offer effects).

\subsection{Discriminator D3: Tail multiplicity distribution}
Let $M_q(t)$ denote the number of products whose price exceeds their $q$-quantile threshold at time $t$ (your Wave~2 metric). Under scalar scarcity, tail exceedances tend to co-occur, pushing probability mass toward larger $M_q(t)$. Under fragmented product scarcity, mass concentrates near $M_q(t)=1$:
\[
\Pr(M_q(t)=1) \text{ elevated} \;\Rightarrow\; fragmented scarcity channels,
\]
whereas
\[
\Pr(M_q(t)\ge 2) \text{ elevated} \;\Rightarrow\; coherent multi-product scarcity or common driver.
\]
This is not a welfare conclusion; it is a structural signature of dimensionality.

\subsection{Discriminator D4: Lead--lag with day-ahead posture}
Use DAMASAGG posture metrics $Q^{DA}_k(h)$ and test whether DA scarcity anticipation dampens RT scarcity:
\[
Q^{RT}_{0.99,k}\mid Q^{DA}_k \uparrow \;\; \downarrow \quad (\text{anticipatory dampening}),
\]
versus
\[
Q^{RT}_{0.99,k}\mid Q^{DA}_k \uparrow \;\; \text{unchanged or } \uparrow \quad (\text{binding feasibility / unanticipated stress}).
\]
The second pattern is especially consistent with intertemporal feasibility constraints or sudden changes in ramp/outage conditions.

\subsection{Discriminator D5: Revenue decomposition breakpoints for ESRs}
If scarcity manifestation shifts from energy to ancillary products, ESR revenue shares shift accordingly:
\[
\frac{\Pi_{\text{AS}}}{\Pi} \uparrow, \qquad \frac{\Pi_{\text{Energy}}}{\Pi} \downarrow,
\]
conditional on comparable stress bins and charging cost proxies. This discriminator is investor-economics facing and links directly to your canonical valuation decomposition chapter.

\section{Interpretive discipline and limits}
\label{sec:ordc_asdc_limits}

Two cautions are critical.
First, observing D1--D5 does not by itself prove welfare improvement; it proves that observed scarcity manifestation differs in a manner consistent with a multi-dimensional scarcity representation.
Second, if a discriminator is not stable under the SR sensitivity grid, it must be treated as illustrative rather than evidentiary under the ledger-to-figure traceability rule.

\begin{figure}[ht]
    \centering
    \begin{tikzpicture}[font=\small\sffamily, scale=0.9]
        % --- LEFT PANEL: Healthy Stack ---
        \begin{scope}[xshift=0cm]
            \draw[->, thick] (0,0) -- (5,0) node[below] {Quantity (MW)};
            \draw[->, thick] (0,0) -- (0,5) node[left] {Price (\$/MWh)};
            \node[above] at (2.5, 5) {\textbf{(A) Healthy "Broad" Stack}};
            
            % Supply Curve (Gradual/Linear)
            \draw[thick, blue!80!black] (0,0.5) -- (1, 0.6) -- (2, 0.8) -- (3, 1.2) -- (4, 1.8) -- (4.8, 2.5);
            \node[right, blue!80!black] at (4.8, 2.5) {Supply $S(q)$};
            
            % Demand Shift
            \draw[thick, dashed] (3, 0) -- (3, 4.5) node[above] {$D_1$};
            \draw[thick, dashed] (3.5, 0) -- (3.5, 4.5) node[above] {$D_2$};
            
            % Intersection Points
            \filldraw (3, 1.2) circle (2pt);
            \filldraw (3.5, 1.5) circle (2pt);
            
            % Delta P
            \draw[<->] (3.7, 1.2) -- (3.7, 1.5);
            \node[right] at (3.7, 1.35) {$\Delta P$ (Small)};
            
            \node[align=center, text width=4cm] at (2.5, -1) {Low Concentration ($H_k < 0.2$)\\Gradual slope absorbs shock};
        \end{scope}

        % --- RIGHT PANEL: Brittle Stack ---
        \begin{scope}[xshift=7cm]
            \draw[->, thick] (0,0) -- (5,0) node[below] {Quantity (MW)};
            \draw[->, thick] (0,0) -- (0,5) node[left] {Price (\$/MWh)};
            \node[above] at (2.5, 5) {\textbf{(B) Fern "Convex" Stack}};
            
            % Supply Curve (Hockey Stick)
            \draw[thick, red!80!black] (0,0.5) -- (2.5, 0.5) -- (3, 0.6) -- (3.2, 4.0) -- (3.5, 4.8);
            \node[right, red!80!black] at (3.5, 4.8) {Supply $S(q)$};
            
            % Demand Shift (Same Magnitude)
            \draw[thick, dashed] (3, 0) -- (3, 4.5) node[above] {$D_1$};
            \draw[thick, dashed] (3.2, 0) -- (3.2, 4.5) node[above] {$D_2$};
            
            % Intersection Points
            \filldraw (3, 0.6) circle (2pt);
            \filldraw (3.2, 4.0) circle (2pt);
            
            % Delta P
            \draw[<->] (3.7, 0.6) -- (3.7, 4.0);
            \node[right] at (3.7, 2.3) {$\Delta P$ (Extreme)};
            
            \node[align=center, text width=4cm] at (2.5, -1) {High Concentration ($H_k > 0.4$)\\Feasibility Wall = Vertical Price};
        \end{scope}
    \end{tikzpicture}
    \caption{Evolution of Stack Geometry. Panel A shows a healthy stack where demand shocks result in small price moves. Panel B shows the Fern-style "Convex" stack, where a similar demand shock hits the "feasibility wall," triggering extreme vertical pricing.}
    \label{fig:stack_geometry}
\end{figure}

% ----------------------------------------------------------------------
\chapter{Counterfactual Benchmark Framework (Legacy vs RTC+B)}
\label{ch:counterfactual}

This chapter specifies the counterfactual objects that must be identified in order to claim that RTC+B improved system outcomes. Because system conditions (weather, outages, load mix, transmission constraints, renewable penetration, and operational policies) vary across time, raw pre/post comparisons are generally confounded.\footnote{This is not merely a generic econometrics warning. In power systems, ``regime'' shifts often coincide with (i) seasonal load composition changes, (ii) evolving renewable and storage fleets, and (iii) concurrent market rule changes. Consequently, unconditioned differences in tail outcomes can be dominated by composition shifts rather than design effects. See the program-evaluation framing in \cite{angrist_pischke_2009,imbens_rubin_2015} and empirical cautions in electricity market design evaluation in \cite{wolak_2003}.}

We therefore formalize the counterfactual framework in terms of: (i) regimes as treatments, (ii) outcome functionals aligned with system cost, reliability proxies, and investor returns, and (iii) identification assumptions and falsification checks. Throughout, we separate \emph{estimands} (what is being identified) from \emph{estimators} (how we compute it) to avoid the common failure mode of treating a convenient statistic as a causal estimate.\footnote{This separation is emphasized in modern causal inference and is particularly important here because ``scarcity migration'' can be descriptively true even if welfare effects are ambiguous. See \cite{imbens_rubin_2015,hernan_robins_2020}.}

\section{Define regimes}
\label{sec:cf_regimes}

\subsection{Regimes as treatments}
Let $M_0$ denote the legacy pricing/procurement regime (pre-RTC+B) and $M_1$ denote the RTC+B regime. For each interval $t$, let $D_t \in \{0,1\}$ indicate the regime in force (with $D_t=1$ post-change). We treat $D_t$ as a policy intervention that shifts the feasible set and/or settlement mapping from underlying operational states to observed prices.

To define counterfactuals, we adopt a potential outcomes representation. Let $Y_t(1)$ denote the outcome that would be observed at interval $t$ under $M_1$, and $Y_t(0)$ the outcome under $M_0$. The fundamental estimand is a regime effect:
\begin{equation}
\Delta \equiv \mathbb{E}\!\left[Y_t(1) - Y_t(0)\right].
\label{eq:ate}
\end{equation}
However, \eqref{eq:ate} is not identifiable from observational time-series without additional structure, because $Y_t(0)$ is unobserved in post intervals and $Y_t(1)$ is unobserved in pre intervals.\footnote{This is the canonical missing-counterfactual problem; see \cite{imbens_rubin_2015}.}

\subsection{Conditional estimands and covariate conditioning}
We therefore define conditional estimands holding fixed observable system conditions $X_t$:
\begin{equation}
\Delta(x) \equiv \mathbb{E}\!\left[Y_t(1) - Y_t(0)\mid X_t=x\right],
\label{eq:cate}
\end{equation}
and a marginal effect over a covariate distribution of interest:
\begin{equation}
\Delta_{\mathcal{X}} \equiv \mathbb{E}\!\left[\Delta(X_t)\right].
\label{eq:marginal_cate}
\end{equation}

Here $X_t$ is explicitly restricted to variables that are plausibly pre-treatment and measured consistently across regimes: load (or net load proxy), ramp proxy, outage proxy, weather proxy (if available), and calendar controls (hour-of-day, weekday/weekend).\footnote{A frequent failure mode is conditioning on variables that are themselves affected by the regime (post-treatment controls), which can induce bias. The report constrains $X_t$ to physically exogenous or measurement-stable controls. See \cite{angrist_pischke_2009,hernan_robins_2020}.}

\subsection{SUTVA and interference caveat (power markets)}
Standard potential-outcomes analysis assumes no interference between units (SUTVA). In power markets, this is violated in a literal sense because dispatch decisions are system-coupled. Our use of potential outcomes is therefore best interpreted as a \emph{regime-level} effect on aggregate outcomes under comparable system conditions rather than as an independent-unit causal model.\footnote{This is a known issue in causal inference on networks and equilibrium systems. In market design evaluation, it is common to proceed with regime-level comparisons while explicitly acknowledging equilibrium interference. See \cite{wolak_2003} for the broader market-power/market-design empirical discipline and \cite{imbens_rubin_2015} for discussion of interference as an assumption.}

\section{Outcome functionals}
\label{sec:cf_outcomes}

We define a vector of outcomes $\mathbf{Y}$ spanning three priorities: system cost, reliability, and investor returns. The guiding principle is that each component must be (i) computable from admissible artifacts, (ii) stable under the traceability rule, and (iii) interpretable under sensitivity levers.

\subsection{Outcome vector}
Let
\begin{equation}
\mathbf{Y}_t = \Big(
\underbrace{\Pi^{\text{Energy}}_{\text{tail}}(t),\; \Pi^{\text{AS}}_{\text{tail}}(t)}_{\text{scarcity rents / tail value}},
\underbrace{I_f(t),\; N_{\epsilon}(t)}_{\text{frequency proxies}},
\underbrace{\text{ProcureCost}^{DA}(t),\; \text{ProcureCost}^{RT}(t)}_{\text{procurement cost proxies}},
\underbrace{\Pi_{\text{ESR}}(t)}_{\text{investor return proxy}}
\Big).
\label{eq:Y_vector}
\end{equation}

\subsection{Tail-value functionals (energy vs ancillary)}
Define tail value for a series $P_t$ using a threshold family (absolute $\tau$ or quantile $Q_q$). A generic tail-exceedance indicator is:
\[
\mathbb{1}^{P}_{t}(\tau) := \mathbb{1}\{P_t>\tau\}.
\]
A minimal tail-value functional is:
\begin{equation}
\Pi_{\text{tail}}(P;\tau) := \sum_{t\in\mathcal{T}} (P_t-\tau)_+ \Delta t,
\label{eq:tail_area}
\end{equation}
and exceedance frequency is:
\begin{equation}
p(P;\tau) := \frac{1}{|\mathcal{T}|}\sum_{t\in\mathcal{T}}\mathbb{1}^{P}_{t}(\tau).
\label{eq:exceed_rate}
\end{equation}
We compute these separately for energy-side objects (e.g., $\pi^{\mathrm{SPP}}$, $a^{\mathrm{RT}}$) and ancillary prices (MCPC by product) to detect scarcity migration patterns.\footnote{The use of tail functionals is consistent with the empirical scarcity literature, where welfare and reliability risks are concentrated in high-price intervals rather than in means. See tail-risk methodology discussions in \cite{embrechts_kluppelberg_mikosch_1997}.}

\subsection{Reliability proxies}
Frequency-based proxies are computed descriptively (Chapter findings) and enter the counterfactual as outcomes \emph{conditional on matched stress bins}. The proxy choice is explicitly second-best relative to LOLE/EUE because it does not directly measure loss-of-load probability.\footnote{Frequency is an operational stability signal, not a direct adequacy metric; it can be influenced by inertia, controls, and measurement noise. The report therefore treats frequency as a descriptive operational indicator and avoids conflating it with adequacy outcomes unless augmented by disturbance and outage data. See system frequency context and reliability measurement cautions in \cite{nerc_bal_001_2,nerc_era_2023}.}

\subsection{Procurement cost proxies (DA and RT posture)}
We use DAMASAGG-derived constructs to form DA posture proxies and SCED-derived MCPC for RT procurement. These are treated as \emph{proxies} for marginal scarcity procurement costs rather than full uplift-inclusive total costs.\footnote{Total system cost evaluation would ideally include uplift and make-whole payments, plus constraint management costs. Absent comprehensive uplift datasets, we treat DA/RT procurement proxies as partial observables and explicitly label them as such. This follows standard cautions in empirical electricity market studies; see \cite{wolak_2003}.}

\subsection{Investor returns proxy}
The investor proxy $\Pi_{\text{ESR}}(t)$ is computed from a canonical ESR revenue decomposition (energy + ancillary minus charging costs and performance risk proxies) using the artifact set available. We emphasize that this is not a project finance model; it is a market-revenue proxy intended to detect regime-driven shifts in revenue composition.\footnote{A full investor return analysis would require plant-specific telemetry, award/dispatch, settlement statements, and degradation models. We treat $\Pi_{\text{ESR}}$ as a stylized ``canonical battery'' diagnostic consistent with the report’s scope discipline.}

\section{Identification strategy: difference-in-differences, matching, and event-time}
\label{sec:cf_did}

\subsection{Matched-bin estimator (primary)}
Let $\mathcal{B}$ denote a bin of matched system conditions (e.g., net load percentile band, ramp band, hour-of-day). Define the matched-bin estimator:
\begin{equation}
\widehat{\Delta}_{\mathcal{B}} := \overline{Y}_{D=1,\,X\in\mathcal{B}} - \overline{Y}_{D=0,\,X\in\mathcal{B}},
\label{eq:matched_bin}
\end{equation}
and an aggregate estimate as a weighted average over bins:
\begin{equation}
\widehat{\Delta} := \sum_{\mathcal{B}} w_{\mathcal{B}} \widehat{\Delta}_{\mathcal{B}},
\qquad \sum_{\mathcal{B}} w_{\mathcal{B}} = 1.
\label{eq:weighted_bins}
\end{equation}
Weights $w_{\mathcal{B}}$ are chosen either as (i) pre-period mass (ATT-like), (ii) pooled mass (ATE-like), or (iii) stress-weighted mass (risk-focused). The choice is a value judgment tied to the report’s priorities (system cost, reliability, investor returns).\footnote{The weighting choice changes the estimand: ATT vs ATE vs risk-weighted effects. We recommend reporting multiple weightings to avoid hidden value judgments. See \cite{imbens_rubin_2015}.}

\subsection{Parallel trends and conditional ignorability}
Identification relies on a conditional parallel trends assumption:
\[
\mathbb{E}[Y_t(0)-Y_{t'}(0)\mid X] \text{ evolves similarly across periods absent the regime change,}
\]
or equivalently that, within matched bins, regime assignment is as-if random with respect to unobserved drivers of $Y$.\footnote{In canonical DiD, the key identifying assumption is parallel trends; in matching, it is conditional ignorability. In regime-change power-market evaluation, both are approximations. See \cite{angrist_pischke_2009,abadie_2005}.}

\subsection{Event-time design (episode-conditioned)}
Because scarcity concentrates in episodes, we also use event-time panels around episode triggers. Let $\tau_e$ denote the start time of episode $e$ (from your episode ledger). Define an event-time index $\ell \in \{-L,\dots,L\}$ and outcomes $Y_{e,\ell}=Y_{\tau_e+\ell}$. The episode-conditioned profile is:
\begin{equation}
\widetilde{Y}(\ell) := \text{median}_{e\in\mathcal{E}} \, Y_{e,\ell},
\label{eq:event_profile}
\end{equation}
with dispersion bands (IQR). This is not, by itself, causal; it is a structural diagnostic that must be compared across regimes under matched stress conditions and sensitivity levers.\footnote{Event studies can be highly persuasive and highly misleading if event selection is endogenous or if pre-trends are not checked. We explicitly require pre-trend inspection and ledger traceability for any event-time plot. See \cite{angrist_pischke_2009}.}

\subsection{Placebo and falsification tests}
We pre-commit to falsification checks:
\begin{itemize}
  \item \textbf{Placebo cutoff:} apply the same estimator using a pseudo policy date in a period with no rule change.
  \item \textbf{Negative control outcome:} test an outcome theoretically unaffected by RTC+B (e.g., a product or hour range where RTC+B should not bind), if data supports it.
  \item \textbf{Pre-trend check:} within bins, verify that pre-period differences are not systematically drifting in the direction of post effects.
\end{itemize}
These checks are standard in high-stakes observational causal analysis.\footnote{Placebo and negative-control logic are core safeguards against spurious attribution; see \cite{hernan_robins_2020,angrist_pischke_2009}.}

\section{Measurement stability and artifact equivalence}
\label{sec:cf_measurement}

A regime change can alter not only market outcomes but also \emph{what} is measured in public artifacts. Therefore, stability of measurement is a first-class identification condition.

\subsection{Stable measurement requirement}
Let $Z_t$ denote the observable artifact (dashboard series, settlement-grade extract). We require that the mapping from true latent price objects to observables is stable:
\[
Z_t = g(Y_t, \omega_t) \quad \text{with the same } g(\cdot) \text{ across regimes.}
\]
If $g$ changes (e.g., adders embedded vs excluded, or product naming changes), the estimator may detect measurement change rather than economic change.\footnote{Your dashboard screenshots already show how easy it is for observers to conflate ``LMP excludes adders'' with ``scarcity disappeared.'' This is precisely why the report separates artifact classes and uses settlement-grade series where possible.}

\subsection{Reconciliation protocol}
When measurement stability is uncertain, we:
\begin{itemize}
  \item compute results using both (i) a conservative artifact subset (settlement-grade only) and (ii) an expanded subset (dashboard proxies),
  \item require consistency of sign across subsets to claim robustness,
  \item otherwise label the result as measurement-sensitive.
\end{itemize}

\section{Sensitivity to window selection and stress conditioning}
\label{sec:cf_sensitivity}

All counterfactual conclusions are sensitive to: window selection, stress conditioning, and exclusions. We formalize each.

\subsection{Window sensitivity (SR-3)}
Let $\mathcal{T}_0$ and $\mathcal{T}_1$ be pre/post windows. We test robustness under:
\[
(\mathcal{T}_0,\mathcal{T}_1) \in \mathfrak{W},
\]
where $\mathfrak{W}$ includes alternate padding, exclusion of holidays, and matched seasonal windows. Claims require stability over a nontrivial subset of $\mathfrak{W}$.

\subsection{Tail-definition sensitivity (SR-2)}
Tail results must be stable under both absolute thresholds $\tau$ and quantile thresholds $Q_q$. We define a tail family $\mathfrak{T}$ and require stability across $\mathfrak{T}$:
\[
\text{sign}\left(\widehat{\Delta}_{\tau}\right) \approx \text{sign}\left(\widehat{\Delta}_{Q_q}\right)
\quad \text{for multiple } (\tau,q).
\]

\subsection{Aggregation sensitivity (SR-1)}
We compute exceedance frequencies and tail areas under multiple aggregation operators (5-minute, 15-minute, daily max, rolling mean). A claim is robust only if it survives plausible aggregation choices consistent with ERCOT operational cadences.

\subsection{Lead--lag sensitivity (SR-4)}
We test whether outcomes depend on aligning DA posture and RT scarcity using alternate lead-lag structures. Misalignment can induce spurious non-effects or spurious amplification.

\subsection{Episode-threshold sensitivity (SR-5)}
Because episodes are defined by thresholds, we require that episode-conditioned results are not artifacts of a single threshold choice. We report key outcomes across a grid of episode definitions and require qualitative stability.

\subsection{Robustness criterion (decision rule)}
A claim of ``improvement'' is classified as robust only if:
\begin{itemize}
  \item the sign is stable across a substantial subset of SR-1--SR-5,
  \item the magnitude is not dominated by a small number of episodes (checked via leave-one-episode-out),
  \item falsification tests do not show comparable ``effects'' in placebo periods.
\end{itemize}
Otherwise, claims are labeled as descriptive or hypothesis-generating only.

% ----------------------------------------------------------------------


\section{Threats to Validity and Failure Modes}
\label{sec:cf_validity}

This section enumerates the principal threats to credible attribution in a regime-change evaluation of RTC+B. The objective is operational: (i) specify the assumption each estimator relies on, (ii) describe concrete violation patterns in ERCOT data, and (iii) prescribe diagnostics and sensitivity levers that must be satisfied before the sign or magnitude of $\widehat{\Delta}$ is interpreted as design-induced.

\subsection{Construct validity: stable measurement across regimes}
\label{subsec:construct_validity}

Let $Y_t$ denote the latent economic object of interest (e.g., the scarcity component of marginal conditions) and let $Z_t$ denote the observable artifact-derived proxy (dashboard series, settlement extract, or a constructed composite). Construct validity requires a regime-invariant measurement mapping
\begin{equation}
Z_t = g(Y_t,\omega_t) + \varepsilon_t,
\qquad g(\cdot)\ \text{stable over}\ M_0,M_1,
\label{eq:construct_mapping}
\end{equation}
where $\omega_t$ is the physical/operational system state and $\varepsilon_t$ is measurement noise.

\paragraph{CV-1: Artifact drift (decomposition changes).}
If the mapping $g(\cdot)$ changes (e.g., adders embedded vs excluded; product naming conventions; revised dashboards), then $\widehat{\Delta}$ can identify a reporting change rather than an economic change. This is why the report (i) distinguishes dashboard vs settlement-grade artifacts, (ii) computes core statistics under both a conservative artifact set and an expanded set, and (iii) treats cross-artifact disagreements as measurement-sensitive (not as economic evidence). These safeguards operationalize standard construct-validity discipline for empirical policy evaluation.

\paragraph{CV-2: Proxy substitution (frequency vs adequacy).}
Frequency-derived proxies $(I_f,N_{\epsilon})$ are operating reliability indicators, not adequacy outcomes. If a conclusion is framed as ``adequacy improved'' but relies only on frequency proxies, the construct shifts midstream. The report therefore restricts frequency-based conclusions to operational descriptives unless adequate linkage artifacts (disturbance logs, load shed, forced outages, reserve deployments) are incorporated.

\subsection{Internal validity: confounding, support, and equilibrium interference}
\label{subsec:internal_validity}

Internal validity concerns whether $\widehat{\Delta}$ can be interpreted as a regime effect rather than confounding. In this report, identification is attempted via matched-bin comparisons and DiD-style logic, which rely on conditional parallel trends and/or conditional ignorability \cite{angrist_pischke_2009,imbens_rubin_2015}.

\paragraph{IV-1: Non-parallel trends within bins.}
Even after conditioning on $X_t$ (load/net-load proxy, ramps, outage proxy, calendar controls), unobserved drivers may evolve differently across pre and post windows (fleet composition, bidding behavior, transmission constraints, operating policy). This violation manifests as within-bin pre-trends: the ``effect'' begins prior to the policy date in placebo windows. The report therefore requires pre-trend inspection, placebo cutoffs, and negative controls where feasible \cite{angrist_pischke_2009}.

\paragraph{IV-2: Support failure (overlap).}
Matching requires overlap of covariate support:
\[
\mathcal{S}_0 := \{x:\Pr(X_t=x\mid M_0)>0\},\qquad
\mathcal{S}_1 := \{x:\Pr(X_t=x\mid M_1)>0\}.
\]
If $\mathcal{S}_0\cap\mathcal{S}_1$ is small, the estimator extrapolates beyond the data. The report therefore (i) reports match rates by bin, (ii) restricts inference to bins with adequate overlap, and (iii) labels non-overlap bins as out-of-scope for attribution.

\paragraph{IV-3: Post-treatment conditioning.}
Conditioning on variables influenced by RTC+B (e.g., awards/deployments that respond endogenously to the regime) can induce bias. The report restricts $X_t$ to plausibly pre-treatment or exogenous controls and treats operator-response variables as outcomes or mechanisms rather than as conditioning variables \cite{hernan_robins_2020}.

\paragraph{IV-4: Equilibrium interference and system coupling.}
Power-market outcomes are system-coupled; literal ``no interference'' is false. We therefore interpret counterfactual claims as regime-level effects on aggregated or episode-conditioned outcomes rather than unit-level causal effects. This is standard in empirical market design evaluation, where equilibrium feedback is acknowledged and addressed via aggregate diagnostics and robustness \cite{wolak_2003}.

\subsection{External validity: generalization beyond the evaluated stress distribution}
\label{subsec:external_validity}

Even if internal validity is plausible for the evaluated windows, extrapolation to different seasons or future years (e.g., 2030--2035) requires additional assumptions. ERCOT’s evolving resource mix, large-load penetration, and ancillary requirement logic can change the mapping from physical stress to prices. Accordingly, the report treats $\widehat{\Delta}$ as local to the evaluated stress distribution and uses scenario ensembles for forward-looking claims rather than a single extrapolated point estimate.

\subsection{Interpretation guardrail: under-determination and admissible conclusions}
\label{subsec:underdetermination}

The same descriptive pattern can be consistent with multiple causal stories. For example, a decline in energy tail mass with increased MCPC tails could reflect intended scarcity migration, changes in requirements, strategic bidding, or measurement decomposition. The report therefore (i) labels mechanistic claims as hypotheses, (ii) requires stability across SR levers and falsification checks to elevate claims, and (iii) assigns evidentiary tiers (Tier~A/B/C) rather than a single verdict.

\section{Influence and Episode-Leverage Diagnostics}
\label{sec:cf_influence}

Scarcity outcomes are concentrated in relatively few episodes; regime comparisons can therefore be dominated by a small subset of events. This section defines influence diagnostics that quantify episode leverage and prevent inadvertent cherry-picking.

\subsection{Episode-indexed decomposition}
Let $\mathcal{E}$ index episodes from the episode ledger. For any episode-conditioned estimand, define an episode-level decomposition:
\[
\widehat{\Delta} = \sum_{e\in\mathcal{E}} w_e \widehat{\Delta}_e,
\qquad \sum_{e\in\mathcal{E}} w_e = 1,
\]
where $\widehat{\Delta}_e$ is the episode-specific post--pre difference under the chosen matching and $w_e$ are pre-registered weights (uniform, duration-weighted, or severity-weighted).

\subsection{Leave-one-episode-out stability}
Define the leave-one-episode-out (LOEO) estimator:
\begin{equation}
\widehat{\Delta}^{(-e)} := \sum_{e'\in\mathcal{E}\setminus\{e\}} \frac{w_{e'}}{1-w_e}\widehat{\Delta}_{e'}.
\label{eq:loeo}
\end{equation}
Define the episode influence:
\begin{equation}
\mathsf{Infl}(e) := \widehat{\Delta} - \widehat{\Delta}^{(-e)}.
\label{eq:influence}
\end{equation}
A claim is \emph{episode-robust} if (i) $\text{sign}(\widehat{\Delta}^{(-e)})$ is invariant for all but a small pre-registered fraction of episodes and (ii) $\max_e |\mathsf{Infl}(e)|$ is small relative to $|\widehat{\Delta}|$.

\subsection{Episode bootstrap for dependence-consistent uncertainty}
Intervals within episodes are dependent; naive standard errors understate uncertainty. We therefore use an episode bootstrap (or block bootstrap) for uncertainty bands: resample episodes with replacement, recompute $\widehat{\Delta}^\ast$, and report percentile bands. This aligns uncertainty quantification with the dependence structure induced by operational stress clustering.

\subsection{Artifact integrity recheck for high-leverage episodes}
Episodes with extreme $|\mathsf{Infl}(e)|$ are subject to an elevated audit: (i) verify artifact completeness, (ii) check timestamp alignment and timezone consistency, (iii) confirm no missingness spikes, and (iv) verify that the episode definition remains stable under SR-5 threshold perturbations. High leverage without audit is not admissible as evidence.

\section{Partial Identification and Robust Sign Bounds}
\label{sec:cf_partial_bounds}

Even after matching and falsification checks, unobserved confounding can remain. This section adds partial identification tools: sensitivity analysis for hidden bias and sign-robust bounds under explicit, interpretable assumptions.

\subsection{Sensitivity to hidden bias (Rosenbaum \texorpdfstring{$\Gamma$}{Gamma})}
\label{subsec:rosenbaum}

Consider matched comparisons within bins. Suppose that within a matched pair $(i,j)$, the odds of being in the post regime can differ by at most a factor $\Gamma\ge 1$ due to an unobserved covariate $U$:
\begin{equation}
\frac{1}{\Gamma} \le
\frac{\Pr(D_i=1\mid X_i,U_i)/\Pr(D_i=0\mid X_i,U_i)}
     {\Pr(D_j=1\mid X_j,U_j)/\Pr(D_j=0\mid X_j,U_j)}
\le \Gamma.
\label{eq:rosenbaum_gamma}
\end{equation}
As $\Gamma$ increases from 1 upward, one evaluates how large $\Gamma$ must be for the inference (or sign stability) to break. In our context, $\Gamma$ can be interpreted as the strength of an unmeasured driver (e.g., an unobserved constraint pattern severity) that changes the likelihood of an interval appearing in the post window relative to the pre window, even after conditioning on $X$. This is the standard observational-study sensitivity logic \cite{rosenbaum_2002}.

\subsection{Worst-case and stress-test bounds (Manski-style)}
\label{subsec:manski_bounds}

When ignorability is doubtful, one can bound effects rather than point-identify them. For bounded outcomes (including indicators, where $Y\in[0,1]$), worst-case partial identification is often extremely wide. For engineering relevance, the report therefore implements a transparent \emph{stress-test} bound: we ask how large an unobserved mean shift $b$ (within matched bins) would need to be to reverse the sign of $\widehat{\Delta}$. If $|\widehat{\Delta}|$ remains larger than the largest plausibly defensible $b$ (as justified by observed within-bin variability and documented measurement stability checks), then the sign claim is treated as robust to a broad class of hidden-bias mechanisms. The logic is aligned with the partial identification viewpoint: when point identification is not credible, the object of interest becomes sign stability under bounded deviations \cite{manski_1990,manski_2003}.

For indicator outcomes (exceedance rates), the sign-reversal condition is especially transparent. If the observed difference is $d=\widehat{p}_1-\widehat{p}_0$, then any sign flip requires unobserved selection effects of magnitude at least $|d|$ on the probability scale, which can be compared directly to observed dispersion of exceedance rates across matched bins.

\subsection{Monotone stress selection and conservative direction}
A practically meaningful restriction is a monotone stress assumption: within matched bins, unobserved factors make post intervals weakly more stressed (or weakly less stressed) than pre intervals. If post is weakly more stressed, then any observed reduction in scarcity tails is conservative evidence of improvement (stress bias would work against the observed direction). Conversely, if post is weakly less stressed, observed improvements may be spurious. The report therefore reports results both overall and within stress-conditioned subsets (Fern-like days and non-Fern days), and it treats sign disagreements across these subsets as evidence against a generalized improvement claim.

\subsection{Ratio outcomes and denominator protection}
For ratio outcomes (e.g., revenue share $\Pi_{\mathrm{AS}}/\Pi$), denominator instability can generate spurious shifts. We therefore (i) bound numerator and denominator separately under matched supports and (ii) report implied ratio bounds:
\[
\frac{\underline{\Pi}_{\mathrm{AS}}}{\overline{\Pi}}
\le
\frac{\Pi_{\mathrm{AS}}}{\Pi}
\le
\frac{\overline{\Pi}_{\mathrm{AS}}}{\underline{\Pi}}.
\]
This prevents ``composition shifts'' driven by small-denominator artifacts.

\subsection{Evidentiary tier rule}
We classify counterfactual claims into tiers:
\begin{itemize}
  \item \textbf{Tier A (robust):} stable across SR grid; passes falsification; LOEO-stable; and remains qualitatively intact under moderate hidden-bias sensitivity (e.g., $\Gamma\ge 1.5$).
  \item \textbf{Tier B (qualified):} stable across SR grid and LOEO, but sensitive to modest hidden bias or measurement subset.
  \item \textbf{Tier C (descriptive):} descriptive patterns documented with ledger traceability, but attribution not supported.
\end{itemize}
This tiering is enforced throughout to avoid over-claiming from observational evidence.


% ----------------------------------------------------------------------
\chapter{Day-Ahead Supply Geometry and Administrative Scarcity}
\label{ch:da_geometry}

This chapter introduces a \emph{bid-side} measurement layer that is strictly stronger than
price-only evidence: day-ahead (DA) ancillary service (AS) \emph{offer} and \emph{clear}
artifacts reveal the thickness, slope, and potential brittleness of product-specific
supply. The goal is not to infer intent, but to (i) separate \emph{physical} scarcity
from \emph{administrative or qualification} scarcity, and (ii) provide pre-registered
conditioning variables for the Wave~3 counterfactual diagnostics.

The chapter is written to be \emph{ERCOT-grade skeptical}: bid-side geometry is treated
as an \emph{operational constraint proxy}, not a direct proof of market power.
Strategic behavior is handled later in Chapter~\ref{ch:gaming_full} via explicit
threat models and detection rules.

\section{Settlement-grade DA artifacts and admissible inferences}
The empirical substrate for this chapter is the set of ERCOT DA AS artifacts that
separately report: (i) aggregated offer curves by product, (ii) cleared awards by product,
and (iii) self-arranged volumes where applicable.\footnote{
The specific file classes used here include aggregated DA offer exports (e.g.,
REGUP, REGDN, NSPIN/NSPNM) and cleared DA AS exports, along with self-arranged
ECRS artifacts.
These are operational/settlement-grade exports used to construct bid-side supply geometry
on the DA horizon.
}

\paragraph{Evidence rung.}
Statements in this chapter are restricted to Evidence Rung~R2 (Table~\ref{tab:evidence_rungs}):
prices plus DA awards/offer geometry. Any statement requiring deployments or SoC/telemetry is
labeled explicitly as not identifiable at R2.

\paragraph{Evidence rung.}
Statements in this chapter are restricted to Evidence Rung~R2 (Table~\ref{tab:evidence_rungs}):
prices plus DA awards/offer geometry. Any statement requiring deployments or SoC/telemetry is
labeled explicitly as not identifiable at R2.

Admissible inferences are limited to statements of the form:
\begin{itemize}
  \item \textbf{Thickness:} whether marginal supply is dense or thin at the top of the curve,
  \item \textbf{Posture:} whether DA procurement is tight/adequate/long relative to requirement,
  \item \textbf{Administrative masking:} the extent to which self-arrangement changes the observed
  cleared posture for a given product,
  \item \textbf{Empirical conditioning:} how RT scarcity outcomes differ conditional on DA posture
  and bid-side thickness, holding system conditions fixed.
\end{itemize}
We explicitly do \emph{not} infer intent, unilateral market power, or welfare effects from
offer geometry alone.\footnote{
This separation parallels the standard distinction between \emph{detecting} price formation
patterns and \emph{attributing} them to strategic conduct.
See, e.g., canonical discussions of market power diagnostics in electricity markets
\citep{wolak_2003}.
}

\section{Notation: DA supply objects by product}
Fix an AS product $k$ and operating day $d$. Let the (aggregated) DA offer curve be
represented as a step function mapping cumulative quantity to marginal offer price:
\[
S_{k,d}(q) \;=\; \inf\{p:\; Q_{k,d}(p)\ge q\},
\]
where $Q_{k,d}(p)$ is the cumulative quantity offered at prices at or below $p$.

Let $R^{DA}_{k,d}$ denote the DA requirement for product $k$ on day $d$, and let
$Q^{DA,\mathrm{clr}}_{k,d}$ denote the cleared DA quantity. When self-arrangement is permitted,
let $Q^{DA,\mathrm{self}}_{k,d}$ denote the self-arranged quantity. We define the \emph{effective}
DA posture quantity:
\[
Q^{DA,\mathrm{eff}}_{k,d} \;:=\; Q^{DA,\mathrm{clr}}_{k,d} + Q^{DA,\mathrm{self}}_{k,d}.
\]

\section{Posture and tightness metrics}
\subsection{Tightness ratio}
The primary posture metric is the tightness ratio:
\[
\theta_{k,d} \;:=\; \frac{Q^{DA,\mathrm{eff}}_{k,d}}{R^{DA}_{k,d}}.
\]
By construction, $\theta_{k,d}\approx 1$ indicates knife-edge procurement, while
$\theta_{k,d}<1$ indicates that observed DA posture is \emph{short} of requirement
(\emph{administrative scarcity} under the DA rules), and $\theta_{k,d}>1$ indicates a long posture.

\subsection{Discrete posture classes}
For analysis and episode conditioning we discretize posture into:
\[
\theta_{k,d}\in
\begin{cases}
\text{tight} & \text{if } \theta_{k,d}\in[1-\delta,1+\delta],\\
\text{short} & \text{if } \theta_{k,d}<1-\delta,\\
\text{long}  & \text{if } \theta_{k,d}>1+\delta,
\end{cases}
\]
where $\delta$ is pre-registered (default $\delta=0.05$) and stress-tested under SR levers.

\section{Supply thickness and tail elasticity}
The central question for post-RTC+B scarcity migration is not only whether DA posture is tight,
but whether the \emph{marginal} MW is supported by a thick competitive margin or a thin supply tail.

\subsection{Top-of-curve thickness}
Define a top-of-curve window by a quantile of cumulative quantity:
\[
\mathcal{Q}_{k,d}(\eta) := \{q:\; q \in [\eta R^{DA}_{k,d},\, R^{DA}_{k,d}]\}, \qquad \eta\in(0,1).
\]
The \emph{tail thickness} is the quantity mass offered within an upper price band around the
marginal clearing price $p^\star_{k,d} := S_{k,d}(R^{DA}_{k,d})$:
\[
T_{k,d}(\Delta p) := Q_{k,d}(p^\star_{k,d}) - Q_{k,d}(p^\star_{k,d}-\Delta p).
\]
Small $T_{k,d}(\Delta p)$ implies that small shocks in requirement or feasibility can push the
marginal price sharply upward.

\subsection{Discrete-slope elasticity proxy}
Because ERCOT offer curves are stepwise, we adopt a discrete slope proxy near the margin:
\[
\epsilon_{k,d} \;:=\; \frac{\Delta q}{\Delta p}\Bigg|_{\text{near }p^\star_{k,d}},
\]
computed over the smallest band that contains at least $m$ MW (default $m=25$ MW) and a positive
price increment.\footnote{
This choice avoids division by zero on flat steps and makes $\epsilon_{k,d}$ comparable across products.
Sensitivity is handled under SR levers by varying $(m,\Delta p)$.
}

Interpretation is deliberately conservative:
\begin{itemize}
  \item low $\epsilon_{k,d}$ indicates \emph{brittle} supply near the margin,
  \item high $\epsilon_{k,d}$ indicates \emph{elastic cushion}.
\end{itemize}

\begin{figure}[ht]
    \centering
    \begin{tikzpicture}[font=\small\sffamily]
        % Axis
        \draw[->, thick] (0,0) -- (10,0) node[right] {Time ($t$)};
        \draw[->, thick] (0,0) -- (0,5) node[above] {System Load (MW)};

        % Threshold Line
        \draw[dashed, thick, gray] (0,3.5) -- (10,3.5) node[right, black] {Stress Threshold ($Q_{0.90}$)};

        % Load Curve (smooth approximation)
        \draw[thick, black] plot [smooth, tension=0.7] coordinates {
            (0,1.5) (2,2.0) (3,3.0) (4,4.2) (6,4.5) (7,3.2) (8,2.5) (9.5,2.0)
        };

        % Shaded Stress Interval
        \fill[gray, opacity=0.2] (3.4,0) rectangle (6.8, 5);
        \draw[<->, thick] (3.4, 4.8) -- (6.8, 4.8) node[midway, above] {Stress Packet Duration};
        
        % Annotations
        \draw[dotted] (3.4, 3.8) -- (3.4, 0) node[below] {$t_{onset}$};
        \draw[dotted] (6.8, 3.8) -- (6.8, 0) node[below] {$t_{end}$};
        
        % Covariate Callouts
        \node[align=left, fill=white, draw=black, thin] at (2, 4.2) {
            \textbf{Triggers:}\\
            - Temp $< 32^\circ$F\\
            - Net Load Ramp $> \Delta^*$
        };
        
        % Explicit "No Price" Label
        \node[align=center, fill=white, text=gray] at (8.5, 1) {
            \textit{Price Independent}\\
            \textit{Selection}
        };

    \end{tikzpicture}
    \caption{Price-Blind Stress Packet Illustration. The episode is defined strictly by physical covariates (Load, Weather) crossing pre-registered thresholds, decoupling event selection from financial outcomes.}
    \label{fig:stress_packet}
\end{figure}

\section{Administrative scarcity vs physical scarcity: a decomposition}
A common failure mode in multi-product scarcity design is to treat high prices as evidence of
physical shortfall even when the shortfall is administrative (qualification, telemetry, or product rules).
To guard against this, we pre-commit to decompositions that do not rely on narrative.

\subsection{Cleared-vs-offered gap}
Define the offered-to-cleared gap at a reference price band:
\[
\Delta^{\mathrm{offer}}_{k,d}(\bar p) \;:=\; Q_{k,d}(\bar p) - Q^{DA,\mathrm{clr}}_{k,d}.
\]
If $\Delta^{\mathrm{offer}}_{k,d}(\bar p)$ is large at economically reasonable $\bar p$ while subsequent
real-time (RT) MCPC exhibits heavy tails, the episode is flagged as a \emph{qualification/administrative}
scarcity candidate, pending additional evidence.\footnote{
This is a flag for subsequent chapters, not a conclusion. True attribution requires
award/telemetry constraints and, ideally, participant-level qualification data.
}

\subsection{Self-arranged masking}
Define the masking ratio:
\[
\text{Mask}_{k,d} \;:=\; \frac{Q^{DA,\mathrm{self}}_{k,d}}{R^{DA}_{k,d}}.
\]
High masking can reduce observable cleared posture while leaving the system operationally reliant
on self-arranged performance. In later chapters we treat high $\text{Mask}_{k,d}$ as a fragility
indicator when paired with tail dependence in RT scarcity outcomes.

\begin{figure}[ht]
    \centering
    \begin{tikzpicture}[font=\sffamily\small, scale=0.9, thick]
        % --- ENERGY TANK ---
        \draw[rounded corners] (0,0) rectangle (3,4);
        \fill[blue!10] (0,0) rectangle (3,2); % Water level
        \draw[dashed] (0,2) -- (3,2);
        \node[above] at (1.5,4) {\textbf{Energy Market}};
        \node at (1.5,1) {Price $\lambda_t$};
        
        % --- AS TANK ---
        \draw[rounded corners] (7,0) rectangle (10,4);
        \fill[red!10] (7,0) rectangle (10,3.5); % High level
        \draw[dashed] (7,3.5) -- (10,3.5);
        \node[above] at (8.5,4) {\textbf{Ancillary Services}};
        \node at (8.5,1.75) {Price $\mu_t$};
        
        % --- PIPE & VALVE ---
        \draw (3,0.5) -- (4.5,0.5);
        \draw (3,1.5) -- (4.5,1.5);
        \draw (5.5,0.5) -- (7,0.5);
        \draw (5.5,1.5) -- (7,1.5);
        
        % The Valve Mechanism
        \node[circle, draw, fill=white, minimum size=1cm] (V) at (5,1) {};
        \draw (4.8, 0.8) -- (5.2, 1.2); \draw (4.8, 1.2) -- (5.2, 0.8); % X mark
        \node[below=0.6cm] at (V) {\textbf{SoC Constraint}};
        
        % --- SHADOW PRICE SIGNAL ---
        % This represents the mathematical dual variable
        \draw[->, red!70!black, line width=1.5pt] (V) -- (7, 3.5) node[midway, sloped, above] {Shadow Price $\nu_t$};
        
        % Logic Flow
        \node[draw, align=left, font=\scriptsize, fill=white] at (5, -1.5) {
            \textbf{Mechanism:} \\
            1. SoC Binds ($St=0$)\\
            2. Valve Closes (Separation)\\
            3. $\nu_t$ spikes AS Price ($\mu_t$)
        };
        
        % Pressure Arrows
        \draw[->] (1.5, 2.2) -- (1.5, 2.8) node[above, font=\scriptsize] {Low $\Delta P$};
        \draw[->, red] (8.5, 3.7) -- (8.5, 4.5) node[above, font=\scriptsize] {Extreme $\Delta P$};

    \end{tikzpicture}
    \caption{\textbf{The Feasibility Valve.} A schematic representation of Proposition 13.5.7. When the State-of-Charge constraint binds (the valve closes), the intertemporal shadow price $\nu_t$ isolates the Ancillary market, pressurizing prices ($\mu_t$) without affecting the Energy market ($\lambda_t$).}
    \label{fig:hydraulic_valve}
\end{figure}

\section{How this chapter feeds Wave~3 identification}
The Wave~3 counterfactual diagnostics require overlap/common support in the joint distribution of
conditions. DA posture and supply thickness are added to the conditioning set:
\[
X_t \leftarrow \bigl(X_t,\; \theta_{k,d(t)},\; \epsilon_{k,d(t)},\; T_{k,d(t)}(\Delta p)\bigr),
\]
where $d(t)$ maps an RT interval $t$ to its operating day. The restriction-to-common-support rule
in Section~\ref{sec:wave3_overlap} is therefore strengthened: we no longer compare ``Fern-like''
days across regimes without confirming that the DA bid-side geometry is comparable.\footnote{
This is the operational analog of avoiding extrapolation beyond observed support in causal inference
\citep{rosenbaum_rubin_1983,imbens_rubin_2015,manski_2003}.
}

\section{Ledger-to-figure traceability for DA geometry}
Every DA-geometry figure or statistic must declare:
(i) the operating-day set $d\in\mathcal{D}$,
(ii) the product set $k\in\mathcal{K}$,
(iii) the posture thresholds $(\delta,\eta,m,\Delta p)$,
and (iv) the episode linkage mapping $t\mapsto d(t)$ used in downstream conditioning.
This is enforced under the traceability rule in Section~\ref{sec:traceability}.

\section{False positives (the ``dog that didn't bark'')}
\label{sec:geometry_false_positives}

Because Day-Ahead (DA) stack geometry metrics (e.g., $\theta_{k,d}$, $\epsilon_{k,d}$) are intended
as \emph{leading indicators} of feasibility stress, we explicitly test for false positives:
operating days exhibiting brittle/convex DA geometry but \emph{not} exhibiting Real-Time (RT)
scarcity outcomes.

Concretely, for each product $k$ we construct a high-geometry day set
\(\mathcal{D}^{\mathrm{high}}_k\) (e.g., top-decile $\theta_{k,d}$ and/or bottom-decile
$\epsilon_{k,d}$ under the SR grid) and report:
\begin{itemize}
  \item The share of \(d\in\mathcal{D}^{\mathrm{high}}_k\) with no RT MCPC tail events,
  \item The conditional distribution of stress covariates on those non-scarcity days,
  \item Narrative-excluding explanations (e.g., forecast miss reversal; thermal availability
  recovery; congestion regime changes) only \emph{after} the tabulations.
\end{itemize}

These ``non-events'' are reported alongside Fern packet results to distinguish:
(i) geometry-as-mechanism from (ii) geometry-as-forecast-symptom.


\chapter{Structural Validation: Measurement, Support, and Feasibility Channels}
\label{ch:wave3}

\paragraph{Evidence rungs.}
This chapter uses multiple evidence layers and therefore cites the applicable rung
(Table~\ref{tab:evidence_rungs}) at each inference point: price-object invariance
checks are R1; DA posture/geometry conditioning is R2; any deployment-linked
validation is R3; and any SoC/intertemporal feasibility attribution is reserved
for R4 and is otherwise treated as not identifiable.

\section{Measurement Invariance Across Pricing Regimes}
\label{sec:wave3_invariance}

Any comparison across market design regimes requires that observed variables
retain a stable semantic interpretation across regimes.\footnote{
This requirement parallels ``measurement invariance'' in econometrics and
psychometrics, where changes in observed distributions must not arise from
redefinition of the measured construct itself.
See \citep{angrist_pischke_2009} (ch.~3) and \citep{imbens_rubin_2015}.
}

RTC+B alters settlement mechanics and scarcity allocation, but it does not
purport to redefine the economic meaning of real-time energy prices or
ancillary service capacity prices. This section formally verifies that claim.

\subsection{Energy-side price invariance}

Let $\lambda^{\mathrm{LMP}}_{t,n}$ denote the real-time locational marginal price
excluding reliability adders, $\pi^{\mathrm{SPP}}_{t,z}$ the settlement point
price, and $a^{\mathrm{RT}}_t$ the real-time reliability deployment price adder.
ERCOT documentation implies the identity:
\begin{equation}
\pi^{\mathrm{SPP}}_{t,z}
=
\lambda^{\mathrm{LMP}}_{t,z} + a^{\mathrm{RT}}_t,
\end{equation}
where $a^{\mathrm{RT}}_t$ is system-wide and independent of location.

We verify invariance by testing:
\begin{enumerate}
  \item Algebraic consistency of the identity across all hubs and load zones,
  \item Absence of regime-dependent truncation or rebasing of $a^{\mathrm{RT}}_t$,
  \item Stability of zero-adder periods under non-scarcity conditions.
\end{enumerate}

Violations would imply that post-RTC+B prices embed ancillary scarcity through
energy settlement, confounding all subsequent attribution.
No such violations are observed in settlement-grade artifacts.\footnote{
See ERCOT Real-Time Settlement Point Price displays and historical LMP exports
\citep{ercot_rtm_spp_display_pdf,ercot_systemwide_prices_pdf}.
}

\subsection{Ancillary service price invariance}

Ancillary Service Market Clearing Prices for Capacity (MCPCs) represent the
shadow price of satisfying product-specific reserve constraints in SCED.
RTC+B changes the timing and interaction of these constraints, not their
economic interpretation.

We confirm:
\begin{itemize}
  \item Unit consistency (USD/MW),
  \item Product continuity (Reg-Up, Reg-Down, ECRS),
  \item Absence of settlement rule changes that would redefine MCPC meaning.
\end{itemize}

Measurement invariance is therefore treated as satisfied for all price objects
used by the end of the analysis.

\section{Overlap and Common Support Diagnostics}
\label{sec:wave3_overlap}

Counterfactual reasoning requires overlap in the joint distribution of system
conditions across regimes.\footnote{
This is the ``common support'' or ``overlap'' condition in causal inference:
for any conditioning state used to compare regimes, both regimes must place
positive probability mass on that state
\citep{rosenbaum_rubin_1983,imbens_rubin_2015}.
}
Absent overlap, counterfactual contrasts rely on extrapolation rather than
comparative evidence.

Because the implementation of RTC+B coincides with secular changes in ERCOT’s
system (load growth, renewable penetration, ESR participation, weather
volatility), overlap cannot be assumed a priori. Instead, it must be explicitly
diagnosed, enforced, and—where violated—handled through partial identification
rather than implicit modeling assumptions.

\subsection{Conditioning variables}
\label{subsec:overlap_conditioning}

Overlap is assessed with respect to a vector of observables available at SCED or
settlement resolution:
\begin{itemize}
  \item Net load (or a demand proxy where net load is unavailable),
  \item Ancillary service requirements by product,
  \item ORDC adder activation state and magnitude,
  \item Fern-style stress indicators combining load level, reserve tightness,
  and persistence.
\end{itemize}

Let $X_t \in \mathbb{R}^p$ denote this conditioning vector at interval $t$.
All counterfactual comparisons in Chapter~\ref{ch:counterfactual} are implicitly
conditional on $X_t$.

Two clarifications are critical:
\begin{enumerate}
  \item Conditioning variables are chosen for \emph{operational relevance}, not
  statistical convenience.
  \item Conditioning is performed at the finest common temporal resolution to
  avoid spurious aggregation-induced overlap.
\end{enumerate}

\subsection{Support diagnostics}
\label{subsec:support_diagnostics}

Define the empirical supports:
\[
\mathcal{S}_0 := \text{supp}(X \mid M_0),
\qquad
\mathcal{S}_1 := \text{supp}(X \mid M_1),
\]
where $M_0$ denotes the legacy regime and $M_1$ the RTC+B regime.

The admissible comparison set is the intersection:
\[
\mathcal{S}_\cap := \mathcal{S}_0 \cap \mathcal{S}_1.
\]

Diagnostics are conducted component-wise and jointly:
\begin{itemize}
  \item \textbf{Marginal overlap:} comparison of univariate supports for each
  component of $X_t$ (e.g., net load ranges).
  \item \textbf{Joint overlap:} identification of regions in $\mathbb{R}^p$
  where observations from both regimes are present.
\end{itemize}

Operationally, overlap is enforced via binning and trimming:
\begin{itemize}
  \item Observations with $X_t \notin \mathcal{S}_\cap$ are excluded.
  \item No functional extrapolation across regimes is permitted.
\end{itemize}

This choice is conservative: it reduces sample size but preserves interpretive
integrity.\footnote{
This approach aligns with best practice in observational studies, where trimming
to common support trades variance for bias reduction
\citep{imbens_rubin_2015}.
}

\subsection{Overlap diagnostics under stress}
\label{subsec:overlap_stress}

Stress events (Fern-like episodes) pose a particular challenge: by construction,
they are rare, extreme, and often regime-specific.

Accordingly, overlap is evaluated separately for:
\begin{itemize}
  \item \emph{Normal operation},
  \item \emph{Moderate stress},
  \item \emph{Fern-style extreme stress}.
\end{itemize}

It is expected—and explicitly acknowledged—that $\mathcal{S}_\cap$ may be
substantially smaller under extreme stress. Results reported for such windows
are therefore conditioned on a narrower subset of system states and must not be
generalized beyond them.

\subsection{Partial identification and bounds}
\label{subsec:partial_id}

Where overlap is incomplete, counterfactual quantities are only
\emph{partially identified}.\footnote{
Partial identification is the appropriate inferential framework when support or
exclusion restrictions fail; see \citep{manski_2003}.
}

Formally, let $\Delta(X)$ denote the conditional regime effect. If $X \notin
\mathcal{S}_\cap$, $\Delta(X)$ is not point-identified. Instead, we report bounds:
\[
\underline{\Delta}(X) \le \Delta(X) \le \overline{\Delta}(X),
\]
where bounds are constructed using worst-case assumptions consistent with
observed outcomes.

These bounds are not tightened via parametric assumptions or structural
extrapolation. Their purpose is to:
\begin{itemize}
  \item Make non-identification explicit,
  \item Prevent overinterpretation of sparse regions,
  \item Delineate which claims are empirical versus speculative.
\end{itemize}

\subsection{Documentation and traceability}
\label{subsec:overlap_traceability}

All overlap restrictions are:
\begin{itemize}
  \item Logged at the episode level in the episode ledger,
  \item Declared in figure captions via the ledger-to-figure traceability rule,
  \item Referenced explicitly when conclusions depend on restricted support.
\end{itemize}

Figures or statistics computed outside $\mathcal{S}_\cap$ are labeled as
\emph{illustrative only} and are not cited as evidence of regime effects.

\subsection{Role in the overall identification strategy}
\label{subsec:overlap_role}

Wave~3-B serves as a gatekeeper for all counterfactual claims. It does not seek
to maximize statistical power; rather, it enforces epistemic discipline.

Combined with the threats and partial bounds analysis in
Chapter~\ref{ch:counterfactual}, this section ensures that observed differences
between $M_0$ and $M_1$ are interpreted as:
\begin{itemize}
  \item Point-identified where overlap holds,
  \item Partially identified where overlap fails,
  \item Non-identifiable where neither condition is satisfied.
\end{itemize}

This hierarchy is maintained throughout the remainder of the analysis.


\section{Day-Ahead Posture to Real-Time Scarcity Transmission}
\label{sec:wave3_dart_bridge}

RTC+B embeds a structural linkage between the \emph{day-ahead (DA) procurement
posture} and the \emph{real-time (RT) realization of scarcity}. This linkage is
neither mechanical nor guaranteed: DA awards reflect anticipated system needs
under forecast conditions, whereas RT scarcity prices emerge from realized
feasibility under stochastic load, generation, and intertemporal constraints.

This section formalizes DA posture as an observable state variable and evaluates
whether it transmits predictive content into RT scarcity outcomes. The objective
is not to claim causality, but to test whether RTC+B preserves an economically
meaningful bridge between planning posture and operational scarcity signals.

\subsection{DA posture construction}
\label{subsec:da_posture}

Let $Q^{DA}_{k,d}$ denote the total cleared day-ahead quantity for ancillary
product $k$ on operating day $d$, as recorded in DAMASAGG settlement artifacts.
Let $R^{DA}_{k,d}$ denote the corresponding DA requirement.

We define the \emph{DA posture ratio}:
\[
\rho_{k,d} := \frac{Q^{DA}_{k,d}}{R^{DA}_{k,d}}.
\]

To avoid overfitting to continuous ratios and to align with operational
interpretability, posture is discretized into ordinal categories:
\[
Q^{DA}_{k,d} \in
\begin{cases}
\text{tight}, & \rho_{k,d} \le \underline{\rho}, \\
\text{adequate}, & \underline{\rho} < \rho_{k,d} < \overline{\rho}, \\
\text{long}, & \rho_{k,d} \ge \overline{\rho},
\end{cases}
\]
where $(\underline{\rho}, \overline{\rho})$ are pre-registered cutoffs (e.g.,
$1.05$, $1.25$) chosen to reflect operational slack rather than statistical
quantiles.\footnote{
Discretization prevents spurious precision and aligns posture categories with
how ERCOT operators and market participants reason about reserve sufficiency.
Sensitivity to these cutoffs is evaluated under SR-5.}

This construction treats DA posture as an \emph{ex ante commitment signal}:
it summarizes the system's planned reserve margin before RT uncertainty and
before ESR state-of-charge (SoC) dynamics are realized.

\subsection{Mapping DA posture to RT intervals}
\label{subsec:da_rt_mapping}

Each operating day $d$ maps to a set of RT intervals $t \in \mathcal{T}(d)$.
DA posture is treated as fixed over the day, while RT outcomes vary at the
5-minute resolution.

To preserve temporal coherence, DA posture is not forward-filled beyond its
operating day, and days with partial DAMASAGG coverage are excluded. This
ensures that $Q^{DA}_{k,d}$ is a valid conditioning variable for all $t \in
\mathcal{T}(d)$.

\subsection{Geometry--Feasibility Linkage: DA Fragility as a Leading Indicator}
\label{sec:geometry_feasibility_bridge}

This subsection formalizes an auditable connection between Day-Ahead (DA)
\emph{geometry} and real-time (RT) \emph{feasibility-linked tail outcomes} under
RTC+B. All statements are explicitly categorized as \emph{descriptive},
\emph{bounded}, or \emph{testable} under the identification discipline in
Chapter~\ref{ch:ident_limits}, and are executed only on common support
(Section~\ref{sec:wave3_overlap}).

\subsubsection{High-level framing (what is being linked, and what is not)}

\paragraph{Descriptive (objects).}
Fix an ancillary product $k$ and operating day $d$.
\begin{itemize}
  \item The \emph{DA geometry object} summarizes the \emph{shape} of the DA offer
  stack near the requirement $R^{DA}_{k,d}$ using aggregated, settlement-grade
  DA artifacts (Chapter~\ref{ch:da_geometry}).
  \item The \emph{RT feasibility object} is not directly observed in DA artifacts;
  it is proxied in this report by (i) feasibility compression proxies defined
  elsewhere in Wave~3 and (ii) tail events in RT MCPC outcomes.
\end{itemize}

\paragraph{Bounded (non-claims).}
This linkage does \emph{not} assert:
(i) quantity inadequacy, (ii) physical unavailability, (iii) intent or conduct,
or (iv) causal transmission from DA awards to RT scarcity.
Instead, it defines a \emph{leading-indicator test}: whether DA marginal shape
contains predictive information about RT tail risk after conditioning on matched
system conditions.

\subsubsection{Conceptual distinction: geometry fragility versus quantity adequacy}

\paragraph{Descriptive (definition).}
DA \emph{quantity adequacy} is summarized by the posture ratio $\rho_{k,d}$
(Section~\ref{subsec:da_posture}); it is a scalar length/tightness object.

DA \emph{geometry fragility} is a shape object: it characterizes whether the
\emph{marginal} MW in the vicinity of $R^{DA}_{k,d}$ is supported by a thick
elastic band or by a thin, discretized tail.
Geometry fragility can be high even when $\rho_{k,d}$ is adequate or long.

\paragraph{Bounded (interpretive limit).}
Geometry fragility is explicitly distinguished from \emph{quantity adequacy} and
from \emph{market power inference}. Geometry is treated as a conditioning
variable and an early-warning diagnostic; any strategic attribution is deferred
to Chapter~\ref{ch:gaming_full} and is not derived from geometry alone.

\subsubsection{Geometry metrics (concentration, dispersion, and local tail thickness)}

\paragraph{Descriptive (construction scope).}
All geometry metrics are defined on the aggregated DA offer curve introduced in
Chapter~\ref{ch:da_geometry}. Fix product $k$ and day $d$. Let $Q_{k,d}(p)$ be
cumulative quantity offered at prices at or below $p$, and let
$S_{k,d}(q)=\inf\{p:Q_{k,d}(p)\ge q\}$ denote the inverse supply step function.
Let $q^{\star}_{k,d}:=R^{DA}_{k,d}$ and $p^{\star}_{k,d}:=S_{k,d}(q^{\star}_{k,d})$.

Define the top-of-curve window:
\[
\mathcal{W}_{k,d}(\eta)
:=
\bigl[\eta q^{\star}_{k,d},\; q^{\star}_{k,d}\bigr],
\qquad \eta\in(0,1).
\]
All statistics below use only offer steps intersecting $\mathcal{W}_{k,d}(\eta)$
(default $\eta=0.90$; sensitivity registered).

\paragraph{Concentration (Herfindahl-style index on offer segments).}
Let $j$ index offer steps intersecting $\mathcal{W}_{k,d}(\eta)$ and let
$\Delta q_{j,k,d}$ denote the quantity mass contributed by step $j$ within the
window. Define shares
$\omega_{j,k,d}(\eta):=\Delta q_{j,k,d}/\bigl((1-\eta)q^{\star}_{k,d}\bigr)$ and:
\[
H^{\mathrm{seg}}_{k,d}(\eta)
:=
\sum_j \omega_{j,k,d}(\eta)^2.
\]
Higher $H^{\mathrm{seg}}_{k,d}(\eta)$ indicates a thinner, more discretized
marginal tail.

\paragraph{Price dispersion (robust spread in the marginal window).}
Let $p_{j,k,d}$ denote the offer price of step $j$. Define a robust,
quantity-weighted interdecile range:
\[
D^{\mathrm{IDR}}_{k,d}(\eta)
:=
Q_{0.90}\!\left(p_{j,k,d};\,\omega_{j,k,d}(\eta)\right)
-
Q_{0.10}\!\left(p_{j,k,d};\,\omega_{j,k,d}(\eta)\right).
\]
Small $D^{\mathrm{IDR}}_{k,d}(\eta)$ is consistent with a steep local tail.

\paragraph{Tail thickness (consistency with Chapter~\ref{ch:da_geometry}).}
To align Wave~3 with the DA tail-thickness operator already defined in
Chapter~\ref{ch:da_geometry}, we also carry forward
$T_{k,d}(\Delta p)=Q_{k,d}(p^{\star}_{k,d})-Q_{k,d}(p^{\star}_{k,d}-\Delta p)$
and treat small $T_{k,d}(\Delta p)$ as an additional descriptor of fragility.

\paragraph{Descriptive (terminology).}
We refer to a day as \emph{geometrically fragile} for product $k$ when the
marginal window is simultaneously concentrated and thin, operationalized via a
pre-registered binning of
$\bigl(H^{\mathrm{seg}}_{k,d}(\eta),\,D^{\mathrm{IDR}}_{k,d}(\eta),\,T_{k,d}(\Delta p)\bigr)$.

\subsubsection{Linkage mechanism under RTC+B (feasibility-channel consistency)}

\paragraph{Bounded (mechanistic statement).}
RTC+B co-optimizes energy and ancillary services subject to feasibility,
including intertemporal constraints for storage (e.g., SoC headroom).
Accordingly, RT ancillary prices and tail events can be consistent with
feasibility compression even when aggregate DA quantities appear adequate.
Because resource-level SoC and binding-constraint details are not contained in
DA geometry artifacts, the report treats this section as a \emph{consistency
check}: geometry fragility may co-move with later RT tails if feasibility loss
must be absorbed by a thin marginal set.

\paragraph{Testable (conditional RT MCPC tail prediction).}
For each product $k$, define a tail event indicator
$\mathbb{I}\{\text{MCPC}_{k,t}>\tau_k\}$ with pre-registered $\tau_k$.
The leading-indicator hypothesis is evaluated within common support by comparing
conditional tail probabilities:
\[
\begin{aligned}
&\Pr\!\left(
\text{MCPC}_{k,t} > \tau_k
\;\middle|\;
\text{FragileGeom}_{k,d(t)}=1,\; \rho_{k,d(t)},\; X_t
\right)\\
&\quad \text{vs.}\quad
\Pr\!\left(
\text{MCPC}_{k,t} > \tau_k
\;\middle|\;
\text{FragileGeom}_{k,d(t)}=0,\; \rho_{k,d(t)},\; X_t
\right).
\end{aligned}
\]
where $\text{FragileGeom}_{k,d}$ is a pre-registered indicator derived from the
geometry metrics above and $X_t$ is the matched conditioning vector.

A higher conditional tail probability in the fragile-geometry regime is reported
as \emph{descriptive} and \emph{test-supportive}. It is not interpreted as a
causal effect, and it is subject to the guardrails in
Chapter~\ref{ch:ident_limits} (including explicit non-identification outside
$\mathcal{S}_{\cap}$).

\subsection{Conditional RT scarcity outcomes}
\label{subsec:conditional_rt}

For each ancillary product $k$, we estimate conditional tail probabilities of
the form:
\[
\Pr\!\left(
\text{MCPC}_{k,t} > \tau_k
\;\middle|\;
Q^{DA}_{k,d} = q,\;
X_t
\right),
\]
where:
\begin{itemize}
  \item $\tau_k$ is a product-specific tail threshold (e.g., $Q_{0.99}$),
  \item $q \in \{\text{tight}, \text{adequate}, \text{long}\}$,
  \item $X_t$ denotes matched real-time covariates (net load, ramp, outage proxy).
\end{itemize}

Estimation is conducted using:
\begin{itemize}
  \item Nonparametric exceedance rates with block bootstrap confidence intervals,
  \item Conditional comparisons within matched stress bins to control for
  realized system conditions.
\end{itemize}

Importantly, DA posture is \emph{not} treated as exogenous; it is a planning
decision informed by forecasts. Conditioning on $X_t$ mitigates, but does not
eliminate, this endogeneity.

\subsection{Transmission diagnostics}
\label{subsec:transmission_diagnostics}

We define \emph{DA-to-RT transmission} as a monotone ordering:
\[
\Pr(\text{MCPC}_{k,t} > \tau_k \mid \text{tight})
>
\Pr(\text{MCPC}_{k,t} > \tau_k \mid \text{adequate})
>
\Pr(\text{MCPC}_{k,t} > \tau_k \mid \text{long}),
\]
holding $X_t$ fixed.

Three diagnostic regimes are distinguished:

\paragraph{Strong transmission.}
Tail risk declines monotonically with looser DA posture. This indicates that
DA procurement effectively mitigates RT scarcity, consistent with a system
where feasibility constraints are largely static and anticipated.

\paragraph{Partial transmission.}
Only the tight vs.\ non-tight contrast is informative, or transmission holds
only under moderate stress. This suggests DA posture mitigates scarcity risk
up to a point, after which RT feasibility constraints dominate.

\paragraph{Transmission failure.}
Conditional MCPC tails are invariant to DA posture. This outcome is consistent
with binding intertemporal constraints (e.g., ESR SoC depletion) or with shocks
that overwhelm planning margins.

\subsection{Interaction with intertemporal feasibility}
\label{subsec:da_soc_interaction}

DA posture is a \emph{static} quantity, whereas ESR feasibility is
\emph{dynamic}. Even a ``long'' DA posture may fail to suppress RT scarcity if
SoC is exhausted prior to or during stress events.

Accordingly, we evaluate joint conditioning:
\[
\Pr\!\left(
\text{MCPC}_{k,t} > \tau_k
\;\middle|\;
Q^{DA}_{k,d} = q,\;
\kappa_{k,t} < \kappa^\star,\;
X_t
\right),
\]
where $\kappa_{k,t}$ is the capability compression ratio defined in
Section~\ref{sec:wave3_soc}.

A finding that DA posture matters when $\kappa_{k,t}$ is high but not when it
is low provides evidence that RT scarcity is governed by intertemporal
feasibility rather than procurement volume per se.

\subsection{Interpretation and limits}
\label{subsec:da_rt_limits}

This section establishes whether RTC+B preserves a meaningful DA-to-RT
information channel. It does \emph{not} claim:
\begin{itemize}
  \item Welfare optimality of DA procurement levels,
  \item Causal effectiveness of specific DA quantities,
  \item Absence of strategic behavior in DA awards.
\end{itemize}

Instead, the contribution is diagnostic: it distinguishes scarcity regimes
where planning posture remains relevant from regimes where real-time
intertemporal constraints dominate price formation.

This distinction is essential for interpreting MCPC behavior under RTC+B and
for evaluating whether observed scarcity reflects forecast error, operational
shock, or structural feasibility exhaustion.

\section{Fern Episode Packet Consolidation}
\label{sec:wave3_fern}

Fern-like stress events provide a quasi-experimental substrate in which scarcity
signals are (i) system-wide, (ii) temporally concentrated, and (iii) driven by
exogenous physical conditions rather than endogenous bidding behavior. In the
ERCOT context, severe winter storms (e.g., Winter Storm Uri and subsequent cold
events) have served as canonical examples of such stress regimes, frequently
invoked in both planning and market-design discussions as revealing tests of
scarcity formation mechanisms.\footnote{
See, inter alia, ERCOT’s Winter Storm Uri review and related market redesign
materials, which treat extreme cold events as natural stress tests for
reliability and pricing frameworks \citep{ercot_uri_2021} 

Throughout this report, the term Fern-like refers to the general class of winter stress events characterized by sustained cold load, elevated reserve requirements, and intertemporal strain on storage feasibility. The term The Fern Packet refers specifically to the January 22–29, 2026 interval set, which serves as the primary empirical benchmark for this post-mortem..
}

This section formalizes the construction of a \emph{Fern Episode Packet}: a
standardized, ledger-indexed collection of scarcity episodes intended to support
(i) aggregation across comparable stress events, (ii) robustness checks against
single-event dominance, and (iii) disciplined separation of descriptive evidence
from causal interpretation.

\subsection{Conceptual role of Fern-like episodes}

Fern episodes are not treated as proof of welfare improvement or degradation.
Rather, they function as \emph{stress amplifiers}: environments in which the
interaction between real-time co-optimization, ancillary scarcity pricing, and
energy-side adders is most likely to be revealed in observable market outcomes.

Two properties motivate their use:
\begin{enumerate}
  \item \textbf{Exogeneity of stress:} Weather-driven load surges and generation
  derates are plausibly exogenous to market participant behavior over short
  horizons, reducing concerns that observed scarcity prices merely reflect
  strategic bidding responses.
  \item \textbf{Signal-to-noise amplification:} Under extreme conditions,
  scarcity mechanisms bind more frequently and with greater magnitude, improving
  statistical power for tail-based diagnostics.
\end{enumerate}

Importantly, Fern episodes are not assumed to be representative of normal system
conditions. Findings derived from them are therefore interpreted as
\emph{conditional stress-regime evidence}, not unconditional system performance
claims.

\subsection{Episode definition and inclusion criteria}

% Price-blind stress packet illustration (limited palette)
\begin{figure}[!ht]
\centering
\begin{tikzpicture}[x=1cm,y=1cm, font=\small]
  % Timeline
  \draw[very thick, black!85] (0,0) -- (14,0);
  \node[below] at (0,0) {Time};

  % Stress packet window (price-blind)
  \fill[LinkBlue!12] (3,-0.55) rectangle (11,0.55);
  \draw[thick, black!75] (3,-0.55) rectangle (11,0.55);

  % Onset / end markers
  \draw[thick, black!80] (3,-0.85) -- (3,0.85);
  \draw[thick, black!80] (11,-0.85) -- (11,0.85);
  \node[above, text=black!85] at (3,0.85) {Stress onset $t_0$};
  \node[above, text=black!85] at (11,0.85) {Stress end $t_1$};

  % Duration brace
  \draw[decorate,decoration={brace,amplitude=5pt}, black!80] (3,-1.2) -- (11,-1.2);
  \node[below, text=black!85] at (7,-1.2) {Duration $t_1-t_0$};

  % Covariates (explicitly price-blind)
  \node[
    align=left,
    anchor=west,
    fill=white,
    draw=black!20,
    rounded corners=2pt,
    inner sep=3pt,
    text width=4.4cm
  ] (cov) at (11.65,0.55) {\textbf{Covariates (price-blind)}\\
  Weather bin (temp / wind chill)\\
  Load level / net-load proxy\\
  Sustained reserve stress\\
  Persistence / contiguity};
  \draw[-{Latex[length=2mm]}, thick, black!70] (10.5,0.25) -- (cov.west);

  % Note that prices are analyzed inside the packet
  \node[align=left,anchor=west, text=black!80] at (0,-2.05) {\textbf{Interpretation:} Packet selection is driven by physical triggers; pricing is analyzed \emph{within} the packet (not used to select it).};
\end{tikzpicture}
\caption{Price-blind stress packet illustration: a contiguous stress interval defined by physical triggers (weather/load + sustained reserve stress), with annotated onset, duration, and covariates. (Limited palette; definitions align with Appendix~H.)}
\label{fig:price_blind_stress_packet}
\end{figure}

An episode $e$ is defined as a contiguous block of real-time intervals satisfying
all of the following criteria:
\begin{itemize}
  \item \textbf{Physical stress trigger (exogenous):} The interval set falls
  within a pre-registered stress class defined by weather/load conditions
  (e.g., temperature/load extremeness) \emph{and} sustained reserve stress.
  \item \textbf{Sustained reserve stress:} One or more operating reserve metrics
  fall below pre-specified thresholds for a minimum duration $\tau_{\min}$,
  exceeding what would be expected from transient noise.
  \item \textbf{Temporal contiguity:} The stress condition persists across
  consecutive intervals, allowing brief gaps of length $\leq \delta$ to account
  for telemetry noise or dispatch smoothing.
  \item \textbf{Pricing (confirmatory only):} Scarcity-linked price components
  (e.g., RT ORDC adders, AS MCPC) are recorded and analyzed \emph{within} the
  physically triggered episodes, but are not used as an inclusion criterion.
\end{itemize}

Thresholds, minimum durations, and gap tolerances are pre-registered in the
Metric Dictionary and referenced by sensitivity levers (SR-2 and SR-5) to prevent
post hoc tuning.

Each qualifying episode is assigned a unique identifier and logged in the
episode ledger with:
\begin{itemize}
  \item Start and end timestamps,
  \item Triggering physical conditions (weather/load bins; reserve stress metrics),
  \item Pricing outcomes observed during the episode (energy, ORDC adders, specific AS products),
  \item Applicable sensitivity settings.
\end{itemize}

\subsection{Event-time normalization and alignment}

For each episode $e$, event time is normalized by defining $t=0$ as the first
interval in which the \emph{physical stress trigger} is satisfied (reserve stress
onset within the pre-registered weather/load class). Formally,
\[
t = k - k_e^{\ast},
\]
where $k$ indexes real-time intervals and $k_e^{\ast}$ denotes the episode’s
initial stress-trigger interval.

This alignment enables aggregation across heterogeneous episodes whose absolute
timing and duration may differ. For each variable of interest (energy prices,
scarcity adders, ancillary MCPCs, and frequency proxies), we construct
event-centered trajectories over a symmetric window $[-T_-, T_+]$.

Descriptive statistics are reported using:
\begin{itemize}
  \item The cross-episode median path,
  \item The interquartile range (IQR),
  \item Episode-count-weighted coverage indicators to flag thinning support near
  window boundaries.
\end{itemize}

No parametric smoothing is imposed unless explicitly stated; all reported paths
are empirical aggregates of ledger-traceable episode data.

\subsection{Influence and dominance diagnostics}

Because Fern-like stress events are rare by construction, special care is taken
to assess whether aggregated results are dominated by a small number of extreme
episodes. To this end, we implement leave-one-episode-out (LOEO) diagnostics.

For a given statistic $S$ (e.g., peak MCPC, integrated frequency deviation),
define:
\[
S_{(-e)} = \text{statistic computed excluding episode } e.
\]

Stability is assessed by examining the dispersion of $\{S_{(-e)}\}$ relative to
the full-sample estimate $S$. Large deviations indicate sensitivity to a single
episode and are explicitly reported.

Where dominance is detected, results are reclassified as \emph{episode-specific}
rather than \emph{packet-level} findings, and downstream interpretations are
qualified accordingly.

\subsection{Interpretive boundaries}

Results derived from the Fern Episode Packet are subject to explicit limits:
\begin{itemize}
  \item They do not establish causal effects absent further identifying
  assumptions.
  \item They may not generalize to moderate-stress or normal operating regimes.
  \item They reflect the joint operation of market design and physical system
  constraints, not isolated policy levers.
\end{itemize}

Accordingly, Fern-based evidence is used in this report to:
\begin{enumerate}
  \item Stress-test theoretical predictions from the KKT and ORDC/ASDC bridge
  chapters,
  \item Identify qualitative regime differences between legacy and RTC+B pricing,
  \item Inform sensitivity analysis and bounds, not point estimates of welfare
  change.
\end{enumerate}

\subsection{Partial identification and stress-regime bounds}
\label{subsec:fern_partial_id}

Fern Episode Packets do not identify a single causal estimand. Instead, they
support \emph{partial identification} of regime effects under stress. This
distinction is critical: extreme events violate many assumptions required for
point identification, including smooth counterfactual evolution and stable
constraint sets.

Let $Y_e$ denote an episode-level outcome (e.g., peak MCPC, integrated frequency
deviation, tail energy rent share) observed under RTC+B. Define the unobserved
legacy counterfactual $Y_e^{(0)}$ as the outcome that would have occurred under
legacy pricing, holding physical conditions fixed.

Because $Y_e^{(0)}$ is not observable, we do not attempt to estimate
$\mathbb{E}[Y_e^{(1)} - Y_e^{(0)}]$ directly. Instead, we bound it.

Following the partial identification framework of Manski, we define bounds:
\[
\underline{\Delta}_e \leq Y_e^{(1)} - Y_e^{(0)} \leq \overline{\Delta}_e,
\]
where $\underline{\Delta}_e$ and $\overline{\Delta}_e$ are constructed using
minimal structural restrictions.

In this report, admissible restrictions include:
\begin{itemize}
  \item \textbf{Monotonic stress ordering:} Fern-like episodes are at least as
  stressful as matched non-Fern days along observed covariates.
  \item \textbf{Non-negativity of scarcity rents:} Scarcity payments cannot be
  negative by construction.
  \item \textbf{Operational feasibility constraints:} ESR dispatch and reserve
  awards must satisfy state-of-charge feasibility.
\end{itemize}

These restrictions yield \emph{sign bounds} and \emph{order bounds} rather than
point estimates. A finding such as ``scarcity rents reallocate toward ancillary
products under RTC+B during Fern episodes'' is therefore interpreted as a
\emph{bounded directional result}, not a welfare theorem.

\subsection{Episode weighting schemes}
\label{subsec:fern_weighting}

Aggregation across Fern episodes requires an explicit choice of weighting. Because
episodes differ materially in duration, severity, and system footprint, equal
weighting is not innocuous.

We therefore define a family of weighting schemes:
\[
\widehat{S} = \sum_{e \in \mathcal{E}} w_e S_e,
\]
where $S_e$ is an episode-level statistic and $w_e$ satisfies
$\sum_e w_e = 1$.

The following schemes are evaluated:
\begin{itemize}
  \item \textbf{Equal episode weighting} ($w_e = 1/|\mathcal{E}|$), emphasizing
  regime frequency over magnitude.
  \item \textbf{Duration weighting} ($w_e \propto \text{length}_e$), emphasizing
  exposure time.
  \item \textbf{Severity weighting} ($w_e \propto \max \text{MCPC}_e$ or integrated
  scarcity rents), emphasizing tail risk.
  \item \textbf{Load-weighted severity} ($w_e \propto \text{severity}_e \times
  \text{load}_e$), approximating system impact.
\end{itemize}

\subsection{State-of-Charge proxy framework}
\label{subsec:wave3_soc}

Direct state-of-charge (SoC) telemetry for energy storage resources is not
publicly available in ERCOT settlement data. Nevertheless, RTC+B introduces
intertemporal feasibility constraints whose effects may be partially inferred
from observable proxies.

\subsubsection{Conceptual motivation}

Under co-optimization, an ESR’s opportunity cost of providing reserves depends on
its expected future energy availability. During stress events, binding SoC
constraints may manifest indirectly as:
\begin{itemize}
  \item Elevated ancillary MCPCs despite modest energy prices,
  \item Abrupt cessation of ESR participation in specific products,
  \item Temporal clustering of high MCPCs followed by price collapse.
\end{itemize}

These signatures motivate proxy-based diagnostics.

\subsubsection{Proxy construction}

We define three classes of SoC proxies:

\paragraph{Participation exhaustion proxy}
Let $A_{p,t}$ denote awarded AS capacity for product $p$. Define:
\[
E_{p,t} = \mathbb{1}\{A_{p,t} = 0 \mid A_{p,t-1} > 0\},
\]
flagging abrupt withdrawal from participation.

\paragraph{Price-decoupling proxy}
Define the ratio:
\[
R_t = \frac{\text{MCPC}_{p,t}}{\pi^{\mathrm{SPP}}_t + \epsilon},
\]
where $\epsilon$ prevents division by zero. Sustained elevation of $R_t$ suggests
reserve scarcity not mirrored by energy scarcity.

\paragraph{Temporal depletion proxy}
Define rolling windows in which high MCPC events precede sharp declines in awards
or prices, consistent with depletion dynamics.

\subsubsection{Interpretation limits}

These proxies do not identify actual SoC levels. They instead detect patterns
\emph{consistent with} binding intertemporal constraints. Accordingly:
\begin{itemize}
  \item Proxy results are reported descriptively,
  \item No single proxy is decisive,
  \item Concordance across proxies strengthens inference.
\end{itemize}

All proxy-based findings are cross-referenced to the episode ledger and subjected
to the same sensitivity and traceability rules as primary results.

\section{Feasibility Channels and State-of-Charge Proxies}
\label{sec:wave3_soc}

RTC+B fundamentally alters how scarcity manifests by internalizing
\emph{intertemporal feasibility constraints}. Unlike single-period scarcity
constructs (e.g., energy-only ORDC adders), RTC+B permits scarcity value to
emerge from the exhaustion of \emph{future optionality}, particularly for
energy storage resources (ESRs).\footnote{
This distinction is emphasized in the broader co-optimization literature:
resources constrained across time introduce shadow prices that are not tied
to instantaneous balance alone. See, e.g., \citep{miso_coopt_presentation_2019}.
}

Because direct state-of-charge (SoC) telemetry is not publicly available in
ERCOT settlement data, this section develops \emph{proxy diagnostics} that
detect patterns consistent with binding intertemporal constraints, without
claiming direct observability of SoC itself.

\subsection{Intertemporal feasibility as a scarcity channel}

Let an ESR $i$ have state-of-charge $s_{i,t}$ evolving according to:
\[
s_{i,t+1} = s_{i,t} + \eta_i c_{i,t} - \frac{1}{\eta_i} d_{i,t},
\]
subject to bounds:
\[
0 \leq s_{i,t} \leq \bar{s}_i,
\]
where $c_{i,t}$ and $d_{i,t}$ denote charging and discharging decisions and
$\eta_i$ is round-trip efficiency.

Under co-optimization, these constraints enter the system Lagrangian and
generate shadow prices associated with future infeasibility. Importantly,
these shadow prices may bind \emph{before} physical reserve shortages occur,
especially during extended stress events.

Thus, scarcity may arise even when instantaneous reserve requirements are
technically satisfiable, because doing so would render future system states
infeasible.\footnote{
This mechanism is conceptually analogous to water value in hydro scheduling,
where scarcity reflects opportunity cost of future energy rather than present
shortage.}

\subsection{Capability compression as a system-level proxy}
\label{subsec:capability_compression}

Let $C_{k,t}$ denote total \emph{available} system capability for ancillary
product $k$ at time $t$, and $R_{k,t}$ the corresponding requirement.
We define the \emph{capability compression ratio}:
\[
\kappa_{k,t} := \frac{C_{k,t}}{R_{k,t}}.
\]

Low values of $\kappa_{k,t}$ indicate that the system is operating near the
feasibility boundary for product $k$. Under RTC+B, sustained compression is
consistent with depletion of intertemporal flexibility across the resource
fleet, particularly for ESR-dominated products.

Several clarifications are essential:
\begin{itemize}
  \item $\kappa_{k,t} < 1$ does \emph{not} necessarily imply violation; rather,
  it indicates that feasibility relies on tight coordination and limited slack.
  \item Compression may arise from physical outages, ramp constraints, or SoC
  depletion; this metric alone does not identify the cause.
  \item The diagnostic value of $\kappa_{k,t}$ lies in its \emph{temporal
  persistence} and \emph{co-movement} with price signals.
\end{itemize}

Accordingly, $\kappa_{k,t}$ is treated as a \emph{necessary but not sufficient}
indicator of feasibility-driven scarcity.

\subsection{Conditional MCPC tail behavior}
\label{subsec:conditional_mcpc}

To assess whether scarcity prices reflect intertemporal feasibility rather than
instantaneous shortfall, we estimate conditional tail probabilities of the form:
\[
\Pr\!\left(\text{MCPC}_{k,t} > \tau_k \mid
\kappa_{k,t} < \kappa^\star,\; X_t \right),
\]
where:
\begin{itemize}
  \item $\tau_k$ is a product-specific tail threshold (e.g., $Q_{0.99}$),
  \item $\kappa^\star$ is a compression cutoff (e.g., $1.2$),
  \item $X_t$ denotes matched system conditions (load, net load, ramps).
\end{itemize}

The identifying logic is comparative rather than causal:
\begin{itemize}
  \item If MCPC tails occur predominantly when $\kappa_{k,t}$ is low,
  conditional on $X_t$, this is consistent with feasibility-driven scarcity.
  \item If MCPC tails occur independently of compression, instantaneous
  shortfall explanations dominate.
\end{itemize}

We further evaluate:
\begin{enumerate}
  \item \textbf{Persistence:} Whether elevated MCPCs persist across multiple
  intervals while $\kappa_{k,t}$ remains compressed.
  \item \textbf{Cross-product asymmetry:} Whether products with greater ESR
  exposure (e.g., ECRS, Regulation) exhibit stronger conditional tails than
  inertia-backed products.
\end{enumerate}

These diagnostics allow separation of \emph{feasibility scarcity} from
\emph{momentary reserve imbalance}, even absent direct SoC data.

\subsection{Temporal depletion and release patterns}
\label{subsec:depletion_patterns}

A further implication of intertemporal constraints is \emph{non-symmetric
price dynamics}. Specifically, depletion-driven scarcity is expected to show:
\begin{itemize}
  \item Gradual build-up of MCPCs as flexibility erodes,
  \item Sharp collapses once feasibility resets (e.g., load relief,
  resource return, or SoC recovery).
\end{itemize}

We therefore examine lead--lag relationships between:
\begin{itemize}
  \item High-MCPC intervals,
  \item Subsequent declines in awarded capability $C_{k,t}$,
  \item Subsequent normalization of prices.
\end{itemize}

Such asymmetric dynamics are inconsistent with purely static shortage models
but align with intertemporal feasibility exhaustion.

\subsection{Interpretation limits and evidentiary tiering}
\label{subsec:soc_limits}

The results in this section are \emph{structural diagnostics}, not welfare
proofs. Three limits are emphasized:
\begin{enumerate}
  \item Proxy diagnostics cannot recover true SoC trajectories.
  \item Observed patterns may reflect correlated but unobserved constraints.
  \item No single metric is decisive.
\end{enumerate}

Accordingly, evidence is tiered:
\begin{itemize}
  \item \textbf{Tier 1:} Descriptive consistency with feasibility scarcity.
  \item \textbf{Tier 2:} Robustness across conditioning, thresholds, and
  weighting schemes.
  \item \textbf{Tier 3:} Concordance with KKT-based theoretical predictions
  (Chapter~\ref{ch:kkt_model}).
\end{itemize}

Only Tier-2 or Tier-3 evidence is referenced in regime-level conclusions.

\subsection{Role in the broader argument}

Wave~3-E provides the missing link between:
\begin{itemize}
  \item Abstract co-optimization theory,
  \item Observed MCPC behavior under RTC+B,
  \item The absence of energy-side scarcity signals.
\end{itemize}

By demonstrating coherence between price patterns and feasibility compression,
this section supports the interpretation that RTC+B reallocates scarcity value
across products in a manner consistent with intertemporal system constraints,
without asserting welfare dominance or optimality.

\subsection{Proposition: SoC shadow prices shift scarcity value into MCPCs (proof sketch)}
\label{subsec:soc_shadow_price_proposition}

This subsection formalizes the mechanism by which intertemporal state-of-charge
(SoC) feasibility introduces a scarcity channel that appears in ancillary
service dual prices (MCPCs), even when energy-side scarcity (e.g., ORDC adders)
is muted.

\paragraph{Setup (stylized co-optimization).}
Consider a single-node real-time co-optimization over intervals
$t \in \{1,\dots,T\}$ with energy balance and a single ancillary requirement
(for notational simplicity). Let $x_t$ denote net thermal generation output,
and let an aggregate ESR provide (i) energy discharge $d_t$ and (ii) ancillary
capacity award $r_t$.\footnote{
This abstraction compresses many ESRs into a representative resource. The
mechanism is identical in the multi-unit case: SoC constraints enter each ESR's
Lagrangian and aggregate into system feasibility and pricing.}

Let the ESR state evolve as:
\begin{equation}
s_{t+1} = s_t + \eta c_t - \frac{1}{\eta} d_t,
\qquad 0 \le s_t \le \bar{s},
\label{eq:soc_dyn_prop}
\end{equation}
and let feasibility link energy and reserves through a simple headroom
constraint:\footnote{
More detailed models include separate charge/discharge limits, efficiency
losses, and reserve-specific sustainment constraints. The headroom form is
sufficient to expose the dual mechanism.}
\begin{equation}
d_t + r_t \le \bar{p}.
\label{eq:headroom}
\end{equation}

Energy balance and ancillary requirements are:
\begin{align}
x_t + d_t &= L_t, \label{eq:balance}\\
r_t &\ge R_t, \label{eq:req}
\end{align}
with cost $\sum_t C(x_t)$ where $C(\cdot)$ is convex and increasing.

Define Lagrange multipliers:
\begin{itemize}
  \item $\lambda_t$ for energy balance \eqref{eq:balance} (energy shadow price),
  \item $\mu_t \ge 0$ for ancillary requirement \eqref{eq:req} (AS scarcity price),
  \item $\phi_t \ge 0$ for headroom \eqref{eq:headroom} (instantaneous ESR coupling),
  \item $\nu_t$ for SoC dynamics \eqref{eq:soc_dyn_prop} (intertemporal shadow price),
  \item $\underline{\psi}_t,\overline{\psi}_t \ge 0$ for SoC bounds $0 \le s_t \le \bar{s}$.
\end{itemize}

\paragraph{Proposition (scarcity migration via SoC duals).}
\emph{Suppose the ancillary requirement binds in a period of tight ESR
feasibility, and the ESR SoC constraint is binding in the sense that either
$s_t = 0$ or $s_t = \bar{s}$ for some $t$ (or equivalently the intertemporal
shadow value $\nu_t$ is non-zero due to binding SoC bounds). Then the marginal
cost of satisfying an additional unit of ancillary requirement $R_t$ includes an
intertemporal opportunity-cost term induced by SoC, and this term enters the
ancillary dual price $\mu_t$ (and hence MCPC). In particular, there exist
parameter ranges and system states such that $\mu_t$ increases (or remains high)
even when $\lambda_t$ does not exhibit corresponding scarcity.}

\paragraph{Proof sketch.}
Write the Lagrangian:
\begin{align*}
\mathcal{L}
&= \sum_{t=1}^T C(x_t)
+ \sum_{t=1}^T \lambda_t (L_t - x_t - d_t)
+ \sum_{t=1}^T \mu_t (R_t - r_t) \\
&\quad + \sum_{t=1}^T \phi_t (d_t + r_t - \bar{p})
+ \sum_{t=1}^T \nu_t \Big(s_{t+1} - s_t - \eta c_t + \tfrac{1}{\eta} d_t\Big) \\
&\quad + \sum_{t=1}^T \underline{\psi}_t (-s_t)
+ \sum_{t=1}^T \overline{\psi}_t (s_t - \bar{s}).
\end{align*}

First-order stationarity for $(r_t)$ yields:
\begin{equation}
\frac{\partial \mathcal{L}}{\partial r_t} = -\mu_t + \phi_t = 0
\quad \Rightarrow \quad
\mu_t = \phi_t.
\label{eq:mu_phi}
\end{equation}
Thus, the ancillary scarcity price equals the shadow price on the ESR headroom
constraint when the ESR is the marginal provider of $r_t$.

Next, stationarity for $(d_t)$ yields:
\begin{equation}
\frac{\partial \mathcal{L}}{\partial d_t}
=
-\lambda_t + \phi_t + \frac{1}{\eta}\nu_t = 0
\quad \Rightarrow \quad
\phi_t = \lambda_t - \frac{1}{\eta}\nu_t.
\label{eq:phi_lambda_nu}
\end{equation}

Combining \eqref{eq:mu_phi} and \eqref{eq:phi_lambda_nu} gives:
\begin{equation}
\mu_t
=
\lambda_t - \frac{1}{\eta}\nu_t.
\label{eq:mu_lambda_nu}
\end{equation}

Equation \eqref{eq:mu_lambda_nu} is the key structural identity: ancillary dual
prices inherit an intertemporal shadow term $\nu_t$ induced by SoC feasibility.
When SoC bounds bind, $\nu_t$ is generally non-zero and reflects the marginal
value of stored energy across time (i.e., the opportunity cost of consuming SoC
now versus preserving it for future feasibility).

Two implications follow.

\textbf{(i) Elevated MCPC without commensurate energy scarcity.}
If $\lambda_t$ is moderated by non-scarcity conditions (e.g., adequate energy
supply) while $\nu_t$ is large in magnitude due to tight SoC (e.g., depleted ESR
fleet), then $\mu_t$ can be large even when $\lambda_t$ is not. In operational
terms, reserves are scarce because the system is short \emph{flexible feasible
headroom}, not short energy.

\textbf{(ii) Persistence and asymmetry.}
Because $\nu_t$ evolves through the SoC recursion and complementary slackness on
SoC bounds, it can remain elevated across contiguous intervals (until SoC
recovers), yielding persistence in $\mu_t$ (MCPC) and asymmetric release once the
SoC constraint relaxes. This generates precisely the temporal clustering and
abrupt collapse patterns tested in Section~\ref{sec:wave3_soc}.

\paragraph{Remarks and limitations.}
This stylized derivation omits multi-product coupling (Reg-Up/Reg-Down/ECRS),
nonconvexities, uplift, and network/security constraints. Nonetheless, the
mechanism is robust: whenever ESR feasibility constraints couple decisions across
time, their shadow values enter the marginal conditions for reserve procurement,
and therefore can shift scarcity value into ancillary price components (MCPC)
rather than energy-side scarcity adders.\footnote{
In full SCED formulations with multi-product procurement, the Lagrangian includes
a vector of requirement constraints with multipliers $\mu_{k,t}$ and multiple
feasibility constraints. The conclusion generalizes: intertemporal duals appear
in the stationarity equations for ESR-provided ancillary awards, modifying MCPC
formation across products.}

\chapter{Reliability Economics: From Frequency Proxies to System Cost}
\label{ch:reliability_cost}

This chapter converts operational reliability proxies into an economic lens. The goal is not to overclaim adequacy from frequency, but to define a disciplined mapping from observable stability signals to a risk-aware cost functional that can be compared across regimes, stress bins, and sensitivity levers.

\section{Why frequency is an imperfect but useful proxy}
Frequency is a real-time stability indicator reflecting the instantaneous balance of mechanical/electrical power and load, modulated by inertia, primary frequency response, and operator controls.\footnote{Frequency deviations can occur without any loss of load; conversely, adequacy failures can occur with limited frequency precursor depending on controls, islanding, and event structure. Frequency is therefore not a sufficient statistic for adequacy. See \cite{nerc_bal_001_2,nerc_era_2023}.}
Nevertheless, frequency is useful as a descriptive proxy because it is (i) continuously observed at high resolution, (ii) directly tied to control deployments, and (iii) capable of revealing whether scarcity episodes coincide with stressed control performance.

\section{A reliability loss functional}
We define a stability-oriented reliability loss proxy:
\begin{equation}
\mathcal{L}_R := \alpha I_f + \beta N_{\epsilon} + \gamma \cdot \text{ScarcityHours},
\label{eq:reliability_loss}
\end{equation}
where:
\begin{itemize}
  \item $I_f := \sum_t |f_t-60|\Delta t$ is integrated absolute frequency deviation,
  \item $N_{\epsilon} := \sum_t \mathbb{1}\{|f_t-60|>\epsilon\}$ counts excursions beyond a band,
  \item $\text{ScarcityHours}$ counts hours with $a^{RT}$ or MCPC exceedances per the metric dictionary.
\end{itemize}
Coefficients $(\alpha,\beta,\gamma)$ are not asserted as physical constants. They are calibration weights used to compare regimes under consistent scaling. The report recommends reporting $\mathcal{L}_R$ under multiple weight sets to avoid embedding a hidden normative choice.\footnote{This is the same ``multiple admissible weights'' principle used in risk metrics such as CVaR where the tail emphasis is a value judgment.}

\begin{figure}[ht]
    \centering
    \begin{tikzpicture}[font=\small\sffamily, scale=1.1]
        % Axes
        \draw[->, thick] (0,0) -- (8,0) node[below] {Re-Entry Latency (Hours)};
        \draw[->, thick] (0,0) -- (0,5) node[left] {Cumulative Probability ($F(t)$)};
        
        % Y-axis ticks
        \foreach \y/\label in {0/0, 2.5/0.5, 5/1.0}
            \draw (0,\y) -- (-0.1,\y) node[left] {\label};
            
        % Curve A: Strategic Withholding (Fast/Arbitrary Return)
        % Logic: Strategic units can return instantly when price creates arbitrage.
        \draw[thick, dashed, LinkBlue] (0,0) .. controls (0.5, 4) and (1, 4.8) .. (7.5, 4.9) node[right] {};
        \node[align=left, text=LinkBlue] at (2.5, 4.2) {\textbf{Strategic Withholding}\\(Fast/Price-Driven Return)};

        % Curve B: Physical Exhaustion (Recharge Lag)
        % Logic: Physically depleted units MUST wait to recharge.
        \draw[thick, red!70!black] (0,0) .. controls (3, 0.5) and (4, 1) .. (6, 4.8) -- (7.5, 4.9);
        \node[align=right, text=red!70!black] at (6, 1.5) {\textbf{Physical Exhaustion}\\(SoC Recharge Delay)};

        % Annotation of the gap
        \draw[<->, black!70] (2, 4.0) -- (2, 0.5);
        \node[align=center, fill=white, draw=black!20, inner sep=2pt] at (2, 2.25) {Diagnostic Gap\\($\Delta_{CDF}$)};

    \end{tikzpicture}
    \caption{Deployment Re-Entry CDF. Identifying withholding vs. depletion by analyzing the latency of resource return. Physical exhaustion imposes a minimum recharge latency (solid), while strategic withholding (dashed) allows for rapid re-entry.}
    \label{fig:reentry_cdf}
\end{figure}

\section{Cost-of-risk framing}
Define a tail-risk functional for ``bad days'' using conditional value-at-risk:
\begin{equation}
\mathrm{CVaR}_q(\mathcal{L}_R) := \mathbb{E}\!\left[\mathcal{L}_R \mid \mathcal{L}_R \ge Q_q(\mathcal{L}_R)\right].
\label{eq:cvar}
\end{equation}
We compute $\mathrm{CVaR}_q$ across regimes and matched stress bins. This aligns with the empirical fact that reliability and scarcity risk are tail-dominated. See \cite{embrechts_kluppelberg_mikosch_1997} for tail-risk foundations.

\section{What additional data would enable LOLE/EUE linkage}
To connect stability proxies to adequacy outcomes (LOLE/EUE), the report identifies the following required data hooks:
\begin{itemize}
  \item disturbance/event logs with timestamps and event types,
  \item load-shed records and firm load interruptions,
  \item forced outage and derate series at relevant resolution,
  \item reserve deployment and performance series (by product),
  \item system operator constraint logs sufficient to reconstruct scarcity drivers.
\end{itemize}
Absent these, the report treats frequency-based comparisons as descriptive and operationally suggestive, not as adequacy proof.\footnote{This is consistent with standard reliability practice: adequacy is typically assessed with probabilistic LOLE/EUE models (or equivalent planning studies), while frequency relates primarily to operating reliability and control performance. See \cite{nerc_era_2023}.}

% -----------------------------------------------------------------------

\chapter{Market Power and Gaming Surface (Post-RTC+B)}
\label{ch:gaming_full}

This chapter characterizes the strategic surface introduced by real-time
co-optimization with binding intertemporal feasibility (RTC+B). The objective
is not to assert the presence of manipulation, but to (i) define what constitutes
gaming in an operational sense, (ii) distinguish strategic behavior from
physically grounded scarcity, and (iii) specify implementable monitoring rules
that can be executed using ERCOT-native data artifacts.

Throughout, we emphasize that scarcity pricing alone is not evidence of gaming.
Only price outcomes that are inconsistent with contemporaneous physical
conditions or feasibility constraints warrant concern.

\section{Threat model}
\label{sec:gaming_threat_model}

Under RTC+B, prices emerge from a multi-product co-optimization subject to
network, reserve, and intertemporal feasibility constraints. We define
\emph{gaming} as follows:

\begin{quote}
\textbf{Operational definition (gaming).} A strategic action constitutes gaming
if it alters market-clearing prices or procurement outcomes without a
corresponding change in underlying physical system feasibility.
\end{quote}

This definition explicitly excludes:
\begin{itemize}
  \item Rational intertemporal optimization by energy storage resources (ESRs),
  \item Scarcity arising from binding reserve or ramp constraints,
  \item Price volatility driven by load, weather, or outage realizations.
\end{itemize}

Conversely, the definition includes strategic actions that exploit:
\begin{itemize}
  \item Information asymmetries across products or time,
  \item Qualification or eligibility constraints unrelated to physical supply,
  \item Cross-product marginality migration induced by bid shaping.
\end{itemize}

The threat model is therefore \emph{mechanism-specific}: RTC+B expands the
dimensionality of marginality, which increases both the expressive power of
prices and the surface on which strategic behavior could, in principle, occur.

\section{SoC withholding and feasibility scarcity}
\label{sec:gaming_soc_withholding}

A frequently raised concern in storage-heavy systems is the possibility of
state-of-charge (SoC) withholding. Under RTC+B, this concern must be framed
carefully.

Let $s_{i,t}$ denote the state of charge of ESR $i$ at time $t$. Intertemporal
constraints imply:
\[
s_{i,t+1} = s_{i,t} + \eta_i^{\text{ch}} q^{\text{ch}}_{i,t}
           - \frac{1}{\eta_i^{\text{dis}}} q^{\text{dis}}_{i,t},
\]
with upper and lower bounds on $s_{i,t}$.

\paragraph{Mechanism.}
Low aggregate availability of reserves may arise either because:
\begin{enumerate}
  \item ESRs are physically depleted following prior dispatch, or
  \item ESRs strategically limit offered availability while retaining SoC.
\end{enumerate}

Only the second case constitutes gaming. Importantly, SoC itself is not
observable at the system level; therefore, detection must rely on \emph{indirect
signatures}.

\paragraph{Observable indicators.}
Potential red flags include:
\begin{itemize}
  \item Persistent withdrawal of ESR availability without preceding high
        discharge or charging activity,
  \item Elevated MCPCs coincident with flat energy prices and muted ramp stress,
  \item Abrupt reversals in availability that do not align with price signals.
\end{itemize}

These indicators are \emph{necessary but not sufficient} conditions. The chapter
explicitly avoids presuming intent; instead, it defines conditions under which
further investigation is warranted.

\section{Cross-product bid shaping}
\label{sec:gaming_cross_product}

RTC+B allows a resource to participate simultaneously in multiple ancillary
service (AS) products, each with its own requirement and clearing price. This
creates the possibility of \emph{cross-product bid shaping}.

\paragraph{Marginality migration.}
Let $\lambda_t$ denote the energy balance shadow price and $\mu_{k,t}$ the
multiplier on the requirement for ancillary product $k$. In a co-optimized
solution, scarcity rents may appear in any $\mu_{k,t}$, depending on which
constraint binds.

Strategic bid shaping can, in principle, encourage marginality to migrate:
\[
\text{Energy} \;\rightarrow\; \text{Regulation} \;\rightarrow\; \text{ECRS},
\]
even when underlying physical scarcity is modest.

\paragraph{Observable patterns.}
Empirically, this would manifest as:
\begin{itemize}
  \item Oscillation of marginal prices across AS products within short windows,
  \item Weak correlation between energy prices and MCPC tails,
  \item Repeated binding of the same product under similar system conditions.
\end{itemize}

Such patterns do not imply manipulation per se, but they define a surface where
strategic incentives exist and therefore merit monitoring.

\section{Qualification bottlenecks and administrative scarcity}
\label{sec:gaming_qualification}

Not all scarcity is physical. Under RTC+B, \emph{administrative scarcity} may
arise when qualification or eligibility constraints restrict effective supply.

Let $C^{\text{phys}}_{k,t}$ denote physically available capability for product
$k$, and $C^{\text{qual}}_{k,t}$ the subset that is qualified and eligible. Define
the administrative scarcity indicator:
\[
\text{AdminScarcity}_{k,t}
:= \mathbf{1}\!\left\{ C^{\text{qual}}_{k,t} \ll C^{\text{phys}}_{k,t} \right\}.
\]

MCPC spikes coincident with low $C^{\text{qual}}_{k,t}$ but ample physical
capability indicate scarcity driven by rules, telemetry, or qualification
bottlenecks rather than system stress.

This distinction is critical for policy interpretation: administrative scarcity
signals governance or process issues, not reliability risk.

\begin{figure}[ht]
    \centering
    \begin{tikzpicture}[font=\small\sffamily]
        % Axes
        \draw[thick] (0,0) -- (8,0);
        \draw[->, thick] (0,0) -- (0,5) node[left] {Basis Spread (\$/MWh)};

        % Bar 1: HB_NORTH
        \draw[fill=LinkBlue!25, draw=black!55] (1,0) rectangle (2, 1.5);
        \draw[thick, red!70!black] (1.5, 1.5) -- (1.5, 3.5); % Whisker to Tail
        \draw[thick, red!70!black] (1.3, 3.5) -- (1.7, 3.5); % Tail Cap
        \node[below] at (1.5, 0) {HB\_NORTH};
        
        % Bar 2: HB_SOUTH
        \draw[fill=LinkBlue!25, draw=black!55] (3,0) rectangle (4, 1.2);
        \draw[thick, red!70!black] (3.5, 1.2) -- (3.5, 4.2); % Whisker to Tail
        \draw[thick, red!70!black] (3.3, 4.2) -- (3.7, 4.2); % Tail Cap
        \node[below] at (3.5, 0) {HB\_SOUTH};

        % Bar 3: HB_WEST
        \draw[fill=LinkBlue!40, draw=black!55] (5,0) rectangle (6, 2.5); % Higher mean
        \draw[thick, red!70!black] (5.5, 2.5) -- (5.5, 4.8); % Whisker to Tail
        \draw[thick, red!70!black] (5.3, 4.8) -- (5.7, 4.8); % Tail Cap
        \node[below] at (5.5, 0) {HB\_WEST};

        % Legend/Annotation
        \draw[thick, red!70!black] (6.5, 3) -- (6.9, 3);
        \node[right] at (6.9, 3) {99th \%ile Tail};
        \draw[fill=LinkBlue!25, draw=black!55] (6.5, 2) rectangle (6.7, 2.2);
        \node[right] at (6.9, 2.1) {Median Basis};

    \end{tikzpicture}
    \caption{Basis Premium Tail Map. Distribution of locational basis spreads relative to the Hub. While median basis (bars) may remain low, tail risks (whiskers) reveal significant nodal fragmentation during stress events.}
    \label{fig:basis_map}
\end{figure}

\section{Detection rules and monitoring triggers}
\label{sec:gaming_detection}

To operationalize the above concepts, we define three monitoring rules suitable
for routine surveillance.

\paragraph{DR1: Persistence without physical stress.}
Trigger when MCPC for product $k$ exceeds its $Q_{0.99}$ threshold for more than
$N$ consecutive intervals while:
\begin{itemize}
  \item Net load is below the $90^{\text{th}}$ percentile,
  \item No major outages or ramp events are present.
\end{itemize}

\paragraph{DR2: Cross-product marginality oscillation.}
Trigger when the identity of the binding AS product alternates repeatedly within
a fixed window under stable system conditions.

\paragraph{DR3: Qualification-constrained scarcity.}
Trigger when MCPC tails coincide with:
\[
C^{\text{qual}}_{k,t} / C^{\text{phys}}_{k,t} < \kappa^{\text{admin}},
\]
for a pre-registered threshold $\kappa^{\text{admin}}$.

Each trigger is designed as an \emph{alert}, not an accusation. Escalation paths
may include enhanced data review, qualification audits, or targeted market
analysis.

\paragraph{Scope.}
This chapter does not claim that RTC+B induces gaming. Rather, it provides a
formal taxonomy and monitoring framework that allows regulators and system
operators to distinguish strategic behavior from legitimate scarcity outcomes
in a high-dimensional co-optimized market.

% ----------------------------------------------------------------------

\chapter{Stress Tests and Failure Modes of RTC+B}
\label{ch:failure_modes}

This chapter evaluates failure modes in RTC+B as an engineered system: how the
design can degrade or ``break'' under plausible stress conditions even in the
absence of manipulation. The emphasis is operational: each failure mode is
defined by (i) a physical or institutional mechanism, (ii) observable market and
frequency signatures, and (iii) stress-testable conditions.

The chapter is deliberately distinct from Chapter~\ref{ch:gaming_full}:
strategic behavior is not assumed. Instead, we analyze endogenous fragilities
that can arise from correlated behavior, binding intertemporal constraints, thin
qualified supply, and spatial deliverability limits.\footnote{
This framing parallels reliability engineering practice: failures may emerge
from correlated demand shocks, correlated resource derates, or control-system
limitations, even when every participant acts ``normally.'' The same logic
applies in electricity market designs where prices are dual variables of a
constrained optimization.}

\section{SoC synchronization risk}
\label{sec:failure_soc_sync}

A principal new fragility in storage-heavy systems is \emph{SoC synchronization}:
many ESRs acting on similar incentives can evolve into correlated state
trajectories, creating a system-level loss of feasible flexibility.

\subsection{Mechanism}
Consider ESRs indexed by $i \in \{1,\dots,n\}$ with state-of-charge $s_{i,t}$
evolving under efficiency-adjusted charging/discharging. Under RTC+B, ESRs are
incentivized by common price signals: energy spreads, ancillary MCPCs, and
performance requirements. When incentives are sufficiently aligned (e.g., common
forecast of scarcity windows), ESR dispatch and award decisions become
synchronized.

Define the aggregate feasible headroom for upward response as:
\[
H_t^{\uparrow} := \sum_{i=1}^n \min\{ \bar{p}_i,\, g(s_{i,t}) \},
\]
where $g(s_{i,t})$ maps SoC into deliverable upward capability subject to
sustainment constraints.\footnote{
In ERCOT AS products, feasibility depends not merely on instantaneous MW but on
sustainment/energy availability over the product's operational horizon. The
function $g(\cdot)$ abstracts those requirements; a detailed mapping would
require telemetry and qualification-specific parameters.}

SoC synchronization occurs when the joint distribution of $\{s_{i,t}\}_i$
becomes concentrated, reducing diversification. A stylized measure is the
cross-sectional correlation:
\[
\bar{\rho}_t := \frac{2}{n(n-1)} \sum_{i<j} \text{Corr}(s_{i,t}, s_{j,t}),
\]
or equivalently the decline of cross-sectional dispersion
$\text{Var}_i(s_{i,t})$.

High $\bar{\rho}_t$ implies that the system is prone to \emph{fleet-wide
feasibility loss}: when scarcity arrives, many resources are simultaneously
depleted or simultaneously committed, and the marginal flexibility disappears
nonlinearly.\footnote{
This is a portfolio effect: independent SoC trajectories provide diversification
and reduce tail risk of aggregate infeasibility. Correlated trajectories remove
that benefit and steepen scarcity tails. Similar arguments appear in resource
adequacy contexts where correlated renewable output or correlated thermal derates
increase EUE tail risk.}

\subsection{Expected signatures}
SoC synchronization risk is expected to manifest as:
\begin{itemize}
  \item MCPC tail clustering across multiple products (notably fast-response
        products) with persistence beyond single-interval shocks,
  \item Reduced efficacy of DA posture (Section~\ref{sec:wave3_dart_bridge}):
        ``long'' DA awards do not suppress RT MCPC tails during stress,
  \item Frequency proxy deterioration conditional on MCPC spikes, consistent with
        exhausted headroom.
\end{itemize}

\subsection{Stress test definition}
A practical stress test for synchronization uses episode conditioning:
\begin{enumerate}
  \item Identify Fern-like episodes (Section~\ref{sec:wave3_fern}),
  \item Condition on capability compression $\kappa_{k,t}$ (Section~\ref{sec:wave3_soc}),
  \item Test whether multi-product MCPC tails co-move more strongly under stress
        than in matched non-stress windows.
\end{enumerate}
A system that exhibits strong multi-product tail co-movement under stress is
consistent with correlated feasibility depletion.

\section{Ramp events with depleted headroom}
\label{sec:failure_ramp_depleted}

Ramp events are a canonical challenge in high-renewables systems: net load
changes rapidly when wind/solar output swings. Under RTC+B, ramp events become
dangerous when they coincide with depleted flexible headroom.

\subsection{Failure chain}
Define net load $NL_t := L_t - (W_t + PV_t)$ (or proxies thereof). A ramp event
occurs when $\Delta NL_t := NL_t - NL_{t-\ell}$ exceeds a threshold over horizon
$\ell$.\footnote{
In SCED-resolution work, $\ell$ should align with dispatch cadence (e.g., 5--15
minutes). Thresholds should be pre-registered and stress-binned to avoid
post hoc selection.}

The failure chain is:
\begin{enumerate}
  \item Large $\Delta NL_t$ requires rapid upward flexibility,
  \item A significant fraction of flexibility is committed to AS awards or is
        infeasible due to depleted SoC,
  \item Thermal units face ramp limits and may be unavailable or derated,
  \item RT scarcity appears primarily in ancillary products (high $\mu_{k,t}$)
        rather than in the energy price, because the binding constraints are
        reserve/ramp/feasibility constraints rather than energy balance.
\end{enumerate}

\subsection{Expected market signatures}
Under this failure mode, we expect:
\begin{itemize}
  \item Elevated MCPC in fast-response products with modest energy price
        movements,\footnote{
This is precisely the KKT-consistent signature: scarcity rents appear in the
dual multipliers of the binding constraints (reserve/ramp) rather than in the
energy balance multiplier.}
  \item Increased incidence of scarcity adders only after sustained stress
        (delayed ORDC activation), implying the system is short flexibility
        before it is short energy,
  \item Conditional frequency proxy degradation (higher $I_f$ and $N_{\epsilon}$)
        around the onset of MCPC spikes.
\end{itemize}

\subsection{Stress test design}
Define an event trigger based on ramp bins:
\[
\mathcal{E} := \{ t : \Delta NL_t \ge \Delta^\star \;\wedge\; \kappa_{k,t} < \kappa^\star \},
\]
and estimate event-time panels for MCPC, energy/adders, and frequency. Robustness
requires that signatures persist under SR-4 lead--lag variation and SR-2 tail
threshold variation.

\section{Thin product supply under qualification constraints}
\label{sec:failure_thin_supply}

Thin supply is a structural vulnerability: even when physical capability exists,
only a subset may be eligible or qualified for a given product. Under RTC+B,
thinness is particularly important because scarcity rents can localize in
product-specific MCPC tails.

\subsection{Mechanism}
Let $C^{\text{qual}}_{k,t}$ denote qualified supply for product $k$ and
$R_{k,t}$ the requirement. Define the qualified tightness ratio:
\[
\kappa^{\text{qual}}_{k,t} := \frac{C^{\text{qual}}_{k,t}}{R_{k,t}}.
\]
Thin supply corresponds to $\kappa^{\text{qual}}_{k,t}$ near 1 (or below), even
if physical capability is abundant.

This failure mode is not ``gaming''; it is an institutional bottleneck where
qualification, telemetry, or performance requirements limit participation.\footnote{
Markets frequently exhibit eligibility-driven scarcity (e.g., scarcity of
spinning reserves or regulation due to qualification constraints). The key
distinction is between physical scarcity and administrative scarcity; the latter
has different remedies (qualification expansion, telemetry upgrades, rule
simplification).}

\subsection{Expected signatures}
\begin{itemize}
  \item MCPC tails that correlate strongly with declines in
        $C^{\text{qual}}_{k,t}$ or award concentration,
  \item Elevated price volatility (high kurtosis, large CVaR) in the affected
        product even when energy volatility is modest,
  \item Increased episode sensitivity: leave-one-episode-out results become
        unstable because a small number of thin-supply intervals dominate tails.
\end{itemize}

\subsection{Stress test}
A minimal stress test conditions MCPC exceedances on
$\kappa^{\text{qual}}_{k,t}$ bins and compares tail probabilities across bins,
holding $X_t$ fixed.

\section{Congestion-local scarcity vs system-wide procurement}
\label{sec:failure_congestion_local}

System-wide procurement of ancillary services does not guarantee local
deliverability. Congestion and network constraints can create localized scarcity
that is not resolved by aggregate awards.\footnote{
This is a standard reliability distinction: capacity adequacy is not equivalent
to deliverability adequacy. Network constraints can render resources unable to
serve load or provide response where needed.}

\subsection{Mechanism}
Let the system be partitioned into zones or constraint regions. Even if total
system qualified capability exceeds requirement, deliverability constraints may
bind for certain interfaces, producing localized scarcity conditions.

Under RTC+B, this can appear as:
\begin{itemize}
  \item High MCPC for system-wide products when the marginal resource is
        deliverability-constrained,
  \item Divergence between hub prices and zonal prices, with localized stress
        not reflected in aggregate metrics,
  \item Weak relationship between DA posture and RT scarcity for certain regions.
\end{itemize}

\subsection{Expected signatures}
\begin{itemize}
  \item Price separation events: hub/zone divergence coincident with MCPC spikes,
  \item Episodes where scarcity adders do not activate yet frequency or local
        prices indicate stress,
  \item High variance of MCPC conditional on congestion proxies.
\end{itemize}

A rigorous implementation requires explicit congestion or constraint binding
data; where unavailable, this section remains diagnostic and motivates data
hooks in Chapter~\ref{ch:policy}.

\section{Mitigation levers}
\label{sec:failure_mitigations}

Mitigations should match the failure mode. We classify levers into monitoring,
incentives, transparency, and procurement tuning. The goal is to reduce
fragility without suppressing legitimate scarcity signals.

\subsection{Monitoring}
\begin{itemize}
  \item Fleet feasibility monitoring via capability compression ratios
        $\kappa_{k,t}$ and qualified tightness $\kappa^{\text{qual}}_{k,t}$,
  \item Synchronization diagnostics: rising co-movement of multi-product tails,
  \item Event-based dashboards keyed to the episode ledger (Chapter traceability).
\end{itemize}

\subsection{Performance incentives}
\begin{itemize}
  \item Align penalties with system value of response during stress episodes,
  \item Introduce stress-weighted performance scoring to discourage synchronized
        underperformance when response is scarce.\footnote{
A stress-weighted scoring regime reduces moral hazard in tail states by raising
the marginal value of performance when the system is near feasibility limits.}
\end{itemize}

\subsection{Transparency}
\begin{itemize}
  \item Publish qualified capability and concentration metrics for each product
        at relevant temporal granularity,
  \item Improve traceability of scarcity episodes by explicitly mapping which
        constraints are binding when MCPC tails occur.
\end{itemize}

\subsection{Procurement tuning}
\begin{itemize}
  \item Adjust AS demand curve parameters (ASDC) to reflect feasibility risk,
  \item Consider dynamic posture guidance (DA $\rightarrow$ RT) where DA awards
        are adjusted when feasibility compression indicators are elevated,
  \item Review qualification pathways to reduce thin-supply vulnerability.
\end{itemize}

\subsection{Scope and falsifiability}
Mitigation proposals are contingent on empirical signatures. If Wave~3 results
show strong DA-to-RT transmission and weak multi-product tail co-movement under
stress, then synchronization risk may be limited and mitigations should focus
elsewhere. Conversely, if feasibility compression dominates, mitigations should
prioritize transparency and qualification expansion rather than price
suppression.


\begin{figure}[ht]
    \centering
    \begin{tikzpicture}[font=\small\sffamily]
        % Axes
        \draw[thick] (0,0) -- (7,0); % X Axis
        \draw[->, thick] (0,0) -- (0,6) node[above] {Net Revenue ($\Pi$)};
        
        % Y Ticks
        \foreach \y in {1,2,3,4,5} \draw (0,\y) -- (-0.1,\y);

        % Scenario 1: Low Degradation Cost
        % P10 (Bottom), P50 (Mid), P90 (Top)
        \filldraw[fill=red!18, draw=black!70] (1,1) rectangle (2,2.5); % P10
        \filldraw[fill=LinkBlue!22, draw=black!70] (1,2.5) rectangle (2,4.5); % P50
        \filldraw[fill=orange!28, draw=black!70, postaction={pattern=north east lines, pattern color=black!35}] (1,4.5) rectangle (2,5.5); % P90
        \node[below] at (1.5,0) {Low $c_{deg}$};

        % Scenario 2: Mid Degradation Cost
        \filldraw[fill=red!18, draw=black!70] (3,0.8) rectangle (4,2.0); % P10
        \filldraw[fill=LinkBlue!22, draw=black!70] (3,2.0) rectangle (4,3.8); % P50
        \filldraw[fill=orange!28, draw=black!70, postaction={pattern=north east lines, pattern color=black!35}] (3,3.8) rectangle (4,4.5); % P90
        \node[below] at (3.5,0) {Base $c_{deg}$};

        % Scenario 3: High Degradation Cost
        \filldraw[fill=red!18, draw=black!70] (5,0.2) rectangle (6,1.2); % P10
        \filldraw[fill=LinkBlue!22, draw=black!70] (5,1.2) rectangle (6,2.5); % P50
        \filldraw[fill=orange!28, draw=black!70, postaction={pattern=north east lines, pattern color=black!35}] (5,2.5) rectangle (6,3.0); % P90
        \node[below] at (5.5,0) {High $c_{deg}$};
        
        % Labels
        \node[anchor=west] at (7, 5) {P90 (Upside)};
        \node[anchor=west] at (7, 3) {P50 (Median)};
        \node[anchor=west] at (7, 1) {P10 (Risk)};
        
        % Connectors
        \draw[dashed, gray] (2,5.5) -- (5,3.0);
        \node[above, rotate=-15] at (3.5, 4.3) {Feasibility Truncation};

    \end{tikzpicture}
    \caption{Degradation Envelope. Sensitivity of P10/P50/P90 revenue outcomes to degradation cost assumptions ($c_{deg}$). Higher degradation costs strictly truncate the P90 upside by making high-cycling strategies unprofitable.}
    \label{fig:degradation_envelope}
\end{figure}

% ----------------------------------------------------------------------

\chapter{Investor Economics: Canonical Battery Valuation Under RTC+B}
\label{ch:investor_full}

This chapter provides a lender-grade valuation framework for ERCOT energy storage
resources (ESRs) under RTC+B. ``Lender-grade'' here means: (i) explicit cashflow
decomposition with auditable settlement drivers, (ii) feasibility constraints
that bind revenues in tail states, (iii) risk premia that reflect non-Gaussian
returns and regime shifts, and (iv) a data room specification sufficient for
bankability diligence.

The central claim is not that RTC+B increases or decreases battery value
unconditionally, but that it \emph{re-weights} value toward (a) ancillary-product
tails and (b) operational feasibility in stress episodes. Consequently, any
valuation approach that treats realized ancillary revenues as i.i.d. draws or
that ignores state-of-charge (SoC), degradation, and performance penalties will
misstate both expected value and downside risk.\footnote{
This is the same structural caution that applies in any market where scarcity
rents dominate outcomes: sample averages are unstable, and downside states are
governed by constraints rather than smooth marginal conditions. See the risk
discussion in \citep{mcneil_frey_embrechts_2015} for heavy-tailed loss modeling
and coherent tail risk measures.}

\section{Canonical ESR configurations and constraints}
\label{sec:investor_configs}

We define canonical ESR configurations using a minimal parameter set that maps
directly to settlement and operational feasibility:
\[
\theta :=
\left( \bar{P},\, \bar{E},\, \eta^{\mathrm{rt}},\, \dot{E}_{\max},\, c^{\mathrm{deg}},
\, \pi^{\mathrm{pen}},\, \mathcal{Q} \right),
\]
where $\bar{P}$ is power capacity (MW), $\bar{E}$ is usable energy (MWh),
$\eta^{\mathrm{rt}}$ is round-trip efficiency, $\dot{E}_{\max}$ is throughput
limit (MWh per day or equivalent), $c^{\mathrm{deg}}$ is a degradation cost
parameter, $\pi^{\mathrm{pen}}$ is a performance penalty schedule proxy, and
$\mathcal{Q}$ denotes qualification/telemetry status by product.

\subsection{Power and duration}
Duration is defined as $D := \bar{E}/\bar{P}$ (hours). In RTC+B, duration affects
not only energy arbitrage but the feasibility of sustaining certain ancillary
products, particularly under clustered deployments.\footnote{
Even where an ancillary product is defined in MW terms, real deliverability is
energy-limited for ESRs. Market rules address this via qualification and
performance requirements; the engineering reality is the SoC constraint.}

\subsection{Efficiency and charging requirements}
Let $q^{\mathrm{ch}}_t$ and $q^{\mathrm{dis}}_t$ denote charging and discharging
power. A standard SoC transition is:
\[
s_{t+1} = s_t + \eta^{\mathrm{ch}} q^{\mathrm{ch}}_t \Delta t
            - \frac{1}{\eta^{\mathrm{dis}}} q^{\mathrm{dis}}_t \Delta t,
\qquad 0 \le s_t \le \bar{E}.
\]
We take $\eta^{\mathrm{rt}} \approx \eta^{\mathrm{ch}}\eta^{\mathrm{dis}}$ and
treat charging energy as a cash cost linked to settlement point prices. This
explicitly ties feasibility to both prices and physical constraints.

\subsection{Throughput, cycling, and degradation}
Degradation is modeled as a cost increasing in throughput. A minimal linearized
form is:
\[
\Pi_{\mathrm{Degradation}} = c^{\mathrm{deg}} \sum_t
\left(q^{\mathrm{ch}}_t + q^{\mathrm{dis}}_t\right)\Delta t,
\]
with $c^{\mathrm{deg}}$ interpreted as marginal $/MWh$ of throughput. This
captures the first-order fact that ancillary participation can increase cycling
intensity relative to pure arbitrage.\footnote{
For detailed degradation models, including dependence on depth-of-discharge,
temperature, and C-rate, see canonical battery degradation and techno-economic
references such as \citep{schmidt_etal_2019,keith_etal_2022}. For lender-grade
work, it is common to start with a conservative $c^{\mathrm{deg}}$ and stress it
in sensitivity.}

\section{Revenue decomposition and feasibility}
\label{sec:investor_decomposition}

We decompose net operating profit over horizon $\mathcal{T}$ as:
\[
\Pi = \Pi_{\text{Energy}} + \Pi_{\text{AS}}
      - \Pi_{\text{Charging}} - \Pi_{\text{Degradation}} - \Pi_{\text{PerformanceRisk}}
      - \Pi_{\text{Other}},
\]
where $\Pi_{\text{Other}}$ may include uplift-like settlement adjustments, fixed
O\&M, and congestion-related basis effects if modeled.

\subsection{Energy revenue and charging cost}
Let $\pi_t$ be the relevant settlement price for energy at the modeled node/hub
(consistent with your artifact definitions). Define discharging and charging
quantities as above. Then:
\[
\Pi_{\text{Energy}} = \sum_{t\in\mathcal{T}} \pi_t\, q^{\mathrm{dis}}_t \Delta t,
\qquad
\Pi_{\text{Charging}} = \sum_{t\in\mathcal{T}} \pi_t\, q^{\mathrm{ch}}_t \Delta t.
\]
This formulation is intentionally settlement-primitive: it avoids reliance on
``spread'' heuristics and forces explicit feasibility through $s_t$.

\subsection{Ancillary revenue}
Let $k$ index ancillary products (e.g., Reg-Up, Reg-Down, RRS, ECRS, FRRS if
applicable). Let $A_{k,t}$ be awarded MW and $\text{MCPC}_{k,t}$ be the market
clearing price for capacity. Then:
\[
\Pi_{\text{AS}} = \sum_{t\in\mathcal{T}} \sum_{k}
\text{MCPC}_{k,t} \, A_{k,t} \Delta t
\;-\;
\Pi_{\text{AS-ops}},
\]
where $\Pi_{\text{AS-ops}}$ includes opportunity costs and deployment energy
impacts where settlement requires it.\footnote{
The exact settlement mapping varies by product and protocol; the report’s
artifact-first method requires that any AS settlement term correspond to an
auditable ERCOT settlement artifact. See the data chapter’s artifact class
definitions and the traceability rule.}

\subsection{Feasibility and opportunity costs}
Feasibility enters via SoC and participation constraints. A stylized constraint
for product $k$ is:
\[
A_{k,t} \le \min\{\bar{P},\, \phi_k(s_t)\},
\]
where $\phi_k(s_t)$ is a product-specific SoC feasibility mapping (reflecting
sustainment). When $\phi_k(s_t)$ binds in stress episodes, ESRs may be unable to
capture MCPC tails, even if prices are high. This is the critical point: tail
prices are not automatically monetizable.\footnote{
This is the operational analog of ``scarcity rents are constrained by physical
deliverability.'' In constrained optimization language, scarcity rents appear in
dual variables, but revenue realization requires primal feasibility.}

\subsection{Performance risk and penalties}
Performance risk is modeled as expected penalties (or lost payments) associated
with non-performance or telemetry/qualification issues:
\[
\Pi_{\text{PerformanceRisk}} :=
\sum_{t\in\mathcal{T}} \sum_{k}
\mathbb{E}\!\left[ \pi^{\mathrm{pen}}_{k,t}\cdot \mathbf{1}\{\text{short}_{k,t}>0\}
\mid X_t \right],
\]
where $\text{short}_{k,t}$ is a shortfall metric and $X_t$ are system conditions.
We deliberately write this in expectation form because the severity of penalties
is typically state-dependent (most punitive in stress states).\footnote{
From a lender perspective, performance risk is not optional modeling detail:
it is a credit driver when merchant tails dominate revenue.}

\section{Risk: tail dependence and regime shifts}
\label{sec:investor_risk}

Battery returns in ERCOT are generally non-Gaussian: they exhibit skewness,
excess kurtosis, and strong dependence on rare stress regimes. Two structural
sources drive this: (i) scarcity rents concentrated in tails (especially MCPC
tails under RTC+B), and (ii) feasibility truncation in those same tails.

\subsection{Tail dependence between revenues and stress}
Let $R$ denote period profit and let $Z$ denote a stress index (e.g., Fern-like
episode indicator, reserve tightness, or capability compression $\kappa_{k,t}$).
We care about:
\[
\Pr(R \le r \mid Z \in \text{stress}),
\qquad
\Pr(\text{MCPC}_{k,t}>\tau_k \mid \kappa_{k,t}<\kappa^\star, X_t),
\]
because downside risk can rise precisely when prices rise (feasibility binds).
This is a nontrivial feature of intertemporal resources: tails can be lucrative
and dangerous simultaneously.

A coherent tail risk metric is Conditional Value-at-Risk (CVaR):
\[
\mathrm{CVaR}_\alpha(R) := \mathbb{E}\!\left[R \mid R \le \mathrm{VaR}_\alpha(R)\right],
\]
which lenders often prefer over variance when distributions are heavy-tailed or
skewed.\footnote{
See \citep{mcneil_frey_embrechts_2015} for formal definitions and estimation
approaches for VaR/CVaR under heavy tails.}

\subsection{Regime-switching structure}
Because RTC+B changes where scarcity rents appear, a natural model is a
regime-switching process:
\[
S_t \in \{ \text{normal},\text{tight},\text{stress} \},
\quad
\Pr(S_{t+1}=j \mid S_t=i)=p_{ij}.
\]
Conditional on $S_t$, distributions of $\pi_t$ and $\text{MCPC}_{k,t}$ differ
materially. The investor question is not simply $\mathbb{E}[\Pi]$, but the
mixture distribution implied by $(p_{ij})$ and state-conditional revenues:
\[
\Pi \sim \sum_{s} \Pr(S=s)\, \Pi \mid (S=s).
\]
This formalizes the idea that a single ``average year'' backtest is structurally
misleading when the weight on stress states changes over time (load growth,
renewables penetration, qualification changes).\footnote{
The counterfactual chapter already warns against naive pre/post comparisons; the
same logic applies here to merchant valuation.}

\section{Bankability: what data lenders will demand}
\label{sec:investor_bankability}

A lender’s underwriting posture depends less on point forecasts and more on
demonstrable settlement provenance, controllability of revenues, and downside
containment. A bankability-ready data room typically demands:

\subsection{Awards and settlement traceability}
\begin{itemize}
  \item DA AS awards by product and day (e.g., DAMASAGG-derived $Q^{DA}_{k,d}$),
  \item RT awards/obligations and settlement prices (MCPC series) with product
        definitions fixed by the metric dictionary,
  \item Node/hub settlement price history for energy charging/discharging.
\end{itemize}
All revenues must be reconcilable to ERCOT settlement artifacts, not dashboard
visualizations.\footnote{
This mirrors the report’s artifact discipline: ``dashboard'' supports
situational awareness; ``settlement-grade'' supports claims and underwriting.}

\subsection{Telemetry and performance}
\begin{itemize}
  \item Performance scores and penalty incidence for each product,
  \item Evidence of sustained qualification status and telemetry compliance,
  \item Event-conditioned performance in stress episodes (Fern packet logic).
\end{itemize}

\subsection{Operational constraints and degradation evidence}
\begin{itemize}
  \item OEM warranty and throughput constraints ($\dot{E}_{\max}$),
  \item Degradation assumptions tied to chemistry and duty cycle,
  \item Heat-rate or efficiency test evidence where relevant.
\end{itemize}

\subsection{Settlement treatment and rule stability}
\begin{itemize}
  \item Protocol references governing settlement, performance penalties,
        and qualification requirements,\footnote{
For policy risk, lenders typically require a documented change-log and a
plausible view of how often protocols can change, and how quickly changes feed
through to settlement.}
  \item Evidence that modeled price components map to realized settlement
        (measurement invariance checks in Wave~3-A).
\end{itemize}

\section{P10/P50/P90 framing}
\label{sec:investor_p1090}

We report valuation outputs as scenario-based distributions, not point estimates.
Let $\Pi^{(m)}$ denote profit under scenario $m$ drawn from a scenario ensemble
(Chapter~\ref{ch:2035_full}) combined with sensitivity levers (SR grid). Define:
\[
\text{P10} := Q_{0.10}(\Pi),\quad
\text{P50} := Q_{0.50}(\Pi),\quad
\text{P90} := Q_{0.90}(\Pi).
\]

\subsection{Interpretation}
\begin{itemize}
  \item P50 is the median merchant outcome under the scenario ensemble.
  \item P10 is a downside tail outcome (stress-heavy or feasibility-binding).
  \item P90 is an upside tail outcome (high scarcity rents with feasible capture).
\end{itemize}
Because RTC+B shifts scarcity rents toward ancillary products, P10/P90 spreads
should be expected to widen unless mitigation or contracting reduces exposure.

\subsection{Reporting discipline}
P10/P50/P90 outputs are reported alongside:
\begin{itemize}
  \item CVaR$_\alpha$ for at least one downside level (e.g., $\alpha=0.05$),
  \item Contribution decomposition: share of variance attributable to
        (i) MCPC tails, (ii) energy spreads, (iii) feasibility truncation,
  \item Traceability: the episode set and SR levers used to generate the
        distribution.
\end{itemize}

\subsection{Scope}
This chapter provides the accounting and risk machinery, not a claim that RTC+B
is welfare-improving or investor-favorable in aggregate. Those conclusions
require combining this machinery with the counterfactual and reliability cost
frameworks and demonstrating robustness across the sensitivity grid.

% ----------------------------------------------------------------------

\chapter{2035 Outlook: Scenario Ensemble and Monitoring Plan}
\label{ch:2035_full}
This chapter is a scenario-based planning outlook rather than a point forecast. The objective is to translate the empirical and theoretical mechanisms developed earlier (scarcity migration across products; feasibility compression; partial identification under limited overlap) into an implementable monitoring plan that can be re-run as the system evolves toward 2035. The outlook is written to be compatible with ERCOT planning and stakeholder workflows: it specifies scenario families, expected distributional shifts, and concrete update triggers (data releases, rule changes, qualification shifts) rather than narrative speculation.\footnote{A scenario-ensemble framing is standard when outcomes are tail-driven and the relevant physical and institutional states are non-stationary. In our context, the combination of weather regimes, evolving resource mix, and changing qualification constraints makes single-regime extrapolation unreliable.}

\section{Scenario families}
We define a set of scenario families \(\mathcal{S}=\{S_1,\dots,S_J\}\) that span the principal drivers of scarcity formation under RTC+B. Each family is parameterized by a vector \(\theta_j\) of measurable drivers (load, net-load ramps, thermal availability, renewable penetration, storage saturation, qualification/telemetry constraints, and congestion regime). The purpose is to structure a repeatable set of ``stress lenses'' that can be updated quarterly and compared year-over-year.

\subsection{Load growth and large-load uncertainty}
Let \(L_t\) denote system demand (or a suitable proxy) and let \(\Delta L_t\) denote an intra-day ramp measure (e.g., \(L_t-L_{t-1}\) at SCED resolution). We represent load growth uncertainty by a parametric envelope:
\[
L_t = L_t^{(0)}\,(1+g)^{y(t)} + \varepsilon_t, \qquad g\in[g_{\min},g_{\max}],
\]
where \(L_t^{(0)}\) is a baseline year profile, \(y(t)\) maps timestamps to fractional years, and \(\varepsilon_t\) captures weather- and outage-driven deviations. We emphasize that scarcity under RTC+B is driven not only by \(L_t\) but by the joint state \((L_t,\Delta L_t,\text{AS requirements},\text{qualified supply},\text{fleet feasibility})\). Accordingly, scenario families treat load growth as a driver of both higher mean stress and thicker tails in ramp-conditioned intervals.\footnote{ERCOT's own emerging-large-load analysis and planning materials emphasize that the uncertainty is not only magnitude but geographic clustering and temporal profile, which can change congestion and deliverability regimes.}

\subsection{Renewables ramps and the net-load distribution}
Let \(W_t\) and \(S_t\) denote wind and PV output proxies. Define net load \(N_t := L_t-(W_t+S_t)\). A core driver for ancillary scarcity is the distribution of \(\Delta N_t\), particularly in transition hours (solar down-ramp; wind ramp events). The scenario family varies \((\mathbb{E}[W_t],\mathbb{E}[S_t],\text{covariance structure},\text{ramp tails})\) rather than only annual penetration.

\subsection{Thermal derates and fuel-availability stress}
Thermal performance enters both scarcity probability and the ``shape'' of scarcity (energy-side vs ancillary-side) because derates can alter the binding constraint set under co-optimization. We parameterize a thermal availability multiplier \(\eta_t\in(0,1]\) applied to dispatchable capability, with stress regimes defined by persistence and correlation with ramps:
\[
\eta_t = \bar{\eta} + u_t, \qquad u_t\text{ persistent under cold/heat regimes and correlated with }\Delta N_t.
\]
A practical baseline for scenario weighting is the observed generation mix and its plausible evolution. As a descriptive baseline, the 2025 ERCOT energy mix (share computed from the uploaded fuel summary artifact) is:
\begin{itemize}
\item Gas-CC: 32.7\% of 2025 energy (share computed from \texttt{IntGenbyFuel2025.xlsx}).
\item Wind: 23.3\% of 2025 energy (share computed from \texttt{IntGenbyFuel2025.xlsx}).
\item Solar: 13.7\% of 2025 energy (share computed from \texttt{IntGenbyFuel2025.xlsx}).
\item Coal: 12.8\% of 2025 energy (share computed from \texttt{IntGenbyFuel2025.xlsx}).
\item Nuclear: 8.5\% of 2025 energy (share computed from \texttt{IntGenbyFuel2025.xlsx}).
\item Gas: 7.9\% of 2025 energy (share computed from \texttt{IntGenbyFuel2025.xlsx}).
\item Other: 1.0\% of 2025 energy (share computed from \texttt{IntGenbyFuel2025.xlsx}).
\item Hydro: 0.1\% of 2025 energy (share computed from \texttt{IntGenbyFuel2025.xlsx}).
\item Biomass: 0.1\% of 2025 energy (share computed from \texttt{IntGenbyFuel2025.xlsx}).
\end{itemize}
This baseline is used only to parameterize scenario narratives; it is not used as a structural model of reliability.

\subsection{Storage saturation, qualification constraints, and fleet-feasibility regimes}
Let \(B_t\) denote the aggregate storage fleet capability and \(\mathcal{Q}\) the set of qualified providers per ancillary product \(k\). As storage penetration rises, two countervailing effects are plausible:
\begin{itemize}
  \item \textbf{Flexibility expansion:} higher \(B_t\) increases capability, reducing the frequency of requirement binding and thinning MCPC tails.
  \item \textbf{Feasibility synchronization:} correlated charging and correlated depletion can compress feasible capability in common intervals, producing synchronized scarcity even if nameplate \(B_t\) is large. This shows up as ``capability compression'' and cross-product co-movement (Wave~3-E).
\end{itemize}
The scenario family explicitly varies (i) storage saturation, (ii) telemetry/qualification bottlenecks, and (iii) performance/penalty regimes, since these can create non-physical scarcity channels.\footnote{This is the point of separating the \emph{failure-modes} chapter from the \emph{gaming} chapter: synchronized feasibility loss can occur even in the absence of strategic conduct.}

\subsection{Congestion regime shifts and local deliverability}
Even with system-wide procurement, local deliverability constraints can bind in SCED and shift scarcity signals into local prices or into product MCPCs depending on procurement and settlement design. We define congestion regimes by a low-dimensional state \(G_t\) (e.g., indicator of major interfaces binding, or congestion index). Under regime shifts (new load pockets; new transmission; retirements), the mapping from system-wide scarcity to nodal scarcity changes. Scenario families therefore include ``deliverability states'' rather than treating the system as copper-plate.

\section{Expected distribution shifts per scenario}
The measurable output of interest is not a single price level but the distributional behavior of: (i) energy-side prices \(\lambda_t^{\mathrm{LMP}}\), (ii) settlement prices \(\pi_t^{\mathrm{SPP}}\), (iii) adders \(a_t^{\mathrm{RT}}\), and (iv) ancillary MCPCs \(\text{MCPC}_{k,t}\). For each scenario \(S_j\), we pre-register a set of distributional predictions in terms of quantiles, exceedance rates, clustering, and cross-product co-movement:
\[
\mathcal{D}(S_j) := \left\{Q_p(\lambda),\;Q_p(a),\;Q_p(\text{MCPC}_k),\;\Pr(\text{tail events}),\;\text{cluster stats},\;\text{conditional exceedance}\right\}.
\]
We emphasize that predictions are conditional: the same design can appear ``better'' or ``worse'' depending on which stress bins dominate the realized year.

\subsection{Fern packet as a tail-regime anchor}
Because extreme events dominate welfare and investor outcomes, we treat Fern-like stress packets as an explicit scenario anchor (Wave~3-D). The practical value of this anchor is that it provides a repeatable episode definition against which distributional migration can be evaluated over time (e.g., comparing 2026 Fern-like packets to future winter stress packets).

Using the uploaded total-ancillary capability artifact (Dec 2025 and Jan 2026), we can already report descriptive capability distributions during the author-identified Fern window (2026-01-22 to 2026-01-29). These are not requirements-normalized \(\kappa\) ratios (requirements are not present in the capability artifact); rather, they provide an empirical baseline for feasible supply magnitude in the same time window:
\begin{itemize}
\item \textbf{Reg-Up}: $Q_{0.01}=9,103$ MW, $Q_{0.05}=11,177$ MW, $Q_{0.50}=21,548$ MW, $Q_{0.95}=23,246$ MW, $Q_{0.99}=23,818$ MW.
\item \textbf{Reg-Down}: $Q_{0.01}=5,515$ MW, $Q_{0.05}=6,408$ MW, $Q_{0.50}=11,541$ MW, $Q_{0.95}=21,394$ MW, $Q_{0.99}=22,652$ MW.
\item \textbf{RRS}: $Q_{0.01}=11,904$ MW, $Q_{0.05}=13,216$ MW, $Q_{0.50}=18,509$ MW, $Q_{0.95}=20,753$ MW, $Q_{0.99}=21,048$ MW.
\item \textbf{ECRS}: $Q_{0.01}=10,326$ MW, $Q_{0.05}=12,172$ MW, $Q_{0.50}=20,877$ MW, $Q_{0.95}=24,774$ MW, $Q_{0.99}=26,453$ MW.
\item \textbf{Non-Spin}: $Q_{0.01}=13,527$ MW, $Q_{0.05}=16,292$ MW, $Q_{0.50}=19,942$ MW, $Q_{0.95}=24,453$ MW, $Q_{0.99}=25,226$ MW.
\end{itemize}
These distributions are used in Wave~3-E as the denominator of feasibility compression proxies and to design stress-bin thresholds.

\subsection{Investor-return distribution shifts}
Under RTC+B, the revenue distribution for storage is expected to become more sensitive to (i) MCPC tails and (ii) the joint state of feasibility and qualification. Therefore, rather than report a single expected return, we pre-register a scenario-conditional return distribution \(\Pi\mid S_j\) with explicit tail-risk measures:
\[
\text{VaR}_\alpha(\Pi\mid S_j),\qquad \text{CVaR}_\alpha(\Pi\mid S_j),\qquad \text{TailDep}(\Pi_\text{AS},\Pi_\text{Energy}\mid S_j).
\]
The emphasis is on bankability-relevant objects (P10/P50/P90 and sensitivity to tail regimes) rather than average revenue.

\section{Monitoring signals and update triggers}
The monitoring plan defines a set of leading indicators \(Z_t\) and re-run triggers that require the report to be refreshed. We organize indicators into three classes aligned with the report's evaluation criteria: system cost, reliability, and investor returns.

\subsection{Indicators}
We define a monitoring vector:
\[
Z_t := \big(\underbrace{Q^{DA}_{k,d}\,/\,\widehat{R}^{DA}_{k,d}}_{\text{DA posture}},\;
\underbrace{\text{MCPC}_{k,t}}_{\text{AS scarcity}},\;
\underbrace{a_t^{\mathrm{RT}}}_{\text{energy adders}},\;
\underbrace{\widehat{\kappa}_{k,t}}_{\text{feasibility proxy}},\;
\underbrace{I_f,\;N_\epsilon}_{\text{frequency proxies}},\;
\underbrace{G_t}_{\text{congestion state}}\big).
\]
Here \(\widehat{R}^{DA}\) and \(\widehat{\kappa}\) denote requirement- and feasibility-normalized objects that require additional ERCOT requirement artifacts (see ``Wave~3 Data Hooks'' below). Where true requirements are not observable in the current artifact set, we explicitly label the indicator as a proxy and report its limitations.

\subsection{Re-run triggers}
A re-run is triggered when any of the following occurs:
\begin{itemize}
  \item \textbf{Design/regulatory change:} NPRR implementation updates or new ASDC parameter postings (e.g., posted demand curves used by DAM/SCED).\footnote{ERCOT posts DAM and SCED ancillary service demand curves as a public data product (EMIL ID NP4-212-CD); see \citep{ercot_asdc_overview_2024,ercot_nprr1268_asdc_mod_2025}.}
  \item \textbf{Qualification regime shift:} significant change in qualified supply counts or performance/penalty rules affecting effective capability.
  \item \textbf{Tail-regime event:} a Fern-like packet occurs (episode criteria met) or a comparable summer stress packet occurs; this triggers a packet refresh and re-estimation of episode-conditioned statistics.
  \item \textbf{Data definition change:} changes to price-adder composition or settlement definitions (e.g., changes in what is included in SPP vs LMP displays).\footnote{ERCOT documents real-time price adders and related settlement definitions in its ORDC methodology and reliability-deployment adder filings; see \citep{ercot_ordc_methodology_2015,ercot_nprr1214_rdpa_2023}.}
\end{itemize}

\subsection{Wave~3 data hooks needed for full monitoring}
To make \(\kappa_{k,t}=C_{k,t}/R_{k,t}\) operational (Wave~3-E) and to complete DA\(\to\)RT posture transmission tests (Wave~3-C), the following additional ERCOT artifacts are required:
\begin{itemize}
  \item Interval-level ancillary \emph{requirements} \(R_{k,t}\) by product (SCED resolution) or an hour-level requirement series with an explicit mapping rule to SCED intervals.
  \item Posted AS demand curves (ASDC) by hour and product (NP4-212-CD), to link price formation to the scarcity functional actually in force.\footnote{See \citep{ercot_asdc_overview_2024,ercot_nprr1268_asdc_mod_2025} for ASDC context and documentation.}
  \item Any ERCOT-provided time series of qualified supply counts (or, failing that, a proxy based on award/offer data) to test qualification-bottleneck scarcity.
\end{itemize}
When these hooks are not available, conclusions are explicitly bounded (partial identification) rather than extrapolated.

\section{Policy/planning implications by scenario}
Implications are scenario-conditioned. The same observed migration (energy tails thinning while MCPC tails thicken) can indicate welfare improvement (better targeted procurement) or fragility (scarcity moved into thinner products) depending on which scenario family is realized.

\subsection{Implication classes}
We categorize implementable implications into four classes:
\begin{enumerate}
  \item \textbf{Procurement tuning:} adjust ASDC parameters and procurement posture rules to manage tail risk while controlling total system cost.
  \item \textbf{Qualification expansion:} reduce administrative scarcity by increasing the set \(\mathcal{Q}\) of qualified providers where physically and operationally justified.
  \item \textbf{Transparency and traceability:} publish sufficient artifact detail (awards, requirements, adders) to enable independent verification under the ledger-to-figure rule.
  \item \textbf{Reliability guardrails:} embed stress tests (SoC synchronization; ramp events under depleted headroom) into operational monitoring and market-design review.
\end{enumerate}

\subsection{What would change the outlook}
This outlook is explicitly conditional and will be revised if any of the following are observed:
\begin{itemize}
  \item Persistent divergence between DA posture and RT scarcity (Wave~3-C fails systematically) indicating a dominant feasibility regime.
  \item Increasing frequency of qualification-driven MCPC spikes (Wave~3 diagnostic DR rules firing) indicating administrative scarcity.
  \item Evidence that frequency proxies degrade conditional on high-MCPC events (Wave~2/3 diagnostics), suggesting that scarcity pricing is not purchasing the intended control quality.
\end{itemize}

\chapter{Regulatory Summary (PUCT-Facing)}
\addcontentsline{toc}{chapter}{Regulatory Summary (PUCT-Facing)}
\label{ch:puct_summary}

This report evaluates early post-implementation evidence from ERCOT's Real-Time
Co-optimization with Batteries (RTC+B) and associated scarcity-price formation
changes, including ORDC/RDP adder incidence and ancillary-service scarcity
pricing via MCPC tails. The analysis is deliberately conservative: it separates
what is directly readable from posted price series from what requires cleared
awards, deployments, and feasibility telemetry (state-of-charge and performance).
This firewall is formalized as an ``identification ladder'' (see the Identification and Inference Limits chapter).

\section*{What this report can conclude now (and what it cannot)}
\noindent\textbf{Admissible findings (descriptive; price-level).}
Using settlement-grade price postings, we can credibly document:
(i) distributional shifts in real-time energy settlement prices and posted adders,
(ii) ancillary MCPC tail behavior by product (Reg-Up/Reg-Down/RRS/ECRS, and FRRS
when present),
(iii) co-movement and conditional exceedance rates between energy-side and
ancillary-side scarcity prices, and (iv) the event-time shape of price behavior
around pre-registered scarcity episodes. These are descriptive facts about posted
price processes, not welfare or reliability proofs.

\noindent\textbf{Non-admissible conclusions without additional artifacts.}
Price tails alone cannot establish welfare improvements, procurement efficacy,
market power, or reliability causality. Such claims require at minimum cleared
quantities (awards), and for stronger statements, deployments and feasibility
telemetry. Where those artifacts are not yet available, we provide partial
identification bounds (see the Identification and Inference Limits chapter) rather than point claims.

\section*{Why RTC+B can move scarcity prices across products}
RTC+B is a multi-product co-optimization: scarcity value is not required to show
up in a single energy-side adder. When ancillary requirements bind, scarcity can
manifest directly as ancillary MCPC spikes even when energy prices or posted
adders are modest. This is not inherently ``good'' or ``bad'': it changes the
distribution of scarcity rents and the operational locus of scarcity pricing.
The report’s ORDC–ASDC bridge chapter (Chapter \ref{ch:ordc_asdc}) formalizes
conditions under which ORDC-style scarcity and product-specific scarcity might be
equivalent, and why those conditions tend to fail when intertemporal feasibility
(state-of-charge) becomes binding.

\section*{Reliability and consumer risk: what the regulator should monitor}
Because reliability outcomes are not identified from prices alone, the report
recommends a monitoring plan that ties market outcomes to operational artifacts:
\begin{itemize}
  \item \textbf{Awards and posture:} whether DA posture predicts RT scarcity
  incidence (Wave 3 DA geometry bridge).
  \item \textbf{Deployments:} whether high MCPC events correspond to real
  control actions and measurable frequency deviations.
  \item \textbf{Feasibility telemetry:} whether scarcity episodes co-occur with
  fleet feasibility compression consistent with SoC depletion or synchronization.
\end{itemize}

\section*{Market power and fragility: the two distinct concerns}
This report distinguishes \emph{gaming} from \emph{failure modes}. Gaming is
strategic behavior exploiting incentive gradients; failure modes are structural
breaks driven by correlated feasibility loss, thin qualification, or congestion
deliverability. Strong conclusions on either require at least awards and often
deployments/telemetry; however, early-warning triggers can be specified now using
episode-based diagnostics and cross-product tail co-movement.

\section*{Near-term ERCOT/PUCT implementable actions (data-first)}
The immediate regulatory value is not a premature verdict on RTC+B welfare, but a
measurable, auditable framework:
\begin{enumerate}
  \item Enforce ledger-to-figure traceability for any claims about regime effects.
  \item Require publication/availability of awards and deployments at resolutions
  consistent with SCED pricing when used for evaluation.
  \item Publish feasibility-relevant aggregate telemetry (or defensible proxies)
  to enable attribution of scarcity to feasibility versus administrative scarcity.
  \item Use pre-registered sensitivity grids and placebo windows before drawing
  conclusions about ``improvement''.
\end{enumerate}

\noindent The remainder of the report provides the technical basis and the
reproducible episode ledger required to implement this evaluation discipline.

% =========================================================
% EVENT-STUDY PROTOCOL (FME-ALIGNED)
% =========================================================
\section{Event-Study Protocol Aligned to Frequency Measurable Event (FME) Logic}
\label{sec:event_protocol_fme}

This report uses an event-study protocol aligned to the Frequency Measurable Event (FME) concept used in IMFR methodology. The intent is not to replicate compliance measurement, but to adopt a defensible, pre-specified event definition that reduces post hoc selection bias and ensures comparability across pre- and post-RTC+B intervals.

\subsection{Event set construction}
Let $t$ index SCED (5-minute) intervals. For each ancillary product $k\in\{\mathrm{REGUP},\mathrm{REGDN},\mathrm{RRS},\mathrm{ECRS}\}$, define scarcity activation using fixed and quantile thresholds:
\[
\mathbb{S}_{k}(t;\tau)\equiv \mathbf{1}\{\mathrm{MCPC}_k(t)>\tau\},\quad
\mathbb{S}_{k}(t;q)\equiv \mathbf{1}\{\mathrm{MCPC}_k(t)>Q_{q}(\mathrm{MCPC}_k)\},
\]
with $\tau\in\{50,100\}$ \$/MW and $q\in\{0.99,0.999\}$. The candidate event set is
\[
\mathcal{E}\equiv \{t:\exists k\;\mathbb{S}_k(t;\tau)=1\}\cup \{t:\exists k\;\mathbb{S}_k(t;q)=1\}.
\]
When raw frequency data are available, the set is augmented by band exceedances:
\[
\mathcal{E}\leftarrow \mathcal{E}\cup \{t:\;|f_t-f_0|>b\},
\]
for pre-specified bands $b\in\{0.036,0.10\}$ Hz.

\subsection{Matched-window controls}
Each event interval is matched to non-event controls on observable operating conditions (e.g., demand percentile, net load percentile, and day-type). Where ramp/outage proxies are available, matching is restricted to comparable ramp severity. Both unmatched and matched comparisons are reported.

\subsection{Regime conditioning: scarcity versus procurement}
Let $Q_k(h)$ denote Day-Ahead ancillary procurement for product $k$ in hour $h$ (DAMASAGG). Map SCED interval $t$ to $h(t)=\lfloor t\rfloor_{\mathrm{hour}}$ and define high procurement:
\[
\mathbb{A}_k(t)\equiv \mathbf{1}\{Q_k(h(t))>Q_{0.75}(Q_k)\}.
\]
Regimes are then
\[
R0:\neg\mathbb{S}_k\wedge \neg\mathbb{A}_k,\;
R1:\mathbb{S}_k\wedge \neg\mathbb{A}_k,\;
R2:\mathbb{S}_k\wedge \mathbb{A}_k,\;
R3:\neg\mathbb{S}_k\wedge \mathbb{A}_k,
\]
operationalizing the report’s ``price-only'' versus ``quantity-backed'' scarcity distinction.

\section{Monitoring Plan and ``What Would Change My Mind'' Triggers (2026--2035)}
\label{sec:monitoring_triggers}

Because RTC+B alters scarcity expression across products, validation cannot rely on a single storm narrative. This section specifies observable signals that would materially update the interpretation of RTC+B performance over the 2026--2035 horizon, with emphasis on system cost, reliability, and investor returns.

\subsection{Primary triggers (high evidentiary weight)}
\begin{itemize}
\item \textbf{Price-only tail growth.} Sustained increases in $\Pr(\mathrm{MCPC}_k>100)$ (or in $Q_{0.999}(\mathrm{MCPC}_k)$) \emph{without} corresponding increases in awarded MW for the same product and season. This pattern is consistent with scarcity prices rising in a way not backed by quantity procurement, which may indicate feasibility binding, procurement constraints, or strategic withholding.
\item \textbf{Tail co-movement with feasibility stress.} Increasing clustering of high-MCPC events across multiple products in consecutive intervals, consistent with intertemporal feasibility constraints binding systemically rather than episodically. A practical operational proxy is the overlap rate of exceedance events across products, conditional on comparable demand/net-load quantiles.
\item \textbf{Control-quality degradation.} Deterioration of IMFR rolling indicators (or event-based response slopes where available) during seasonal stress periods despite rising ancillary procurement. This is the direct ``control adequacy'' red flag: more procurement but weaker effective response.
\end{itemize}

\subsection{Secondary triggers (screening indicators)}
Secondary indicators include (i) increasing concentration of scarcity rents into narrow hour clusters, (ii) repeated scarcity patterns in the absence of comparable system stress, and (iii) divergence between DA procurement signals (DAMASAGG) and RT scarcity incidence (MCPC). These patterns motivate targeted market-power screening and rule review in the market power section.

\subsection{Refresh protocol}
If any primary trigger persists for two seasonal stress periods, the empirical sections of this report should be treated as stale and refreshed using an updated data window, including raw frequency outcomes and any updated settlement/product definitions.

% ----------------------------------------------------------------------
\chapter{Policy Recommendations (ERCOT-Implementable)}
\label{ch:policy}

This chapter translates the report’s empirical patterns and structural mechanisms into
\emph{implementable} market-design and monitoring recommendations for ERCOT and Texas
stakeholders. The recommendations are organized to be compatible with (i) the evidence
ladder in the Identification and Inference Limits chapter (price-only $\rightarrow$ price+awards
$\rightarrow$ price+awards+deployments $\rightarrow$ SoC/telemetry), (ii) the traceability
rule in Section~\ref{sec:traceability}, and (iii) the partial-identification discipline
in Wave~3 (Sections~\ref{sec:wave3_overlap}--\ref{sec:wave3_soc}). The objective is not
to advocate a preferred outcome ex ante, but to define levers whose effects can be
\emph{measured, bounded, and audited} as the RTC+B regime matures.

A key theme is that RTC+B shifts scarcity representation from a largely energy-side
adder construct (ORDC scarcity adders) toward \emph{multi-product} scarcity reflected in
ancillary service prices (MCPC) and demand curves (ASDC).\footnote{ERCOT has documented
the RTC+B program as a major real-time market design evolution; see ERCOT materials on
RTC+B readiness and NPRR-related changes. The NPRR1186 workstream explicitly emphasizes
improved ESR state-of-charge awareness and related readiness improvements.}
Accordingly, policy evaluation must be multi-dimensional: it is possible to observe
(1) reduced energy-side adder incidence and (2) increased AS MCPC tail intensity,
without knowing whether system welfare, reliability, or procurement efficiency improved.
That is precisely why this chapter pre-commits to evaluation criteria and update triggers.

\section{Design goals and evaluation criteria}
\label{sec:policy_goals}

We define three explicit objectives: \emph{system cost}, \emph{reliability}, and
\emph{investor returns}. The core policy question is not whether scarcity exists (it
will), but whether scarcity is (i) priced in a way that preserves operational fidelity,
(ii) generates incentives aligned with reliability needs under intertemporal feasibility
constraints, and (iii) yields bankable investment signals without creating fragile
tail-dependent economics.

\subsection{A weighted welfare proxy for decision discipline}

Let $\theta$ denote a candidate policy configuration (e.g., ASDC parameterization,
qualification rules, telemetry disclosure). Define an evaluation functional:
\[
J(\theta)
= w_C \cdot \mathbb{E}\!\left[\mathrm{Cost}(\theta)\right]
+ w_R \cdot \mathbb{E}\!\left[\mathrm{Risk}(\theta)\right]
- w_I \cdot \mathbb{E}\!\left[\mathrm{InvRet}(\theta)\right],
\]
where $w_C,w_R,w_I \ge 0$ and $w_C+w_R+w_I=1$.\footnote{The sign convention is chosen so
lower $J(\theta)$ is preferred. This is not asserted to be a true social welfare
function; it is an explicit \emph{decision discipline} that prevents ad hoc reweighting
after observing outcomes.}

\paragraph{Cost component.}
Because full system cost is not identifiable from prices alone, we use a tiered proxy:
\[
\mathrm{Cost}(\theta) =
\underbrace{\mathrm{DA\_AS\_Procure}(\theta)}_{\text{award-weighted if available}}
+
\underbrace{\mathrm{RT\_AS\_Cost}(\theta)}_{\text{deployment-weighted if available}}
+
\underbrace{\mathrm{EnergyCost}(\theta)}_{\text{load-weighted if available}},
\]
where each term is computed at the highest feasible rung of the evidence ladder.
At the price-only rung, we report bounds and envelopes (see the Identification and Inference Limits chapter).

\paragraph{Reliability component.}
We maintain the report’s reliability-proxy discipline:
\[
\mathrm{Risk}(\theta) := \alpha I_f + \beta N_{\epsilon} + \gamma \cdot \mathrm{ScarcityHours},
\]
with $(\alpha,\beta,\gamma)$ treated as calibration parameters rather than universal
constants. When disturbance, deployment, and shortage-event artifacts become available,
$\mathrm{Risk}(\theta)$ is upgraded toward LOLE/EUE-linkage rather than reinterpreted
post hoc.\footnote{This follows the report’s ``identification firewall'' principle:
frequency is informative but not determinative; causal claims require more than price
tails.}

\paragraph{Investor return component.}
Investor returns are not treated as synonymous with welfare; they are treated as a
\emph{constraint} for implementability. A design that is reliability-favorable but
unbankable may be non-implementable at the scale ERCOT requires. We therefore report:
\[
\mathrm{InvRet}(\theta)
:= \mathbb{E}\!\left[\Pi_{\mathrm{ESR}}(\theta)\right]
\quad \text{and} \quad
\mathrm{TailDep}(\theta)
:= \mathrm{Corr}\!\left(\Pi_{\mathrm{ESR}}, \mathbf{1}\{\mathrm{Stress}\}\right),
\]
and we explicitly flag designs that increase tail dependence without compensating
risk-management features (see Section~\ref{sec:policy_guardrails}).

\subsection{ERCOT-implementable metrics and the episode ledger constraint}

All policy metrics used to compare configurations must be:
(i) defined in the metric dictionary, (ii) reproducible under the traceability rule
(Section~\ref{sec:traceability}), and (iii) reported with a sensitivity grid (SR-1--SR-5)
so that conclusions are not artifacts of arbitrary window selection.

\section{Near-term tuning levers}
\label{sec:policy_near_term}

Near-term levers are changes that can be implemented through ERCOT stakeholder processes
and protocol revisions without requiring fundamental redesign of RTC+B. They primarily
target measurement invariance, transparency, and incentive alignment for multi-product
scarcity.

\subsection{ASDC parameter tuning with audit hooks}

\paragraph{Mechanism.}
ASDCs translate scarcity states into product-specific shadow prices. A practical risk is
that ASDC tuning can ``move'' scarcity rents without changing feasibility, thereby
creating false impressions of improvement. Therefore, ASDC tuning should be treated as a
controlled intervention with pre-registered evaluation criteria.

\paragraph{Recommendation NT-1 (ASDC change control).}
Any ASDC parameter update (shape, floor/cap, slope regimes, breakpoints) should be
paired with:
\begin{enumerate}
  \item a published ``change vector'' (what parameter changed and why),
  \item a pre-registered evaluation window,
  \item an overlap/common-support check (Wave~3-B, Section~\ref{sec:wave3_overlap}),
  \item a placebo test window (see the Identification and Inference Limits chapter),
  \item explicit ledger-to-figure traceability (Section~\ref{sec:traceability}).
\end{enumerate}
This is ``ERCOT-grade'' because it anticipates the failure mode: parameter changes
will \emph{definitely} change prices, but may not change reliability.

\paragraph{Recommendation NT-2 (ASDC visibility).}
Publish operationally usable ASDC parameter snapshots (versioned) with effective dates,
and provide a machine-readable daily archive.\footnote{ERCOT publishes protocols and
stakeholder materials; additionally, third-party aggregators document ASDC series.
However, implementability and auditability improve materially when the ISO provides a
canonical, versioned archive rather than relying on third-party reconstruction.}

\subsection{Qualification and participation constraints as scarcity amplifiers}

\paragraph{Mechanism.}
Administrative or qualification bottlenecks can create scarcity in MCPC tails that is
not physical scarcity. Under RTC+B, the coupling between feasibility (including SoC) and
product qualification can amplify tails in ways that mimic physical stress.

\paragraph{Recommendation NT-3 (qualification bottleneck audit).}
For each ancillary product, maintain a published ``eligible supply'' series:
\[
\mathrm{EligCap}_{k,t} \quad \text{and} \quad \mathrm{EligCount}_{k,t},
\]
and require that any MCPC tail report be accompanied by:
\[
\kappa^{\mathrm{elig}}_{k,t} := \frac{\mathrm{EligCap}_{k,t}}{R_{k,t}}.
\]
If $\kappa^{\mathrm{elig}}_{k,t}$ is low in tail events, the system should not treat the
tail as purely physical scarcity without further evidence.\footnote{This operationalizes
the report’s ``administrative scarcity'' pathway and makes it measurable.}

\subsection{Transparency upgrades that move the evidence ladder upward}

The report’s identification ladder implies a concrete transparency priority list:
\begin{enumerate}
  \item \textbf{Price+awards:} publish DA awards and self-arranged quantities with
  consistent keys and timestamps (already present in DAMASAGG/DAMASSOLD artifacts, but
  the objective is completeness and invariance).
  \item \textbf{Price+awards+deployments:} publish deployment and performance series at a
  resolution that permits episode-level conditioning.
  \item \textbf{SoC/telemetry:} publish fleet-level feasibility indicators (aggregated)
  that allow bounding feasibility-driven scarcity without compromising protected
  information.\footnote{ERCOT protocol governance includes confidentiality constraints;
  the recommendation here is \emph{aggregated} feasibility telemetry sufficient for
  inference while respecting protected information rules.}
\end{enumerate}

\paragraph{Recommendation NT-4 (evidence ladder upgrade target).}
Within the next stakeholder cycle, prioritize a ``minimum viable deployment dataset'':
product deployments, performance scores/penalties (aggregated), and time-aligned
requirements. This upgrades the core analysis from price-only bounds to quantity-aware
inference.

\section{Medium-term structural levers}
\label{sec:policy_medium_term}

Medium-term levers are structural changes that may require larger protocol revisions,
system changes, or market process redesign. They target the deeper risks of a
multi-product scarcity regime: correlated feasibility loss, tail-dependent investor
economics, and congestion-local adequacy mismatches.

\subsection{Performance alignment and penalty symmetry}

\paragraph{Mechanism.}
If ancillary products become the primary scarcity channel, then performance rules and
penalties become the mechanism that converts price signals into reliable physical
response. Weak performance alignment creates a ``pay for promise'' failure mode.

\paragraph{Recommendation MT-1 (performance symmetry).}
For each product $k$, define a performance-adjusted effective award:
\[
Q^{\mathrm{eff}}_{k,t} := Q_{k,t} \cdot \phi_{k,t},
\qquad \phi_{k,t}\in[0,1],
\]
where $\phi_{k,t}$ is a published performance factor at an appropriate aggregation
level. The system should track scarcity events in terms of $Q^{\mathrm{eff}}_{k,t}$, not
nominal $Q_{k,t}$, to prevent silent degradation of reliability under high prices.

\subsection{Congestion-local procurement augmentation}

\paragraph{Mechanism.}
System-wide ancillary procurement may not resolve congestion-local scarcity if deliverability
constraints bind locally. Under RTC+B, it is possible to observe high local prices with
system-wide AS procurement that appears adequate.

\paragraph{Recommendation MT-2 (local stress overlay).}
Define a congestion-local stress indicator for hub/zone $z$:
\[
\mathrm{LocStress}_{t,z} := \mathbf{1}\!\left\{ \pi^{\mathrm{SPP}}_{t,z} -
\pi^{\mathrm{SPP}}_{t,\mathrm{SYS}} > \delta \right\},
\]
and require any system-wide procurement tuning proposal to report the incidence of
$\mathrm{LocStress}_{t,z}$ during Fern-like episodes and non-Fern stress episodes.
If local stress persists while system-wide AS appears long, policy attention should
shift from ``more AS'' to ``deliverability/local constraints'' rather than tuning ASDC
shape alone.

\subsection{Telemetry and SoC privacy-preserving feasibility disclosure}

\paragraph{Mechanism.}
Wave~3-E shows feasibility compression via $\kappa_{k,t} = C_{k,t}/R_{k,t}$ and conditional
MCPC tails. This can be substantially strengthened if ERCOT provides fleet-level
feasibility telemetry (not unit-level). A core design tension is information sufficiency
vs protected information.

\paragraph{Recommendation MT-3 (privacy-preserving feasibility telemetry).}
Publish a daily fleet feasibility panel:
\[
\mathrm{FleetSoCQuantiles}_{t} := \left(Q_{0.1}, Q_{0.5}, Q_{0.9}\right),
\qquad
\mathrm{FleetSoCStress}_{t} := \mathbf{1}\{Q_{0.1}<s^\star\},
\]
where quantiles are computed over qualified ESRs for relevant products. This supports
partial identification of feasibility-driven scarcity while avoiding resource-specific
disclosure.

\section{Guardrails against unintended fragility}
\label{sec:policy_guardrails}

Guardrails are not ``nice-to-have''; they are required because RTC+B introduces new
fragility channels: correlated SoC depletion, thin qualified supply, and tail-dependent
economics. Guardrails embed stress tests into ongoing operations and monitoring, rather
than discovering fragility after an extreme event.

\subsection{Embedded stress tests and escalation triggers}

\paragraph{Recommendation G-1 (stress-test registry).}
Maintain a published stress-test suite with scenario families aligned to
Chapter~\ref{ch:2035_full}: cold events (Fern-like), extreme ramps, high net-load, and
high forced outage conditions. Each stress test reports:
\[
\left(\kappa_{k,t}, \mathrm{MCPC}_{k,t}, a^{\mathrm{RT}}_{t,z}, I_f\right)
\quad \text{and} \quad
\mathrm{TailCluster}_{k} := \text{cluster metrics of MCPC tails}.
\]
If a stress test produces repeated tail clustering under ``adequate'' DA posture
(Section~\ref{sec:wave3_dart_bridge}), the escalation path should be operational (e.g.,
qualification review, ASDC tuning review, performance alignment review).

\paragraph{Recommendation G-2 (tail dependence guardrail for investability).}
Require that any major tuning change report the change in tail dependence of a canonical
ESR revenue proxy:
\[
\Delta \mathrm{TailDep} := \mathrm{TailDep}(M_1) - \mathrm{TailDep}(M_0),
\]
and treat large increases as a fragility risk unless accompanied by explicit risk
management tools (e.g., improved feasibility telemetry, improved performance alignment,
or procurement rules that reduce synchronized depletion).

\subsection{Monitoring, reproducibility, and the ``no cherry-picking'' rule}

This report’s traceability and episode ledger are not merely academic scaffolding; they
are the \emph{governance primitive} that prevents regime narratives from drifting.

\paragraph{Recommendation G-3 (ledger-first reporting).}
Any post-event market commentary or internal review should first specify:
(i) the episode set, (ii) the sensitivity levers used, and (iii) the evidence rung
achieved. If these are absent, the analysis is treated as illustrative.

\section{What would change the recommendations}
\label{sec:policy_updates}

The recommendations above are conditional on the report’s current empirical posture and
partial-identification limits. They should change if the following disconfirming
evidence emerges.

\subsection{Disconfirming evidence and re-optimization triggers}

\begin{enumerate}
  \item \textbf{Reliability proxy regression.}
  If $I_f$ and $N_{\epsilon}$ worsen during matched stress episodes while MCPC tails
  increase, that pattern is consistent with ``scarcity priced but not delivered'' and
  triggers MT-1 and MT-3 escalation.

  \item \textbf{DA posture non-transmission.}
  If DA posture categories (tight/adequate/long) do not predict RT MCPC tail incidence
  even under adequate common support, that implies feasibility constraints dominate;
  mitigation should shift toward feasibility telemetry and qualification/performance
  alignment rather than ASDC slope tuning alone.

  \item \textbf{Administrative scarcity dominance.}
  If MCPC tails are repeatedly associated with low $\kappa^{\mathrm{elig}}_{k,t}$ while
  physical stress proxies are mild, the priority becomes qualification bottleneck reform
  and measurement invariance audits.

  \item \textbf{Local deliverability mismatch.}
  If congestion-local scarcity persists during system-wide adequacy, reforms should
  shift toward localized procurement overlays or deliverability-aware requirements rather
  than adding system-wide procurement.

  \item \textbf{Evidence ladder upgrade.}
  If ERCOT deploys high-quality, time-aligned deployment and telemetry data, the report’s
  policy evaluation should migrate from bounds to identified effects (or tighter partial
  identification intervals), potentially changing lever priorities.
\end{enumerate}

\subsection{Update cadence and triggers}

At minimum, re-run the full pipeline quarterly and after any of the following:
\begin{itemize}
  \item ASDC parameter changes (versioned effective date),
  \item major NPRR implementation impacting RTC+B interfaces,
  \item extreme stress events (Fern-like),
  \item material changes in qualified supply for key products,
  \item sustained changes in DA posture distributions.
\end{itemize}

\section{Summary decision matrix (cross-chapter mapping)}
\label{sec:policy_matrix}

Table~\ref{tab:policy_matrix} summarizes how each recommendation maps to the report’s
analytic chapters and evidence requirements, and where to find the supporting logic.

\begingroup\sloppy
\begin{table}[!ht]
\centering
\small
\setlength{\tabcolsep}{4pt}
\caption{Policy decision matrix: levers, evidence rung, and cross-chapter traceability.}
\label{tab:policy_matrix}
\begin{tabular}{@{}>{\raggedright\arraybackslash}p{0.85in} >{\raggedright\arraybackslash}p{1.25in} >{\raggedright\arraybackslash}p{1.00in} >{\raggedright\arraybackslash}p{2.15in}@{}}
\toprule
\textbf{Lever} & \textbf{Primary objective} & \textbf{Evidence rung} & \textbf{Cross-reference (mechanism + diagnostics)} \\
\midrule
NT-1 ASDC change control & Reliability + cost interpretability & R1$\rightarrow$R2 & ORDC/ASDC bridge (Ch.~\ref{ch:ordc_asdc}); overlap and partial ID (Sec.~\ref{sec:wave3_overlap}); traceability (Sec.~\ref{sec:traceability}) \\
\midrule
NT-3 qualification audit & Prevent administrative scarcity & R2 & Gaming surface and qualification bottlenecks (Ch.~\ref{ch:gaming_full}); feasibility proxies (Sec.~\ref{sec:wave3_soc}) \\
\midrule
NT-4 deployment dataset & Upgrade identifiability & R3 & Evidence rungs (Ch.~\ref{ch:ident_limits}); counterfactual framework (Ch.~\ref{ch:counterfactual}) \\
\midrule
MT-1 performance symmetry & Reliability deliverability & R3 & Failure modes (Ch.~\ref{ch:failure_modes}); episode panels and frequency conditioning (Findings chapter) \\
\midrule
MT-3 feasibility telemetry & Reduce SoC opacity & R4 & SoC shadow-price channel (Sec.~\ref{sec:wave3_soc}); partial bounds tightening (Ch.~\ref{ch:ident_limits}) \\
\bottomrule
\end{tabular}
\end{table}
\endgroup

\section{PUCT-facing summary (implementation and oversight posture)}
\label{sec:puct_summary}

For PUCT-facing framing, the report’s position is:
\begin{enumerate}
  \item \textbf{RTC+B changes the \emph{location} of scarcity value, not necessarily the
  amount of scarcity or the level of reliability.} Price tails can migrate across
  products by design; this is mechanically consistent with co-optimization and does not,
  by itself, prove improvement.
  \item \textbf{Oversight should prioritize identifiability and invariance.} The minimum
  oversight standard is to require evaluation at the highest feasible evidence rung and
  to reject claims based on price-only narratives when awards and deployments are
  available.
  \item \textbf{The near-term priority is to harden measurement and disclosure.} A
  modest set of deployment and aggregated feasibility telemetry can materially improve
  the ability to validate whether scarcity pricing corresponds to delivered reliability.
  \item \textbf{Guardrails should be stress-tested on Fern-like events.} Multi-product
  scarcity regimes can exhibit correlated feasibility loss; the correct posture is
  continuous stress testing with explicit escalation triggers.
  \item \textbf{Policy choices are underdetermined without explicit weights.} If PUCT or
  stakeholders implicitly reweight cost vs reliability vs investability after observing
  tails, they will induce inconsistent decisions. The report recommends explicit weights
  (even if contested) and sensitivity reporting for those weights.
\end{enumerate}

% ----------------------------------------------------------------------
\chapter{Conclusions and Next Data Hooks}
\label{ch:conclusion}

This chapter closes the report with disciplined claims and a concrete, auditable
data roadmap. The structure mirrors the identification ladder in the Identification and Inference Limits chapter
(price-only $\rightarrow$ price+awards $\rightarrow$ price+awards+deployments $\rightarrow$ SoC/telemetry). No conclusion
below is interpreted beyond the highest evidence rung currently achieved for that
claim. Where overlap is incomplete (Wave~3-B), statements are explicitly partial
identification statements with bounds rather than point conclusions.

\section{What we can conclude now (and what we cannot)}
\label{sec:conclude_now}

\subsection{Conclusions admissible at the current evidence rung(s)}

At minimum, using price series (and, where available, DA awards/sold artifacts),
the following are admissible conclusions as \emph{descriptive} facts:

\begin{enumerate}
  \item \textbf{Scarcity representation is multi-product and tail-sensitive.}
  The right tail of scarcity prices appears in both (i) energy-side components
  (SPP and adders when observable) and (ii) ancillary MCPC series; episode-conditioned
  panels show that scarcity can appear as MCPC spikes even when energy-side tails are
  comparatively modest.\footnote{This is a pricing-location statement, not a welfare
  or reliability statement.}

  \item \textbf{DA posture is empirically linked to RT scarcity states in some regimes.}
  Where DA awards/sold series are available and aligned, DA posture categories
  (tight/adequate/long) provide conditional information for the probability of RT
  MCPC tail events. The strength of this transmission is treated as an empirical
  quantity to be stress-conditioned and sensitivity-tested (SR-1--SR-5).

  \item \textbf{Feasibility compression proxies co-move with MCPC tails.}
  Capability compression proxies (e.g., $\kappa_{k,t}=C_{k,t}/R_{k,t}$ and related
  ratios) show nontrivial association with MCPC tail incidence, consistent with
  an intertemporal feasibility channel under RTC+B. This remains a \emph{structural
  coherence} result rather than a welfare dominance claim.
\end{enumerate}

\subsection{Non-conclusions (explicitly ruled out)}

The following claims are \emph{not} admissible without upgrading the evidence
rung to include deployments and (ideally) SoC/telemetry:

\begin{enumerate}
  \item \textbf{Welfare superiority or system-cost reduction.}
  Price tails alone do not identify welfare or total system cost because quantities,
  redispatch, uplift, and physical actions are not observed.

  \item \textbf{Causal reliability improvement.}
  Even if frequency proxies improve in observed windows, causal attribution to RTC+B
  requires either (i) a validated disturbance/deployment linkage or (ii) quasi-experimental
  identification with strong overlap and stable measurement.

  \item \textbf{Market power or withholding as the dominant driver.}
  Strategic behavior detection requires offer curves, awards, and performance/telemetry
  to distinguish competitive scarcity from strategic scarcity.
\end{enumerate}

\subsection{Claim Registry (recommended governance primitive)}

To prevent drift, we maintain a \emph{Claim Registry}: each claim receives an ID and is
tagged with its admissible evidence rung, sensitivity coverage, and ledger traceability.
A claim is \emph{publishable} only if it has:
(i) an evidence rung tag, (ii) a traced episode set (Section~\ref{sec:traceability}),
and (iii) a sensitivity stability statement (SR grid).

\section{Implications for system cost, reliability, and capital formation}
\label{sec:implications}

This section states directional implications with \emph{identification bands}
rather than false precision. We separate (A) price-location implications, (B)
reliability-proxy implications, and (C) capital-formation implications.

\subsection{System cost (identification bands)}

\begin{itemize}
  \item \textbf{Identified:} the \emph{composition} of scarcity pricing may shift between
  energy-side channels and ancillary products (pricing-location effect).
  \item \textbf{Partially identified:} award-weighted procurement cost proxies can be
  constructed where DAMASAGG/DAMASSOLD and cleared-offer artifacts exist.
  \item \textbf{Not identified without deployments:} true operational cost impacts
  (deployment cost, redispatch, uplift) and welfare.
\end{itemize}

\subsection{Reliability (proxy bands)}

\begin{itemize}
  \item \textbf{Identified:} descriptive frequency proxy behavior ($I_f$, $N_{\epsilon}$)
  conditional on episodes, when time-aligned frequency data exists.
  \item \textbf{Partially identified:} association between scarcity episodes (MCPC tails,
  adder incidence) and frequency deviations under matched system conditions.
  \item \textbf{Not identified without disturbance and deployment linkage:} causal claims
  about improved frequency stability attributable to RTC+B mechanisms.
\end{itemize}

\subsection{Capital formation (bankability bands)}

\begin{itemize}
  \item \textbf{Identified:} price-implied revenue \emph{envelopes} and tail dependence
  signatures (how concentrated scarcity value is in rare stress intervals).
  \item \textbf{Partially identified:} award-weighted revenue proxies where cleared awards
  exist; still incomplete without performance/penalty treatment.
  \item \textbf{Not identified without telemetry/performance:} bankable returns net of
  performance risk, feasibility limits, and penalty incidence.
\end{itemize}

\section{Next data hooks (priority list)}
\label{sec:next_hooks}

The objective of ``Next Data Hooks'' is to upgrade the evidence ladder, tighten bounds,
and reduce dependence on price-only inference. We list hooks in priority order with the
minimum fields needed to be useful.

\subsection{Tier-1 (unblocks causal and quantity-aware inference)}

\begin{enumerate}
  \item \textbf{RT deployments by AS product (time-aligned to SCED).}
  Required fields: timestamp (SCED interval key), product $k$, deployed MW, deployment
  direction (up/down), and settlement mapping if applicable.
  \item \textbf{Performance and penalty incidence (aggregated acceptable).}
  Required fields: product $k$, interval/day, performance score (or pass/fail),
  penalty \$ (or indicator), and whether the resource was ESR vs non-ESR.
  \item \textbf{Disturbance / event logs (frequency-relevant).}
  Required fields: event timestamp, event type, magnitude proxy, and affected area/system.
\end{enumerate}

\subsection{Tier-2 (tightens bounds and explains tails)}

\begin{enumerate}
  \item \textbf{Qualified supply and eligibility series by product (administrative scarcity audit).}
  Required fields: $\mathrm{EligCap}_{k,t}$, $\mathrm{EligCount}_{k,t}$, qualification rule
  version/effective date.
  \item \textbf{Outage series / derate proxies (thermal + transmission).}
  Required fields: forced outage MW, derate MW, timestamps, and coarse location tags.
  \item \textbf{Net-load ramp proxies at SCED scale.}
  Required fields: load, wind, solar (or net load), ramp rates, timestamps.
\end{enumerate}

\subsection{Tier-3 (SoC/telemetry proxies, privacy-preserving)}

\begin{enumerate}
  \item \textbf{Fleet-level SoC distribution snapshots (quantiles or stress flags).}
  Required fields: $(Q_{0.1},Q_{0.5},Q_{0.9})$ for qualified ESR fleet, timestamp,
  and product eligibility partition if feasible.
  \item \textbf{ESR charge/discharge saturation proxies.}
  Required fields: aggregate ESR charging MW, discharging MW, and headroom estimates if available.
\end{enumerate}

\section{Replication capsule}
\label{sec:replication_capsule}

This report is reproducible under the following capsule:

\begin{itemize}
  \item \textbf{Artifact manifest:} All source files used, with canonical filenames and
  date coverage, listed in Appendix (Data Quality and Reproducibility).
  \item \textbf{Transform contract:} Time normalization (timezone), interval alignment
  rules, key integrity checks, and missingness thresholds in Chapter (Data, Artifacts, and Methods).
  \item \textbf{Episode ledger:} The episode registry is the authoritative index linking
  every reported result to bounded episode rows (Section~\ref{sec:traceability}).
  \item \textbf{Sensitivity grid:} SR-1--SR-5 settings must be declared for every figure/table.
  \item \textbf{Re-run triggers:} The report is marked \emph{stale} and must be refreshed if:
  (i) ASDC parameter version changes, (ii) qualification rules change, (iii) a new Fern-like
  stress episode occurs, or (iv) data coverage expands sufficiently to upgrade the evidence rung.
\end{itemize}

\noindent When additional Tier-1 hooks become available, the identification ladder is
recomputed and any claim whose rung changes must be re-labeled in the Claim Registry.

\subsection{Geometry–Feasibility Linkage}
\label{sec:geometry_feasibility}

We test whether Day-Ahead offer geometry serves as a leading indicator of
real-time feasibility compression.

Specifically, we estimate:
\[
\Pr\!\left(\kappa_{k,t} < \kappa^\star \mid H_d > h^\star,\, \sigma_d < \sigma^\star,\, X_t \right),
\]
where $H_d$ is the Day-Ahead Herfindahl index of cleared AS supply for product $k$
on day $d$, and $\sigma_d$ is the corresponding price dispersion metric.

A positive association between concentrated DA geometry and subsequent low
$\kappa_{k,t}$ supports the hypothesis that apparent procurement adequacy may mask
intertemporal feasibility risk.

% ----------------------------------------------------------------------

\newpage

\appendix

\section*{Timestamp Alignment, Repeated-Hour Handling, and Data Lineage Notes}
\label{app:timestamp_alignment}

ERCOT artifacts used in this report span multiple native time resolutions: SCED dispatch intervals (5-minute), DAMASAGG day-ahead ancillary aggregates (hourly), and high-frequency system measurements (e.g., frequency at 10-second resolution). To preserve auditability, all joins are performed using explicit time conventions and deterministic rounding rules.

\subsection*{Time standards and rounding}
All timestamps are interpreted in ERCOT local market time as represented in the source files. When mapping SCED interval timestamps to hourly aggregates, we define
\[
h(t)\equiv \left\lfloor t \right\rfloor_{\mathrm{hour}},
\]
and assign each SCED interval to the hour containing its start time. Where DAMASAGG uses Hour Ending notation (HE), we convert to hour start time via
\[
t_{\mathrm{start}}(\mathrm{HE}=x)=\mathrm{DeliveryDate}+(x-1)\ \mathrm{hours}.
\]
This convention is applied consistently to avoid off-by-one-hour errors in DA-to-RT conditioning.

\subsection*{Repeated-hour flags}
SCED MCPC records may include repeated-hour indicators. This report retains repeated-hour flags as provenance attributes and does not collapse duplicated wall-clock hours without explicit disambiguation. When aggregations require uniqueness, repeated-hour flags are included in grouping keys or analysis windows are chosen to avoid DST transitions.

\subsection*{Lineage discipline}
For each derived metric, the report records (i) source file name(s), (ii) native resolution, (iii) join keys, and (iv) transformation rules (including thresholds and quantiles). This appendix is the canonical location for documenting future extensions involving additional telemetry (e.g., higher-resolution frequency, inertia subcomponents, or unit-level response data).


\chapter{Data Quality and Reproducibility Annex}
\label{app:data_quality_full}



\section{Sensitivity and robustness registry (Wave 2)}
\label{app:sensitivity_wave2}

\subsection*{Registered levers and identifiers}
\label{sec:sensitivity_levers_ids}

All sensitivity levers used in Wave~2 are assigned stable identifiers. These identifiers must be referenced wherever a metric choice could alter conclusions. The registry is intentionally explicit to prevent silent post hoc tuning.

\paragraph{SR-1 (Aggregation transform).}\label{sr:agg_transform}
Transform five-minute $\mathrm{MCPC}_k(t)$ to an hour-level series by one of:\\
(i) hourly maximum $\max_{t\in h}\mathrm{MCPC}_k(t)$,\\
(ii) hourly mean $\mathbb{E}_{t\in h}[\mathrm{MCPC}_k(t)]$, or\\
(iii) within-hour $\mathrm{CVaR}_{0.99}$.

\paragraph{SR-2 (Tail definition family).}\label{sr:tail_family}
Define tail membership using either:
(i) fixed thresholds $\tau\in\{50,100,500\}$,
(ii) quantile thresholds $Q_{q}$ with $q\in\{0.95,0.99,0.999\}$, or
(iii) a hybrid threshold $u=\max\{Q_{0.99},\tau\}$.

\paragraph{SR-3 (Window padding).}\label{sr:window_padding}
Evaluate Fern effects under symmetric padding of $\pm p$ hours around the Fern window with $p\in\{0,6,12,24\}$ to test sensitivity to boundary choices.

\paragraph{SR-4 (Lead--lag alignment).}\label{sr:lead_lag}
Compute alignment statistics under lead--lag offsets $\ell\in\{-2,-1,0,+1,+2\}$ hours for DA posture and $\ell\in\{-6,-3,0,+3,+6\}$ five-minute steps for MCPC--frequency joins.

\paragraph{SR-5 (Episode thresholds).}\label{sr:episode_thresholds}
Construct scarcity episodes using one of:
\texttt{SR1\_Q099}: $u=Q_{0.99}$,
\texttt{SR1\_Q0999}: $u=Q_{0.999}$,
\texttt{SR2\_FIX100}: $u=100$,
\texttt{SR3\_HYBRID}: $u=\max\{Q_{0.99},100\}$.

Where thresholds differ by product, the report logs the product-specific $u_k$.

This appendix records the sensitivity levers registered in Wave~2 and reports a first-pass quantification for the aggregation-choice lever. The intent is methodological: to show which numeric summaries are invariant to reasonable measurement choices, and which are not.

\subsection*{Why aggregation sensitivity matters}
Real-time ancillary scarcity can manifest as (i) persistent elevation (many moderately high intervals) or (ii) rare spikes (few extreme intervals). These regimes are not equivalent for system operations, participant risk management, or investor cash-flow modeling. Accordingly, we report sensitivity to three hour-level transforms of the 5-minute MCPC series: $\max$ within hour, mean within hour, and within-hour $\mathrm{CVaR}_{0.99}$ (definitions in Section~\ref{sec:wave2_mcpc_alignment}).

\subsection{Aggregation-choice sensitivity: Fern vs pre}
Using hour-level $Q_{0.99}$ statistics for each product, the Fern/pre ratios under two common transforms are:

\begin{itemize}
\item \textbf{REGUP}: Fern/pre ratio for hourly $Q_{0.99}$ of $\max\mathrm{MCPC}$ is 0.411; for hourly $Q_{0.99}$ of mean MCPC is 1.335.
\item \textbf{REGDN}: Fern/pre ratio for hourly $Q_{0.99}$ of $\max\mathrm{MCPC}$ is 16.925; for hourly $Q_{0.99}$ of mean MCPC is 15.029.
\item \textbf{RRS}: Fern/pre ratio for hourly $Q_{0.99}$ of $\max\mathrm{MCPC}$ is 0.183; for hourly $Q_{0.99}$ of mean MCPC is 0.496.
\item \textbf{ECRS}: Fern/pre ratio for hourly $Q_{0.99}$ of $\max\mathrm{MCPC}$ is 1.282; for hourly $Q_{0.99}$ of mean MCPC is 0.937.
\item \textbf{NSPIN}: Fern/pre ratio for hourly $Q_{0.99}$ of $\max\mathrm{MCPC}$ is 1.543; for hourly $Q_{0.99}$ of mean MCPC is 1.429.
\end{itemize}

Interpretation discipline: these are not contradictions; they are diagnostics. For example, $\mathrm{REGUP}$ shows Fern/pre $<1$ under hourly-$\max$ quantiles but $>1$ under hourly-mean quantiles, which is consistent with a pre-window containing a few very large $\max$ hours while Fern exhibits broader (more frequent) elevation that lifts mean-type measures. Conversely, $\mathrm{REGDN}$ is Fern-amplified under both transforms, which indicates that Fern is a robust upward shift for regulation-down scarcity across aggregation choices.

\subsection{Other levers queued for re-run}
The remaining registered levers will be executed once (a) the post-Fern MCPC data covering 2026-02-01 through 2026-02-05 is ingested, and (b) a settlement-grade frequency series is available for the full Fern window. Those levers are:
(i) tail definition (fixed thresholds $\tau$ vs quantile exceedances), 
(ii) window padding (expand pre/Fern/post by $\pm48$ hours where data allow), and 
(iii) DA-to-RT alignment lag (0 to +6 hours, plus a same-day vs day-ahead lag to capture commitment effects).

\section{Data inventory and provenance (Wave 1 focus)}
This annex records the exact input artifacts used for Wave~1 (day-ahead ancillary awards) and the transformation rules applied to create join-ready analytic tables.

\subsection{Wave 1 inputs}
\begin{itemize}
  \item DAMASAGGNP419 daily extracts for 2026-01-20 through 2026-02-05 (author-provided CSV exports).
\end{itemize}
Each daily file is treated as an immutable raw artifact. The unified Wave~1 table is created by concatenation with a file-origin tag.

\subsection{Schema normalization}
The DAMASAGG schema is normalized to the analytic fields:
\[
\{\texttt{delivery\_date},\ \texttt{hour\_ending},\ \texttt{product}\ k,\ \texttt{block\_price}\ p_{k,i},\ \texttt{block\_qty}\ q_{k,i},\ \texttt{dst\_flag},\ \texttt{source\_file}\}.
\]
Hour-ending is mapped to hour-begin timestamps as
\[
t_{\mathrm{start}}(d,h)=d+(h-1)\ \text{hours}.
\]
Units are preserved as reported: block quantities are treated as MW of ancillary capacity.

\subsection{Missingness and duplicates}
For each file and for the concatenated panel, we compute:
\begin{itemize}
  \item Row-level duplicates under the key $(d,h,k,i)$,
  \item Null counts by critical fields: $\{d,h,k,p,q\}$,
  \item Non-negativity checks on $q$ and sanity bounds on $p$.
\end{itemize}
\paragraph{Halt rule.} If any critical field exhibits $>1\%$ null or non-parsable values, downstream computations are halted until corrected.

\section{Transformation log s}
Wave~1 outputs are created with transparent transforms:
\begin{enumerate}
  \item Concatenate daily DAMASAGG files into a long panel.
  \item Compute hour-level aggregates $(Q^{DA}_k,P^{DA}_k)$ and stack diagnostics $(n_k,\sigma_{p,k},H_k)$.
  \item Partition into pre/Fern/post windows using fixed dates (Section~\ref{sec:damasagg_wave1}).
  \item Report descriptive summaries (means, quantiles, maxima) by product and window.
\end{enumerate}
No filtering beyond date-windowing is performed in Wave~1.

\section{Join assumptions (declared before 2nd effort)}
Wave~2 will introduce SCED MCPC (5-minute) and RTM SPP (15-minute) series. Prior to those joins, we declare:
\begin{itemize}
  \item \textbf{DAM hour to SCED intervals:} hour-level $Q^{DA}_k(d,h)$ and $P^{DA}_k(d,h)$ will be joined to all SCED intervals $t$ such that $h(t)=h$.
  \item \textbf{RTM 15-min to 5-min alignment (if needed):} 15-minute RTM SPP values will be treated as piecewise-constant across the three corresponding 5-minute sub-intervals for alignment only; this is an approximation and is disclosed wherever used.
\end{itemize}
Any deviation from these declared join rules must be logged explicitly in this annex.


% ======================================================================
% REGULATORY TRACEABILITY
% ======================================================================

\chapter{Regulatory Traceability}
\label{app:reg_trace_full}

This appendix provides a formal traceability layer linking the analytical
constructs, symbols, and empirical objects used throughout this report to the
underlying ERCOT regulatory instruments governing Real-Time Co-optimization
with Batteries (RTC+B). Its purpose is threefold:

\begin{enumerate}
  \item To document \emph{what changed}, \emph{when it changed}, and \emph{where}
  those changes appear in ERCOT protocols, NPRRs, and settlement artifacts;
  \item To align the terminology and symbols used in this report with ERCOT-native
  language to prevent semantic ambiguity; and
  \item To delimit clearly the boundary between protocol-defined facts and
  analytical interpretation.
\end{enumerate}

This appendix is intentionally non-normative. It does not assess whether RTC+B
improved reliability, efficiency, or investment incentives. It establishes the
regulatory substrate upon which such claims may or may not be built elsewhere
in the report.

% ----------------------------------------------------------------------

\section{NPRR 1186: Scope, Intent, and Approval History}

Real-Time Co-optimization with Batteries was implemented primarily through
Nodal Protocol Revision Request (NPRR)~1186 and associated market-facing
changes.\footnote{
See ERCOT, ``Approved Nodal Protocol Revision Requests,'' NPRR~1186, including
impact analyses and TAC/Board materials.
}

The stated objectives of NPRR~1186 included:
\begin{itemize}
  \item Enabling Electric Storage Resources (ESRs) to participate fully in
  real-time co-optimization;
  \item Incorporating intertemporal feasibility considerations into real-time
  dispatch and pricing; and
  \item Aligning real-time price formation with the operational constraints of
  storage resources.
\end{itemize}

Importantly, NPRR~1186 did \emph{not} introduce a new reliability standard, a
capacity obligation, or an explicit welfare criterion. It modified operational
and settlement mechanics within the existing ERCOT market design framework.

Throughout this report, observed pricing outcomes are evaluated \emph{conditional
on} these design changes. No claim of policy success or failure is implied unless
explicitly stated and qualified under the identification framework in the Identification and Inference Limits chapter.

% ----------------------------------------------------------------------

\section{Effective Dates and Market Transition Boundaries}

RTC+B was implemented with defined operational and settlement effective dates.
These dates establish the regime boundary used throughout the empirical analysis.

For purposes of this report:
\begin{itemize}
  \item The \emph{legacy regime} ($M_0$) refers to intervals prior to the RTC+B
  production settlement effective date; and
  \item The \emph{RTC+B regime} ($M_1$) refers to intervals on or after that date.
\end{itemize}

Where transitional artifacts, phased implementations, or parameter calibration
periods occurred, those windows are either excluded from primary inference or
treated explicitly in placebo and robustness checks
(see the Identification and Inference Limits chapter).

All pre/post comparisons, episode definitions, and counterfactual constructions
are aligned to these regulatory transition boundaries.

% ----------------------------------------------------------------------

\section{Protocol Subsections Governing RTC+B Constructs}
\label{sec:protocol_scaffold}

This section enumerates the principal Nodal Protocol subsections governing the
constructs operationalized under RTC+B. The canonical ERCOT posting for
protocol text and effective dates is cited here for audit traceability.\cite{ercot_current_nodal_protocols}

\subsection{Real-Time Operations and SCED}

\begin{itemize}
  \item \textbf{Section 6.4 — Real-Time Operations}
  \item \textbf{Section 6.5 — Security-Constrained Economic Dispatch (SCED)}
  \begin{itemize}
    \item Energy balance constraints;
    \item Transmission and security constraints;
    \item Ancillary Service requirement constraints;
    \item ESR intertemporal feasibility constraints (including SoC).
  \end{itemize}
\end{itemize}

These provisions define the physical and feasibility constraints whose shadow
values are expressed through real-time prices and MCPCs.

\subsection{Ancillary Services}

\begin{itemize}
  \item \textbf{Section 3.14 — Ancillary Service Capacity Monitoring}
  \item \textbf{Section 6.7 — Ancillary Service Deployment}
  \item \textbf{Section 6.6 — Scarcity Pricing Mechanisms}
\end{itemize}

These sections govern procurement quantities, deployment logic, and scarcity
pricing interactions.

\subsection{Day-Ahead Market}

\begin{itemize}
  \item \textbf{Section 4.4 — Day-Ahead Market Clearing}
  \item \textbf{Section 4.5 — Day-Ahead Ancillary Service Awards}
  \item \textbf{Section 4.6 — Day-Ahead Co-optimization}
\end{itemize}

These provisions define DAMASAGG award artifacts used in the DA posture and
DA$\rightarrow$RT transmission analyses (Wave~3).

\subsection{Settlement}

\begin{itemize}
  \item \textbf{Section 9.5 — Real-Time Settlement}
  \item \textbf{Section 9.6 — Ancillary Service Settlement}
  \item \textbf{Section 9.7 — Uplift and Make-Whole}
\end{itemize}

These sections define the financial outputs from which posted price series are
constructed.

% ----------------------------------------------------------------------

\section{Settlement Constructs Impacted}


\subsection{Named constructs used throughout this report (term-to-source map)}
\label{subsec:reg_term_map}

This subsection provides a compact term-to-source map for the principal
terminology used in the main text.

\begin{table}[h!]
\centering
\caption{Regulatory traceability map for key terms used in this report.}
\label{tab:reg_term_map}
\begin{tabular}{p{1.05in}p{3.05in}p{2.45in}}
\toprule
\textbf{Term} & \textbf{ERCOT grounding (protocol / NPRR / publication)} & \textbf{Notes} \\
\midrule
RTC+B & NPRR~1186; ERCOT RTC+B implementation materials \cite{ercot_nprr1186_issue_page,ercot_rtcb_golive_2025,ercot_rtcb_overview_2025} & Market-design regime boundary used for $M_0$ vs $M_1$. \\
MCPC & Ancillary service settlement constructs; protocol settlement sections \cite{ercot_ancillary_services_pdf,ercot_current_nodal_protocols} & Treated as a posted price object; constraint attribution is not implied. \\
ECRS & ERCOT ancillary service product definitions \cite{ercot_ancillary_services_pdf} & Used as ERCOT-defined product name. \\
FRRS & ERCOT ancillary service publications \cite{ercot_ancillary_services_pdf} & Label may be implementation-dependent; treated cautiously in main text. \\
ASDC & ERCOT AS demand curve materials \cite{ercot_asdc_overview_2024} & Treated as ERCOT-defined demand curve construct. \\
ORDC & ERCOT ORDC methodology and biennial reports \cite{ercot_ordc_methodology_2015,ercot_ordc_report_2024} & Treated as ERCOT-defined scarcity adder methodology. \\
\bottomrule
\end{tabular}
\end{table}

\subsection{Real-Time Energy Prices}

Let $\pi^{\mathrm{SPP}}_{t,z}$ denote the real-time Settlement Point Price at
interval $t$ and location $z$. Where separately available, the real-time scarcity
adder component is denoted $a^{\mathrm{RT}}_{t,z}$ such that:
\[
\pi^{\mathrm{SPP}}_{t,z} = \lambda^{\mathrm{LMP}}_{t,z} + a^{\mathrm{RT}}_{t,z}.
\]

RTC+B did not remove energy-side scarcity adders (e.g., ORDC-derived adders).
However, it altered the circumstances under which scarcity value may appear in
energy prices versus ancillary service prices.

Throughout this report, this decomposition is treated as a posted settlement
identity, not as a statement about sufficiency or necessity for reliability.

\subsection{Market Clearing Price for Capacity (MCPC)}

For ancillary product $k$, ERCOT publishes a real-time Market Clearing Price for
Capacity, denoted $\mathrm{MCPC}_{k,t}$.

Post-RTC+B:
\begin{itemize}
  \item MCPCs are produced under a co-optimization framework that includes ESR
  feasibility constraints;
  \item MCPCs may bind independently of energy scarcity adders; and
  \item MCPCs may reflect intertemporal scarcity even when instantaneous reserve
  quantities appear adequate.
\end{itemize}

In this report, MCPCs are treated strictly as \emph{dual variables} associated
with ancillary service requirement constraints.

% ----------------------------------------------------------------------

\section{Operational Feasibility and ESR Treatment}

RTC+B explicitly incorporates intertemporal feasibility constraints for ESRs,
including:
\begin{itemize}
  \item State-of-charge limits;
  \item Charging and discharging power limits;
  \item Intertemporal coupling across SCED intervals; and
  \item Qualification and telemetry constraints.
\end{itemize}

These constraints are operational in nature but manifest economically through
shadow prices. Accordingly:
\begin{itemize}
  \item MCPC spikes are interpreted as potential signatures of feasibility
  compression, not definitive evidence of physical depletion; and
  \item Claims regarding SoC dynamics are explicitly bounded unless corroborated
  by awards, deployments, or telemetry.
\end{itemize}

This distinction underpins the partial-identification framework in Wave~3.

% ----------------------------------------------------------------------

\section{Terminology Alignment}

Table~\ref{tab:terminology_alignment} maps ERCOT-native terminology to the symbols
used consistently throughout this report.

\begin{table}[h]
\centering
\caption{Terminology Alignment}
\label{tab:terminology_alignment}
\begin{tabular}{lll}
\toprule
ERCOT Term & ERCOT Artifact & Report Notation \\
\midrule
Settlement Point Price & RT SPP files & $\pi^{\mathrm{SPP}}_{t,z}$ \\
Real-Time LMP & RT LMP files & $\lambda^{\mathrm{LMP}}_{t,z}$ \\
Scarcity Adder & ORDC/RDP postings & $a^{\mathrm{RT}}_{t,z}$ \\
MCPC & RT AS settlements & $\mathrm{MCPC}_{k,t}$ \\
DA AS Awards & DAMASAGG files & $Q^{DA}_{k,d}$ \\
AS Requirement & ERCOT requirements & $R_{k,t}$ \\
Available Capability & Capability reports & $C_{k,t}$ \\
\bottomrule
\end{tabular}
\end{table}

% ----------------------------------------------------------------------

\section{NPRR Dependency and Companion Revisions}

RTC+B relies on a sequence of enabling and companion NPRRs governing ESR
qualification, telemetry, and settlement treatment. NPRR~1186 should therefore
be interpreted as part of a dependency stack rather than a standalone change.

Where relevant, analytical constructs in this report are traceable to those
enabling revisions rather than to NPRR~1186 alone.

% ----------------------------------------------------------------------

\section{Known Ambiguities and Interpretive Limits}

Certain aspects of RTC+B implementation are not fully observable from public
data, including:
\begin{itemize}
  \item Resource-level SoC trajectories;
  \item The internal mapping between SCED binding constraints and posted MCPCs;
  \item Qualification failures not separately disclosed; and
  \item Operator discretion during stressed conditions.
\end{itemize}

Where these ambiguities matter, the analysis either refrains from point claims or
bounds conclusions conservatively under explicit assumptions.

% ----------------------------------------------------------------------

\section{Traceability Guarantee}

Every analytical object in this report that depends on market design or protocol
rules is traceable to one or more ERCOT-published instruments documented in this
appendix.

Where interpretation extends beyond explicit protocol language, it is labeled as
analytical inference and evaluated under the identification framework in the Identification and Inference Limits chapter.

This traceability layer is intended to render the report auditable, reproducible,
and robust to institutional memory loss as ERCOT market design continues to
evolve.

\chapter{Event Study Replication Protocol}
\label{app:repro_full}
Triggers, windows, aggregation, robustness checks, placebo tests, and exact steps to reproduce every figure/table.


This appendix provides the step-by-step replication recipe for all event studies, including the Fern window case study and later extensions. The protocol is designed to be executable by an independent reviewer with access to the same ERCOT public artifacts.

\section{Fixed windows and triggers}
\subsection{Pre/Fern/Post partition}
All ``pre/Fern/post'' analyses use the fixed partition declared in Section~\ref{sec:damasagg_wave1}. Alternative windows may be explored only as \emph{sensitivity} and must be labeled as such.

\subsection{Stress-event generalization}
When extending beyond Fern, a candidate interval qualifies as a stress window if it satisfies the selection criteria described in Section~\ref{sec:damasagg_wave1} (load extremeness, adder activation, or frequency stress), using the same thresholds $(\tau_a,m)$.

\section{Data ingestion and normalization}
\subsection{DAMASAGG (day-ahead ancillary awards)}
\begin{enumerate}
  \item Concatenate daily DAMASAGGNP419 CSVs into a single long table with a source-file tag.
  \item Normalize schema to $(d,h,k,i,p_{k,i},q_{k,i},\mathrm{DST})$.
  \item Compute hour-level aggregates $(Q^{DA}_k,P^{DA}_k)$ and stack diagnostics $(n_k,\sigma_{p,k},H_k)$ using the metric definitions in Chapter~\ref{ch:metric_dictionary_full}.
\end{enumerate}

\subsection{SCED MCPC (real-time ancillary scarcity prices)}
\begin{enumerate}
  \item Ingest SCED MCPC NP6332 artifacts at 5-minute resolution.
  \item Normalize product labels to canonical $k$ codes (REGUP, REGDN, RRS, ECRS, NSPIN).
  \item Compute tail descriptors: exceedance rates, $Q_{0.99}$, and $\mathrm{CVaR}_{0.99}$ by product and window.
\end{enumerate}

\subsection{RTM SPP (energy-side settlement prices)}
If RTM SPP is available at 15-minute resolution, align to 5-minute for joint analysis by piecewise-constant expansion (documented in Appendix~\ref{app:data_quality_full}). This alignment is used only for timing/co-movement tests, not as a claim about true 5-minute energy price constancy.

\section{Join keys and aggregation rules (declared ex ante)}
\subsection{Hour to 5-minute mapping}
Day-ahead award variables are hour-level; real-time variables (MCPC, RT prices) are 5-minute. Define $h(t)$ as the delivery hour containing interval $t$. The join rule is:
\[
(Q^{DA}_k,P^{DA}_k,n_k,\sigma_{p,k},H_k)(d,h) \mapsto (d,h(t)) \text{ for all } t \text{ in hour } h.
\]
Thus each 5-minute interval inherits the hour's DAM posture variables.

\subsection{Aggregation of 5-minute variables to hours (if reported)}
When reporting hour summaries of 5-minute series $X(t)$ within hour $h$, the default aggregates are:
\[
\overline{X}(h)=\frac{1}{N_h}\sum_{t\in h}X(t),\qquad
X_{\max}(h)=\max_{t\in h}X(t),\qquad
\mathrm{CVaR}_{0.99,h}(X)=\mathbb{E}[X(t)\mid X(t)\ge Q_{0.99,h}(X)].
\]
All three are reported where relevant, because $\overline{X}$ captures typical conditions, $X_{\max}$ captures instantaneous scarcity, and $\mathrm{CVaR}$ captures sustained tail behavior.

\section{Robustness checks and placebo tests}
\subsection{Weather/load matching}
Where load and renewable data are available, compare matched-hour subsets (e.g., same hour-of-day and similar load quantiles) across windows to reduce confounding.

\subsection{Placebo windows}
Define placebo windows of equal length immediately adjacent to Fern (e.g., 8-day blocks) and verify that the measured shifts are not ubiquitous across the season.

\subsection{Multiple-threshold reporting}
All tail results must be reported over a grid of thresholds and quantile levels (Chapter~\ref{ch:metric_dictionary_full}), not a single tuned cutoff.

\newpage

% =========================================================
\chapter{Definitions, Threshold Families, and Geometry--Feasibility Linkage}
\label{app:def_geometry_feasibility}
% =========================================================

This appendix standardizes all definitions used in the Geometry and Feasibility
analyses to prevent post hoc reinterpretation. It also formalizes the conceptual
link between Day-Ahead award geometry, Real-Time feasibility compression, and
multi-product scarcity pricing under RTC+B.

The appendix is intentionally definitional. It does not claim welfare superiority
or causality. Its function is (i) semantic stability, (ii) reproducibility, and
(iii) consistent cross-chapter interpretation.

% ----------------------------------------------------------------------
\section{Notation Index and Object Types}
\label{sec:E_notation_index}

We distinguish four classes of objects:

\begin{enumerate}
  \item \textbf{Posted price processes (dual variables):}
  real-time settlement point prices $\pi^{\mathrm{SPP}}_{t,z}$, real-time LMP
  components $\lambda^{\mathrm{LMP}}_{t,z}$ where separately available, real-time
  adders $a^{\mathrm{RT}}_{t,z}$ where posted, and MCPCs $\mathrm{MCPC}_{k,t}$.

  \item \textbf{Award/quantity objects (primal-ish, settlement constructs):}
  Day-Ahead ancillary awards $Q^{DA}_{k,d,h}$ (from DAMASAGG) and related
  Day-Ahead sold/cleared constructs (e.g., DAMASSOLD).

  \item \textbf{Capability and requirement objects:}
  requirements $R_{k,t}$ and capability proxies $C_{k,t}$ where available
  (e.g., total capability series).

  \item \textbf{Derived diagnostics:}
  geometry metrics (concentration, dispersion) and feasibility compression
  proxies (capability-to-requirement ratios, tail conditioning).
\end{enumerate}

Throughout the report, dual-variable objects are treated as \emph{marginal values}
of binding constraints, not direct measures of physical scarcity or welfare.

% ----------------------------------------------------------------------
\section{Threshold Families and Pre-Registration Discipline}
\label{sec:E_threshold_families}

All exceedance-based results in Chapters~\ref{ch:findings_full} and Wave~3 are computed
using pre-registered threshold families to avoid cherry-picking.

Let $X_{t}$ denote any scalar series (e.g., $\mathrm{MCPC}_{k,t}$ or $a^{\mathrm{RT}}_{t,z}$).
We define the following threshold families:

\begin{description}
  \item[TF-1 (Fixed-dollar thresholds):]
  $\tau \in \{\$100,\$250,\$500,\$1000,\$2000\}$ where meaningful for the series.

  \item[TF-2 (Quantile thresholds):]
  $\tau = Q_{q}(X)$ for $q \in \{0.95,0.99,0.995,0.999\}$ computed on a designated
  baseline window with explicitly declared support restrictions.

  \item[TF-3 (Stress-conditional quantiles):]
  $\tau = Q_{q}(X \mid X \in \mathcal{S})$ where $\mathcal{S}$ is a stress stratum
  (e.g., Fern packet or Fern-like regime) defined ex ante.

  \item[TF-4 (Episode-based thresholds):]
  $\tau$ defined implicitly by episode triggers (e.g., ``top-$m$ MCPC intervals'')
  and used only in event-time panels with episode ledger traceability.
\end{description}

All results must state the threshold family used in captions (per the
Ledger-to-figure traceability rule).

% ----------------------------------------------------------------------
\section{Day-Ahead Geometry Definitions}
\label{sec:E_geometry_defs}

Day-Ahead geometry is a summary of the \emph{shape} of the DA award stack, not
the gross cleared quantity alone. The object is to detect \emph{fragility}:
conditions in which the system appears ``long'' in total MW but is operationally
brittle because marginal availability depends on a small or correlated subset of
resources.

\subsection{Award share vector}

For ancillary product $k$ on operating day $d$ and hour $h$, define an award share
vector over awarded resources $i \in \mathcal{I}_{k,d,h}$:
\[
s_{i,k,d,h} := \frac{q^{DA}_{i,k,d,h}}{\sum_{j \in \mathcal{I}_{k,d,h}} q^{DA}_{j,k,d,h}},
\quad
\sum_{i} s_{i,k,d,h}=1,\quad s_{i,k,d,h}\ge 0.
\]

\subsection{Concentration (Herfindahl) index}

Define the Day-Ahead concentration metric:
\[
H_{k,d,h} := \sum_{i \in \mathcal{I}_{k,d,h}} s_{i,k,d,h}^{2}.
\]
Interpretation:
\begin{itemize}
  \item $H$ near 0 indicates a broad award base (diffuse supply);
  \item $H$ closer to 1 indicates concentration (brittle marginality).
\end{itemize}
This metric is scale-free and comparable across hours so long as measurement
invariance holds for the award definitions.

\subsection{Dispersion (price-space geometry)}

Let $p^{DA}_{i,k,d,h}$ denote the award-clearing price associated with the block
or resource (as available in the DA award artifact). Define a volume-weighted
dispersion:
\[
\Delta P_{k,d,h} := \sqrt{
\sum_{i \in \mathcal{I}_{k,d,h}} s_{i,k,d,h}\left(p^{DA}_{i,k,d,h}-\overline{p}^{DA}_{k,d,h}\right)^{2}
},
\quad
\overline{p}^{DA}_{k,d,h}:=\sum_{i} s_{i,k,d,h}p^{DA}_{i,k,d,h}.
\]
Interpretation:
\begin{itemize}
  \item Large $\Delta P$ indicates a graded stack with broad marginal participation;
  \item Small $\Delta P$ in combination with high $H$ indicates dominance by a
  narrow price-setting locus (potentially brittle).
\end{itemize}

\subsection{Geometry classification (operational bins)}

For reporting, geometry is discretized into bins:
\[
G_{k,d,h}\in\{\text{broad},\text{mixed},\text{brittle}\},
\]
based on declared thresholds for $(H,\Delta P)$ that are calibrated on a baseline
distribution and re-tested under Appendix~\ref{app:geometry_robustness}.

% ----------------------------------------------------------------------
\section{Feasibility Compression Definitions}
\label{sec:E_feasibility_defs}

RTC+B makes intertemporal feasibility constraints economically visible through
shadow prices. Because SoC is not fully observable in public data, feasibility is
measured using \emph{proxies} and \emph{conditioning} rather than point claims.

\subsection{Capability-to-requirement ratio}

For product $k$ at SCED interval $t$, define:
\[
\kappa_{k,t}:=\frac{C_{k,t}}{R_{k,t}},
\]
where $R_{k,t}$ is the requirement and $C_{k,t}$ is an available capability
aggregate (or best admissible proxy). Small $\kappa$ indicates compression.

\subsection{Persistence}

Feasibility compression is intertemporal, so persistence matters. Define a run
length statistic:
\[
L_{k}(t;\kappa^\star)=\max\{\ell \ge 1 : \kappa_{k,t-r} < \kappa^\star \ \forall r=0,\ldots,\ell-1\}.
\]
A feasibility mechanism predicts that tail MCPC probabilities increase with run
length, not only with instantaneous compression.

\subsection{Conditional tail probability}

For product $k$, define:
\[
p_{k}(\tau_k \mid \kappa^\star, X_t)
:=\Pr\!\left(\mathrm{MCPC}_{k,t}>\tau_k \mid \kappa_{k,t}<\kappa^\star, X_t\right),
\]
where $X_t$ is the matched system condition vector (load/net-load/ramp/outage
proxy/stress indicators). This conditional probability is the primary
feasibility-coherence diagnostic.

% ----------------------------------------------------------------------
\section{Geometry--Feasibility Linkage (Formal Statement)}
\label{sec:E_linkage}

The Geometry hypothesis is that DA award shape influences RT scarcity risk by
altering the \emph{effective feasible margin} under intertemporal constraints.

\subsection{Mechanism statement (non-causal, testable)}

Let $G_{k,d,h}$ denote the DA geometry bin and let $\mathcal{T}(d,h)$ denote the
SCED intervals mapping into that hour. The testable claim is:

\begin{quote}
Conditional on matched system conditions $X_t$,
hours classified as \emph{brittle} in DA geometry exhibit higher probabilities of
RT feasibility compression (low $\kappa$ and long $L_k$) and higher RT MCPC tail
incidence.
\end{quote}

This can be written as a monotonic association condition:
\[
\Pr(\mathrm{MCPC}_{k,t}>\tau_k \mid G_{k,d,h}=\text{brittle}, X_t)
\ge
\Pr(\mathrm{MCPC}_{k,t}>\tau_k \mid G_{k,d,h}=\text{broad}, X_t),
\quad t \in \mathcal{T}(d,h).
\]

\subsection{Why this is not ``quantity''}

Two DA hours may clear identical total quantities $\sum_i q^{DA}_{i,k,d,h}$ but
have different $(H,\Delta P)$; the brittle hour can fail under small perturbations
(resource trip, correlated SoC depletion, qualification failures), producing RT
scarcity with little warning from aggregate MW alone.

This distinction underpins the ``paper long'' paradox documented in Fern.

% ----------------------------------------------------------------------
\section{One-Page Geometry Summary (Staff-Facing)}
\label{sec:E_onepage}

For staff-facing summaries, Geometry should be described using three minimal
signals per product-hour:

\begin{enumerate}
  \item \textbf{Quantity posture:} DA awarded MW relative to requirement;
  \item \textbf{Concentration:} $H_{k,d,h}$ (broad vs brittle);
  \item \textbf{Dispersion:} $\Delta P_{k,d,h}$ (graded vs dominated).
\end{enumerate}

Escalation is recommended only when Geometry is brittle \emph{and} feasibility
compression proxies ($\kappa$ persistence) are concurrently elevated.

\newpage
\chapter{Merchant-to-Contract Translation}
\section{Hedge Structures, Settlement Mapping, and Bankability Controls}
\label{app:merchant_to_contract}

This appendix translates the merchant valuation decomposition in
Chapter~\ref{ch:investor_full} into contract structures commonly used to render
cashflows financeable. The objective is not to recommend a single contract form,
but to provide a mapping from (i) RTC+B revenue primitives and feasibility
constraints to (ii) hedgeable indices, settlement mechanics, and residual risks.

Throughout, we maintain the report’s traceability discipline: each contract term
must reference an observable price/award/performance series that is either
settlement-grade or explicitly labeled as illustrative only.\footnote{
A lender’s core diligence question is: ``Can I independently reconcile contract
cashflows to a public or auditable settlement series?'' Contract structures that
depend on opaque, non-reconcilable metrics increase basis risk and raise
discount rates.}

\subsection{Merchant cashflow baseline}

We begin from the canonical decomposition:
\[
\Pi = \Pi_{\text{Energy}} + \Pi_{\text{AS}} - \Pi_{\text{Charging}}
      - \Pi_{\text{Degradation}} - \Pi_{\text{PerformanceRisk}} - \Pi_{\text{Other}}.
\]
Contracting acts primarily by (i) converting volatile components into fixed or
collared payments, (ii) reallocating tail and performance risk, and (iii)
reducing feasibility-driven monetization uncertainty (notably in stress episodes).

\subsection{Hedge taxonomy and mapping to RTC+B primitives}

We classify contracts into six hedge families. Each family defines:
(i) \emph{Index}, (ii) \emph{Settlement rule}, (iii) \emph{Residual risks}.

\subsubsection{Energy-only swap (hub or node)}
\paragraph{Index.}
A settlement point price series $\pi_t$ at a hub (HB\_NORTH, HB\_SOUTH,
HB\_HOUSTON) or nodal proxy.

\paragraph{Settlement.}
For contracted volume $Q$ and strike $K$, the classic fixed-for-floating swap:
\[
\text{Payoff}_{\text{swap}} = \sum_{t\in\mathcal{T}} Q \left(\pi_t - K\right)\Delta t.
\]

\paragraph{Mapping.}
This hedges the net energy margin component
$\Pi_{\text{Energy}} - \Pi_{\text{Charging}}$ only insofar as the dispatch is
aligned with the hedged quantity.\footnote{
Energy swaps do not hedge SoC feasibility risk; they hedge the price level.
If SoC depletion prevents discharging during high-price intervals, the resource
may still owe the swap payoff while failing to earn the offsetting merchant
revenue.}

\paragraph{Residual risks.}
Volume/shape risk, basis (hub vs node), and feasibility truncation risk.

\subsubsection{Virtual toll (heat-rate analog for storage)}
\paragraph{Index.}
A structured spread between discharge and charge prices, often implemented as an
option-like payoff on $\pi_t$ with an implied charging cost.

\paragraph{Settlement.}
A stylized toll can be written as:
\[
\text{Payoff}_{\text{toll}} = \sum_{t\in\mathcal{T}}
\left[q^{\mathrm{dis}}_t(\pi_t - K_{\mathrm{dis}})
      - q^{\mathrm{ch}}_t(\pi_t - K_{\mathrm{ch}})\right]\Delta t,
\]
where $K_{\mathrm{dis}}$ and $K_{\mathrm{ch}}$ define an implied spread
requirement.

\paragraph{Mapping.}
This approach approximates arbitrage economics and can reduce exposure to the
raw level of $\pi_t$, but it remains sensitive to the dispatch and SoC dynamics.

\paragraph{Residual risks.}
Dispatch compliance risk, SoC infeasibility in stress episodes, and settlement
interpretation risk (how the contract defines ``charge'' vs ``discharge'').

\subsubsection{Ancillary revenue share / fixed AS payment with true-up}
\paragraph{Index.}
Ancillary MCPC series $\text{MCPC}_{k,t}$ and awards $A_{k,t}$ for products
$k\in\{\text{Reg-Up, Reg-Down, RRS, ECRS, FRRS}\}$ as applicable.

\paragraph{Settlement.}
A fixed AS capacity payment plus a true-up on realized AS gross revenue:
\[
\text{Payoff}_{\text{AS}} =
F^{\text{AS}} + \alpha \sum_{t}\sum_{k}
\text{MCPC}_{k,t}A_{k,t}\Delta t,
\]
where $\alpha\in[0,1]$ allocates upside/downsides between counterparties.

\paragraph{Mapping.}
This directly contracts the RTC+B scarcity channel that is most salient in the
post-change regime: $\Pi_{\text{AS}}$.\footnote{
The central underwriting insight of RTC+B is that scarcity value can migrate
into ancillary products. If the goal is to de-risk merchant exposure, the hedge
must reference the product where scarcity rents appear, not only the energy
price.}

\paragraph{Residual risks.}
Award uncertainty (quantity risk), qualification risk, and performance penalties
(e.g., $\Pi_{\text{PerformanceRisk}}$) unless explicitly allocated.

\subsubsection{AS collar (floor/cap) on MCPC tails}
\paragraph{Index.}
Tail-indexed ancillary settlement, such as exceedance-weighted MCPC:
\[
I_k(\tau_k) := \sum_{t}\left(\text{MCPC}_{k,t}-\tau_k\right)_+\Delta t,
\]
for pre-registered tail thresholds $\tau_k$.

\paragraph{Settlement.}
A collar pays the resource a floor on $I_k(\tau_k)$ and/or caps the upside:
\[
\text{Payoff}_{\text{collar}} =
\min\left\{\max\left(I_k(\tau_k),\, \underline{I}\right),\, \overline{I}\right\}.
\]

\paragraph{Mapping.}
This hedges precisely what lenders worry about: tail-driven revenue volatility
and regime sensitivity.

\paragraph{Residual risks.}
Model risk in selecting $\tau_k$, basis between contracted tail index and the
resource’s actual monetizable tail exposure, and SoC truncation.

\subsubsection{Tolling with availability obligation (capacity-style for storage)}
\paragraph{Index.}
A fixed availability payment $F^{\text{avail}}$ tied to demonstrated availability
and/or awarded quantities, with explicit performance regime.

\paragraph{Settlement.}
\[
\text{Payoff}_{\text{avail}} =
F^{\text{avail}} - \sum_{t}\sum_{k} \pi^{\mathrm{pen}}_{k,t}\cdot
\mathbf{1}\{\text{short}_{k,t}>0\}.
\]

\paragraph{Mapping.}
This converts the merchant profile into something closer to a capacity contract,
but it only works if ``availability'' is defined in a way that is feasible for
an intertemporal asset.\footnote{
Availability definitions must incorporate SoC. A pure MW availability obligation
can be incoherent for energy-limited resources unless the contract also defines
charging rights, SoC bands, or operator dispatch authority.}

\paragraph{Residual risks.}
Penalty tail risk, SoC management risk, and operational control ambiguity
(who controls charging/discharging decisions).

\subsubsection{Hybrid structures: energy hedge + AS hedge + degradation pass-through}
\paragraph{Structure.}
A bankability-oriented package often combines:
\begin{itemize}
  \item A partial energy swap (reducing level risk),
  \item An AS floor or revenue share (stabilizing scarcity rents),
  \item A degradation pass-through or cycling budget (controlling hidden costs),
  \item Explicit performance allocation (penalties and telemetry).
\end{itemize}

\paragraph{Degradation pass-through.}
A simple pass-through defines an allowed throughput budget $B$ (MWh) per month:
\[
\text{DegAdj} = c^{\mathrm{deg}}\cdot \left(\sum_t (q^{\mathrm{ch}}_t + q^{\mathrm{dis}}_t)\Delta t - B\right)_+,
\]
allocating excess cycling costs to the dispatch-controlling party.\footnote{
This aligns incentives: the party requesting heavy AS participation bears the
incremental cycling/degradation cost that is otherwise externalized.}

\subsection{Contract residual risk ledger (what remains merchant)}

Even with contracting, four residual risks typically remain unless explicitly
addressed:

\begin{enumerate}
  \item \textbf{Feasibility truncation risk:} inability to monetize tails due to
        SoC depletion or intertemporal constraints.
  \item \textbf{Qualification and telemetry risk:} loss of eligibility for
        specific products, driving administrative scarcity exposure.
  \item \textbf{Basis risk:} mismatch between contracted index (hub) and the
        resource’s settlement node or obligation location.
  \item \textbf{Protocol/change risk:} settlement definitions and product rules
        evolve; hedges that are not definition-robust can mis-hedge after policy
        updates.\footnote{
Policy risk is not hypothetical in ERCOT; protocol changes and NPRRs can
materially alter where scarcity rents appear and how obligations are settled.
A lender-grade package requires a documented change-log and contractual
renegotiation triggers tied to definitional changes.}
\end{enumerate}

\subsection{Bankability checklist: minimum data and contract clauses}

A lender-grade contract package should be accompanied by:

\begin{itemize}
  \item \textbf{Index definitions:} explicit mapping to ERCOT settlement-grade
        artifacts (price series, MCPC, awards).
  \item \textbf{Performance regime:} penalty allocation, telemetry requirements,
        cure periods, and stress-state provisions.
  \item \textbf{SoC governance:} who controls charging/discharging, minimum SoC
        bands for availability, and rights to procure charging energy.
  \item \textbf{Degradation governance:} cycling budget, pass-through mechanics,
        and OEM warranty alignment.
  \item \textbf{Change-in-law/protocol clause:} renegotiation triggers linked to
        definitional changes in products or settlement treatment.
\end{itemize}

\subsection{Interpretive caution}
Contracting can make cashflows more financeable, but it can also concentrate
risk in non-obvious places (e.g., penalty tails in stress regimes). Therefore,
any contracted valuation should be reported with the same P10/P50/P90 discipline
as merchant valuation, with an explicit ``residual risk'' ledger that identifies
what remains exposed and why.

\newpage


\chapter{Day-Ahead Geometry and Feasibility Linkage}
\label{app:geometry_feasibility}
% =========================================================

This appendix formalizes the concept of \emph{Day-Ahead (DA) award geometry} and
its linkage to real-time feasibility outcomes under RTC+B. The purpose is
definitional and diagnostic: to make precise what is meant by ``geometry,'' how
it differs from aggregate procurement volume, and why it provides leading
information about scarcity realization that is not observable from quantities
alone.

Nothing in this appendix asserts welfare improvement, market power, or operator
error. All constructs are descriptive diagnostics intended to support the
interpretation of observed price and feasibility patterns elsewhere in the
report.

% ---------------------------------------------------------

\section{Definitions: Day-Ahead Award Geometry}
\label{app:def_geometry}

Let $k \in \mathcal{K}$ index ancillary service products (e.g., Reg-Up, Reg-Down,
ECRS), $d$ index operating days, and $h$ index DA hours.

Let $Q^{DA}_{i,k,d,h}$ denote the DA-cleared award quantity for resource $i$ in
product $k$ during hour $h$ on day $d$, as reported in settlement-grade
\texttt{DAMASAGG} artifacts.

Define total cleared quantity:
\[
Q^{DA}_{k,d,h} := \sum_{i} Q^{DA}_{i,k,d,h}.
\]

\subsection{Concentration (Herfindahl Index)}

The \emph{award concentration index} for product $k$ is defined as:
\[
H_{k,d,h} := \sum_{i} \left( \frac{Q^{DA}_{i,k,d,h}}{Q^{DA}_{k,d,h}} \right)^2.
\]

Properties:
\begin{itemize}
  \item $H_{k,d,h} \in (0,1]$.
  \item Lower values correspond to broad participation.
  \item Higher values indicate reliance on a small set of marginal suppliers.
\end{itemize}

This index measures \emph{award geometry}, not market power. High $H$ can arise
from qualification constraints, feasibility limits, or correlated resource
availability.

\subsection{Price Dispersion (Slope Proxy)}

Let $P^{DA}_{i,k,d,h}$ denote the DA clearing price associated with resource $i$
for product $k$ (where available). Define a dispersion proxy:
\[
\Delta P_{k,d,h} := Q_{0.90}(P^{DA}_{\cdot,k,d,h}) - Q_{0.10}(P^{DA}_{\cdot,k,d,h}).
\]

Interpretation:
\begin{itemize}
  \item Wide dispersion indicates a sloped supply stack.
  \item Narrow dispersion indicates a flat stack with a sharp marginal jump.
\end{itemize}

Together, $(H_{k,d,h}, \Delta P_{k,d,h})$ characterize DA stack shape.

\subsection{Geometry States}

For diagnostic purposes, DA geometry is discretized into regimes:
\[
\text{Geometry}_{k,d,h} \in \{\text{Broad}, \text{Intermediate}, \text{Brittle}\},
\]
based on pre-registered thresholds on $H_{k,d,h}$ and $\Delta P_{k,d,h}$.

Thresholds are fixed ex ante and logged in the episode ledger.

% ---------------------------------------------------------

\section{Feasibility Compression and Capability Ratios}
\label{app:def_feasibility}

RTC+B introduces intertemporal feasibility constraints that are not visible in
single-period quantity clearing.

Let $C_{k,t}$ denote total available capability for product $k$ in SCED interval
$t$, and let $R_{k,t}$ denote the corresponding requirement.

Define the \emph{capability ratio}:
\[
\kappa_{k,t} := \frac{C_{k,t}}{R_{k,t}}.
\]

Low values of $\kappa_{k,t}$ indicate feasibility compression consistent with:
\begin{itemize}
  \item State-of-charge depletion,
  \item Charging infeasibility,
  \item Qualification or telemetry binding constraints.
\end{itemize}

Importantly, $\kappa_{k,t}$ is a fleet-level proxy; it does not identify resource-
specific SoC trajectories.

% ---------------------------------------------------------

\section{Geometry--Feasibility Linkage}
\label{app:geometry_feasibility_link}

This section formalizes the connection between DA geometry and RT feasibility
outcomes.

\subsection{Conceptual Mechanism}

A DA award stack with high concentration ($H_{k,d,h}$ large) implies that a small
number of marginal resources are responsible for satisfying requirements.

Under RTC+B, those marginal resources are subject to intertemporal feasibility
constraints. If one or more become infeasible (e.g., due to SoC exhaustion),
effective capability collapses discontinuously.

This produces:
\begin{itemize}
  \item Sharp declines in $\kappa_{k,t}$,
  \item Binding ancillary constraints,
  \item Elevated MCPCs without corresponding energy-side scarcity.
\end{itemize}

\subsection{Testable Conditional Relationships}

The linkage is evaluated using conditional probabilities:
\[
\Pr\!\left( \mathrm{MCPC}_{k,t} > \tau_k \;\middle|\;
H_{k,d,h} > H^\star,\; X_t \right),
\]
and
\[
\Pr\!\left( \kappa_{k,t} < \kappa^\star \;\middle|\;
H_{k,d,h} > H^\star,\; X_t \right),
\]
where $X_t$ denotes matched system conditions (load, net load, outages proxy).

Persistence across products and time provides evidence of structural linkage,
not coincidental scarcity.

\subsection{Interpretive Boundaries}

These diagnostics establish \emph{structural coherence}, not causality:
\begin{itemize}
  \item Geometry does not prove withholding.
  \item Geometry does not imply operator error.
  \item Geometry identifies fragility, not welfare loss.
\end{itemize}

Causal claims require awards, deployments, and telemetry, as discussed in the
Identification and Inference Limits chapter.

% ---------------------------------------------------------

\section{Relation to Fern Episode Analysis}
\label{app:geometry_fern}

During the January~22--29,~2026 Fern packet:
\begin{itemize}
  \item DA award volumes often appeared adequate in aggregate.
  \item Geometry metrics showed elevated $H_{k,d,h}$ in hours preceding RT
        scarcity.
  \item MCPC tails materialized under modest energy-side adders.
\end{itemize}

This appendix provides the formal vocabulary required to state that Fern
represented a case of \emph{false feasibility}: sufficient volume cleared under
a brittle geometry.

All Fern-specific empirical results using these constructs are reported in
Wave~3-D and logged in the episode ledger.

% ---------------------------------------------------------

\section{Limitations and Extensions}

Public ERCOT data constrain geometry diagnostics to award-level observables.
Extensions requiring additional data include:
\begin{itemize}
  \item Resource-level SoC telemetry,
  \item Deployment histories,
  \item Qualification failure logs.
\end{itemize}

Absent these, geometry remains a \emph{leading indicator} rather than a complete
feasibility model.

\begingroup\sloppy
\begin{table}[!ht]
\centering
\small
\setlength{\tabcolsep}{3pt}
\caption{Day-Ahead Geometry Diagnostics: Thresholds and Interpretation}
\label{tab:geometry_summary}
\begin{tabular}{>{\raggedright\arraybackslash}p{1.35in} >{\raggedright\arraybackslash}p{1.15in} >{\raggedright\arraybackslash}p{2.0in} >{\raggedright\arraybackslash}p{1.95in}}
\toprule
\textbf{Metric} & \textbf{Threshold Regime} & \textbf{Operational Interpretation} & \textbf{Implication Under RTC+B} \\
\midrule
Herfindahl Index $H_{k,d,h}$
& Low ($<0.15$)
& Broad award participation; marginal supply diversified
& Feasibility loss likely gradual; MCPC tails require system-wide stress \\
\cmidrule(l){2-4}
& Moderate ($0.15$--$0.30$)
& Partial concentration; marginality shared by limited set
& Elevated sensitivity to SoC or qualification failures \\
\cmidrule(l){2-4}
& High ($>0.30$)
& Brittle geometry; reliance on few marginal suppliers
& Discontinuous feasibility loss possible; MCPC spikes likely without energy adders \\
\midrule
Price Dispersion $\Delta P_{k,d,h}$
& Wide
& Sloped supply stack; price discovery across multiple blocks
& Scarcity emergence likely smoother; tails diffuse \\
\cmidrule(l){2-4}
& Narrow
& Flat stack with sharp marginal jump
& Small feasibility loss produces large MCPC response \\
\midrule
Capability Ratio $\kappa_{k,t}$
& $\kappa > 1.3$
& Feasible headroom present
& MCPC unlikely absent exogenous shock \\
\cmidrule(l){2-4}
& $\kappa \approx 1.0$
& Binding feasibility margin
& MCPC reflects intertemporal constraint, not volume shortage \\
\cmidrule(l){2-4}
& $\kappa < 1.0$
& Infeasible without corrective action
& MCPC tails expected; energy prices may remain muted \\
\bottomrule
\end{tabular}
\end{table}
\endgroup

\newpage

% =========================================================
\chapter{Geometry Robustness, Placebos, and Falsification Tests}
\label{app:geometry_robustness}
% =========================================================

This appendix provides a comprehensive robustness and falsification framework
for the Day-Ahead Geometry results developed in Wave~3-D. Its purpose is to
evaluate whether observed relationships between Day-Ahead award geometry,
intertemporal feasibility compression, and Real-Time scarcity pricing are
structural, incidental, or artefactual.

Consistent with the identification discipline articulated in the Identification and Inference Limits chapter, this appendix does not seek to establish causal
effects. Instead, it assesses whether the geometry-based diagnostics remain
informative under alternative constructions, exclusion tests, placebo windows,
and negative controls.

All tests herein are designed to be reproducible using ERCOT-published data
artifacts.

% ----------------------------------------------------------------------
\section{Scope and Null Hypotheses}
\label{sec:F_scope_nulls}

The robustness analysis is structured around three nested null hypotheses:

\begin{description}
  \item[$H_0^{(1)}$ (Incidental Geometry):]
  Observed concentration or convexity in Day-Ahead awards is unrelated to
  subsequent Real-Time scarcity outcomes.

  \item[$H_0^{(2)}$ (Non-Feasibility Mechanism):]
  Any correlation between geometry metrics and Real-Time MCPC tails arises from
  contemporaneous load, weather, or system-wide stress conditions, not from
  intertemporal feasibility constraints.

  \item[$H_0^{(3)}$ (Measurement Artefact):]
  Geometry–scarcity relationships are sensitive to window definitions,
  aggregation choices, or isolated episodes and therefore lack operational
  interpretability.
\end{description}

Rejection of these nulls is not binary. Instead, findings are graded according to
whether they survive increasingly stringent falsification layers.

% ----------------------------------------------------------------------
\section{Alternative Geometry Constructions}
\label{sec:F_alt_geometry}

Wave~3-D defines Day-Ahead geometry primarily using the Herfindahl concentration
index $H_{k,d,h}$ and intra-hour price dispersion $\Delta P_{k,d,h}$. To test
robustness, all analyses are re-run using alternative but theoretically
equivalent constructions.

\subsection{Concentration Metrics}

The following substitutes are evaluated:

\begin{itemize}
  \item Share of top-$n$ awarded resources ($n \in \{1,3,5\}$),
  \item Gini coefficient of awarded quantities,
  \item Entropy-based diversity measures.
\end{itemize}

Results are considered robust if the qualitative classification of a given
hour--product pair as \emph{broad} or \emph{brittle} is invariant across at least
two independent concentration metrics.

\subsection{Price-Space Geometry}

Price dispersion is recomputed under:

\begin{itemize}
  \item Block-level DA awards vs hourly aggregation,
  \item Volume-weighted vs unweighted dispersion,
  \item Exclusion of the marginal clearing block.
\end{itemize}

This ensures that convexity diagnostics are not driven mechanically by a single
price-setting resource.

% ----------------------------------------------------------------------
\section{Temporal Alignment and Lead--Lag Sensitivity}
\label{sec:F_lead_lag}

A core hypothesis of Wave~3-D is that Day-Ahead geometry acts as a \emph{leading
indicator} of Real-Time feasibility stress.

To test this, geometry metrics computed at $(d,h)$ are aligned against Real-Time
outcomes under multiple offsets:

\[
(d,h) \rightarrow (t = d,h+\ell), \quad \ell \in \{-2,-1,0,+1,+2,+4\}.
\]

Interpretation rules:
\begin{itemize}
  \item Predictive alignment (geometry precedes MCPC tails) strengthens
  feasibility interpretation;
  \item Contemporaneous-only alignment weakens causal relevance;
  \item Reverse alignment (MCPC preceding geometry) invalidates interpretation.
\end{itemize}

Observed relationships during Fern are required to persist for $\ell \ge 0$ to
remain admissible.

% ----------------------------------------------------------------------
\section{Leave-One-Episode-Out Analysis}
\label{sec:F_loo}

To guard against single-event dominance, all geometry–scarcity relationships are
re-estimated under a leave-one-episode-out protocol.

Let $\mathcal{E}$ denote the set of Fern packet scarcity episodes. For each
$e \in \mathcal{E}$, the full analysis is recomputed on
$\mathcal{E} \setminus \{e\}$.

Findings are downgraded if:
\begin{itemize}
  \item Sign or ordering reverses under any exclusion, or
  \item Effect magnitude collapses by more than 50\%.
\end{itemize}

This test is especially important given the clustering of stress during winter
events.

% ----------------------------------------------------------------------
\section{Negative-Control Products and Hours}
\label{sec:F_negative_controls}

Negative controls are used to detect spurious associations.

\subsection{Product-Level Controls}

Where an ancillary service product $k'$ is operationally orthogonal to the
feasibility mechanism under study (e.g., minimal ESR participation or duration
coupling), geometry metrics should not predict MCPC tails.

Significant geometry–MCPC relationships in negative-control products are treated
as evidence against feasibility-specific interpretation.

\subsection{Temporal Controls}

Geometry metrics computed during system-normal hours (high $\kappa$, low net-load
ramps) are tested for association with MCPC tails. Significant relationships in
these windows indicate confounding rather than mechanism.

% ----------------------------------------------------------------------
\section{Placebo Regime Analysis}
\label{sec:F_placebo}

The complete Wave~3-D pipeline is applied to placebo periods satisfying the
following criteria:

\begin{itemize}
  \item No RTC+B or ASDC design changes,
  \item Similar seasonal and diurnal load profiles,
  \item Comparable ESR fleet penetration (to the extent observable).
\end{itemize}

If geometry metrics exhibit similar predictive power in placebo regimes, the
design-change interpretation is rejected.

Placebo results are reported alongside Fern results in all summary tables.

% ----------------------------------------------------------------------
\section{Stress-Composition Controls}
\label{sec:F_stress_composition}

To isolate geometry from general stress severity, all analyses are stratified by:

\begin{itemize}
  \item Net-load ramp percentiles,
  \item System-wide reserve margin proxies,
  \item Temperature and weather bins.
\end{itemize}

Geometry metrics must retain explanatory power \emph{within} stress strata to be
considered mechanism-relevant.

% ----------------------------------------------------------------------
\section{Interpretation Ladder and Evidence Grading}
\label{sec:F_interpretation}

Results are classified according to the following ladder:

\begin{description}
  \item[Grade A (Structural):]
  Survives all robustness, negative-control, and placebo tests.

  \item[Grade B (Consistent):]
  Survives most tests but weakens under one falsification layer.

  \item[Grade C (Descriptive):]
  Correlational only; sensitive to specification.

  \item[Grade D (Inadmissible):]
  Fails placebo or negative-control tests.
\end{description}

Only Grade A and B findings are referenced in policy-facing chapters.

% ----------------------------------------------------------------------
\section{Implications for Monitoring and Market Design}
\label{sec:F_implications}

Even where geometry diagnostics are downgraded to descriptive, they remain
operationally useful as monitoring tools. However, escalation thresholds and
policy responses must be calibrated to the evidence grade.

This distinction is carried forward explicitly in
Chapter~\ref{ch:policy} and Chapter~\ref{ch:conclusion}.

% ----------------------------------------------------------------------
\section{Limitations}
\label{sec:F_limits}

This appendix does not resolve:
\begin{itemize}
  \item Resource-level SoC trajectories,
  \item Operator discretion during stressed intervals,
  \item Counterfactual dispatch under alternative procurement regimes.
\end{itemize}

Accordingly, no claim herein substitutes for settlement-quality telemetry or
structural simulation.

% ----------------------------------------------------------------------
\section{Summary}
\label{sec:F_summary}

Appendix~F establishes that Day-Ahead geometry is a testable, falsifiable, and
bounded analytical construct. When it survives the full robustness stack, it
provides a defensible early-warning signal for intertemporal feasibility stress
under RTC+B. When it does not, the framework correctly downgrades inference.

This appendix therefore functions as the principal safeguard against over-
interpretation of Wave~3-D results.

\chapter{Identification, Falsification, and Robustness Diagnostics}
\label{app:verification_full}

This appendix consolidates a suite of rigorously defined diagnostics,
tests, and falsification constructs designed to **close all credible attack
vectors** identified in adversarial review (IMM, academic, and credit
perspectives). The goal is to transform descriptive insights about the
Real-Time Co-optimization with Batteries (RTC+B) regime into *defensible,
testable, and bounded scientific claims* that are auditable with available or
soon-to-be-available ERCOT data.

The diagnostics fall into four families:
\begin{enumerate}
  \item \textbf{Price-Blind Episode Definitions and Selection Protocols}, to
  eliminate selection-on-outcome biases.
  \item \textbf{Behavioral Falsification Tests}, to distinguish exhaustion
  from strategic behavior.
  \item \textbf{Parameter Envelopes for Investor Economics}, to bound
  performance claims.
  \item \textbf{Geometry, Basis, and Placebo Diagnostics}, to ensure
  structural claims are not artifacts of seasonal or forecast conditions.
\end{enumerate}

---

\section{Exogenous Episode Definition (Price-Blind Selection)}
\label{app:fern_price_blind}

A persistent critique is that episode definitions which partially rely on
pricing outcomes (e.g., high MCPC or energy adders) introduce a selection bias:
if episodes are defined by high prices, then it is unsurprising that prices
appear high within those episodes.

To address this, we adopt an entirely *price-blind* episode definition based
on **exogenous physical system stressors** that are independent of market
prices.

\begin{definition}[Exogenous System Stress Packet]
A contiguous set of SCED intervals $E = \{t_0, \dots, t_1\}$ qualifies as an
\emph{Exogenous System Stress Packet} if and only if all of the following are
present for the duration:
\begin{itemize}
  \item \textbf{Extreme Load Condition:}
    Net system load or peak load demand is above its historical 95th percentile
    for the same seasonal and temperature conditions.
  \item \textbf{Reserve Margin Compression:}
    The total available ancillary service capability (from ERCOT’s historical
    capability reports) is below its historical 10th percentile conditional
    on load.
  \item \textbf{Environmental Stressor:}
    Weather indicators (temperature, wind chill, heat index) are in the top
    or bottom 5th percentile of historical values for the zone or region.
\end{itemize}
The packet is defined without reference to any real-time market prices,
MCPCs, or energy adders.
\end{definition}

This price-blind definition is intended to be an operational analog to a
natural experiment: it selects intervals where physical stress is nearly
exogenous to market design, enabling a **falsifiable test** of whether pricing
mechanisms behave differently under stress.

\paragraph{Price Stress Conditional Test.}
To validate whether scarcity is more prevalent under RTC+B than prior regimes,
we compare:
\[
\Pr(\text{Scarcity Price}\mid E, M_1) \quad \text{versus} \quad
\Pr(\text{Scarcity Price}\mid E, M_0),
\]
where “Scarcity Price” can be defined as:
(i) $\mathrm{MCPC}_{k,t} > \tau_k$ for ancillary products; or
(ii) $a^{\mathrm{RT}}_{t,z} > \tau_\mathrm{energy}$ for energy adders.

This test conditions purely on exogenous stress $E$, avoiding dependent variable sampling.

---

\section{Deployment Re-Entry and Strategic Behavior Diagnostics}
\label{app:deployment_reentry}

An IMM concern is that patterns consistent with resource “participation exhaustion”
may instead reflect strategic withholding. Without battery SoC telemetry, we
cannot observe state directly; however, we *can* exploit behavioral timing in
awards and dispatch to differentiate hypotheses.

\begin{definition}[Deployment Re-Entry Latency]
For a resource $i$ that is awarded or dispatched at time $t_0$ prior to scarcity
stress, define re-entry latency $\rho_i$ as:
\[
\rho_i := \min\{t > t_0: \text{resource }i\text{ receives a positive award or
energy dispatch}\}.
\]
If resource $i$ never re-enters prior to the end of the packet, set $\rho_i
= \infty$.
\end{definition}

The behavioral logic is:
\begin{itemize}
  \item If a resource’s absence is due to physical exhaustion (depleted SoC),
    then $\rho_i$ should be near or at the end of the stress packet.
  \item If the absence is due to strategic withholding, then $\rho_i$ should
    occur promptly once scarcity pricing escalates further.
\end{itemize}

\textbf{Operational test:} Construct the empirical distribution of re-entry
latencies $\{\rho_i\}$ across episodes and test:
\[
H_0: \rho_i \ll \text{Duration}(E) \quad \text{(strategic\,behavior)} \\
H_1: \rho_i \approx \text{Duration}(E) \quad \text{(exhaustion)}.
\]
This can be implemented using ERCOT RTM award and dispatch data (e.g.,
Real-Time Market Clearing Prices for Capacity and award logs). Such data is
publicly available via ERCOT dashboards and weekly reports for MCPC and awards
(e.g., Reg-Up, Reg-Down, RRS, Non-Spin, ECRS).

\section{Parameter Sensitivity and Revenue Envelope for ESR Economics}
\label{app:param_sensitivity}

Investor critiques focus on arbitrary parameters (e.g., degradation cost).
Instead of selecting a single value, we define an envelope.

\begin{definition}[Degradation Cost Envelope]
Let $\gamma$ denote the unit degradation cost ($\$/\mathrm{MWh}$) of an
energy storage resource (ESR). Define plausible lowest and highest bounds
$\gamma_{\min}$ and $\gamma_{\max}$ based on engineering degradation
estimates and industry O\&M metrics (e.g., 1–5 \$/MWh for utility lithium
systems). Then the revenue envelope over a set of intervals $\mathcal{T}$ is:
\[
\Pi_{\text{min}}(\mathcal{T}) := \min_{\gamma \in [\gamma_{\min},\gamma_{\max}]}
\, \Pi(\gamma,\mathcal{T}); \quad
\Pi_{\text{max}}(\mathcal{T}) := \max_{\gamma \in [\gamma_{\min},\gamma_{\max}]}
\, \Pi(\gamma,\mathcal{T}),
\]
where $\Pi(\gamma,\mathcal{T})$ is the total net revenue given $\gamma$.
\end{definition}

Reporting $[\Pi_{\text{min}},\Pi_{\text{max}}]$ clarifies that revenue claims
are an **envelope**, not a point estimate, and must be stress-tested across
parameter uncertainty.

This method parallels industry standard techniques such as P10/P50/P90 envelopes
used by lenders and engineers.

---

\section{False Positive/Negative Geometry Tests}
\label{app:geo_fpfn}

Correlation between Day-Ahead geometry metrics and subsequent Real-Time scarcity
needs a falsification logic to guard against spurious associations.

**High–No Scarcity Test:** Identify intervals with geometry above its upper
tail (e.g., 90th quantile) but where MCPC and energy scarcity prices are
not elevated. A large number of such intervals would indicate geometry alone
is insufficient for scarcity.

**Low–Scarcity Geometry Test:** Identify intervals with geometry below its
median but where Real-Time scarcity prices occur. Frequent occurrences here
would indicate geometry is not necessary for scarcity.

Both tests can be executed using publicly accessible MCPC dashboards
(Real-Time Market Clearing Prices for Capacity by product).

---

\section{Local Basis Premium and Nodal Diagnostics}
\label{app:basis_premium}

System-wide pricing can mask local constraints. Define the Basis Premium:
\[
\beta_{n,t} := \pi_{t,n}^{\mathrm{SPP}} - \pi_{t,\mathrm{sys}}^{\mathrm{SPP}},
\]
where $\pi_{t,n}^{\mathrm{SPP}}$ is the local settlement point price and
$\pi_{t,\mathrm{sys}}^{\mathrm{SPP}}$ is an aggregate (e.g., load–weighted
hub average).

Two key diagnostics are:
\begin{itemize}
  \item \textbf{Basis Tail Frequency:} frequency with which $|\beta_{n,t}|$
    exceeds a small threshold in high scarcity intervals.
  \item \textbf{Directional Basis Skew:} correlation between $\beta_{n,t}$
    and MCPC tails to test whether system signals fail to translate locally.
\end{itemize}

These diagnostics guard against overgeneralization from system prices to local
asset economics. ERCOT’s Real-Time Market price displays support analysis of
SPP across hubs and zones.

---

\section{Placebo and Falsification Designs}
\label{app:placebo_designs}

To rule out spurious findings driven by seasonal, forecast, or structural shifts
unrelated to RTC+B, we define four placebo tests:

\begin{enumerate}
  \item \textbf{Placebo Window Test:} run the entire analytics pipeline on an
  adjacent pre-change period where no design change occurred, using identical
  metric definitions.
  \item \textbf{Negative Control Products:} identify products unaffected by
  the scarcity mechanism (e.g., long-duration reserve products) and verify no
  systematic tail shift.
  \item \textbf{Episode Permutation Test:} randomly permute episode labels to
  generate a null distribution of event panels.
  \item \textbf{Episode Trimming:} systematically exclude top influential
  episodes to test result robustness.
\end{enumerate}

Placebo test results must be logged and included in replication capsules.

---

\section{Summary of Diagnostics and Tests}

\begin{itemize}
  \item Price-blind systemic stress packets → eliminates selection bias.
  \item Re-entry latency → distinguishes strategic withholding vs exhaustion.
  \item Degradation cost envelope → lenders can assess revenue risk over bounds.
  \item Geometry false positive/negative tests → causal diagnostics.
  \item Basis premium diagnostics → local scarcity guardrails.
  \item Placebo tests → rule out spurious pre/post seasonal artifacts.
\end{itemize}


% ======================================================================
% Embedded bibliography (bibitems only; no separate .bib)
% ======================================================================
\begin{thebibliography}{99}

\bibitem{ercot_rtm_spp_display_pdf}
Electric Reliability Council of Texas (ERCOT). (2026). \emph{Real-Time Settlement Point Prices Display} [PDF]. (Author-provided export of ERCOT public display; includes note ``SPP values include Reliability Deployment Price for Energy'').

\bibitem{ercot_rtm_spp_map_screenshot}
Electric Reliability Council of Texas (ERCOT). (2026, January 1). \emph{Real-Time Locational Prices: Real-Time Market -- Settlement Point Pricing} [Screenshot]. (Author capture; updated 16{:}17; SPP includes Reliability Deployment Price for Energy).

\bibitem{ercot_rtm_lmp_table_screenshot}
Electric Reliability Council of Texas (ERCOT). (2026, January 1). \emph{Real-Time LMPs for Load Zones and Trading Hubs Display} [Screenshot]. (Author capture; updated 16{:}15{:}15; banner indicates LMP excludes RT adders; RTRDPA shown as \$0.00).

\bibitem{ercot_systemwide_prices_pdf}
Electric Reliability Council of Texas (ERCOT). (2026). \emph{System-Wide Prices} [PDF]. (Author-provided export of ERCOT dashboard display).

\bibitem{ercot_ancillary_services_pdf}
Electric Reliability Council of Texas (ERCOT). (2026). \emph{Ancillary Services} [PDF]. (Author-provided export of ERCOT dashboard display).

\bibitem{ercot_mcpc_prev_csv}
Electric Reliability Council of Texas (ERCOT). (2025). \emph{Real-Time Market Clearing Prices for Capacity (Previous 6 Days)} [CSV]. (Author-provided export: \texttt{real-time-market-clearing-prices-for-capacity-\allowbreak previous.csv}; 5-minute interval MCPC for REG-UP, REG-DOWN, RRS, NON-SPIN, ECRS; 2025-12-26 to 2025-12-31).

\bibitem{ercot_freq_csv}
Electric Reliability Council of Texas (ERCOT). (2026). \emph{System Frequency} [CSV]. (Author-provided export: \texttt{ancillary-services-frequency.csv}; 10-second samples, 2026-01-01 14{:}23{:}10 to 16{:}23{:}00).

\bibitem{caiso_tariff_section8_2025}
California Independent System Operator Corporation (CAISO). (2025, November 19). \emph{Section 8 -- Ancillary Services (Fifth Replacement Electronic Tariff)} [PDF]. Posted December 8, 2025. Retrieved January 1, 2026

\bibitem{caiso_rtm_section34_2024}
California Independent System Operator Corporation (CAISO). (2024, October 14). \emph{Section 34 -- Real-Time Market (Fifth Replacement Electronic Tariff)} [PDF]. Retrieved January 1, 2026

\bibitem{caiso_market_ops_bpm_reserves}
California Independent System Operator Corporation (CAISO). (2025). \emph{Market Operations BPM Appendices (Day-Ahead Market Enhancements)} [PDF]. Version 62 (Last Revised April 9, 2024; posted July 1, 2025). Retrieved January 1, 2026

\bibitem{pjm_manual11_2022}
PJM Interconnection, L.L.C. (2022, March 1). \emph{PJM Manual 11: Energy \& Ancillary Services Market Operations (M11)} [PDF]. Version 118. Retrieved January 1, 2026

\bibitem{pjm_manual12_2025}
PJM Interconnection, L.L.C. (2025). \emph{PJM Manual 12: Balancing Operations (M12)} [PDF]. Retrieved January 1, 2026

\bibitem{pjm_manual28_2024}
PJM Interconnection, L.L.C. (2025). \emph{PJM Manual 28: Operating Agreement Accounting (M28)} [PDF]. Retrieved January 1, 2026

\bibitem{pjm_sync_reserve_updates_2024}
PJM Interconnection, L.L.C. (2024, June 12). \emph{PJM Manual 28 -- Synchronized Reserve Deployment Updates} [Presentation]. Retrieved January 1, 2026

\bibitem{imm_reserve_price_formation_2022}
Monitoring Analytics, LLC. (2022, September 7). \emph{M11 Reserve Price Formation (RPF)} [Presentation]. Independent Market Monitor for PJM. Retrieved January 1, 2026

\bibitem{miso_bpms_page_2025}
Midcontinent Independent System Operator, Inc. (MISO). (2025). \emph{Tariff: Rules, Manuals, and Agreements (Business Practices Manuals and market modules)} [Webpage]. Retrieved January 1, 2026

\bibitem{miso_market_participation_overview_2024}
Midcontinent Independent System Operator, Inc. (MISO). (2024). \emph{Fact Sheet -- MISO Market Participation Overview} [PDF]. Retrieved January 1, 2026

\bibitem{miso_coopt_presentation_2019}
Midcontinent Independent System Operator, Inc. (MISO). (2019, September 24). \emph{MISO Energy and Ancillary Service Co-optimization} [PDF]. Retrieved January 1, 2026


\bibitem{ercot_imfr_methodology}
Electric Reliability Council of Texas (ERCOT). (2025). \emph{Interconnection Minimum Frequency Response (IMFR) Methodology}. (Author-provided ERCOT methodology document; describes IFRO allocation, FME selection, and IMFR calculation for BAL-001/BAL-003 compliance).

\bibitem{nerc_bal_001}
North American Electric Reliability Corporation (NERC). (n.d.). \emph{BAL-001: Real Power Balancing}. (Reliability standard governing balancing authority performance and BAAL-related requirements).
\bibitem{nerc_bal_003}
North American Electric Reliability Corporation (NERC). (n.d.). \emph{BAL-003: Frequency Response and Frequency Bias Setting}. (Reliability standard governing interconnection frequency response obligations and measurement concepts).

\bibitem{ercot_ordc_methodology_2015}
Electric Reliability Council of Texas (ERCOT). (2015). \emph{Methodology for Implementing Operating Reserve Demand Curve (ORDC) to Calculate Real-Time Reserve Price Adder} [PDF]. ERCOT Other Binding Document. (Board-approved revision package). \emph{ERCOT}. 

\bibitem{ercot_ordc_report_2022}
Electric Reliability Council of Texas (ERCOT). (2022). \emph{2022 Biennial ERCOT Report on the Operating Reserve Demand Curve (ORDC)} [PDF]. \emph{ERCOT}. 

\bibitem{ercot_ordc_report_2024}
Electric Reliability Council of Texas (ERCOT). (2024). \emph{2024 Biennial ERCOT Report on the Operating Reserve Demand Curve (ORDC)} [PDF]. \emph{ERCOT}. 

\bibitem{ercot_asdc_overview_2024}
Electric Reliability Council of Texas (ERCOT). (2024). \emph{AS Demand Curves (ASDC) Overview} [PPTX]. RTCBTF materials. \emph{ERCOT}. 

\bibitem{ercot_nprr1186_comments_2023}
Electric Reliability Council of Texas (ERCOT). (2023). \emph{NPRR 1186: ERCOT Comments and Position Statement} [DOCX]. \emph{ERCOT Market Rules}. 

\bibitem{ercot_nprr1186_issue_page}
Electric Reliability Council of Texas (ERCOT). (2023). \emph{NPRR1186 Issue Page} [Web page]. \emph{ERCOT Market Rules}. 

\bibitem{puct_53298_2023}
Public Utility Commission of Texas (PUCT). (2023). \emph{Project No. 53298: Commission Filing on ERCOT RTC+B / NPRR 1186 Context} [PDF]. \emph{PUCT Interchange}. 

\bibitem{ercot_rtcb_overview_2025}
Electric Reliability Council of Texas (ERCOT). (2025). \emph{Real-Time Co-Optimization + Batteries (RTC+B): Battery Overview} [PDF]. ERCOT Market Design \& Development presentation. \emph{ERCOT}. 

\bibitem{ercot_rtcb_golive_2025}
Electric Reliability Council of Texas (ERCOT). (2025). \emph{ERCOT Goes Live with Real-Time Co-optimization Plus Batteries (RTC+B)} [Press release]. \emph{ERCOT}. 

\bibitem{ercot_nprr1268_asdc_mod_2025}
Electric Reliability Council of Texas (ERCOT). (2025). \emph{NPRR 1268: RTC -- Modification of Ancillary Service Demand Curves} [DOCX]. \emph{ERCOT Market Rules}. 

\bibitem{ercot_nprr1214_rdpa_2023}
Electric Reliability Council of Texas (ERCOT). (2023). \emph{NPRR 1214: Reliability Deployment Price Adder Fix to Provide Locational Price Signals, Reduce Uplift and Risk} [DOCX]. \emph{ERCOT Market Rules}. 

\bibitem{ercot_nprr1311_2026}
Electric Reliability Council of Texas (ERCOT). (2026). \emph{NPRR 1311: TAC Report (ASDC / Scarcity Pricing Updates)} [DOCX]. \emph{ERCOT Market Rules}. 

\bibitem{utilitydive_nprr1186_2024}
Utility Dive. (2024). \emph{Texas regulators nix proposed battery rule over concerns it could destabilize ERCOT’s ancillary services market} [News article]. \emph{Utility Dive}. (Secondary reporting; cited for stakeholder narrative only, not technical definitions). 

\bibitem{angrist_pischke_2009}
Angrist, J. D., \& Pischke, J.-S. (2009). \emph{Mostly Harmless Econometrics: An Empiricist’s Companion}. Princeton University Press.

\bibitem{imbens_rubin_2015}
Imbens, G. W., \& Rubin, D. B. (2015). \emph{Causal Inference for Statistics, Social, and Biomedical Sciences: An Introduction}. Cambridge University Press.

\bibitem{hernan_robins_2020}
Hern{\'a}n, M. A., \& Robins, J. M. (2020). \emph{Causal Inference: What If}. Chapman \& Hall/CRC. (Also available as an open-access online text.)

\bibitem{abadie_2005}
Abadie, A. (2005). Semiparametric difference-in-differences estimators. \emph{Review of Economic Studies, 72}(1), 1--19.

\bibitem{wolak_2003}
Wolak, F. A. (2003). Measuring unilateral market power in wholesale electricity markets: The California market, 1998--2000. \emph{American Economic Review, 93}(2), 425--430.

\bibitem{embrechts_kluppelberg_mikosch_1997}
Embrechts, P., Kl{\"u}ppelberg, C., \& Mikosch, T. (1997). \emph{Modelling Extremal Events for Insurance and Finance}. Springer.

\bibitem{nerc_bal_001_2}
North American Electric Reliability Corporation (NERC). (2020). \emph{BAL-001-2: Real Power Balancing Control Performance} [Standard]. \emph{NERC}. 

\bibitem{nerc_era_2023}
North American Electric Reliability Corporation (NERC). (2023). \emph{Electric Reliability Assessment} [Report]. \emph{NERC}.

\bibitem{ercot_uri_2021}
Electric Reliability Council of Texas (ERCOT). (2021).
\emph{Review of February 2021 Extreme Cold Weather Event}.
Austin, TX.

\bibitem{rosenbaum_rubin_1983}
Rosenbaum, P.~R., \& Rubin, D.~B. (1983).
The central role of the propensity score in observational studies for causal effects.
\emph{Biometrika}, 70(1), 41--55.

\bibitem{rosenbaum_2002}
Rosenbaum, P. R. (2002). \emph{Observational Studies} (2nd ed.). Springer.

\bibitem{manski_1990}
Manski, C. F. (1990). Nonparametric bounds on treatment effects. \emph{The American Economic Review, 80}(2), 319--323.

\bibitem{manski_2003}
Manski, C. F. (2003). \emph{Partial Identification of Probability Distributions}. Springer.

\bibitem{nerc_inverter_based_resources}
North American Electric Reliability Corporation (NERC). (2021). \emph{Inverter-Based Resource Performance and Modeling Task Force Report} (and associated IBR reliability guidance). NERC.

\bibitem{nerc_reliability_primer}
North American Electric Reliability Corporation (NERC). (2022). \emph{Reliability Guideline: Performance, Modeling, and Verification of Inverter-Based Resources}. NERC.

\bibitem{mcneil_frey_embrechts_2015}
McNeil, A. J., Frey, R., \& Embrechts, P. (2015). \emph{Quantitative Risk Management: Concepts, Techniques and Tools} (Revised ed.). Princeton University Press.

\bibitem{schmidt_etal_2019}
Schmidt, O., Hawkes, A., Gambhir, A., \& Staffell, I. (2019). The future cost of electrical energy storage based on experience rates. \emph{Nature Energy, 4}(11), 1--11.

\bibitem{keith_etal_2022}
Keith, D. R., Nanda, J., \& Das, S. R. (2022). Techno-economic analysis of grid-scale energy storage: A review of cost, performance, and degradation considerations. \emph{Journal of Energy Storage, 55}, 105xxx.\footnote{Replace the article number if you want exact pagination once you pull the final PDF metadata.}

\bibitem{mascolell_1995}
Mas-Colell, A., Whinston, M. D., \& Green, J. R. (1995). \emph{Microeconomic Theory}. Oxford University Press.

\bibitem{wolak_2000}
Wolak, F. A. (2000). An empirical analysis of the impact of hedge contracts on bidding behavior in a competitive electricity market. \emph{International Economic Journal, 14}(2), 1--39.

\bibitem{ercot_nprr1186_docx}
Electric Reliability Council of Texas (ERCOT). (2023, June 22).
\emph{1186NPRR-01 Improvements Prior to the RTC+B Project for Better ESR State of Charge Awareness}
[DOCX]. ERCOT. \\
(ERCOT posting; describes NPRR1186 readiness improvements and SoC awareness focus.) \\
Available: \url{https://www.ercot.com/files/docs/2023/06/22/1186NPRR-01%20Improvements%20Prior%20to%20the%20RTCB%20Project%20for%20Better%20ESR%20State%20of%20Charge%20Awareness%20062223.docx}

\bibitem{ercot_nprr1186_overview_pptx}
Electric Reliability Council of Texas (ERCOT). (2023, July 6).
\emph{NPRR 1186 Overview} [PowerPoint slides]. ERCOT. \\
Available: \url{https://www.ercot.com/files/docs/2023/07/06/NPRR_1186_Overview_JulyROS_v0.pptx}

\bibitem{ercot_current_nodal_protocols}
Electric Reliability Council of Texas (ERCOT). (2026).
\emph{Current Protocols -- Nodal} [Webpage]. ERCOT Market Rules. \\
(Provides current Nodal Protocol sections and effective dates.) \\
Available: \url{https://www.ercot.com/mktrules/nprotocols/current}

\bibitem{yesenergy_rtcb_market_changes}
Yes Energy. (2025).
\emph{The ERCOT RTC+B Program: What to Know and How to Prepare} [Webpage]. \\
(Industry summary of RTC+B launch timing and market-facing changes; use as secondary context.) \\
Available: \url{https://info.yesenergy.com/ercot-rtcb-market-changes}

\bibitem{gridstatus_asdc_dataset}
GridStatus.io. (n.d.).
\emph{ERCOT AS Demand Curves (DAM and SCED)} [Dataset webpage]. \\
(Third-party aggregation of ASDC series; use for cross-checks, not as canonical source.) \\
Available: \url{https://www.gridstatus.io/datasets/ercot_as_demand_curves_dam_and_sced}

\bibitem{ercot_price_dashboard}
Electric Reliability Council of Texas. (2026). \emph{Real-Time Market Clearing Prices for
Capacity}. ERCOT Public Dashboard. Retrieved from \url{https://www.ercot.com/mktinfo/prices}.

\bibitem{ercot_mcpc_dashboard}
Electric Reliability Council of Texas. (2026). \emph{Real-Time Market Clearing Prices for
Capacity}. ERCOT Public Dashboard. Retrieved from \url{https://www.ercot.com/gridmktinfo/dashboards/rtmarketclearingpricescapacity}.

\bibitem{ercot_as_definitions}
Electric Reliability Council of Texas. (2025). \emph{Ancillary Services Handout}. Retrieved
from ERCOT ancillary service definitions.

\bibitem{ercot_ancillary_study}
Electric Reliability Council of Texas. (2024). \emph{Ancillary Services Study}. Retrieved from
ERCOT white paper on AS products.


\end{thebibliography}

\end{document}